\chapter{Sobre a coleção Índigo}

Nas últimas décadas, narrativas adotadas tanto por setores do campo conservador quanto progressista criaram preconceitos, estereótipos e polarizações a respeito do Estado de Israel, do sionismo e do judaísmo. A percepção do lugar que Israel passou a ocupar na conjuntura política e no imaginário social brasileiro embasou a formação do Instituto Brasil"-Israel, o \textsc{ibi}, que se propõe a expor a complexidade do país e a pluralidade da comunidade judaica, abrindo caminhos para o diálogo e a diversidade de opiniões.

Para lidar com o assunto e aprofundar as discussões, o \textsc{ibi} considera a produção de conhecimento acadêmico ou, de forma mais ampla, intelectual, de suma importância: muitas vezes, é nesse contexto que são construídos debates que, mais tarde, ganham importância na sociedade.

A falta de livros, dissertações e teses em português sobre temas relacionados a Israel incentivou essas publicações. São textos escolhidos pela comissão editorial do \textsc{ibi} em parceria com a Ayllon Editora, e trazem ideias e pensadores que contribuem na qualificação do debate público. Bem vindos à Coleção Índigo.

\medskip

\hfill\textit{Instituto Brasil-Israel}