\chapter*{}
\thispagestyle{empty}
\vfill{}

\begin{flushright}
\textit\small{Yitzhak Rabin não era um homem carismático, mas sim um capitão lógico e
capaz. Ele não havia sido brindado com a paixão poética de Ben"-Gurion, nem com
a cálida graciosidade de Levi Eshkol. Ele não tinha a simplicidade arrebatadora de Golda Meir, nem a energia
populista de Menachem Begin. A multidão nunca respondeu a ele cantando \textit{Rabin, Rabin}. Comportando"-se como um engenheiro cuidadoso e um navegador preciso, sua personalidade incorporava o espírito de uma nova Israel, um país que
buscava não a redenção, mas soluções.}\\
\smallskip
\textsc{amos oz, 1996}
\end{flushright}

\chapter[Prólogo]{Prólogo \subtitulo{A morte de Yitzhak Rabin,\\ a vida de Yitzhak Rabin}}

A maioria das mortes representa simplesmente o final de uma vida. Um
assassinato político, entretanto, é distinto de qualquer outra morte. É
uma morte que adquire seu próprio significado, uma morte com
consequências. Um assassinato não é tão somente o ponto final na vida de
uma pessoa, mas o início de uma nova realidade, criada pela própria
morte. Conforme a morte violenta for lembrada e comemorada, o líder
assassinado muitas vezes se tornará o tema de uma nova mitologia,
lançando uma nova perspectiva sobre sua vida e sua carreira.

Na sequência do assassinato de Yitzhak Rabin em 1995, um público
israelense chocado saiu em busca de contextos e precedentes. O
assassinato de Abraham Lincoln foi enfaticamente lembrado; O poema ``O
Captain! My Captain!'' de Walt Whitman foi novamente traduzido para o
hebraico e para ele foi composta uma nova melodia. Mas a analogia estava
errada. O corpo do capitão de Whitman jazia no navio quando este chegou
ao porto. Lincoln havia completado sua missão e seu assassinato foi um
ato de vingança contra suas realizações. Uma analogia muito mais próxima
seria a tentativa da direita radical francesa e dos colonos argelinos de
assassinar Charles de Gaulle, para interromper as negociações de paz com
a Frente de Libertação Nacional (\textsc{fln}).\footnote{Michael Karpin e Ina Friedman, \textit{Assassinato em nome de Deus: o plano para matar
Yitzhak Rabin}. Tel Aviv: Zemorá"-Bitan, 1999, p. 40 {[}em hebraico{]}.} Se tivessem sido bem"-sucedidos,
o assassinato poderia ter abortado a solução para o problema
argelino. Fato é que o assassino de Rabin, Yigal Amir, um judeu ortodoxo
fanático, cuja família era de origem iemenita, inspirou"-se em Jean"-Marie
Bastien"-Thiry, o oficial francês acusado da tentativa de assassinar de Gaulle e executado em 1963. 
Amir via uma semelhança entre a situação
francesa no auge da crise argelina e as dificuldades enfrentadas por
Israel nos primeiros anos da década de 1990. Para ele, Rabin era a
versão israelense de de Gaulle: um herói de guerra que havia se desviado
do rumo certo e tinha de ser assassinado antes que se abdicasse de uma
valiosa parte do território nacional.

Há uma outra analogia interessante: o assassinato do presidente John F.
Kennedy. Assim como no caso de Kennedy, houve incitamento contra Rabin
antes de sua morte, incluindo acusações de traição, entre outras
piores. Antes da chegada de Kennedy a Dallas, local onde foi
assassinado, circularam panfletos com sua foto alegando ser ele
``procurado por atividades traiçoeiras contra os Estados
Unidos''.\footnote{Arthur M. Schlesinger Jr., \textit{A Thousand Days: John F. Kennedy in the
White House}. Boston: Houghton Mifflin, 1965, p. 1029. {[}Ed. bras. \textit{Mil dias: John Fitzgerald Kennedy na Casa Branca}. Rio de Janeiro: Civilização Brasileira, 
1966. 2 vols.{]}} De forma similar, o assassinato de Kennedy gerou
uma sensação de nostalgia por uma era gloriosa que havia se perdido, a
imagem e o mito de Camelot que supostamente havia existido. O escritor
Norman Mailer escreveu: ``Durante algum tempo sentimos que o país era
nosso. Agora pertence novamente a eles.''\footnote{\textit{Ibid}., p. 1027.} Após o assassinato
de Rabin, seus seguidores, a ``geração das velas'' e muitos outros também
sentiram a falta do que foi visto como a era de ouro da história e da
política israelenses. Milhares de jovens portando velas acesas
mantiveram"-se em vigília próximo à residência de Rabin e no local de sua
morte. Nos vinte anos seguintes houve uma manifestação anual em 4
de novembro, o dia do assassinato, com a participação de uma multidão.
No final de 2015, em antecipação ao vigésimo aniversário do assassinato,
era evidente em Israel a onda de saudades de Rabin, refletindo tanto a
sensação de perda quanto a insatisfação com a liderança do país naquele
momento e sua incapacidade de lidar com o endêmico problema palestino.

O homicídio de Rabin também ressaltou o forte contraste entre o
``nós'' e ``eles'' na sociedade israelense. Amir assassinou alguém cuja
vida e carreira representavam a essência da elite formadora da sociedade
israelense: judeus oriundos da Europa oriental, o movimento trabalhista,
o Palmach (unidade militar de elite anterior à criação de Israel) e as
forças armadas; um homem laico, do norte de Tel Aviv. Os anos que
antecederam sua morte foram definidos por uma espécie de guerra
cultural: um choque entre os colonos, a direita radical e grande parte
da comunidade ortodoxa frente ao setor laico e moderado do público
israelense, não somente em relação ao processo de paz mas também a respeito
dos rumos mais amplos do país. Assim como com o assassinato de Kennedy,
o de Rabin provocaria, por anos, um impacto dramático nessa guerra
cultural. Mas, apesar de todas essas e outras consequências,
o que definiu o legado de Rabin não foi sua morte, mas sua vida --- suas ações e decisões. 
O impacto e o legado de Kennedy foram delineados pela Crise
dos Mísseis de Cuba, o episódio da Baía dos Porcos, o discurso de
Berlim, o envolvimento na Guerra do Vietnã e a aura de \textit{glamour}
por ele criada. O legado de Lincoln foi definido pelo fim da escravidão
e pela preservação da União e deu aos Estados Unidos um modelo de poder
presidencial. O legado de Rabin é definido pela política de paz de seu
segundo mandato, pelas decisões ousadas que tomou em relação às negociações
com a Síria e com os palestinos e pela elevada qualidade de sua liderança.

A vida de Rabin é um conto fascinante de um israelense nato que cresceu
junto à elite que precedeu a fundação desse Estado, movendo"-se através dos agora
conhecidos estágios da escola do movimento trabalhista, uma escola
agrícola, o Palmach, a guerra de 1948 e uma carreira militar. Os
talentos e a perseverança de Rabin --- e o ocasional golpe de sorte ---
conduziriam Rabin ao topo da pirâmide militar e finalmente ao posto de
primeiro"-ministro. Mas não seria uma ascensão fácil nem tranquila. Rabin
não possuía o carisma de líderes como Moshe Dayan e Yigal Alon, que se
destacaram desde a juventude. Rabin ascendeu lentamente, tornando"-se um
verdadeiro líder somente na década de 1980. Seu primeiro mandato, na
década de 1970, foi prejudicado por sua incapacidade de atrair o público
israelense. Somente após sua impressionante performance como ministro da
Defesa, na década de 1980, promovendo uma combinação exclusiva de autoridade e
credibilidade, permitiu seu retorno do exílio político para
reconquistar o posto de primeiro"-ministro.

Em seu segundo mandato, sua liderança transformou"-o em um estadista.
Rabin demonstrou habilidade para tomar decisões ousadas, históricas,
para ir até mesmo contra seus próprios instintos e conduzir o povo. E o
sucesso de Rabin continua a ilustrar, até hoje, um aspecto crucial da
política israelense atual: que uma política de paz efetiva pode ser
aplicada --- por um líder de centro, confiável, detentor de credenciais
na área de segurança capazes de persuadir uma população israelense
ansiosa para fazer concessões e assumir os riscos necessários para o avanço
em direção à paz. Em um país que ainda enfrenta os mesmos problemas
fundamentais da era Rabin, sente"-se ainda a falta de um líder da
estatura e qualidades de Yitzhak Rabin.

\chapter[A criação de um soldado, 1922--48]{A criação de um soldado, 1922--48}
\markboth{A criação de um soldado}{}

``Com exceção de sua inteligência e tenacidade, Rabin não tinha nada para ser
embaixador. Era taciturno, tímido, introspectivo e desdenhava
conversa fiada; possuía muito poucos dos atributos normalmente
associados à diplomacia. Entediava"-se com pessoas repetitivas e chegava
a ofender"-se com banalidades; infelizmente, para Rabin, ambos não eram
raridades em Washington. Ele odiava a ambiguidade, cerne da diplomacia.
Mas sua integridade e sua capacidade analítica brilhante, que permitiam
ir ao amago do problema, eram incríveis.''\footnote{Henry Kissinger, \textit{White House Years} {[}Anos na Casa Branca{]}, vol. 1. Boston: Little,
Brown and Company, 1979, p. 355.}
Foi assim que Henry Kissinger hábil e sutilmente descreveu Yitzhak
Rabin, com quem trabalhou intimamente em Washington de 1969 a 1973.
Essas qualidades, que o tornaram um embaixador improvável mas muito
eficiente, faziam dele um político ainda mais improvável. Mas algumas
dessas mesmas qualidades ajudam a explicar a transformação de um
político esquisito em um estadista impressionante.

Rabin nasceu em Jerusalém em 1922, filho de Rosa Cohen e Nechemia
Rubichev (depois Rabin). Sobre sua infância, escreveu: ``Meu trajeto foi
decisivamente determinado pela personalidade de meus pais, pela
inspiração de nosso lar e da escola onde estudei. Durante minha infância
me vi praticamente guiado para uma vida ligada à agricultura, uma vida
no \textit{kibutz}; se houvessem me dito que me tornaria um militar, haveria
reagido como se fosse algo ridículo''.\footnote{Yitzhak Rabin, 
\textit{O lar de meu pai}. Tel Aviv: Hakibutz Hameuchad, 1970, p. 8 {[}em hebraico{]}.} Ambos
os pais de Rabin nasceram no império russo, e foram radicalizados por
sua sinistra autocracia, chegando à Palestina logo após o final da
Primeira Guerra Mundial. Rosa, apelidada de ``Rosa, a Vermelha'', era a
dominante. Sua personalidade marcante está ilustrada em uma fotografia
que a mostra marchando em uma parada de 1\textsuperscript{o} de maio em
Tel Aviv, de queixo erguido, seu rosto e seus olhos expressando
determinação. Rosa nasceu em Gomel em 1890, filha de Yitzhak Cohen, um
próspero judeu ortodoxo antissionista, de quem Rabin herdaria o nome.
Desde jovem destacou"-se como uma individualista determinada,
antissionista, de convicções esquerdistas e populistas. Era avessa a
grupos estruturados, organizados e, portanto, não se filiou nem aos
Revolucionários Sociais Russos nem ao antissionista Bund Judaico, de
esquerda. Rompendo com o padrão de uma jovem de família judaica ortodoxa
tradicional, Rosa graduou"-se em uma escola politécnica russa,
recusando"-se a aceitar o apoio financeiro de seu próspero pai e
escapando de casa para assistir aulas durante o shabat. Seu radicalismo
de esquerda foi canalizado para um populismo de estilo russo, ajudando
os pobres e necessitados. Vivia entre os trabalhadores russos e era
adorada por eles, cortando lenha nas florestas pertencentes ao
grão"-príncipe, que eram arrendadas a sua família. Era um estilo de vida
perigoso, que a colocava na mira da polícia secreta czarista e, na
sequência, da comunista. Uma genuína radical, desapontou"-se com o regime
comunista. Rosa dirigia uma fábrica em São Petersburgo (depois
Leningrado) que se transformou em uma indústria de munições. Em 1919,
quando foi demitida por sua recusa em filiar"-se ao partido, os
trabalhadores entraram em greve. Ela se viu em dificuldades, sem
trabalho e assinalada como perigosa em um período politicamente
tumultuado.

Não sendo sionista, Rosa decidiu visitar sua família em Jerusalém e
verificar se poderia encontrar seu lugar na Palestina. Escreveu em
ídiche para Berl Katznelson, um dos líderes do movimento trabalhista
que conhecia através de suas conexões familiares, e pediu que a
aconselhasse sobre ``a possibilidade de a Terra de Israel solucionar o
seu problema.'' Preocupada com a ideia de abandonar uma vida que
conhecia, Rosa lhe perguntou se ele achava que ela conseguiria se
adaptar. ``Ao viajar para a Terra de Israel tive de romper com um estilo
de vida para abraçar outro; não há caminho de volta'', escreveu.\footnote{Carta de Rosa Cohen
escrita em 11 nov. 1919. Coleção privada.} Seu tio, Mordechai Ben
Hillel Hacohen, havia se mudado com sua família para a Palestina em 1903
e vivia em Jerusalém; Rosa planejou morar com ele por um tempo. Em
dezembro de 1919, ela embarcou em Odessa no \textsc{ss} Ruslan, um navio
que se tornaria famoso na mitologia sionista. A bordo havia um grupo de
pioneiros que se dirigia a Kinneret, um \textit{kibutz} às margens do lago
Tiberíades.\footnote{A partida de Rosa da Rússia e sua chegada na Palestina são
descritas em detalhes em um livro sobre ela, de autoria de Eliezer Smoli,
um conhecido educador e escritor do período anterior à criação do Estado
de Israel. Smoli foi o professor de Rabin no primeiro grau e decidiu
escrever um livro sobre Rosa após a morte prematura desta. Ver Eliezer
Smoli, \textit{Rosa Cohen}. Tel Aviv: {[}s.n.{]}, 1940 {[}em hebraico{]}.} 
O \textsc{ss} Ruslan atracou em Yafo em 19 de dezembro de 1919.

Ao chegar a Jerusalém, Rosa foi apresentada por seu primo David Hacohen
a Moshe Shertok (mais tarde, Sharett), que viria a ser ministro de
Relações Exteriores e primeiro"-ministro de Israel. Sharett deu a Rosa
uma carta, para que levasse a sua irmã, membro do Kibutz Kinneret, na
qual pedia que cuidasse da recém"-chegada. Na carta, Sharett descreve Rosa
como ``uma jovem importante, uma engenheira, socialista mas não
bolchevique, que havia trabalhado durante alguns anos em uma fábrica
bolchevique perto de Petrogrado {[}\ldots{}{]} Há anos que ela não vê rostos judaicos
e está ansiosa para isso. Também está preocupada com a vida em
comunidade, com a qual não está acostumada. Está muito solitária aqui. Sente"-se
sufocada na casa dos Cohen --- você conhece o tipo de moça russa
inteligente, da alta burguesia, que rompeu todos os laços com a família
e seu círculo social e não mais os tolera''.\footnote{Carta de Sharett à sua irmã, sem data. Coleção privada.}

Ainda que tenha chegado à Palestina como uma não sionista, Rosa aos
poucos tornou"-se uma fervorosa apoiadora do movimento. Após uma breve
estada em Kinneret, mudou"-se para Jerusalém para morar com seus
parentes. Os árabes palestinos provocaram um tumulto na cidade velha de
Jerusalém em 1920, e Rosa se engajou na ajuda aos residentes judeus, como
enfermeira e combatente; havia adquirido experiência em autodefesa
durante os \textit{pogroms} na Rússia. Mudou"-se então para Haifa e atuou
na organização do trabalho judaico no porto local, obtendo sustento
através do trabalho na loja de seus parentes.

Pouco se sabe sobre os primórdios da vida do pai de Rabin, Nechemia
Rubichev. Filho de uma família pobre de um vilarejo próximo a Kiev,
engajou"-se em atividades revolucionárias contra o governo do czar.
Emigrou para os Estados Unidos para evitar a prisão, estabelecendo"-se em
St. Louis, onde trabalhou como alfaiate e atuou como ativista nos
sindicatos judaicos. Em 1917, Nechemia tentou se alistar na Brigada
Judaica, organizada por Zeev Jabotinsky para lutar ao lado dos
ingleses na Palestina, mas foi rejeitado por razões médicas devido a um
defeito na perna. Nechemia mudou seu sobrenome para Rabin e conseguiu
se alistar em outra junta. Chegou à Palestina e lá ficou. Em 1920, como
um dos membros originais da primeira encarnação da Haganá (organização
judaica de autodefesa na Palestina mandatária), Nechemia tomou parte na
defesa do bairro judeu da cidade velha de Jerusalém, atacado pelos
violentos manifestantes árabes. Foi onde conheceu Rosa, que
oferecia sua ajuda e apoio médico. Nechemia foi preso pelas autoridades
britânicas, que reprimiram as manifestações e prenderam os defensores
judeus, acusados de portarem armas. Rosa e Nechemia casaram"-se em 1921 e
estabeleceram"-se temporariamente em Haifa. Rosa mudou"-se para Jerusalém
para estar próxima de sua família às vésperas do nascimento de seu filho
Yitzhak, em 1\textsuperscript{o} de março de 1922. A jovem família
mudou"-se então para Tel Aviv em 1923, onde Rosa trabalhava em um banco e
Nechemia era funcionário da Companhia de Eletricidade Palestina. Em 1925 nasceria
sua filha Rachel.

A família viveu modestamente em uma série de apartamentos de quarto e
sala, descritos como ``espartanos'' por amigos de infância de Rabin.
Seus pais, além de trabalharem, estavam totalmente engajados em
atividades públicas: ambos estavam envolvidos com a Haganá; Nechemia
participava de atividades sindicais; Rosa era membro do conselho municipal de Tel
Aviv e de diversas outras organizações beneficentes. Aparentemente, a
família dava preferência a valores e não a emoções; as crianças
cresceram de maneira independente, e Yitzhak muitas vezes cuidava de sua
irmã. Sexta"-feira era a noite de reunião da família. O pequeno
apartamento, pobremente mobiliado, sediava numerosos encontros de
sindicalistas, membros da Haganá e visitantes de outras cidades. Rosa
era muito ativa, conhecida e respeitada, mas recusava filiações
partidárias e tampouco queria assumir algum cargo público de responsabilidade. Ela sofria
do coração, e seus filhos constantemente temiam perdê"-la. Morreria cedo,
em 1937, aos 47 anos, quando Yitzhak tinha quinze. Seu
funeral foi um grande evento público, prestigiado por milhares de
pessoas, entre os quais David Ben Gurion, líder da Agência Judaica, a
principal organização da comunidade judaica antes da criação do Estado
de Israel. Ben Gurion já era então o líder inquestionável da comunidade
judaica.

Em muitos aspectos, a infância de Rabin foi típica dos garotos
envolvidos com o movimento trabalhista da Palestina mandatária. Ele
frequentou a escola primária afiliada ao movimento trabalhista,
juntou"-se ao movimento juvenil, continuou seus estudos em uma escola
agrícola em um \textit{kibutz} a leste de Tel Aviv e depois estudou no colégio
agrícola Kadoorie, uma das melhores escolas do país, localizada na base
do monte Tabor, na Galileia. A escola Kadoorie --- fruto de doações de uma rica
família de Hong Kong, que construiu duas escolas na Palestina durante o
Mandato Britânico, uma para rapazes judeus e outra para árabes --- era famosa pelo
elevado padrão acadêmico e pelo seu código de honra. Os professores
abandonavam a sala durante as provas, confiando em que os alunos não
``colariam''. Rabin floresceu em Kadoorie; foi um desenvolvimento
tardio, afetado por dificuldades de leitura e escrita durante os dois
primeiros anos da escola primária. A questão foi resolvida com a chegada
de Eliezer Smoli, um escritor e professor inspirador. Esse
desenvolvimento tardio foi típico na vida de Rabin e se repetiria em
outras fases. Com a ajuda de Smoli, Rabin passou nos difíceis exames de
admissão da escola Kadoorie e destacou"-se como um aluno excepcional;
na formatura, recebeu um premio especial das mãos do alto comissário
britânico. Rabin teve direito a uma bolsa do governo que lhe permitiria
estudar engenharia hidráulica na Califórnia, mas os eventos na Palestina
e a eclosão da Segunda Guerra Mundial interferiram em seus planos.
Durante a Revolta Árabe (1936--1939), Rabin tornou"-se ciente dos problemas
de segurança enfrentados pela comunidade judaica e recebeu os primeiros
treinamentos no manejo de armas. Após formar"-se, não pôde ir para a
Califórnia, mas morou e trabalhou com seus amigos em vários
\textit{kibutzim}, ainda que não tenha se filiado a nenhum deles. Em suas
memórias e outros relatos sobre sua infância e juventude, Rabin menciona
o fato, mas não o explica. Aparentemente, o traço individualista de seu
caráter impediu"-o de juntar"-se ao coletivo.

Houve, entretanto, outra dimensão importante na vida de Rabin nessa
época: durante o verão, quando não estava na escola, Rabin ficava em
Jerusalém, aos cuidados do já mencionado tio de sua mãe, Mordechai Ben Hillel Hacohen.
Ben Hillel era um tipo impressionante, escritor, intelectual e rico
empresário. Sua família mais próxima e mais ampla incluía alguns dos
elementos mais proeminentes da comunidade judaica que antecedeu a
criação do Estado de Israel, o Ishuv. O filho de Ben Hillel, David
Hacohen, vivia em Haifa e tornou"-se o elemento de ligação com a
inteligência britânica durante a Segunda Guerra Mundial. Uma de suas
filhas casou"-se com Arthur Ruppin, chefe do departamento econômico da
Agência Judaica. Ben Hillel e sua família eram ligados às famílias de
Moshe Shertok, Eliyahu Golomb e Dov Hoz, alguns dos líderes mais
importantes da época. Durante suas estadias na casa do tio"-avô, Rabin ficava
em contato com um ambiente ao qual nunca esteve acostumado. A casa dos
Hacohen era espaçosa e elegante. Em contraste com o pequeno e parcamente
mobiliado apartamento de Rabin em Tel Aviv, tinha uma grande biblioteca
que ele e seu primo Rafael Rupin tinham que arrumar, como parte de
seus afazeres durante o verão. Rabin até se juntava a seu primo nas
quadras de tênis, um esporte totalmente desconhecido do ambiente
proletário de Tel Aviv.

Durante sua última estada com seus parentes em Jerusalém, Rabin enviou
uma carta reveladora a sua amiga Hanna Guri (Rivlin). Eles pertenciam a
um pequeno grupo de estudantes chamado Telem, todos alunos das mesmas
escolas em Tel Aviv e Givat Hashlosha e eram membros do mesmo movimento
juvenil trabalhista. Era um grupo coeso; seus membros discutiam questões
pessoais e genéricas com a seriedade e franqueza típica dos movimentos
juvenis da época. A carta de Rabin para Guri esclarece como ele mesmo
percebia"-se tímido e reservado, dois dos elementos que definiam sua
personalidade. Em 6 de agosto de 1937 escreveu: 

\begin{quote}
Sou eu o único membro
do Telem que fico quieto? Não importa. Isto não me exime da
responsabilidade de articular, mas há razões que dificultam a
articulação {[}\ldots{}{]} Se aqueles que se mantém em silêncio quiserem fazer parte
da sociedade, terão de expressar seus sentimentos e, se não o fizerem,
é porque a sociedade os impediu de falar; e se houve tentativas de sua
parte, foram sempre recebidas com desprezo pela sociedade {[}\ldots{}{]} Pode ser
que tenha uma sensação de inferioridade por achar que os membros não se
interessam por mim.\footnote{Carta de Ravin a Hanna Guri (Rivlin). Coleção privada.}
\end{quote}

Em 1941 Rabin alistou"-se no Palmach. Criado pela liderança judaica 
naquele mesmo ano, o Palmach, acrônimo de Unidades Hebraicas de Choque, tinha
dois objetivos. O primeiro era criar uma força militar permanente, já que a
Haganá tinha um pequeno contingente, mas não uma força militar
definitiva. Em 1941 havia grande expectativa de uma invasão alemã. O
marechal de campo alemão Erwin Rommel avançava no norte da África e,
antes que fosse derrotado pelo marechal de campo Bernard Montgomery em
El Alamein, tudo indicava que suas tropas conquistariam o Egito e
invadiriam a Palestina. As seis companhias do Palmach não tinham por
objetivo impedir o avanço do exercito de Rommel, mas simplesmente
dificultá"-lo, enquanto a comunidade judaica tentaria se defender nas
montanhas do Carmel. Naquela época, a Haganá e o Palmach colaboravam
estreitamente com os britânicos. Em junho de 1941, enquanto os
britânicos se preparavam para invadir a Síria e o Líbano controlados por
tropas leais à França de Vichy, times da Haganá foram encarregados de
apoiá"-los com ações de reconhecimento e sabotagem. Rabin se encontrava
no Kibutz Ramat Iochanan e foi consultado pelo chefe da segurança local
se estava disposto a participar como voluntário de uma missão. Rabin
concordou e foi entrevistado por Moshe Dayan, uma das jovens estrelas da
Haganá. Foi o primeiro encontro dos dois, cujos caminhos se cruzariam
frequente e significativamente, com mais aspectos negativos do que
positivos. Rabin mencionou a entrevista em suas memórias: ``Ele me
perguntou sobre o tipo de arma que poderia utilizar; disse"-lhe que
estava familiarizado com o revolver, o rifle e a granada de mão, mas
nada de maior calibre ou mais sofisticado. Fez mais algumas perguntas e
então murmurou secamente, `você serve'\,''.\footnote{Yitzhak Rabin, 
\textit{The Rabin Memoirs} {[}As memórias de Rabin{]}. Boston: Little, Brown and Company, 1979, p. 10.
\textit{The Rabin Memoirs} é a tradução inglesa das memórias de Rabin, publicadas em hebraico com o título
\textit{Pinkas Sherut}. A tradução não é
completa e há diferenças entre as versões em inglês e hebraico.} Rabin juntou"-se ao
time de Dayan quando este entrou no Líbano em 7 de junho, dando apoio à
unidade australiana encarregada da invasão. Foi nessa operação que Dayan
perdeu seu olho: enquanto olhava através de um binóculo, foi atingido
por um franco"-atirador francês e desde então usou a viseira que faria
parte de sua \textit{persona} e de sua imagem. Como membro júnior do grupo,
Rabin recebeu a missão de subir nos postes telefônicos e cortar os
cabos, sua primeira experiência em combate. Em seguida, juntou"-se ao
recém"-criado Palmach.

Rabin escalou a hierarquia até se tornar chefe de operações, --- na
verdade o braço direito de Yigal Alon, comandante do Palmach. Após a
criação do Palmach em 1941, a liderança judaica teve de lidar com a
manutenção de uma força militar. Limitações orçamentárias
levaram o Palmach a ser adotado pelo movimento \textit{kibutziano}, especialmente
o Hakibutz Hameuchad, um grupo identificado com a Facção \textsc{b} --- um dos
componentes do Mapai, partido dominante no movimento trabalhista
antes da criação de Israel e nos primórdios da política israelense. O
líder da Facção \textsc{b}, Yitzhak Tabenkin, era considerado um rival de Ben
Gurion (em 1944 a Facção \textsc{b} se separaria do Mapai e formaria um novo
partido denominado Achdut Haavodá). Ben Gurion havia se
tornado o líder do Ishuv, a comunidade judaica da Palestina mandatária, e
não apreciava o Palmach. Ele acreditava que a melhor opção
para a nova geração da comunidade judaica na Palestina durante a Segunda
Guerra Mundial era o alistamento no exército britânico que lutava na
Europa. Essa seria uma forma de contribuir com a guerra contra os
nazistas e ajudar milhares de jovens judeus a adquirir experiência
militar atuando com um grande exército moderno. Ben Gurion, um político
e um estadista, também se preocupava com a orientação política
do Palmach. De qualquer forma, as unidades do Palmach ficaram em sua
maioria nos \textit{kibutzim}, alternando"-se entre o treinamento e o trabalho.
Serviu para que a comunidade, pressionada financeiramente, pudesse
manter uma força militar permanente.

Rabin foi enviado ao primeiro curso de comandantes de esquadrão do
Palmach, liderado por Alon, um homem carismático e atraente, alguns anos
mais velho que Rabin. Natural de Kfar Tavor, uma colônia agrícola aos
pés do monte Tabor, Alon era membro do Kibutz Guinossar, localizado na
margem do lago Tiberíades e, assim como Rabin, formado na escola
Kadoorie. Este seria um dos mais importantes relacionamentos do início
da carreira de Rabin, pois Alon foi o responsável por sua rápida
ascensão: descobriu os talentos de Rabin, cultivou"-os e tornou"-o seu
oficial de operações e vice"-comandante. Em 1942, Rabin foi nomeado
instrutor e então promovido a comandante de pelotão. Um dos outros
comandantes lembrou"-se dele, anos mais tarde, como alguém
``definido por seu pensamento original, não atado a padrões
convencionais, apto a refletir, levantando temas que nem todos estavam
dispostos a aceitar, descartando pessoas conhecidas e populares e
cultivando as pessoas que o entendiam e seguiam seu caminho. Em suma,
era um jovem e sério comandante, que fazia seu trabalho, mas sem ser
considerado nada especial''.\footnote{Testemunho de Mickey Haft, sem data. Museu do Palmach.}

Em 1944, o Palmach passou de uma organização formada por companhias para
uma formada por batalhões. Rabin foi promovido a instrutor de batalhão,
o que na pratica significava atuar como vice"-comandante;
em 1945 já comandava um grande curso para comandantes de esquadrão.
Nessa posição Rabin começou a se destacar. Seu profundo conhecimento do
comando e de questões militares, bem como seu talento para o ensino
transformaram o curso em uma experiência memorável para muitos e
ajudaram a construir a reputação de Rabin. Um dos formados considerou"-o
o mais significante curso que havia feito em sua carreira, o que incluía
uma passagem pela prestigiada École de Guerre, na França.

Em 1945, Rabin viveu sua primeira experiência como comandante de uma
operação militar em larga escala. Nos seus primórdios, as atividades do
Palmach basearam"-se na colaboração com os ingleses contra a esperada
invasão alemã. Uma vez passado o perigo, a cooperação teve fim. Os
britânicos tornaram"-se rivais e as operações
do Palmach passaram a tê"-los como alvo. A imigração ilegal judaica para
a Palestina era, naquela época, o maior ponto de atrito entre o Ishuv e
as autoridades britânicas. Para o Ishuv, a ideia de que os britânicos
pudessem impedir o acesso de sobreviventes do Holocausto europeu era
abominável. Para os britânicos, a política era compatível com seu papel
de árbitros do conflito entre judeus e árabes na Palestina. Imigrantes
ilegais capturados a bordo de navios que se dirigiam à Palestina eram
internados em um campo em Atlit, ao sul de Haifa. O Palmach recebeu do
quartel"-general da Haganá a missão de atacar o campo, libertar os
detentos e distribuí"-los entre as aldeias judaicas. O comandante de
batalhão Nahum Sarig comandava a operação e Rabin era um de seus vices.
A manobra foi um sucesso. Em sua próxima missão, Rabin e seus homens
foram encarregados de destruir ferrovias britânicas, o que fizeram com
êxito e sem baixas. Sua terceira operação foi o ataque à delegacia de
polícia britânica de Jenin. A essa altura, a colaboração com os
britânicos já havia se transformado em um relacionamento hostil: os
antigos parceiros na guerra contra a Alemanha nazista eram agora vistos
como os apoiadores dos árabes palestinos, um obstáculo para a criação
de um estado judeu na Palestina. Mas Rabin feriu"-se em um acidente de
motocicleta e ficou acamado por vários meses. Enquanto se recuperava,
Rabin escreveu com frequência para sua irmã Rachel, que ainda jovem
havia se filiado ao Kibutz Menara, na fronteira com o Líbano, onde vive
até hoje. As cartas de Rabin refletem sua relação calorosa com a irmã e
um lado leve e bem"-humorado que compartilhava somente com seu circulo
mais íntimo --- em contraste com sua \textit{persona} mais habitual, séria
e rude. Em 17 de janeiro de 1946 escreveu: ``Considerando que meu tempo
`limitado' é agora racionalmente dividido entre a ociosidade absoluta e
a semiociosidade, finalmente encontrei tempo para escrever o
trabalho de literatura que você agora tem à frente. Primeiramente, devo
detalhar o objeto principal: minha gloriosa perna envolta em gesso, uma
das criações honoráveis do Dr. Pizer''. Outra carta é escrita como
paródia de estilo bíblico e uma outra é assinada ``Seu irmão manco''.

Em 29 de junho de 1946, data conhecida na história israelense como o Sábado
Negro, as autoridades britânicas prenderam centenas de comandantes da
Haganá e líderes do Ishuv em um campo de detenção em Rafah, no norte do
Sinai. Rabin, ainda engessado, estava entre os presos e passaria vários
meses na prisão britânica, durante os quais sua perna continuaria sendo
tratada. O escritor israelense Nathan Shacham, também um membro do Palmach
encarcerado, impressionou"-se com a presença séria, dominante de Rabin. Ele
registrou seu primeiro encontro com Rabin em Rafah:

\begin{quote}
A admiração que senti por ele quando o encontrei pela primeira vez não
diminuiu quando o conheci melhor. Apesar de não gostar de seu linguajar
rude e ofensivo, ainda que não fosse intencional, e dos gestos desdenhosos de
sua mão (às vezes até mais ofensivos), e apesar de me opor firmemente a
várias de suas declarações públicas, em todos esses anos, meu respeito
por ele não diminuiu. Sua honestidade compensava todas as suas falhas {[}\ldots{}{]}
eu o vi pela primeira vez no campo de detenção de Rafah {[}\ldots{}{]} estava atrás
da cerca, para encontrar um importante membro do setor de segurança,
preso no outro campo {[}\ldots{}{]} Ao seu lado mancava um jovem louro, com a perna
engessada {[}\ldots{}{]} Senti"-me desconfortável, sob o olhar gélido do jovem
desconhecido {[}\ldots{}{]} O olhar severo do jovem louro obrigou"-me a ser direto e
conciso. Mais tarde me disseram que ele era Yitzhak Rabin, um jovem
comandante cuja reputação o precedia. Eu não era um jovem inocente da
resistência que se impressionava com pessoas famosas {[}\ldots{}{]} ainda assim,
fiquei profundamente impressionado com a seriedade daquele jovem louro
de olhos azuis {[}\ldots{}{]} Ele não proferiu palavra, mas encarou"-me com seu olhar
frio --- não exatamente meu rosto, mas olhando para algum lugar, próximo
à direita de minha orelha direita {[}\ldots{}{]} De qualquer forma, enquanto olhava
para aquele ponto à direita de minha orelha --- onde sempre se fixava
quando ouvia os outros, a não ser que tivesse chegado à conclusão que o
outro havia dito o suficiente e então o encarava, dizendo"-o claramente e
sem palavras --- ele conseguia, assim pensava eu, ler"-me como um livro
aberto e viu em mim aquela frivolidade que impedia que fossemos amigos.
Homens de ação não são amigos de pessoas com imaginação, que vem suas
experiências de fora.\footnote{Nathan Shacham, \textit{Shalom Chaverim: Pages from a Private Collection} {[Shalom Chaverim: páginas de uma coleção privada]}. Tel Aviv: Dvir, 2004, p. 135--6 {[}em hebraico{]}.}
\end{quote}

No final de 1946, após cinco meses de detenção, Rabin foi libertado de
Rafah quando os britânicos decidiram encerrar aquela fase do conflito
com a liderança do Ishuv. Rabin foi nomeado comandante do segundo
regimento do Palmach, que se expandia como parte da preparação do Ishuv
para a já antecipada colisão inevitável com os árabes palestinos e os
estados árabes. Ben Gurion criou a pasta de segurança na Agência Judaica
e tornou"-se, \textit{de facto}, o ministro da Defesa do Ishuv. Ele havia
chegado à conclusão, anos antes, de que a criação de um estado judeu na
Palestina levaria à guerra com os palestinos e que os árabes se
juntariam a eles. Ele assumiu a pasta da Defesa para estar pronto para
essa guerra, e os preparativos intensificaram"-se. Esse esforço marcou a
tensão existente entre a liderança do Palmach e os oficiais que haviam
retornado da atuação junto ao exército britânico. A tradição do Palmach
enfatizava a espontaneidade e o combatente como indivíduo, enquanto a
tradição da Brigada Judaica no exército britânico enfatizava a atuação
bem ensaiada. Ben Gurion simpatizava com a tradição britânica. Ele
desdenhava os valores militares do Palmach e suspeitava de sua relação
com seus rivais dos movimentos trabalhista e \textit{kibutziano}. Essas tensões
mantiveram"-se durante e logo após a guerra de 1948. No início dos anos
1950, Ben Gurion não perderia nenhuma oportunidade de promover os
oficiais oriundos do exército britânico e outros exércitos regulares,
postergando a promoção de oficiais do Palmach. Somente mais tarde, na
década de 1960, os oficiais do Palmach que continuaram nas forças
armadas assumiriam um papel primordial na sua liderança.

Rabin não ficou muito tempo no posto de comandante de batalhão. Alon,
que tinha em alta conta a capacidade de Rabin como planejador militar e
oficial de comando, levou"-o ao quartel"-general do Palmach como oficial
de operações e seu vice. O primeiro comandante do Palmach era Yitzhak
Sadeh, uma figura heróica, mais velho e dotado de experiência militar
russa. A liderança política preferia Alon como comandante do Palmach e
mantinha Sadeh como comandante em campo. No quartel"-general do Palmach,
a principal responsabilidade de Rabin no final de 1947 e início de 1948
era o planejamento dos comboios de suprimentos militares e civis que
viajavam ao longo da estrada sinuosa das colinas da Judeia, ascendendo
da costa até Jerusalém e outras aldeias e povoações judaicas sitiadas.
Esta continuou sendo a responsabilidade de Rabin até ser nomeado
comandante da Brigada Harel do Palmach. Assegurar os comboios
significava defendê"-los, assim como tentar capturar as áreas e aldeias
ao longo da estrada para Jerusalém. Em determinados momentos, a brigada
de Rabin foi enviada a Jerusalém para participar da batalha pela cidade.

A Guerra de Independência de Israel foi um marco na carreira política de
Rabin e um dos períodos formadores mais importantes de sua vida,
transformando"-o de oficial mediano do Palmach em um dos mais conhecidos
oficiais do alto escalão das Forças de Defesa de Israel (\textsc{idf}). Rabin participaria em algumas das
campanhas mais difíceis e importantes da guerra, das quais emergiria com uma
rica experiência política e militar. A luta em Jerusalém,
especialmente na estrada para Jerusalém onde sua brigada sofreria
pesadas perdas, causariam um profundo impacto de longo prazo no jovem
oficial.

A Guerra de Independência durou mais de um ano e foi um esforço caro e
difícil. O Ishuv perdeu 1\% de sua população, seis mil vítimas entre seus
seiscentos mil habitantes. Em certos momentos, especialmente no início da
primavera de 1948, os judeus pareciam perder a guerra. A
guerra civil na Palestina eclodiu oficialmente após a Resolução da
Partilha da Organização das Nações Unidas (\textsc{onu}) 
em 29 de novembro de 1947, mas a onda de violência já
havia se iniciado antes. A guerra civil entre as comunidades árabe e
judaica durou de novembro de 1947 até 15 de maio de 1948. Durante essa
primeira fase, forças palestinas irregulares e voluntários árabes
atacaram povoados judaicos isolados e emboscaram veículos judaicos, com
um empenho especial para bloquear o acesso a Jerusalém. A cidade havia
sido internacionalizada no Plano de Partilha, mas ambos os lados
investiram seus esforços tentando garantir o controle sobre a cidade
entendendo, corretamente, que Jerusalém era a chave para o futuro do
país. Para os palestinos era %\protect\hypertarget{OLE_LINK1}{}{}
relativamente fácil bloquear o acesso a Jerusalém: eles controlavam o
topo das montanhas que davam para o leito seco do rio, justamente onde
passava a estrada sinuosa que levava até lá. Obter reforços e
suprimentos para a cidade sitiada representava um enorme desafio para a
liderança política e militar do Ishuv, e o trabalho de Rabin no quartel
general do Palmach consistia no planejamento desses comboios. Em 15 de
abril de 1948, Rabin foi nomeado comandante da nova Brigada Harel (não
era na verdade uma brigada completa, pois compunha"-se somente de dois e
não de três batalhões). Durante os dois meses seguintes, Rabin e seus
homens engajaram"-se em algumas das mais ferozes batalhas da guerra.

Missões impressionantes e perigosas foram cumpridas por Rabin e seus
homens, certamente fruto da guerra, mas também de manobras políticas
entre os militares. Lidar com dirigentes difíceis e com a discórdia dentro
e fora das fileiras era um dos muitos desafios e, em varias ocasiões
durantes aquelas duras semanas, Rabin questionou a pertinência das
ordens que recebeu.

Um dos dois comandantes de batalhão de Rabin, Yosef (Yosef'le) Tabenkin
--- filho de Yitzhak Tabenkin, o líder do Hakibutz Hameuchad e da Achdut
Haavodá (uma das facções políticas do movimento trabalhista), e um
príncipe do movimento \textit{kibutziano} --- considerava"-se superior a Rabin e
recusava"-se a aceitar sua autoridade.\footnote{Esses testemunhos estão nos arquivos do Museu do Palmach.} Rabin e Tabenkin haviam
se enfrentado no passado e, na verdade, a liderança havia nomeado Rabin
em parte por sua incapacidade de trabalhar com Tabenkin. Tabenkin
simplesmente não cumpria as ordens e instruções dadas por Rabin, e seu
regimento não atuava a maior parte do tempo como uma unidade da Brigada
Harel, mas sim de forma independente. Era uma das razões pelas quais a
brigada não era utilizada como uma entidade completa, mas como uma série
de unidades menores. Em Jerusalém, a Brigada Harel foi enviada para
apoiar os esforços da Brigada Etzioni, comandada por David Shealtiel,
outro comandante difícil de ser controlado e de competência militar
duvidosa. Adicionava"-se a isso o fato complicador de que até 15 de maio o
exército britânico ainda estava presente em Jerusalém e, em diversas
ocasiões, obstruiu os esforços das \textsc{idf}. Após 15 de maio, a Legião Árabe,
força militar jordaniana, entrou na disputa. Era o melhor dos exércitos
árabes, e o rei Abdullah da Jordânia estava determinado a apossar"-se da
maior parte possível de Jerusalém.

Rabin e seus homens tiveram sucesso em sua missão na parte sul da
cidade, mas não na parte norte, nem em seus acessos. Shealtiel
pressionou Rabin a apoiar seus esforços para salvar o bairro judaico na
cidade velha de Jerusalém; Rabin acreditava que o plano de Shealtiel era
totalmente equivocado mas, como compartilhavam o comando, não podia
recusar seu pedido. Assim, os homens de Rabin protestaram mas
participaram da tentativa de Shealtiel, um fracasso retumbante. Os
combatentes da Harel retiraram"-se do bairro judaico e este finalmente
rendeu"-se à Legião Árabe. O episódio se tornou uma
controvérsia entre Rabin e Shealtiel e deixou em Rabin uma cicatriz
permanente.

A luta na estrada para Jerusalém cobrou da Brigada Harel um preço
enorme. O nível de baixas era de 50\% e os homens de Rabin ficaram
atordoados e em estado de choque. O escritor Yoram Kaniuk, um de seus
soldados, escreveu em seu romance \textit{1948}: ``antes de partir para a
batalha, costumávamos dizer aos mais velhos no \textit{kibutz}: `cavem as covas,
porque estamos a caminho'\,''.\footnote{Yoram Kaniuk, \textit{1948}. Tel Aviv: Yedi'ot Achronot Books, 2010, p. 131 {[}em hebraico{]}.} Rabin descreveria em suas
memórias que os combatentes da Harel ficaram tão atordoados que, em 14
de maio de 1948, ``quando um velho rádio no Kibutz Maaleh Hachamisha,
localizado a algumas milhas de Jerusalém, nos transmitiu a voz de Ben
Gurion proclamando a criação do Estado de Israel, nossas tropas exaustas
esforçaram"-se para captar o que significavam suas palavras. Um soldado,
encolhido em um canto e completamente exausto, abriu parcialmente um olho e
pediu: `Ei, desliguem. Estou desesperado por dormir. Podemos escutar
amanhã estas belas palavras' {[}\ldots{}{]} Nenhum de nós jamais havia imaginado que
receberíamos assim o nascimento de nosso estado''.\footnote{Rabin,
\textit{The Rabin Memoirs}, \textit{op}. \textit{cit}., p. 29.}

Rabin emergiu da luta por Jerusalém experiente, mas amargurado. Conforme
relatou em suas memórias, não podia entender porque a liderança do
Ishuv não havia se preparado melhor e mais cedo para a guerra
inevitável: 

\begin{quote}
Este 20 de maio de 1948 foi para mim um dia amargo, um dia
de introspecção. Durante o período dos comboios, da dura luta em
Jerusalém, antes e depois da invasão dos exércitos árabes regulares, fui
atormentado pela questão: por que esta guerra nos surpreendeu tão mal
preparados? Foi necessário?\footnote{\textit{Ibid}., p. 45.}
\end{quote}

Às vésperas da primeira trégua, em 11 de junho de 1948, Alon renomeou
Rabin. Alon e o voluntário norte"-americano coronel Mickey Marcus (Stone)
receberam de Ben Gurion o comando de uma força relativamente grande e a
missão de garantir o acesso a Jerusalém.\footnote{David Daniel Marcus 
(mais conhecido como Mickey Marcus, ou coronel Stone) é lembrado na história israelense
por duas razões: primeiramente, como coronel do exército
dos Estados Unidos, foi o voluntário de mais alta patente que veio do
exterior auxiliar as \textsc{idf} na Guerra de 1948 e recebeu de Ben Gurion um
posto do mais alto escalão. Em segundo lugar, devido a sua trágica morte
acidental. A história de sua vida foi resgatada tanto no romance \textit{The Hope} {[}A esperança{]} de
Herman Wouk quanto no filme \textit{Cast a Giant Shadow} {[}A Sombra de um
Gigante{]}, no qual Kirk Douglas atuou no papel de Marcus.} O objetivo era controlar
posições"-chave mais distantes, como a cidade palestina de Ramallah e a
fortaleza de Latrun.

Desde o início da guerra, Ben Gurion havia dado especial importância ao
controle judaico de Jerusalém e dos caminhos que levavam a ela e, antes
da trégua, queria garantir que Israel teria livre acesso à cidade. Latrun
era um ponto chave na estrada de Tel Aviv para Jerusalém; as tentativas
anteriores das \textsc{idf} para conquistá"-la haviam falhado e sua captura
tornou"-se a obsessão de Ben Gurion. Em 10 de junho, Alon e Marcus haviam
concluído que suas forças estavam exaustas e sem condições de
atacar Latrun. Também já sabiam que um caminho alternativo fora
encontrado, apelidado de Caminho de Burma, e estava sendo preparado para
uso das \textsc{idf}. Ambos os oficiais superiores temiam confrontar Ben Gurion
pessoalmente com a noticia de que não podiam e não queriam lançar mais
um ataque contra Latrun, e pediram a Rabin para fazê"-lo. Rabin sabia que
seria exposto à ira do ``Velho'', e tinha razão. Ben Gurion ficou tão
furioso que disse ao jovem Rabin que ``Yigal Alon deveria ser
executado''.

Quando o coronel Stone foi morto em um trágico acidente por um soldado
das \textsc{idf} que estava de guarda e o confundiu com um invasor, Rabin se
tornou o vice e chefe de operações de Alon. O período da primeira trégua
foi utilizado para o planejamento da fase seguinte da campanha de
controle do acesso a Jerusalém, assim como o controle da área entre
Jerusalém e Tel Aviv pertencente ao novo Estado. O comando das \textsc{idf}
estava preocupado com a possibilidade de uma ação do exército jordaniano
que pudesse atravessar essa área e ameaçar Tel Aviv e a região da costa.
Mas os planos e as preparações durante esse período foram ofuscados por
duas questões políticas complicadas, ambas envolvendo Rabin.

Uma foi o caso \textit{Altalena}. O \textit{Altalena} era um navio enviado
da França para Israel pelo Irgun, ou Etzel, acrônimo de Irgun Tzevai
Leumi (em português: Organização Militar Nacional na Terra de Israel), 
a resistência de direita afiliada ao movimento revisionista e
liderada por Menachem Begin. O navio estava carregado com armas
fornecidas pelo governo francês. Os franceses foram convencidos a dar suporte a
essa organização de direita para enfraquecer o componente esquerdista da
liderança israelense e para apoiar os esforços militares do Irgun em
Jerusalém, minando assim a posição de Abdullah, visto pelos franceses
como um agente britânico. O Irgun havia se dissolvido após a criação do
Estado de Israel, com exceção de Jerusalém, que havia sido
internacionalizada no Plano de Partilha. O \textit{Altalena} chegou à
costa israelense, ao norte de Tel Aviv, em 20 de junho de 1948 e ancorou em Tel
Aviv em 22 de junho, nas proximidades do Hotel Ritz, onde se localizava o quartel
general do Palmach. Ben Gurion exigiu a rendição total dos combatentes
que estavam no navio de maneira categórica, afirmando que, para que Israel
sobrevivesse, a autoridade do Estado não poderia ser contestada e
milícias e exércitos privados não poderiam ser tolerados. Há até hoje
muita discussão sobre o caso \textit{Altalena}, mas não se discute o fato
de que os homens do Irgun desembarcaram do navio e iniciou"-se uma luta
entre eles e a pequena força presente na sede do Palmach. A
possibilidade de uma guerra civil judaica era real e assustadora. Rabin
estava por acaso no quartel general do Palmach, visitando sua namorada
Leah. Como oficial de mais alto escalão no local, assumiu o comando da
luta contra os homens do Irgun. No final, o papel de Rabin no caso
\textit{Altalena} foi pouco importante, limitado à luta na praia, enquanto
os principais atos foram protagonizados por Alon e Ben Gurion. Este último
ordenou o bombardeio do navio pela artilharia das \textsc{idf} e o navio foi
afundado. Ele se referia com orgulho a esta ação, denominando de Canhão
Sagrado o canhão que afundou o \textit{Altalena}. Após o final da luta na
praia, foi dado a Alon o comando de uma operação mais ampla contra o
Irgun na área de Tel Aviv.

O episódio do \textit{Altalena} se manteria como um ponto controverso de
tensão entre a direita e a esquerda na política israelense das décadas
seguintes. Em uma peculiar reviravolta do destino, a relação de Ben
Gurion com Begin, líder do Irgun, melhorou após 1967. Em junho desse ano,
Begin e seu partido juntaram"-se a um governo de união nacional, no qual
mantiveram"-se até o verão de 1970, o que os levou ao centro da
política israelense. Ben Gurion já havia se aposentado e estava acima
das rotineiras disputas políticas locais. Por essa razão, sua
participação no caso foi reduzida na mitologia da direita
e sua responsabilidade no afundamento do \textit{Altalena} foi transferida
para outros; primeiramente para Yisrael Galil e, mais tarde, em meados da
década de 1990 --- quando Rabin estava sendo demonizado pela direita
israelense por ter assinado os Acordos de Oslo --- foi adicionada à lista
de pecados de Rabin.

Rabin esteve ainda menos envolvido em um outro grande conflito político daquele
período. Este iniciou"-se com uma disputa durante a nomeação do
comandante do \textit{front} central, considerado crucial na etapa seguinte da
guerra. A questão era, mais uma vez, a atitude hostil de Ben Gurion em
relação ao Palmach. A liderança das \textsc{idf} queria a nomeação de Alon. Desde
a sua criação, o Palmach esteve afiliado ao Hakibutz Hameuchad e à Achdut
Haavodá e seu líder, Yitzhak Tabenkin, era rival de Ben Gurion. A Achdut
Haavodá juntou"-se ao Hashomer Hatzair em janeiro de 1948 para formar o
\textsc{mapam}, acrônimo em hebraico de Partido dos Trabalhadores Unidos. O
Hashomer era um movimento sionista marxista, enquanto o Achdut Haavodá
combinava o socialismo com nacionalismo israelense. Para Ben Gurion, esse
Partido dos Trabalhadores Unidos era um movimento político
pró"-soviético suspeito, e o Palmach era, de certa forma, seu exército
privado. Ben Gurion respeitava os atributos militares e a capacidade de
liderança de Alon, mas tinha restrições a sua ambição e caráter. Por outro lado,
Mordechai Maklef, o escolhido por Ben Gurion, havia servido
no exército britânico. Quando Ben Gurion, enfrentando a liderança das
\textsc{idf}, insistiu em nomeá"-lo chefe do Comando Central, vários generais
apresentaram suas renúncias. Uma comissão de cinco membros do gabinete
foi nomeada para mediar. O comitê, que se reuniu entre 3 e 6 de julho e
convocou diversas testemunhas --- entre elas, Rabin ---, assumia uma função
muito mais ampla que a relação de Ben Gurion com Alon. Seu
questionamento concentrou"-se no fracasso de Rabin em salvar a cidade
velha de Jerusalém, e este queixou"-se da decisão tomada naquele momento
de cancelar o comando unificado das operações militares na cidade:

\begin{quote}
Acredito que no início da operação havia um comando único, Yitzhak
Sadeh estava lá, era o líder da ``Equipe Jebusita''. De qualquer forma,
acredito que esse comando era importante naquele momento. Não fui eu
quem exigiu que o comando acima de mim e de Shealtiel fosse removido.
Acredito que tenha sido equivocado dar a David Shealtiel a autoridade de
emitir ordens, já que ele era responsável somente pela cidade. Difícil
dizer se houve falhas militares, porque se algo for dito terá que ser
provado e, para isso, toda a questão da defesa de Jerusalém terá que ser
esclarecida.\footnote{Yemima Rosenthal (org.), \textit{Yitzhak Rabin, primeiro"-ministro de
Israel, 1974--1977 e 1992--1995: documentos selecionados}, vol. 1. Jerusalém,
2005, p. 23 {[}em hebraico{]}.}
\end{quote}

Rabin acreditava que a importância do episódio residia no aumento da
tensão e falta de confiança entre Ben Gurion e o Palmach, e a sombra que
esta lançava sobre seus principais comandantes. Como vice e protegido de
Alon, Rabin não podia evitar ser afetado pela atitude negativa de Ben
Gurion em relação a ele; sua promoção nas \textsc{idf} foi, portanto, retrasada
em 1948 e nos anos seguintes.

Quando os combates recomeçaram após o final da primeira trégua em 11 de
julho de 1948, duas decisões críticas haviam sido tomadas: investir os
principais esforços das \textsc{idf} no Comando Central e não no sul, e entregar
o comando da operação a Alon. A operação foi denominada \textsc{lrlr} (Lydda ---
Ramleh --- Latrun --- Ramallah), mais tarde sendo chamada de Operação
Danny. Foi conduzida em duas fases: a conquista de Lydda e Ramleh, para
consolidar o controle do centro do território do jovem país, e depois,
de Latrun e Ramallah, para controlar a estrada que leva a Jerusalém. A
primeira parte da operação foi um sucesso total, e dois de seus aspectos
ficaram registrados na memoria coletiva israelense. Um deles foi o
ousado ataque de Dayan, que facilitou a conquista de Lydda, um evento
que facilitou a criação de sua reputação como um comandante brilhante e
não convencional no campo de batalha. O outro foi a expulsão em larga
escala dos civis árabes.

A questão da expulsão marca o debate político e historiográfico
sobre certo e errado no que tange à guerra de 1948 e a probidade
moral de Israel. Como oficial do alto escalão, Rabin não esteve
diretamente envolvido nos combates, mas teve sim um papel na expulsão.
Ele escreveu abertamente sobre o tema em suas memórias, em uma parte do
livro que foi censurada em 1979 pelo comitê ministerial encarregado de
autorizar memórias e relatos escritos por servidores do Estado.
Entretanto, o tradutor do livro para o inglês, Peretz Kidron, um
ativista radical de esquerda, vazou as páginas censuradas para a
imprensa internacional:

\begin{quote}
Enquanto lutávamos, tivemos de lidar com um problema preocupante {[}\ldots{}{]}: o
fato de que a população civil de Lyddah e Ramleh chegava a cinquenta mil
pessoas. Nem Ben Gurion propôs uma solução, e durante as discussões no
quartel general operacional manteve"-se em silêncio, como era seu hábito
em tais situações. Estava claro que não poderíamos deixar a população
hostil e armada de Lod em nossa retaguarda, onde poderia colocar em
perigo a rota de suprimento para a Brigada Yiftah, que avançava para o
leste. Saímos e Ben Gurion nos acompanhou. Alon repetiu a pergunta: ``O
que deve ser feito com a população?'' Ben Gurion gesticulou e disse, %deve, e não ceve, certo?
``Expulse"-os.''

Alon e eu conversamos a respeito. Eu concordava que era essencial
expulsar os habitantes. Nós os conduzimos a pé até a estrada para Beit
Horon, assumindo que a Legião {[}jordaniana{]} seria obrigada a assumir a
responsabilidade por eles. ``Expulsão'' é um termo que soa duro.
Psicologicamente, foi uma de nossas ações mais difíceis.\footnote{David Shippler, \textit{New York Times}, 23 out. 1979.}
\end{quote}

O caso de Lydda transformou"-se num marco da discussão sobre o problema
dos refugiados palestinos. Aproximadamente trinta anos depois, o escritor
israelense Ari Shavit dotou de proporções quase místicas o destino de
	Lydda, no livro \textit{Minha Terra Prometida}: ``Lydda é nossa `caixa
	preta.' Nela se encontram os segredos do sionismo. A verdade é que o
	sionismo não podia suportar Lydda {[}\ldots{}{]} Se o sionismo sobrevivesse, Lydda
não o faria. E vice versa. Em retrospecto, tudo é muito
claro.''\footnote{Ari Shavit, \textit{My Promised Land: The Triunph and Tragedy of Israel}. Nova York: Spiegel \& Grau, 2013. {[}Ed. bras. \textit{Minha Terra Prometida: o Triunfo e a Tragédia de Israel}. São Paulo: Três Estrelas, 2016{]}. Ver também Ari Shavit, ``Lydda, 1948: A City,
a Massacre, and the Middle East Today'' [Lydda, 1948: uma cidade, um massacre e o Oriente Médio hoje],
\textit{New Yorker}, 21 out. 2013.}

Qualquer que seja a perspectiva adotada na segunda década do século \textsc{xxi},
estará longe das considerações dos líderes políticos e militares em
julho de 1948, quando discutiu"-se o tema da população civil de
Lydda.\footnote{Anita Shapira, \textit{Yigal Alon, filho nativo: uma biografia} (Tel Aviv:
Hakibutz Hameuchad, 2004), p. 371--7 {[}em hebraico{]}.}

A segunda parte da Operação Danny fracassou: tanto Latrun quanto
Ramallah continuaram sob controle jordaniano até 1967, bem como parte da Cisjordânia.
Mas a operação consolidou as relações entre Alon e Rabin como
seu chefe de operações e vice"-comandante \textit{de facto}. Conforme
descrito pela biógrafa de Alon, 

\begin{quote}
A combinação entre ele (Alon) e Rabin
era ideal. Alon fornecia a liderança; seu otimismo infinito irradiava 
autoconfiança e também ousadia no planejamento. Rabin,
por outro lado, era aquele que transformava as ideias em planos
operacionais precisos e detalhados, levando em conta os custos e
benefícios. Em conjunto, formavam um time vencedor. Para os soldados, Alon
era um grande líder militar por quem estavam dispostos a fazer um
esforço extra. Rabin era um excelente número dois, mas sem as qualidades
que faziam com que Alon fosse admirado por seus soldados.\footnote{\textit{Ibid}., p. 382.}
\end{quote}

A segunda trégua, após 21 de julho de 1948, foi aproveitada pelas \textsc{idf}
como um preparo para a fase seguinte da guerra. O equilíbrio havia,
então, se alterado claramente a favor de Israel. Armas novas e melhores
haviam chegado, mais homens foram mobilizados e 
implementaram"-se as lições da rodada anterior de confrontos. As \textsc{idf} logo
passariam à ofensiva, explorando a vantagem de que agora dispunham, para
encerrar a guerra com uma clara vitória. Mas a trégua poderia ser --- e foi ---
utilizada também com outras finalidades.

Rabin aproveitou o intervalo no conflito para casar"-se com sua namorada
de muitos anos, Leah Schlosberg. Rabin era um jovem bonito, considerado
atraente por muitas das jovens do Palmach e dos \textit{kibutzim} onde passava
algum tempo, mas se apaixonou pela atraente e vivaz Schlosberg.
Namoraram durante vários anos e eram vistos como o casal"-símbolo do
Palmach. O passado de Leah --- nascida na Alemanha, em uma família da
classe média"-alta --- era dramaticamente diferente do que vivera o noivo.
Ela era determinada e culta, apreciava a arte e a literatura e raramente
deixava de expressar sua opinião. A união dos dois seria um sucesso. Mais
tarde, quando Rabin tornou"-se chefe do estado maior, embaixador e
primeiro"-ministro, Leah também se tornaria uma figura pública,
florescendo nos círculos de Washington onde se distinguia como uma
anfitriã diplomática ativa e bem sucedida. Tendo em conta o caráter
introvertido e tímido de Rabin, Leah teve um importante papel na
condução da vida social do casal. Rabin escreveu sobre a relação em suas
memórias:

\begin{quote}
O nosso foi um romance da época da guerra. Começou com um encontro
casual em uma rua de Tel Aviv, em 1944: um olhar, uma palavra, um tremor
interno, e mais um encontro. Mas havia obstáculos para ampliarmos a
relação. Eu estava no Palmach e tinha poucas folgas. Nos aproximamos em
1945, quando Leah se incorporou ao Palmach e serviu no batalhão no qual
eu era o vice"-comandante --- uma das poucas vezes em nossa vida conjunta
em que ela esteve sob \textit{meu} comando. Fiz o melhor que pude para
visitar com frequência o \textit{kibutz} onde ela servia e fizemos diversos
passeios em minha moto {[}\ldots{}{]} Quando começamos a fazer planos, os britânicos
vieram e me prenderam, e nosso único contato passou a ser através de
cartas. Então eclodiu a Guerra de Independência, uma luta amarga e com
grande número de baixas. Os planos pessoais foram postos de lado e Leah
conseguiu finalizar seus estudos no seminário de professores. Durante a
segunda trégua, decidimos aproveitar a oportunidade e selamos nosso
vínculo.\footnote{Rabin, \textit{The Rabin Memoirs}, \textit{op}. \textit{cit}., p. 35--6.}
\end{quote}

Como era característico daqueles anos, o jovem casal não podia custear
um apartamento próprio e mudou"-se para um quarto no apartamento
dos pais de Leah, no centro de Tel Aviv. Somente em 1952 mudaram"-se para
sua própria casa em um projeto residencial para oficiais graduados das
\textsc{idf}.

O segundo evento ocorreu no Kibutz Naan, iniciando"-se em 14 de setembro
de 1948. Algumas dúzias de oficiais graduados do Palmach foram
convidados para encontrar"-se com Ben Gurion. O Velho adoçou a pílula ao
elogiar o Palmach pela contribuição que havia feito durante a guerra,
para então dizer a sua audiência que não havia razão para a manutenção
de um comando à parte: o Estado tinha um exército e o Palmach deveria
fazer parte dele. Suas três brigadas seriam mantidas, mas colocadas sob
a total autoridade de seus respectivos comandos regionais. Rabin evitou
tomar parte no debate que se seguiu. Concordou em princípio com a
posição de Ben Gurion, ainda que acreditasse que o comando do Palmach
deveria ser mantido, recebendo uma missão especial com o objetivo de
preservar seu legado e seu espírito. Mas percebeu que a apresentação de
uma visão tão complexa seria entendida como apoio à decisão de Ben
Gurion. Conforme as suspeitas, este não foi o final do processo e Ben
Gurion mais tarde dissolveria o Palmach completamente. Para ele, a ação
era compatível com sua atuação durante o episódio do \textit{Altalena}.
Para que o novo Estado sobrevivesse, deveria haver somente um exército e
exércitos privados não poderiam ser tolerados.

No final de setembro de 1948, Israel estava pronta para voltar à luta e
encerrar a guerra. A escolha por levar as batalhas para a frente sul era
óbvia, tendo o Egito tomado controle do Neguev, a grande área árida e
pouco habitada no sul do país. Essa era a reserva geográfica de Israel, e
o Mar Vermelho em sua extremidade setentrional oferecia um futuro acesso
naval para a Ásia e a África. Folke Bernadotte, o mediador da \textsc{onu} que
seria assassinado pelo grupo Stern --- um movimento nacionalista radical e
ilegal ---, recomendou em seu relatório publicado em 20 de setembro, após
sua morte, que o sul do Neguev fosse retirado de Israel em troca de
alguma outra parte do país. O Egito e possivelmente a Grã"-Bretanha viam
o sul do Neguev como uma ponte ligando o Egito, então ainda um
protetorado britânico, ao leste do mundo árabe. Havia chegado a hora de
agir.

Quatro brigadas foram destinadas à frente sul. Alon era o comandante e
Rabin, mais uma vez, seu vice e chefe de operações. Mas a divisão de
trabalho entre eles agora seria modificada e Rabin teria um papel mais
importante e uma função mais saliente nestas operações. Antes visto como
braço direito de Alon, ele começou a ser visto como planejador militar
de primeira classe por seus próprios méritos, como o oficial que planejava
meticulosamente as principais operações no sul e então supervisionava
sua implementação.

Entre outubro de 1948 e março de 1949, as \textsc{idf} completariam três grandes
operações no sul: Yoav, de 15 a 22 de outubro de 1948; Chorev, de 22 de
dezembro de 1948 a 7 de janeiro de 1949; e Uvda, de 6 a 10 de março de
1949. A operação Yoav cumpriu sua missão e abriu o caminho para o
Neguev, abrindo uma brecha nas forças egípcias. A operação Chorev, que
buscava expulsar o exército egípcio do território da Palestina
mandatária, foi bem"-sucedida, com exceção do ``bolsão de Faloujah'',
onde um força egípcia significativa ficou sitiada. E as tropas
israelenses vitoriosas adentraram o território nacional egípcio na
Península do Sinai. Sob pressão internacional, e sempre ciente dos
limites políticos das ações militares, Ben Gurion instruiu os
desgostosos Alon e Rabin a retirarem suas tropas para a fronteira
internacional. Durante as operações Yoav e Chorev, as \textsc{idf} enfrentaram
uma forte oposição egípcia, mas o exercito israelense agora operava em
outra escala, conduzindo complexas operações terrestres, aéreas e
navais.

O sucesso da operação Chorev finalmente levou os egípcios à mesa de
negociação. Em 12 de janeiro de 1949 iniciaram"-se as negociações de
armistício no Hotel Roses, na ilha grega de Rhodes, conduzidas por Ralph
Bunche, o novo mediador da \textsc{onu}. O vice"-comandante das \textsc{idf}, Yigael Yadin,
liderou o time militar que fez parte da delegação israelense. Rabin era
membro do time, como representante da frente sul e como o especialista
da delegação para os temas em discussão. Alon, ainda magoado sob o
impacto da ordem de retirada do Sinai, recusou"-se a participar. Rabin
também estava insatisfeito com todo o processo e com a sensação que as
\textsc{idf} foram destituídas da oportunidade de derrotar definitivamente os
egípcios e assim obter uma posição de vantagem nas negociações. Durante
esses acordos, houve recusa à demanda egípcia de desmilitarização da área
conhecida como Aujah al"-Hafir. Os diplomatas do Ministério do Exterior
assumiram uma posição mais branda. Davam grande importância à assinatura
de um armistício com os principais Estados árabes e estavam dispostos a
fazer concessões em questões militares. A perspectiva e o humor de Rabin
durante os encontros são visíveis em uma carta que escreveu a Alon em
10 de fevereiro de 1949:

\begin{quote}
Estou convencido, por tudo que vi e ouvi aqui, que os egípcios
necessitam muito de um acordo de armistício, que libere a Brigada do
Bolsão (a que estava cercada no bolsão de Faloujah) e reduza as forças
no setor costeiro. Na minha opinião, qualquer concessão feita agora
seria prematura e desnecessária. Penso que temos mais espaço de
manobra do que os egípcios e podemos aguentar mais do que eles em uma guerra
de nervos. Estou quase certo de que seremos bem sucedidos. E, se não, sempre
poderemos fazer uma concessão {[}\ldots{}{]} Estou farto da política e da
diplomacia.\footnote{Carta a Yigal Alon, 10 fev. 1949.}
\end{quote}

As negociações egípcio"-israelenses foram concluídas em 24 de fevereiro
de 1949. Rabin e seus colegas foram derrotados. Como uma espécie de
concessão, permitiram que Rabin voltasse em 20 de fevereiro, quatro dias
antes da assinatura, poupando"-o de assinar um acordo ao qual se opunha.

Foram essas negociações em Rhodes a primeira experiência diplomática de
Rabin e sua primeira interação real com árabes não palestinos. Ele levou
dos encontros uma lição que o acompanharia nas décadas seguintes: não
era do interesse de Israel negociar com um coletivo árabe. Nesse tipo de
negociação, a dinâmica grupal tendia a seguir a liderança do Estado mais
radical e Israel obteria melhores resultados lidando individualmente com
os Estados árabes.

Havia uma segunda lição aprendida nas negociações de Rhodes: o Egito era
o único país com um senso de \textit{raison d'Etat} {[}razão de Estado{]}. Seus líderes foram os
primeiros a reconhecer que era de seu interesse encerrar a guerra contra
Israel: o acordo foi firmado em 24 de fevereiro de 1949. Outros Estados
árabes, especialmente a Síria, não se apressaram e as negociações se
estenderam até julho de 1949. Contudo, mesmo durante as conversações, Israel
concluiu seu avanço rumo ao sul --- através da operação Uvda --- e
conquistou todo o Neguev, garantindo o acesso ao Mar Vermelho com o
controle de Umm Rashrash, onde hoje se localiza a cidade de Eilat. Foi
uma façanha logística incrível, que levou uma grande força militar a
cruzar com sucesso um território desconhecido.

Rabin era tenente"-coronel quando a guerra terminou em 1949, vice de Alon
no Comando Sul. Havia emergido com uma reputação de planejador militar e
oficial de comando de primeira. Tinha adquirido ampla experiência no
campo de batalha, em postos de importância no comando e à mesa de
negociações. Mas seu legado mais duradouro foi obtido na dura
luta pela estrada para Jerusalém. Conforme escreveu repetidas vezes em
suas memórias, Rabin sentia que as \textsc{idf} não haviam se preparado
adequadamente para a guerra de 1948 e que, por isso, seus homens e
outros pagaram um alto preço. Ele jurou que tal despreparo não voltaria
a se repetir.

\chapter[Da Independência até a Guerra dos Seis Dias, 1949--67]{Da Independência até a Guerra\\ dos Seis Dias, 1949--67}
\markboth{Da Independência até a Guerra}{}

À medida que transcorriam as últimas semanas da guerra e uma nova rotina
pós"-guerra começava a emergir, as tensões entre Ben Gurion e a liderança
do Palmach atingiam um novo nível. Alguns dos comandantes mais graduados
do Palmach foram expulsos das \textsc{idf}, enquanto outros, em protesto,
decidiram renunciar. Rabin decidiu ficar. Em suas memórias, explicou:

\begin{quote}
Encontrando"-me em uma encruzilhada de minha vida pessoal, senti um
profundo senso de responsabilidade moral, uma espécie de dívida de honra
com aqueles homens que, com seus corpos e sua coragem, bloquearam o
avanço árabe {[}\ldots{}{]} Nos mais trágicos momentos da guerra {[}\ldots{}{]} eu e muitos de
meus companheiros assumimos um compromisso pessoal {[}\ldots{}{]} Dedicaríamos nossa
vida para garantir que o Estado de Israel nunca mais enfrentasse alguma
agressão sem estar preparado {[}\ldots{}{]} Construímos um exército
poderoso.\footnote{Yitzhak Rabin, \textit{The Rabin Memoirs} {[}As memórias de Rabin{]}, edição expandida. Berkeley:
University of California Press, 1996, p. 45.}
\end{quote}

Mas havia algo mais. Por um lado, havia poucas opções no país, que era
jovem e pobre. Rabin se identificava com o Palmach, mas também era o
independente solitário --- o filho de Rosa --- que não se juntou a um
kibutz ou ao Achdut Haavodá e que tomava suas próprias decisões. Era o
mesmo Rabin que se mantivera calado no encontro marcado por Ben
Gurion com os líderes do Palmach no Kibutz Na'an, para discutir suas
diferenças. Rabin na verdade concordava com a decisão de Ben Gurion de
dissolver o comando do Palmach, mas não queria alinhar"-se ao ``Velho''
opondo"-se a seus camaradas. Rabin se identificava com o Achdut Haavodá,
mas evitou filiar"-se ao partido. Em meados da década de 1940, escreveu a
sua irmã: 

\begin{quote}
Não sou membro do partido e, apesar de identificar"-me com
alguns dos temas do Achdut Haavodá, não consigo enxergar nele nenhuma
visão, nenhuma vontade de tomar a iniciativa, liderar e mostrar o
caminho; é na verdade uma entidade mediana, que deseja mostrar um
trajeto um tanto distinto daquele apresentado pela atual
liderança.\footnote{Carta não datada. Coleção privada.}
\end{quote}

Nada disso afetou sua lealdade a Alon, que foi removido por Ben Gurion
de seu posto no Comando Sul e substituído por Dayan. Alon estava na
Europa visitando o exercito francês e, apesar de estar sendo minado em
casa, inexplicavelmente prolongou sua visita. Sua carreira chegou
ao ápice durante a guerra de 1948 e, logo depois, começou a declinar.
Esse episódio representava uma indicação precoce de uma profunda mudança
na personalidade de Alon. Rabin o manteve informado enquanto este
esteve no exterior e lutou por ele, descrevendo a Alon seus esforços
para defendê"-lo ao citar uma carta que escreveu a Yaacov Dori, chefe do
Estado"-Maior, em 12 de outubro de 1949: ``Exigindo uma explicação por
escrito em relação a você. Exigi também, verbalmente, um encontro com
ele. Apesar de três solicitações ao seu ajudante de campo em um mesmo
dia, não obtive resposta.'' As cartas enviadas a Alon foram escritas em
um código primitivo, e o nome codificado escolhido para Dayan foi ``o
comerciante de antiguidades.'' Eles eram críticos de Dayan. Em uma carta
para Alon em 23 de outubro, Rabin escreveu: ``Dayan apareceu. Estou
rompendo com ele. Ele não tem ideia. Na minha opinião, ele não tem a
mínima compreensão militar quando se trata de formações acima de uma
companhia ou um batalhão. Totalmente sem tato ao tratar com as
pessoas''.\footnote{Rosenthal (org.), \textit{Yitzhak Rabin, primeiro"-ministro de Israel}, \textit{op}. \textit{cit}., p. 58.}

A antipatia era mútua. Dayan não perdeu tempo em destituir Rabin de seu
posto como vice no Comando Sul. Rabin foi nomeado comandante da
12\textsuperscript{a} Brigada, o que no padrão daquela época equivalia a
uma brigada de blindados. Mas a posição não durou muito. A
12\textsuperscript{a} Brigada era vista como uma brigada do Palmach, e
foi dispersada sob o pretexto de que as todas unidades blindadas
deveriam se concentrar na 7\textsuperscript{a} Brigada. Rabin enviou
uma carta furiosa ao chefe do Estado"-Maior criticando a decisão: ``Pergunto"-me por que e para quê?'', escreveu. ``Ocorre"-me que possa
haver duas razões: ou argumentos pessoais contrários a mim, que não
posso entender, ou argumentos contra a brigada, por causa de sua origem
no Palmach.''\footnote{\textit{Ibid}., p. 52.}

As tensões em torno do tema chegaram ao auge em 24 de
outubro, o dia marcado para uma manifestação do Palmach, concebida tanto
como um evento de despedida quanto um ato de protesto. O Estado"-Maior
das \textsc{idf} emitiu uma ordem proibindo oficiais do serviço ativo de
participarem daquilo que viam como um evento político. Para Ben Gurion
era muito importante que Rabin não participasse, já que era o oficial do
Palmach do mais alto escalão nas \textsc{idf}. Sua participação --- ou ausência ---
seria notada. Ben Gurion, que tinha uma queda por Rabin, pode ter
desejado protegê"-lo das consequências de desobedecer abertamente a uma
ordem. Ele o convidou para uma discussão em sua casa no final da tarde
daquele dia. Conforme Rabin descreveu em suas memórias, quando se
preparava para sair o ``Velho'' o convidou para ficar para o jantar.
Rabin recusou e foi para a manifestação, sabendo que pagaria o preço de
seu ato de rebeldia. Ficar nas \textsc{idf} era uma coisa; lealdade a seus amigos
e colegas era outra.

Ben Gurion ficou revoltado e a conduta de Rabin custou"-lhe uma década de
promoções. Mas ele não seria o único político a se irritar com o
incidente. Em uma reunião de gabinete, Moshe Haim Shapira, o líder do Partido Mizrahi (mais
tarde o Partido Nacional Religioso, Hamafdal na sigla em hebraico), alegou que
``Rabin não é tão somente um jovem confundido por dois comandos --- ele
esteve com Ben Gurion, discutiu com ele, abriu seu coração e, quando
enfrentou o dilema do que é mais importante --- a disciplina partidária
ou a militar --- optou pela disciplina partidária. Significa que o vice"-comandante
de uma frente, em seu primeiro teste, quando confrontado
com tal escolha, decidiu"-se pelo partido. E o que acontecerá se o
partido lhe ordenar que lastime o governo --- ele mais uma vez terá de
fazer a escolha entre o partido e o governo, e não sei como ele se sairá
nesse teste {[}\ldots{}{]} Digo abertamente: não estou disposto a depositar o
destino de uma brigada nas mãos de um comandante que possa violar a
disciplina dessa maneira.'' Esse não foi o ultimo embate entre Shapira e
Rabin.

O incidente teve fim quando Dori, o chefe do Estado"-Maior, adotou medidas
disciplinares em 21 de outubro de 1949. O encontro de Dori com Rabin,
transcrito nos arquivos, foi um momento raro: poucos oficiais de alto
escalão a caminho do comando supremo são enquadrados por terem
desrespeitado a uma ordem. As respostas de Rabin foram curtas, secas e
objetivas:

\begin{quote}
Dori: Você foi convocado para um julgamento disciplinar por quebra de
disciplina. Você recebeu uma ordem proibindo a sua participação na
manifestação?

Rabin: Recebi a ordem.

Dori: Você participou?

Rabin: Sim.

Dori: Como o não cumprimento da ordem se coaduna com o juramento
prestado de obedecer a todas as instruções que lhe foram dadas pelo
comando supremo?

Rabin: Não se coaduna. Reconheço haver descumprido a ordem. Tinha um
senso de lealdade pessoal em relação a meus amigos {[}\ldots{}{]} foi uma questão de
amizade {[}\ldots{}{]} senti que tinha de encontrar meus amigos.

Dori: Houve outra ordem?

Rabin: Não houve outra ordem, foi uma decisão pessoal. É verdade que
deixei de cumprir uma ordem, mas não recebi nenhuma ordem de outra
fonte.

Dori: Como isso se coaduna com o juramento?

Rabin: De fato havia uma ordem, mas não há lei que proíba um
soldado de participar de manifestações como cidadão quando está de folga.
\end{quote}

Rabin recebeu uma severa reprimenda e foi resgatado dessa encrenca por
Haim Laskov, que o convidou para juntar"-se a ele e depois substituí"-lo
como comandante da Escola para Comandantes de Batalhão, um novo curso
militar fundado logo depois do final da guerra, como parte da
institucionalização das \textsc{idf}. Rabin convidou vários de seus colegas do
Palmach para se juntarem a ele como instrutores. Para alguns, isso representou
um incentivo para ficarem, ou retornarem às \textsc{idf}. Laskov, que havia sido
major na Brigada Judaica do exército britânico, era o modelo do
ex"-oficial britânico que Ben Gurion preferia, como antítese do oficial
do Palmach. Mas, apesar de seus distintos antecedentes militares, ficou
famosa a relação harmoniosa entre Laskov e Rabin. Sua colaboração e o
papel desempenhado por seus colegas foram cruciais para a absorção das
lições da guerra de 1948, assim como das duas diferentes tradições
militares, na doutrina específica das \textsc{idf}. Os cinquenta estudantes da
escola se tornariam, ao longo dos anos seguintes, o cerne do comando das
\textsc{idf}, cujos comandantes e alunos desenvolviam, em conjunto, um novo idioma
profissional em hebraico. A abordagem meticulosa de Rabin às vezes
irritava os estudantes, mas estes aceitavam e respeitavam sua
autoridade.\footnote{Efraim Inbar, \textit{Rabin and Israel's National Security} {[Rabin e a segurança nacional de Israel]} (Baltimore: Johns
Hopkins University Press, 1999), p. 60--1.} Rabin tinha uma memória excelente e combinava
aptidões analíticas com a atenção ao detalhe. Ele insistia na precisão e
na perfeição, qualidades nem sempre apreciadas por seus subordinados.

O posto seguinte de Rabin seria o de chefe do departamento de operações
do quartel"-general, onde ficou de janeiro de 1951 a dezembro de 1952.
Era um posto"-chave na divisão mais importante do quartel"-general. O
departamento era responsável por três áreas: operações, segurança
corrente e a organização e mobilização de reservistas. Os documentos
redigidos por Rabin durante o período refletem o seu domínio de temas
militares, doutrinas, comando e controle e táticas.\footnote{Rosenthal, \textit{Yitzhak Rabin, primeiro"-ministro de Israel, 1974--1977 e 1992--1995}, \textit{op}. \textit{cit}., p. 72--120.} Seus
comentários sobre uma pesquisa preparada pelo departamento de
planejamento das \textsc{idf} são um claro exemplo de sua atuação nessa função.
Ele era perspicaz e incisivo, demonstrou profundo conhecimento do
papel da força aérea na guerra moderna e insistia em que a função e a
atuação da força aérea fossem determinados pelo Estado"-Maior e não pela
própria força aérea. A principal contribuição de Rabin ao trabalho e
metodologia das \textsc{idf} durante esse período foi na realização da transição
do planejamento militar de 1948 para um tipo de planejamento mais
ordenado e sistemático, necessário para as \textsc{idf} do pós"-guerra, ilustrada
por um artigo que escreveu em 24 de dezembro de 1951, intitulado ``Guerra e
Planejamento nas \textsc{idf}''. Rabin estabeleceu a diferença entre planejamento
estratégico e operacional, resolvendo assim o dilema enfrentado por todo
exército moderno. O planejamento para um cenário hipotético podia
produzir rigidez; o desafio do planejador estava em manter um elemento
de flexibilidade, que permitisse à liderança do exército lidar com a
realidade enfrentada quando um plano militar antecipadamente preparado
precisa ser implementado em circunstancias imprevistas.

Durante esse período, Rabin construiu sua posição e reputação como
pensador e planejador militar, além de oficial de comando inusualmente
eficiente. O general Israel Tal, legendário arquiteto do corpo de
blindados das \textsc{idf}, descreveu Rabin como a ``mais elevada autoridade em
questões militares.''\footnote{Inbar, \textit{Rabin and Israel's National Security}, \textit{op}. \textit{cit}., p. 59.} Outro general das \textsc{idf}, Elad Peled,
escreveu: ``Yitzhak era considerado como o profissional militar mais
graduado pelos representantes da geração de 1948 nas \textsc{idf}. {[}\ldots{}{]} Ele era
ouvido em todo exercício e manobra, em qualquer discussão de que
participava. Sabiam que ele era o `Admor' {[}o rabino hassídico{]}, o
profissional''. \footnote{Elad Peled em palestra no Centro Rabin, 24 out. 2012.}

A ascensão de Rabin na lista de promoções das \textsc{idf} foi demonstrada pela
iniciativa do chefe do Estado"-Maior, Mordechai Maklef, ao nomeá"-lo
para o posto de chefe de operações em 1953 --- tornando"-o seu vice em lugar
de Dayan, com quem Maklef não se dava. Mas Ben Gurion, que apreciava e
tinha profunda estima por Dayan e talvez ainda magoado com a
desobediência de Rabin durante a manifestação do Palmach, abortou a
iniciativa de Maklef.

Naquele momento, Rabin e sua família estavam em Camberley, na Inglaterra,
passando o ano de 1953 na Escola de Comando do Exército inglês. Rabin
foi um dos primeiros oficiais a serem mandados para uma escola militar
do Ocidente, como parte de uma política designada a proteger as \textsc{idf} do
potencial provincialismo de um país pequeno e isolado, bem como para integrar Israel,
ainda que não formalmente, no \textit{establishment} de defesa
ocidental. Para ele, era uma oportunidade de expor"-se ao mundo
exterior, aprender com base na experiência de um grande exército,
melhorar o seu domínio do inglês e desfrutar da companhia de sua esposa
Leah e de sua filha Dalia (seu filho Yuval nasceria em 1955).

Com a volta de Rabin, Dayan, agora chefe do Estado"-Maior,
demonstrou não guardar mágoas ao nomeá"-lo chefe da Divisão de Instrução
do quartel"-general, promovendo"-o a general de brigada. Na verdade, Dayan
fez um grande esforço para convencer Pinchas Lavon, o ministro da
Defesa, a concordar com a promoção de Rabin. Rabin ainda não era considerado
politicamente confiável pela liderança do Mapai, devido à questão
disciplinar da manifestação do Palmach. Dayan disse a Lavon que a
divisão de instrução não tinha tropas sob seu comando e portanto Rabin
não poderia, ainda que quisesse, deixar de cumprir uma ordem com a qual
não concordasse. Lavon ainda resistiu, salientando que as \textsc{idf} tinham,
naquele momento, um número excessivo de generais de brigada. Dayan lhe
escreveu que concordava com ele, pedindo que fizesse uma exceção no caso
de Rabin. Lavon finalmente concordou. O biógrafo de Dayan, Shabtai
Teveth, que conheceu tanto Dayan quanto Rabin, acerta ao elogiar Dayan
nesse episódio da promoção de Rabin, apesar da incômoda relação entre os
dois. De acordo com o autor, Dayan estava determinado a transformar o
espírito das \textsc{idf} e via em Rabin o candidato apropriado para fazer de
suas ideias um sistema de instrução ordenado e efetivo. Apesar do
apoio político de Dayan, havia uma tensão permanente entre os dois. Eles
conseguiam trabalhar efetivamente em conjunto, mas mantinham um
relacionamento pessoal difícil.

Segundo Teveth,

\begin{quote}
Parte da explicação pode ser encontrada na diferença de personalidades.
Rabin não apreciava o \textit{modus operandi}de Dayan. Mas havia também
diferenças características entre um comandante de campo e um oficial do
comando, entre um líder carismático, natural e um excelente
profissional, tecnocrata, entre um homem que funciona de acordo com seus
sentidos e visão prática e um outro, racional, baseado na lógica, entre
alguém confortável no meio de outras pessoas e alguém tímido. A isso
deve ser acrescentada a tendência de Rabin de fazer comentários críticos
e irônicos sobre seus superiores {[}\ldots{}{]} Foi Rabin que cunhou a frase
que perseguiu Dayan por muito tempo --- ``Ele é um utilizador e não um
construtor da força''.\footnote{Shabtai Tevet, \textit{Moshe Dayan: uma biografia} (Tel Aviv: Schocken Books,
1971), p. 404--5 {[}em hebraico{]}.}
\end{quote}

Na cultura organizacional das \textsc{idf}, era normal que os oficiais servissem
três ou quatro anos em uma posição sênior para depois avançarem. O posto
seguinte de Rabin, como comandante"-geral do Comando Norte, que ocupou de
abril de 1956 a abril de 1959, era muito diferente. Em lugar de um posto
sênior no comando, Rabin comandava agora uma entidade territorial,
responsável pela segurança da parcela do território que fazia fronteira
com a Síria e o Líbano. Seu principal desafio era administrar a difícil
relação fronteiriça com a Síria. Os Acordos de Armistício de 1949 eram
entendidos naquela época como temporários, a serem substituídos por
acordos de paz, o que não ocorreu. O conflito se intensificava. A
política síria se militarizava e se radicalizava: havia sofrido quatro
golpes militares entre 1949 e 1954. Em meados de 1955, sob um aparente
governo parlamentar, a influência e poder políticos eram controlados por
facções militares e partidos ideológicos. Os acordos relativos ao
trabalho agrícola e à utilização da água no entorno do Lago Tiberíades e
no rio Jordão eram complexos e bizarros e fonte de contínua fricção.
Israel e a Síria se enfrentariam várias vezes em relação ao acesso sírio
ao Lago Tiberíades, a principal fonte de água doce de Israel. O Acordo
de Armistício de 1949 com a Síria definia como zona desmilitarizada o
pequeno território a oeste da fronteira internacional entre a Síria e a
Palestina, ocupado pelo exercito sírio no final da guerra. A soberania
israelense sobre o território era limitada e a Síria se opunha ao
direito israelense de cultivar as terras em zonas desmilitarizadas que
haviam sido propriedade dos árabes antes da guerra de 1948. Os tratores
israelenses eram alvejados de cima do Golã sempre que operavam
nessas glebas. Rabin era um fiel guardião da posição israelense frente
aos sírios e emergiu de seus três anos no Comando Norte com a percepção
da Síria como um inimigo amargo, e um profundo senso do significado
tático e estratégico das Colinas do Golã. Como oficial general
comandante do Comando Norte, Rabin não se envolveu na campanha do Sinai
de 1956, que serviu a Dayan como trampolim para a fama e a glória. A
campanha do Sinai foi travada no Sul e não teve repercussão nas relações
entre Israel e a Síria.

O final de seu mandato no Comando Norte em 1959 representou um ano
importante para Rabin. Por um tempo, pareceu ser o final de sua carreira
militar. Dayan, em seu último ano como chefe do Estado"-Maior das \textsc{idf},
tentou direcionar a saída de Rabin do exército ao sugerir que
ambos se matriculassem na Universidade Hebraica. Rabin
naquele momento insistiu em seu interesse em estudar exclusivamente no
exterior: seu velho sonho de estudar engenharia hidráulica nos Estados Unidos.
Dayan, com a prerrogativa de chefe do Estado"-Maior, rejeitou a opção e
Rabin continuou no exército. Graças à sua relação amigável com Laskov, o
sucessor de Dayan, Rabin conseguiu que Laskov o enviasse para um curso
na Universidade de Harvard em Cambridge, Massachusetts, ao final de seu
serviço no Norte. Rabin e sua família se preparavam para partir no verão
de 1959, mas, em abril daquele ano, um exercício de convocação de
reservistas foi mal administrado pelas \textsc{idf}, causando o temor de uma
guerra. Ben Gurion destituiu o chefe da divisão de comando das \textsc{idf} e também o
diretor da inteligência militar, e Rabin foi nomeado chefe da divisão de
comando no quartel"-general. Ele agora detinha a terceira posição mais
importante nas \textsc{idf} após o chefe do Estado"-Maior Laskov e seu vice Zvi
(``Chera'') Tsur, tornando"-se um sério candidato à sucessão de Laskov.

Em sua nova posição como chefe da divisão de comando, Rabin juntou"-se ao
pequeno grupo que formulava a política de segurança nacional de Israel,
expandindo sua atividade além dos limites estritos das questões
militares. Ben Gurion estava ocupado com a construção de uma ``aliança
periférica'' uma parceria entre Israel e seus vizinhos regionais, a
Turquia, o Irã e a Etiópia, em oposição à União Soviética e ao
pan"-arabismo revolucionário do presidente egípcio Gamal Abdel Nasser.
Rabin fez várias viagens à Etiópia, oferecendo conselhos e instruções a
seus militares, e enquanto a ministra das Relações Exteriores, Golda Meir,
cultivava relações com a África, Rabin foi enviado ao Congo. Tomou parte
no debate sobre a política de aquisições e viu"-se discordando do
influente diretor"-geral do Ministério da Defesa, Shimon Peres. O
conflito entre a liderança civil das instituições de defesa e os
escalões militares superiores é endêmico em Israel --- e em quase todos
os países. Neste caso, Rabin argumentou que as decisões relativas a
aquisições deveriam ser feitas pela liderança militar e não pela
burocracia civil. Havia vários outros temas sobre os quais Rabin e Peres
tinham visões opostas. Enquanto Peres defendia uma orientação europeia
(aquisições da França e Alemanha), Rabin e outros queriam tornar os
Estados Unidos, ainda relutantes, em uma fonte principal de equipamento
militar. Peres, considerado o pai da indústria militar israelense,
acreditava na produção local, enquanto Rabin preferia comprar
equipamento pronto no exterior. Peres era a principal força por trás do
projeto do reator nuclear de Dimona, que gerava dúvidas em Rabin e seus
amigos Galil e Alon, da facção do Achdut Haavodá. Rabin acreditava em
dissuasão convencional e preferia alocar os recursos limitados de que
Israel dispunha para a construção das \textsc{idf} em lugar de um dispendioso
projeto nuclear. Ele mudou de opinião em 1963, quando se convenceu de
que Israel teria dificuldade em manter uma corrida armamentista
convencional interminável com o mundo árabe. Estas seriam as origens da
rivalidade e inimizade entre Rabin e Peres, que se tornaram um tema tão
importante na política israelense dos anos 1970.

O principal evento militar desse período ficou conhecido como Rotem: a
remilitarização do Sinai pelo Egito em 1960. Na época (fevereiro de
1958 --- setembro 1961), o Egito havia se unido à Síria e criado a
República Árabe Unida (\textsc{rau}). A escalada das tensões na fronteira
sírio"-israelense, causada pela fricção relacionada à zona
desmilitarizada, e o temor de um ataque israelense maciço levaram o
Egito a enviar um grande número de tanques ao Sinai, violando os acordos
assinados em 1957. Israel foi surpreendida, mas a tensão cessou, já que
nenhuma das partes tinha interesse em confronto militar. Foi um caso bem"-sucedido
de administração de crise. Um dos maiores esforços de Rabin nos
anos seguintes seria dedicado ao desenvolvimento da capacidade de
inteligência necessária para evitar uma nova surpresa.

O final prematuro do mandato de Laskov como chefe do Estado"-Maior foi
outro marco no lento avanço de Rabin ao topo da pirâmide das \textsc{idf}. Ben
Gurion apreciava Laskov, mas seus conflitos com Peres e vários altos
oficiais tinham um preço. Rabin acreditava que merecia suceder Laskov
como chefe do Estado"-Maior, mas Ben Gurion optou por Tsur. Ele convidou
Rabin para um encontro, para explicar sua decisão. Rabin sabia que Peres
havia pressionado pela escolha de Tsur, o que não foi mencionado por Ben
Gurion. Em uma carta pessoal enviada a seu amigo Uzi Narkis, na época
adido militar em Paris, Rabin escreveu: 

\begin{quote}
Ben Gurion ofereceu uma ampla
gama de explicações, começando com a senioridade de Chera etc. {[}\ldots{}{]}
Também mencionou os dois ``pecados'' --- primeiro o fato de que havia
desobedecido à sua ordem e ido à manifestação do Palmach {[}\ldots{}{]} segundo, ele
acha que sou demasiado cauteloso {[}\ldots{}{]} Depois lhe perguntei se deveria
continuar nas \textsc{idf} em meu posto atual {[}\ldots{}{]} caso se criasse uma relação
pessoal indesejada (com Tsur). Em resumo, ele quase saltou de sua
cadeira e me perguntou como me atrevia a pensar tais coisas etc. etc.
De qualquer forma, ficou claro que ele concorda que, além de meu cargo
como chefe da divisão de comando, serei nomeado também vice"-chefe do
Estado"-Maior, mas a oferta não veio dele.\footnote{Rosenthal, \textit{Yitzhak Rabin, primeiro"-ministro de Israel}, \textit{op}. \textit{cit}., p. 265.}
\end{quote}

Rabin realmente foi nomeado vice"-chefe do Estado"-Maior, mais um passo em
sua árdua jornada ao posto de chefe do Estado"-Maior. Mas a última etapa
do percurso não foi fácil. Sua pergunta a Ben Gurion sobre a possível
tensão com Tsur havia sido premonitória. No início de dezembro de 1960,
Tsur mencionou Rabin positivamente a Ben Gurion. Este anotou em
seu diário, em 8 de dezembro, que --- ``na opinião de Tsur --- é Rabin e não
Haim (Laskov) quem administra o exército, e ele é um bom administrador.
Conhece o trabalho. Trabalha em silêncio''. Mas, uma vez que Tsur
substituiu Laskov, as relações com Rabin se deterioraram. Tsur não o
queria como seu vice e, após um período de trabalho conjunto marcado por
desavenças, informou"-o de que desejava substituí"-lo. Ironicamente, Ben
Gurion foi o principal apoiador de Rabin nesse conflito com Tsur e,
quando este quis destituí"-lo, Rabin insistiu que fossem até Ben Gurion
para solucionar o conflito. Rabin agora estava decidido a se tornar
chefe do Estado"-Maior e manteve o foco. Em 26 de fevereiro de 1963, Ben
Gurion anotou em seu diário sobre a possibilidade de uma ``licença
acadêmica'' que ofereceu a Rabin: ``Planejo manter Tsur em
seu posto por mais dois anos e, se mantiver minha posição, --- o posto de
chefe do Estado"-Maior está reservado para ele, para Yitzhak''. Mas Rabin
rejeitou a ideia da licença e decidiu manter"-se firme. Ele bem sabia que
seria demasiado arriscado esperar por uma promoção estando no exterior.

Foi uma decisão acertada. Na primavera de 1963, Ben Gurion abandonou a
política. Antes de retirar"-se, pediu a seu sucessor, Levi Eshkol, que
respeitasse a promessa que havia feito a Rabin. Eshkol concordou e,
mais ainda, decidiu encerrar o mandato de Tsur após três anos.
Rabin se tornou chefe do Estado"-Maior em janeiro de 1964.

Naquela época, a liderança sênior das \textsc{idf} era composta de um pequeno
grupo de homens capazes e ambiciosos, alguns dos quais eram amigos,
outros rivais implacáveis. A competição pelos postos mais elevados era
dura, mas moderada por uma sensação de pertencimento a um mesmo clube.
Algumas dessas relações, boas e más, mantiveram"-se ao longo das décadas
seguintes com vários altos oficiais superiores e militares entrando na
política e continuando a competir por poder e influência. As relações de
Rabin com Alon, Dayan, Peres, Ariel Sharon, Ezer Weizman, Haim Bar"-Lev e
Aharon Yariv foram forjadas nas décadas de 1940 e 1950 e continuaram a
ser parte de seu universo ainda na década de 1990.

Após constituir seu governo em 26 de junho de 1963, Eshkol teve duas
importantes reuniões com o comando"-geral das \textsc{idf}, durante as quais
recebeu detalhados informes sobre a situação do exército e seus planos.
No segundo encontro, realizado em 8 de julho, Weizman, comandante da
Força Aérea, argumentou que as fronteiras atuais de Israel representavam
um perigo e ameaçavam a segurança da Força Aérea e de suas bases. As
\textsc{idf}, disse ele, deveriam esforçar"-se por expandir as fronteiras, mesmo
se isso fosse incompatível com a ``abordagem política.'' Rabin, que como
chefe do Estado"-Maior apresentou uma plataforma para discussão, tinha
outra visão: ``Não acreditamos que, para implementar a missão do Estado,
para garantir sua existência, a condição seja melhorar suas fronteiras
{[}\ldots{}{]} Vemos como funções das \textsc{idf} --- {[}\ldots{}{]} como principal prioridade a defesa do
Estado, sua integridade territorial e a proteção de seus direitos
soberanos, em seu território e fora dele. Se é este o objetivo, melhor
alcançá"-lo sem uma guerra''.\footnote{\textit{Ibid}., p. 318--21.}

Essa declaração refletia a concepção de Rabin sobre a segurança nacional do
país e sua abordagem da questão de utilização da ação militar:
essencialmente defensiva. Ele acreditava em dissuasão mais do que no uso
preemptivo da força. Tendo em conta o pequeno tamanho de Israel, a
vantagem árabe em territórios e o tamanho dos exércitos árabes, Rabin
adotava a doutrina que rezava que, em caso de guerra, Israel não deveria
basear"-se na defesa e sim na rápida transferência da guerra para o
território inimigo. Como chefe do Estado"-Maior, Rabin construiria a
força e supervisionaria o planejamento para implementar seu conceito da
segurança nacional de Israel.

O lento avanço de Rabin rumo ao topo da pirâmide das \textsc{idf} teve um
resultado positivo: ele tornou"-se chefe do Estado"-Maior como o líder
indiscutível, poderoso e legítimo do alto escalão das \textsc{idf}. Rabin era o
mais velho chefe do Estado"-Maior das \textsc{idf} naquela época, e foi o primeiro
oficial a assumir o posto depois de fazer quarenta anos de idade. Tendo servido em várias
funções nas \textsc{idf}, Rabin havia acumulado experiência e profundidade que
agregavam à autoridade que já possuía junto a colegas e subordinados.

Rabin montou um time que até hoje é considerado o melhor comando"-geral
das \textsc{idf}. Ezer Weizman, Haim Bar"-Lev, Aharon Yariv, Zvi Zamir, Yeshayahu
Gavish, Amos Horev, e David Elazar eram alguns de seus membros proeminentes. Rabin também fez
questão de incluir três oficiais heterodoxos: Ariel Sharon, Matti Peled
e Israel Tal. Era uma fórmula que fornecia originalidade e frescor a um
time composto de excelentes generais tradicionais. A promoção de Sharon
cumpria a promessa feita por Rabin a Ben Gurion de cuidar de Sharon. Ben
Gurion tinha ciência do viés problemático do caráter de Sharon --- era
considerado indisciplinado e capaz de mentir ---, mas o admirava como um
grande soldado, e Rabin compartilhava dessa opinião. Em 1964,
Sharon era coronel, comandante de uma base sem importância, e parecia
estar no triste final de sua carreira. Rabin o nomeou como vice do
general comandante do Comando Norte. Rabin disse a Sharon que, caso se
portasse bem durante um ano, seria promovido. O ano passou sem problemas e
ele então nomeou Sharon como chefe de instrução das \textsc{idf} e o promoveu a
general.

Como chefe do Estado"-Maior, Rabin continuou a combinar uma visão
abrangente com um excepcional domínio dos detalhes. Às vésperas e depois
das operações, dirigia"-se ao campo e estudava os mínimos detalhes de cada
exercício, através de conversas com oficiais subalternos e simples
soldados. Famoso por sua falta de elegância e impaciência com longas
discussões nas reuniões do comando, Rabin demonstrava uma paciência não
usual e elegante calma ao tratar com soldados.

Entre 1949 e 1964, Rabin ocupou diversas posições que tiveram um papel
importante na construção das \textsc{idf} e na formação de sua doutrina. Ele
continuou, naturalmente, a fazê"-lo como chefe do Estado"-Maior da \textsc{idf}. De
especial importância seria a sua colaboração com Weizman no
desenvolvimento de Kurnass, o plano secreto para a destruição das foças
aéreas árabes, elaborado como o primeiro lance caso uma nova guerra se
tornasse inevitável.

\section{Rumo à guerra}

Quando Rabin iniciou seu mandato como chefe do Estado"-Maior das \textsc{idf}, em
janeiro de 1964, a estimativa da inteligência israelense era de que o
mundo árabe não iniciaria uma nova guerra no futuro próximo. Em janeiro
de 1964, boa parte do exército egípcio estava atolado no Iêmen, em um
esforço desesperado para ajudar o governo republicano a derrotar uma
rebelião tribal apoiada principalmente pela Arábia Saudita. Em seu
primeiro dia na função, Rabin apresentou aos membros de seu comando
geral sua visão da posição da segurança nacional israelense: ``Acredito
que não há perigo de um ataque imediato a Israel, enquanto pudermos
manter um equilíbrio de poder razoável, no mínimo no nível atual. É
importante alterar o equilíbrio a nosso favor. Acredito que seja
possível manter o equilíbrio de poder, o que na prática significaria a
relutância e a falta de motivação para envolver"-se conosco também no
futuro''.\footnote{\textit{Ibid}., p. 337.}

Entretanto, três cenários se desenvolviam na época e alterariam a
equação israelo"-palestina, culminando, três anos e meio depois, na crise
de maio de 1967 e na Guerra dos Seis Dias. Eram esses o declínio do
nasserismo e do governo de Nasser no Egito; o renascimento do
nacionalismo palestino; e a radicalização da política síria.

A fracassada aventura no Iêmen era somente um dos aspectos do declínio
do nasserismo após seu auge, no final da década de 1950. A Síria
abandonou a união com o Egito em setembro de 1961, o que o
estudioso americano Malcolm Kerr chamou de ``a Guerra Fria árabe'' do
início dos anos 1960 contra os Estados árabes conservadores e os regimes
baahtistas da Síria e do Iraque. O Egito tornou"-se mais um cliente do que
um aliado da União Soviética, e Nasser tinha dificuldades em lidar com
os desafios domésticos políticos e econômicos. Um desses desafios, não
totalmente compreendido em Israel na época, era a perda de controle de
Nasser sobre os militares, que se tornaram domínio privado de seu vice,
o marechal de campo Abdel-Hakim Amer.

Durante o auge do nasserismo, a maioria dos palestinos eram seus
ardentes apoiadores. Sentiam que sua redenção viria quando ele
unisse o mundo árabe e desenvolvesse a força formidável, necessária para
derrotar Israel. Mas, com o início do declínio do nasserismo, uma
variedade de grupos palestinos, alguns deles de orientação islâmica,
começaram a se organizar para atuar por conta própria --- especialmente a
Fatah, de Yasser Arafat. Fatah (em árabe, ``conquista'') é o acrônimo
invertido de Movimento Árabe para a Libertação da Palestina. Fundado no
final dos anos 1950, e mantendo"-se na ilegalidade até 1965, o Fatah era
um grupo nacionalista palestino com um toque de islamismo, cujo objetivo
era ressuscitar a dimensão palestina do conflito árabe"-israelense após a
derrota e a dispersão de 1948--49.

Mas o elemento mais importante, que levou à crise de maio de 1967, foi a
radicalização da política síria. Os três anos e meio de uma união
fracassada com o Egito, de fevereiro de 1958 a setembro de 1961,
reviveram o senso de singularidade síria. A Síria se separou do Egito
mas a secessão foi mais simples do que a construção de um regime estável. O
Egito se recusou a reconhecer uma nova Síria independente e Nasser ainda
tinha influência suficiente para negar legitimidade à Síria. Ele taxou
os novos líderes sírios de traidores da causa do nacionalismo
pan"-árabe, criticando"-os por solapar seu resultado mais importante: a
primeira união entre dois Estados árabes.

Em março de 1963, um grupo de oficiais do partido Baath deu um golpe de
estado na Síria. Assim como o
nasserismo, o Baath era a favor da união árabe e de uma versão do
árabe do socialismo, mas, após um período de estreita colaboração no final
dos anos 1950, Nasser e o Baath tornaram"-se rivais implacáveis. Tendo
tomado o poder na Síria, o governo Baath apoiava"-se em uma limitada
base política, apoiada por uma facção militar e pela pequena ala esquerda
radical do partido, provocando o ressentimento da maioria, composta de árabes
muçulmanos sunitas. No final de 1963, o grupo tentou transformar um
desafio em uma oportunidade. Israel anunciou que estava por completar o
aqueduto nacional, bombeando água do Lago Tiberíades e levando"-a ao sul
do país. O projeto era visto pelo mundo árabe como um dramático ponto de
inflexão porque, com a água que estaria disponível na região árida do
sul de Israel, o país poderia aumentar sua população e consolidar sua
existência. Nos anos 1950, os governantes conservadores da Síria foram
bem"-sucedidos ao impedir os primeiros esforços desse projeto. O que
poderiam fazer os seus sucessores radicais para superar a humilhação de
fracassar e de não impedir a nova encarnação do projeto? A solução
encontrada pela liderança do Baath sírio foi anunciar que iriam à
guerra para impedi"-lo. Não era meramente uma fórmula para salvaguardar as
aparências, mas uma ameaça direcionada ao Egito: se os novos líderes
radicais da Síria iniciassem uma guerra contra Israel e fossem
derrotados, poderia o Egito manter"-se fora da guerra? Era um risco que
Nasser não podia assumir. Para conter os sírios, ele convocou uma cúpula
árabe em janeiro de 1964.

O objetivo da conferência era lidar com a nova ameaça israelense. Nasser
almejava um consenso árabe que contivesse a Síria. Mas o alcance da
cúpula árabe ampliou"-se: a resposta não lidaria somente com o
projeto do aqueduto israelense, mas formularia uma estratégia abrangente
contra o inimigo sionista. Foram tomadas três decisões principais:
primeiro, desviar os tributários do Jordão; segundo, criar um ``comando
árabe unificado'', com o fim de proteger os trabalhos de desvio das
águas; e terceiro, formar a Organização para a Libertação da Palestina
(\textsc{olp}) com o objetivo de criar, pela primeira vez desde 1949, uma
entidade palestina que transformasse a questão palestina de um problema
de refugiados em um confronto de entidades contra Israel. Como seus
rivais do Baath, Nasser tentava transformar a adversidade em uma
vantagem. Além de simplesmente restringir a Síria, ao criar o Comando
Árabe Unificado, dependente do poder militar egípcio, ele tentava
exercer pressão sobre a Arábia Saudita --- o que lhe permitiria uma
retirada honrosa do Iêmen. A Arábia Saudita teria o ônus de concordar
com uma solução política no Iêmen, que permitisse o retorno das tropas
egípcias, para serem colocadas sob o novo Comando Árabe Unificado. A
formação da \textsc{olp} estava voltada não só contra Israel mas também contra a
Jordânia, um dos inimigos conservadores do Egito. Abdullah, o avô do rei
Hussein da Jordânia, havia anexado a Cisjordânia e dado cidadania
jordaniana aos palestinos. A formação de uma nova entidade palestina
representava uma ameaça direta à Jordânia, ao demandar a lealdade de sua
população palestina.

A Cúpula do Cairo foi a primeira de uma série de conferências que se
tornariam o marco das relações inter"-árabes. As principais decisões da
cúpula foram rapidamente implementadas: foi criado o Comando Árabe
Unificado, sob o comando de um general egípcio; a \textsc{olp} foi formada,
liderada por Ahmad Shuqeiri, um \textit{protégé} egípcio; e foram feitos
os planos para desviar os tributários do Jordão na Síria e no Líbano.

A implementação dessas resoluções exacerbou o conflito árabe"-israelense
e levou à crise de maio de 1967. A criação da \textsc{olp} estimulou a emergência
dos grupos palestinos autênticos que haviam começado a se cristalizar na
década de 1950. Sua atitude frente aos países árabes era complexa:
queriam e necessitavam desse apoio, mas aprenderam em 1948 que tinham de
controlar seu próprio destino para não serem utilizados como um
instrumento da política árabe. Em janeiro de 1965 a primeira ação
terrorista foi praticada pelo Fatah, quando tentou sabotar o aqueduto
nacional israelense. O ato marcou o início do caminho da organização
rumo à ressureição do movimento nacional palestino e à sua transformação
em um ator independente no contexto da política árabe.

Com a Síria tendo iniciado os trabalhos para o desvio das fontes
do Jordão, uma nova dimensão se somava ao conflito com Israel. A
frente síria fora a linha de confronto mais ativa do conflito
árabe"-israelense desde 1949. A maioria dos incidentes relacionava"-se a
disputas sobre a zona desmilitarizada e direitos de pesca no Lago
Tiberíades. Israel estava determinada a impedir que a Síria completasse
os trabalhos de desvio das águas, ainda que tivesse de usar a força. Em
1964 e 1965, as \textsc{idf}, lideradas e incitadas por Rabin, demonstraram que
podiam destruir o equipamento sírio com disparos de tanques, limitando
o confronto e um conflito em larga escala. A Síria retaliou,
adotando o Fatah e apoiando suas atividades terroristas, originadas
primordialmente em território jordaniano. Não era somente uma maneira de
vingar"-se de Israel mas também uma demonstração da alegação do governo
sírio de que havia adotado uma ``guerra popular de libertação'' contra
Israel. O porta"-voz do regime argumentava que, como no Vietnã, a forma
de enfrentar um inimigo militarmente superior era o abandono dos métodos
de guerra clássicos e a adoção de métodos populares. Era uma
mal"-disfarçada crítica ao Egito e seus métodos tradicionais. Ao
encorajar o Fatah a cometer seus atos terroristas a partir da Jordânia, a
Síria buscava provocar um conflito entre Israel e seu rival árabe
conservador, reduzindo assim o risco de uma retaliação israelense contra
ela própria.

Em pouco tempo foi criado um ciclo de violência. A destruição por
parte de Israel dos equipamentos sírios gerava uma retaliação síria, fosse
através das linhas de armistício ou de um esquadrão terrorista do Fatah.
Israel retalharia o ataque e uma nova rodada de violência teria início.
A Síria facilmente intensificou o conflito ao bombardear as aldeias
israelenses a partir de suas posições vantajosas nas Colinas do Golã.
Em novembro de 1964, Rabin obteve a permissão de Eshkol para utilizar a
força aérea e silenciar a artilharia síria. Rabin argumentou que, dada a
maciça utilização do poderio aéreo norte"-americano no Vietnã, Washington
teria dificuldade em criticar Israel por fazer uso de sua superioridade
aérea para proteger sua população civil.

O golpe de Estado na Síria em 23 de fevereiro de 1966 colocou no poder a
ala radical do Baath e seus aliados militares, escalando mais uma vez o
conflito. O novo regime era mais radical do que o anterior
e mais claramente dominado pelas comunidades minoritárias do
país, alauitas, drusos e ismaelitas. Apoiava"-se em uma base extremamente
estreita e pareceram várias vezes estar à beira do colapso. Seus líderes
estavam dispostos a engajar"-se em uma política mais aventureira,
iniciando ataques diretos e indiretos contra Israel, que preocupavam os
seus patronos soviéticos, assim como o Egito. Nasser se preocupava com
duas possibilidades: o colapso do regime e sua possível substituição por
um regime conservador aliado a seus rivais árabes, e uma guerra aberta
contra Israel que o obrigasse a ``descer do muro''. Estava, portanto,
disposto a reconhecer a legitimidade do Estado sírio, reestabelecendo
relações diplomáticas, então avançou, assinando com Damasco um pacto de
defesa em novembro de 1966.

Se Nasser esperava que essas concessões servissem para acalmar as
tensões, decepcionou"-se rapidamente. Em agosto de 1966 houve um grande
confronto sobre o Lago Tiberíades e, em 7 de abril de 1967 seis jatos da
força aérea síria foram derrubados pelos israelenses perto de Damasco.
Um círculo vicioso perfeito foi alimentado pelos líderes sírios
radicais, pela imprudência e a convicção de Rabin e seus generais de que
respostas cada vez mais violentas eram a única forma efetiva de conter
os sírios. A ala moderada do gabinete, liderada pelos ministros do
Partido Nacional Religioso, opunha"-se consistentemente a esta linha de
atuação --- especialmente ao emprego da força aérea --- mas, com raras
exceções, Eshkol tendia a aceitar as preferencias de Rabin. Em distintos
períodos da historia de Israel a formulação da política de segurança
nacional seria dividida entre o primeiro"-ministro, o ministro da Defesa
e o chefe do Estado"-Maior. Entre 1964 e 1967, Eshkol, como primeiro"-ministro 
e ministro da Defesa, tendia a atender a Rabin, que se tornou o
principal arquiteto da política de segurança nacional.
Caracteristicamente, Rabin lidou tanto com as questões mais amplas
quanto com detalhes, tais como o uso de tanques para alvejar tratores,
engajados no esforço para desviar os tributários do rio Jordão.

Havia uma dimensão jordaniana no processo. Em novembro de 1966, um
artefato explosivo, colocado por um esquadrão palestino enviado da Síria
através da Jordânia, ceifou a vida de quatro soldados israelenses.
Israel tinha claro que o esquadrão havia vindo da Síria, mas decidiu
retaliar não contra a Síria, mas contra a Jordânia; alegou que esta era
responsável por vedar suas fronteiras e seu território contra tais
atividades. O ataque retaliatório israelense contra o posto de polícia
da aldeia de Samu, próxima a Hebron, no sul da Cisjordânia, foi uma
operação maciça e um enorme erro de avaliação. A Síria e o Egito,
inimigos árabes radicais da Jordânia, criticaram o regime hachemita
jordaniano conservador pró"-ocidental por sua incapacidade de defender o
território nacional e sua população, e conclamaram a introdução de um
serviço militar compulsório. Isso causaria a palestinização do exército
jordaniano e a destruição do pilar do regime, ou seja, a lealdade dos
oficiais beduínos de suas forças armadas. A Jordânia reagiu provocando o
regime nasserista e acusando"-o de ``esconder"-se embaixo da saia dos
soldados da \textsc{onu}'' que em 1957 foram estacionados na fronteira com Israel
e nos Estreitos de Tiran. Era um caso clássico da ``Guerra Fria árabe''
de Kerr. O vínculo entre a briga inter"-árabe e o conflito
árabe"-israelense contribuiu fortemente para a escalada que culminou na
crise de maio de 1967.

A liderança militar egípcia não estava disposta a aceitar a humilhação
causada pela provocação jordaniana. Após a derrota egípcia em junho de
1967, o marechal de campo Abdel-Hakim Amer e vários outros altos
oficiais foram julgados por uma corte marcial no Egito por seus atos
antes e durante a guerra. Os protocolos da corte marcial esclareceram as
decisões egípcias anteriores aos combates e revelaram que, durante uma
viagem ao Paquistão, em novembro de 1966, Amer enviou um telegrama a
Nasser exigindo a remoção da força de paz da \textsc{onu} e a remilitarização da
Península do Sinai. O argumento de Amer era de que o exército egípcio
poderia conter as \textsc{idf} no Sinai, caso Israel se decidisse pela guerra em
virtude do colapso do regime de segurança instalado após a Guerra do
Sinai. Um paragrafo paralelo nas memórias de Rabin indica que os órgãos
de defesa em Israel estavam cientes da posição de Amer, mas não lhe
deram a devida importância. Até a primavera de 1967 ainda se acreditava
que o Egito não seria arrastado a uma guerra ``por causa de um trator
sírio'', conforme Nasser havia declarado.

Rabin concordava com esta avaliação. Acreditava que Israel havia
encontrado a solução para o desafio do desvio das águas, mas não tinha
uma resposta adequada para o apoio sírio às atividades terroristas. Ele
e seus colegas do Estado"-Maior chegaram à conclusão, no final de 1966 (e
a mantiveram até maio de 1967), de que somente uma operação militar em larga
escala contra a Síria colocaria um ponto final nesse desafio e que o
Egito não se envolveria na luta. Em uma entrevista que deu ao semanário
das \textsc{idf} às vésperas do ano novo judaico, em setembro de 1966, Rabin
explicou que, ao contrário do Egito, da Jordânia e do Líbano, que se
opunham a atividades terroristas, a Síria as patrocinava, o que
demandava um tipo diferente de resposta. ``A resposta às atividades da
Síria'', disse, ``tem de ser dirigida aos perpetradores e contra o
regime que os suporta {[}\ldots{}{]} O objetivo deve ser a mudança das decisões do
regime e a remoção da motivação para tais atividades. O problema com a
Síria é, em essência, um confronto com o regime.''

Rabin não ameaçou exatamente derrubar o regime Baath, mas sua
declaração foi amplamente entendida dessa forma, gerando uma tormenta de
protesto e uma reprimenda de Eshkol a seu chefe de Estado"-Maior. Mas ele
o fez de forma privada e então emitiu uma declaração mais amena em 18 de
setembro, dizendo: ``Israel não intervém nas questões domesticas de
outros países e seus regimes''.

Não foi a primeira vez que Eshkol repreendeu Rabin. Em setembro de 1964,
escrevera a Rabin uma carta ``absolutamente privada'' criticando"-o
por ter feito na mídia comentários de natureza diplomático"-políticas.
Tais comentários, escreveu, não deveriam ser feitos, a menos que fossem
absolutamente essenciais e, sendo o caso, era necessário ``coordená"-los
com os atores relevantes.'' O assistente militar de Eshkol anotou em seu
diário que a reprimenda serviu também a outro propósito sutil: apesar de
sua boa relação, Eshkol invejava a popularidade de Rabin e explorou seu
deslize para repreendê"-lo e instruí"-lo a solicitar autorização prévia
para dar entrevistas. Mas o primeiro"-ministro ``pegava leve'' em seu
relacionamento com Rabin e deixou que o diretor"-geral de seu ministério,
Yaacov Herzog, negociasse o texto da declaração do gabinete relativa à
entrevista. Rabin não se impressionou com a insatisfação de
Eshkol ou com a crítica da mídia. Em 26 de setembro de 1966, declarou em um
encontro do comando geral das \textsc{idf}: ``emiti uma ordem para que todo nosso
pessoal não se envolva em discussão, debate, aula ou instrução; para se
manterem fora do tema. Deixem que escrevam o que quiserem: amanhã haverá
outro jornal. A atividade de relações públicas das \textsc{idf} continuará,
apesar de eu ter pedido que, na primeira fase, aqueles que se
expressarem o façam com muita cautela. De qualquer forma, em antecipação
ao dia da campanha do Sinai, haverá aparição publica de oficiais pois
até o momento não recebi instruções para impedi"-las e não tomarei a
iniciativa de fazê"-lo''.\footnote{\textit{Ibid}., p. 417.} Rabin respeitava a subordinação
militar ao nível político e não concordava plenamente com a repercussão
diplomática e política de declarações para a imprensa, mas estava
determinado a manter o controle de todos os aspectos das atividades das
\textsc{idf}.

\section{A crise de maio de 1967}

Na segunda semana de maio de 1967, eclodiu uma crise que, quase
imediatamente levou à Guerra dos Seis Dias. O dia da independência, 14
de maio, é o feriado laico mais importante de Israel. Em maio de 1967, as
preparações para o feriado foram ofuscadas por tensões e controvérsias.
O governo havia decidido realizar em Jerusalém a tradicional parada
militar. A decisão foi contestada pela Jordânia, que argumentou ser esta
uma violação do acordo de armistício de 1949, assinado entre os
dois países. O governo respondeu reduzindo a parada militar e dividindo
a celebração em duas partes: uma cerimonia festiva no estádio da
Universidade Hebraica em 14 de maio e uma pequena parada militar em 15
de maio. A decisão foi criticada por Ben Gurion, que alegou que seu
sucessor, Eshkol, mostrava"-se ``fraco em questões de segurança''. A
controvérsia aconteceu enquanto aumentavam as tensões nas fronteiras
síria e jordaniana. Os planejadores da cerimônia festiva que seria
realizada em 14 de maio decidiram incluir no programa a leitura de um
poema escrito em 1956 pelo poeta nacional israelense, Nathan Alterman,
que foi na verdade o prenúncio da campanha do Sinai, deflagrada alguns
meses depois. Os versos de Alterman, transmitidos através dos
alto"-falantes, foram um grave alerta ao mundo árabe e, em
retrospectiva, também proféticos: ``Ó Araby, calcule o seu trajeto a
tempo. A corda está se pondo mais e mais fina {[}\ldots{}{]} descarte os sonhos
alucinados, esta pode ser a última hora''. Naquele mesmo momento, Rabin
recebeu e transmitiu a Eshkol as primeiras noticias relativas à
passagem de unidades blindadas egípcias, do território egípcio para a
península do Sinai.

A remilitarização do Sinai, uma clara violação dos acordos de 1957, logo
expandiu"-se e foi conduzida com alarde. As unidades egípcias moviam"-se
abertamente, algumas através das ruas do Cairo, à vista de todo o mundo.
Eshkol e Rabin concordavam com a avaliação de que esta era uma repetição
das medidas tomadas pelo Egito em 1960 no caso Rotem, e Rabin ordenou às
\textsc{idf} que tomassem medidas limitadas de precaução. Hoje sabemos que a ação
egípcia de maio de 1967 pode ter sido provocada por informações falsas,
fornecidas pala União Soviética, de que Israel havia concentrado pelo
menos onze brigadas, planejando um grande ataque contra a Síria. A
motivação de Moscou em fornecer estas informações ao Egito e à Síria não
ficaram claras até hoje: tinha por objetivo induzir o Egito a repetir a
manobra Rotem, e assim impedir Israel de agir contra seu cambaleante
cliente Sírio? Ou era mais sinistra? De qualquer forma, produziu"-se uma
avalanche de eventos que ninguém havia previsto. Uma onda de entusiasmo
varreu o Egito e o mundo árabe na esteira da remilitarização do Sinai.
Nasser e seus associados sentiram que o nasserismo, que havia declinado
por anos, experimentava uma revitalização. Embriagado por seu ego e
encorajado, Nasser abandonou a cautela. Exigiu que a \textsc{onu} removesse suas
forças de paz da fronteira egípcio"-israelense no Sinai, concentrando"-as
em Gaza. Mas não exigiu que elas fossem retiradas de Gaza ou do
Estreito de Tiran. O secretário"-geral da \textsc{onu}, U Thant,
inexplicavelmente optou por informar o governo egípcio que não
acreditava em paliativos e, quando os egípcios ampliaram a demanda
incluindo o Estreito de Tiran, U Thant aceitou"-a. Previsivelmente, o
passo seguinte foi uma declaração de embargo aos navios israelenses e a
determinados produtos destinados ao porto de Eilat em navios
não israelenses. Israel definiu o embargo como um \textit{casus belli}
e a guerra parecia iminente, se não inevitável.

Israel encontrava"-se então em meio a uma severa crise. Discussões
prévias com o governo de Lyndon Johnson em Washington indicavam não
haver apetite da parte dos Estados Unidos para engajar"-se e aliviar a crise. Com o
passar do tempo, o governo parecia tender ao início de um diálogo
com um Egito fortalecido. A Síria era a aliada do Egito, e a Jordânia
parecia desmoronar com a pressão; uma guerra em três frentes tinha de
ser levada em conta. Rabin e seus colegas do quartel"-general enfrentavam
sozinhos um grande desafio. As reservas foram mobilizadas e vários
planos foram revistos para o que parecia ser uma guerra inevitável.
Rabin havia estruturado as \textsc{idf} para lidar com um ataque árabe em várias
frentes, e havia vários planos contingenciais para diversos cenários. Mas
Israel não tinha condições de manter um grande número de reservistas
mobilizados por muito tempo; o custo para a economia do país era
proibitivo. Havia indicadores de que o Egito poderia não aguardar um
ataque israelense, optando por iniciar os combates. Outros indicadores
provocavam o temor de um ataque aéreo contra o reator nuclear israelense
em Dimona. Abriu"-se uma ampla brecha entre as opiniões do gabinete e do
quartel general. Os ministros moderados, que haviam criticado o ativismo
do exército desde 1964, eram agora veemente críticos de um quartel"-general
que, em sua opinião, levara o país à beira de um desastre. Rabin,
até recentemente respeitado pelo gabinete e visto por seus membros quase
como um deles, um virtual ministro da Defesa, perdeu muito de sua
estatura devido às avaliações imprecisas da conduta árabe e a sensação
de que havia levado as \textsc{idf} e o país à beira do abismo. Os generais, por
outro lado, tinham uma visão completamente diferente da situação.
Especialmente após o bloqueio do Estreito de Tiran, acreditavam que Israel
deveria entrar em guerra: o tempo não favorecia os israelenses. Também
estava em jogo a capacidade do país de dissuadir seus inimigos árabes.
Esta já havia se erodido ao não responderem ao desafio inicial do Egito
e continuaria a diminuir enquanto o governo hesitasse e buscasse uma
solução diplomática. O consenso entre os generais era de que Israel
poderia vencer a guerra.

O maior defensor de uma solução diplomática era o principal diplomata do
país, o ministro das Relações Exteriores, Abba Eban. Anteriormente,
havia surgido uma controvérsia entre Eban e os militares (especialmente
a inteligência militar) a respeito da interpretação da real posição de
Washington. Normalmente, os líderes e porta"-vozes norte"-americanos
alertavam Israel a não optar pela guerra, e Eban alegava que suas
declarações deviam ser tomadas ao pé da letra. A inteligência militar
atestava que tais advertências deveriam ser vistas com cautela e que o
presidente e outros membros do governo na verdade esperavam que Israel
resolvesse a crise por si só; os Estados Unidos não tinham nem a intenção nem o
desejo de participar de uma flotilha que rompesse o bloqueio. Mas nem o
presidente daquele país nem qualquer alto oficial queria deixar registrado
que havia encorajado outra nação a engajar"-se em uma guerra. Assim,
optaram por oferecer a seus interlocutores israelenses uma mensagem
dúbia, alertando Israel contra a guerra, mas dando sinais indiretos de
que deveria agir por si só com o objetivo de romper o bloqueio e
eliminar a ameaça egípcia mais ampla.

Eban se opunha à opção militar. Ele convenceu Eshkol de que deveria
viajar a Paris e Washington para buscar uma solução com a ajuda de
Charles de Gaulle e Johnson. Sua viagem adiou a decisão de iniciar a
guerra e tornou"-se fonte de novas controvérsias, baseadas em seu
relatório, que pontuava a promessa de Johnson de organizar uma força naval
multinacional para romper o bloqueio. Os céticos convenceram Eshkol a
confirmar, enviando a Johnson uma mensagem pessoal agradecendo"-o por
esse posicionamento. Johnson enviou seu assessor de Segurança Nacional,
Walt Rostow, para comunicar ao ministro na embaixada israelense que o
presidente não havia feito tal promessa.\footnote{Asaf Siniver, \textit{Abba Eban: A Biography} {[}Abba Eban: uma biografia{]}. Nova York / London: Overlook
Duckworth, 2015, p. 213--45.}

A sequencia de eventos criou um problema para Rabin. Ele havia
transformado as \textsc{idf} em uma força militar formidável, liderada pelo
melhor comando e capaz de defender Israel de uma coalizão árabe.
Juntamente com Weizman, ajudou a construir uma força aérea capaz de
implementar o Kurnass, o plano secreto para a eliminação das forças
aéreas árabes em um ataque preemptivo que daria às \textsc{idf} uma vantagem
inicial massiva, caso a guerra se tornasse inevitável. Por três anos e
meio, Rabin atuou como um chefe do Estado"-Maior qualificado e popular,
negociando as relações entre os generais e o gabinete e sendo visto por
ambos como um deles. Mas agora ele estava desconfortavelmente prensado
entre os políticos e os generais, e sua importância havia diminuído em
face da crise e da ausência de uma solução clara. Ele mesmo acreditava
que Israel deveria ir à guerra, mas se absteve de insistir com Eshkol ou
com o gabinete. Sua costumeira precaução e o conceito de sua função e de
sua posição impediram"-no de fazê"-lo. Tendo perdido sua posição quase
ministerial, e como Eshkol não era um ministro da Defesa no sentido mais
amplo do termo, Rabin não tinha uma figura política qualificada em quem
se apoiar e com quem compartilhar a responsabilidade.

Em 22 de maio, um dia antes do anúncio do bloqueio egípcio, Rabin buscou
o conselho de duas pessoas que respeitava. A primeira foi Ben Gurion.
Ele foi cordial; Rabin o colocou a par da situação militar e ficou
constrangido ao descobrir como o grande líder havia se alienado das
novas realidades do país e da capacidade das \textsc{idf}. O ``Velho'' foi muito
crítico, especialmente de Eshkol mas não poupou Rabin: ``Você levou o
país a uma situação muito difícil, você é responsável. Não devemos
iniciar uma guerra. Estamos isolados''.\footnote{Rabin, \textit{The Rabin Memoirs},
\textit{op}. \textit{cit}., p. 75--6.}

Na sequência, Rabin encontrou"-se com Dayan. É sabido que a relação entre eles
há muito era tensa; em 1965, Dayan juntou"-se a Ben Gurion quando
este abandonou o Mapai para formar um novo partido, o Rafi. Como membro
do Rafi, Dayan estava na oposição e, como membro dos comitês de relações
exteriores e defesa do parlamento israelense (Knesset), havia criticado as políticas de Rabin e
Eshkol desde meados da década de 1960. Mas, tensões à parte, Rabin
nutria estima por Dayan. Entretanto, esse encontro não forneceria nem
conforto nem orientação. Conforme relata em suas memórias, a crítica não
lhe agradou, apesar de Dayan, ao contrário de Ben Gurion, não ter se
expressado em termos pessoais. De acordo com Dayan, o governo se
equivocava ao testar a liderança de Nasser no mundo árabe e encurralá"-lo
ao agir contra a Síria e a Jordânia. Ao fazê"-lo, alegou, Israel forçava
Nasser a defender seu prestígio em seu próprio país assim como no mundo
árabe e gerava uma grave escalada no Oriente Médio. Do ponto de vista de
Dayan, Nasser reagiria com uma escalada, bloqueando os estreitos, o que
levaria Israel a agir.\footnote{\textit{Ibid}.}

No dia seguinte, 23 de maio, após o bloqueio do Estreito de Tiran, Rabin teve
um confronto ainda mais duro, dessa vez com o ministro do Interior,
Moshe Haim Shapira, líder do Hamafdal, o Partido Nacional Religioso.
Naquela época, Hamafdal era um partido
moderado e Shapira, membro do comitê ministerial de Defesa, era o mais
consistente opositor de uma ação militar. Após a reunião do comitê, Rabin
teve um encontro com ele e sofreu uma avalanche de críticas. ``Você
realmente pensa que o time Eshkol"-Rabin deveria ser mais audaz e
corajoso que o time Ben Gurion"-Dayan?'', perguntou. ``Por quê? {[}\ldots{}{]} Ben
Gurion não iniciou uma guerra apesar de os egípcios terem encorajado
atividades terroristas contra Israel e terem armado e protegido os
terroristas. Quando Ben Gurion foi à guerra, em 1956? Somente com a
convergência de dois fatores: Israel não estava só {[}\ldots{}{]} a França e a
Inglaterra se incumbiram de destruir a força aérea e a marinha
egípcias {[}\ldots{}{]} as marinhas francesa e britânica defenderam a costa de Israel
e a segurança da população civil foi garantida. Como você se atreve a
envolver"-se em uma guerra em que todas as condições são adversas a
nós?''\footnote{\textit{Ibid}., p. 80--2.} Era o mesmo Shapira que havia exigido uma reação mais
dura contra Rabin, após a manifestação do Palmach. Seja por rancor ou
como reflexo de sua moderação, ficou clara a severidade de suas
críticas.

Os três duros encontros cobraram um pesado preço de Rabin. Ele teve
dificuldade em lidar com as crescentes pressões em várias frentes: mesmo
após o bloqueio do Estreito, o gabinete mantinha"-se relutante em votar
pela guerra, enquanto os generais tornavam"-se mais agitados e
militantes. Rabin se remoía em dúvidas e sensação de culpa,
relativas a seu papel na escalada de 1964--1967, em especial a operação
Samu e algumas de suas declarações. Mencionou a seu vice,
Weizman: ``não posso escapar da sensação de que, em conjunto com o nível
político, tenho minha parte em haver levado Israel a uma situação
difícil --- a mais difícil desde a Guerra de Independência''.

Na tarde de 23 de maio, Rabin sucumbiu ao impacto da combinação da
exaustão física e ansiedade aguda. Ele escreveu em suas memórias, em 1979,
que não podia explicar a si mesmo ``porque cheguei à exaustão física e
mental''. Ele as atribuiu ao peso de uma semana de trabalho
interminável, à falta de sono, ao fumo desenfreado e a uma sensação de
culpa pelos erros cometidos. Naquela noite, convidou Weizman para ir a
sua casa e o que se seguiu envolve uma controvérsia. De acordo
com Rabin, seu interesse era o de compartilhar suas ideias e
preocupações ``tentando aliviar sua enorme angústia ao expor"-se a outra
pessoa. Sou fechado e, naquele momento tinha uma enorme necessidade''.
Ele contou a Weizman que ``estava com uma sensação de fadiga e
angústia'', que se via como corresponsável por levar Israel à crise do
momento. Abrindo"-se totalmente, perguntou a Weizman se deveria renunciar.
Weizman o demoveu da ideia e se foi. Leah ligou para o médico"-chefe das
\textsc{idf}, que deu a Rabin uma injeção para que dormisse. No dia 25, Rabin
reassumiu suas responsabilidades. O episódio foi mantido em segredo, e
seus colegas foram informados de que Rabin havia sucumbido a um
``envenenamento por nicotina''.

A versão de Weizman sobre o evento é bem diferente e menos elogiosa de
Rabin. Suas relações sempre tiveram altos e baixos e, mais tarde, a
decisão de Rabin de apontar Haim Bar"-Lev e não Weizman como vice"-chefe
do Estado"-Maior não serviu para melhorá"-las. Em suas memórias, Weizman
descreve que, desde a eclosão da crise em meados de maio, sentiu que
``a condição e a estabilidade de Yitzhak Rabin, chefe do Estado"-Maior,
estavam se deteriorando. Manifestava"-se através de mudanças de resoluções,
expressões de ansiedade em relação ao futuro e a incapacidade de tomar
decisões. Rabin gerava insegurança ao seu redor. Era visível em
encontros com o primeiro"-ministro e sessões do Estado"-Maior''. De acordo
com ele, ao visitar Rabin em sua casa viu ``uma pessoa
deprimida, destruída''. Rabin assumiu a responsabilidade por seus erros,
disse Weizman, e disse"-lhe que queria renunciar, oferecendo"-lhe o cargo
de chefe do Estado"-Maior. Weizman galantemente rejeitou a oferta e
disse"-lhe que se recompusesse, que se tornasse o vitorioso chefe do
Estado"-Maior de guerra iminente, que seria bem"-sucedida. É uma versão
problemática dos eventos. Weizman, como vice, teria certamente atuado em
lugar de Rabin até que este se recuperasse, mas a nomeação de um chefe
do Estado"-Maior das \textsc{idf} era prerrogativa do gabinete.

No dia 24, enquanto Rabin se recuperava, Weizman moveu"-se ativamente
como seu substituto e convocou o Estado"-Maior para lidar com os planos
de guerra. Rabin reassumiu o comando em 25 de maio, mas levou algum
tempo até que retomasse o controle total. Seus colegas e o primeiro"-ministro
o apoiaram, e o incidente foi mantido em segredo até 1974,
quando Weizman o tornou público, para apoiar Peres em sua disputa com
Rabin.

Quando Rabin retornou a seu posto, o gabinete ainda não estava pronto
para a guerra. A pressão pública e política havia aumentado e a
credibilidade de Eshkol desabou quando ele gaguejou em um discurso à nação.
Eshkol foi obrigado a abdicar da pasta da Defesa e a dar algum cargo a
Dayan. Este declarou inicialmente que queria tornar"-se o oficial
comandante do Comando Sul. Eshkol concordou, na esperança de que assim
pudesse aliviar a pressão para nomeá"-lo ministro da Defesa. Rabin ficou
obviamente descontente com a ideia, ciente de que Dayan não aceitaria
sua autoridade como seu superior nominal, mas não se opôs. Finalmente,
Rafi e o partido de direita Gahal, uma união do nacionalista Herut com o
liberal"-conservador Sionistas Gerais, liderado por Begin, juntaram"-se à
coalizão para formar um governo de união nacional, e Dayan foi nomeado
ministro da Defesa. Bar"-Lev foi nomeado vice"-chefe do Estado"-Maior. Essa
grande leva de mudanças abriu o caminho para a decisão de ir à guerra,
facilitada pelo envio a Washington de Meir Amit, chefe do Mossad (Instituto para Informações e Operações Especiais, a
agencia de inteligência nacional), para determinar através de seus
próprios canais que o governo Johnson não se opunha à ação israelense.

Uma vez iniciada, a Guerra dos Seis Dias revelou"-se um sucesso militar
brilhante. Seu impacto foi dramaticamente ampliado pelo nítido contraste
entre a ansiedade do período anterior à guerra e a magnitude e rapidez
da vitória. Em seis dias, as \textsc{idf} destruíram três forças aéreas árabes e
conquistaram o Sinai, as Colinas do Golã e a Cisjordânia do Egito, da
Síria e da Jordânia.

Apesar do sucesso da guerra, a colaboração e coordenação entre Rabin e
Dayan esteve longe de ser perfeita. Dayan reclamou que as \textsc{idf} chegaram
ao Canal de Suez contra sua vontade, enquanto Rabin descobriu, \textit{pós
facto}, que Dayan, sem informá"-lo e no último momento havia ordenado
(por telefone) ao comandante do Comando Norte que capturasse as Colinas
do Golã. Mas a magnitude da vitória obscureceu todos os problemas. E
havia glória suficiente para ambos, Dayan e Rabin. O auge para Rabin
após a guerra foi o discurso que proferiu em 28 de junho de 1967, no
anfiteatro da Universidade Hebraica de Jerusalém, no Monte Scopus. O
Monte Scopus era o \textit{campus} original da universidade e manteve"-se como um
enclave na área de Jerusalem controlada pela Jordânia após 1948. Em uma
manifestação da nova realidade de Israel e de Jerusalém, a cerimônia
inicial da universidade foi realizada naquele local, agraciando com um
doutorado honorário o vitorioso chefe do Estado"-Maior das \textsc{idf}. Rabin foi
muito elogiado por seu discurso de aceitação do título, que dizia:

\begin{quote}
Toda a nação foi envolvida por uma onda de alegria, mas, ainda assim,
encontramos repetidamente um estranho fenômeno entre os combatentes.
Eles não podem estar plenamente felizes e, mais de um elemento de
tristeza e perplexidade soma"-se à sua celebração, e alguns deles nem mesmo
celebram. Os combatentes nas linhas de frente viram com seus próprios
olhos não somente a glória da vitória mas também seu preço --- seus
camaradas caíram junto a eles, cobertos de sangue. E sei que o terrível
preço pago pelo inimigo também afetou profundamente a muitos deles. É
possível que o povo judeu nunca tenha sido educado ou acostumado a
sentir a alegria da conquista e da vitória. É, portanto, recebida com
sensações ambíguas.
\end{quote}

Somente nos anos que se seguiram ficou clara a magnitude dos problemas
criados pela vitória. O enorme triunfo dos militares é visto, em
retrospectiva, como uma bênção duvidosa. Israel obteve uma grande
vitória, e uma enorme crise foi transformada em uma nova realidade ---
com Israel emergindo como uma potência militar e um importante ator
regional, tendo capturado territórios três vezes maiores do que sua própria
área. Os territórios capturados na guerra finalmente forneciam a carta
de negociação para um acordo de paz, que faltava desde 1949. Mas, com a
subsequente ausência de progresso em direção a um acordo, os territórios
capturados tornaram"-se também um empecilho. A conquista da Cisjordânia
fez reemergirem ideias e sentimentos dormentes desde 1948 e gerou uma
onda messiânica que transformou o sionismo religioso de uma ala moderada
do sistema político israelense em um movimento e um partido radical
nacionalista.

O debate a respeito da entrega desses territórios, especialmente a
Cisjordânia e a Faixa de Gaza, partes da histórica Terra de Israel, tem
desde então dividido a sociedade e o sistema político israelenses. Em
uma manifestação da ironia da história, Rabin, o arquiteto da grande
vitória militar de 1967, foi o primeiro"-ministro convocado para lidar com
essa bênção duvidosa, e seu esforço custou"-lhe a vida.

\chapter[Embaixador em Washington, 1968--73]{Embaixador em Washington, 1968--73}
\markboth{Embaixador em Washington}{}

O posto de embaixador israelense em Washington, como bem sei, é um posto
diplomático incomum. Desde a Segunda Guerra Mundial, a embaixada em
Washington tem sido o posto mais importante e mais graduado nos
ministérios das Relações Exteriores. Tendo em conta a importância crucial
das relações entre Israel e os Estados Unidos, o embaixador em
Washington é, na maioria dos casos, um emissário pessoal do primeiro"-ministro:
um confidente leal ou uma figura pública e não
necessariamente um diplomata profissional. Embaixadores bem"-sucedidos
tendem a não atuar como diplomatas tradicionais mas sim como elementos
ativos da política em Washington, tornando"-se familiares no Congresso e
na mídia do país, interagindo com os mais altos escalões da Casa
Branca, do Departamento de Estado e do Pentágono. A peculiaridade dessa
posição é devida à importância dos temas relacionados ao Oriente Médio e a
Israel na agenda política e de segurança nacional norte"-americana, bem como à
influencia (real ou imaginada) exercida pelo embaixador israelense sobre
a comunidade judaica dos Estados Unidos. Muito depende da postura e da
habilidade de cada embaixador, e os poderosos em Washington não perdem
tempo em descobrir se o embaixador israelense é um canal eficaz de
comunicação com o primeiro"-ministro.

Yitzhak Rabin foi, sem dúvida, um dos embaixadores israelenses mais
eficazes. No período que se seguiu à Guerra dos Seis Dias, Rabin teve
de calcular a fase seguinte de sua carreira. Ele desejava ingressar na
política e tornar"-se membro do governo, mas sentia que sua transição da
esfera militar para a política seria mais fácil se contasse com a
experiência que uma temporada como embaixador em Washington poderia lhe
proporcionar. Ele pediu a um Eshkol surpreendido que o nomeasse para
o posto, e foi atendido, apesar da objeção inicial de Eban, então
ministro das Relações Exteriores.

Rabin era especialmente qualificado para a missão, apesar de seu domínio
do inglês ser meramente adequado. Era uma figura conhecida, o vitorioso
chefe do Estado"-Maior da Guerra dos Seis Dias, fortemente ligado ao
primeiro"-ministro e ao restante do sistema político, conhecia bem os
principais temas da agenda Israel"-Estados Unidos, que incluíam o futuro
dos territórios conquistados por Israel em junho de 1967 e a busca por
sofisticados armamentos fabricados pelos Estados Unidos. Para
a comunidade judaica, Rabin era um herói militar, o arquiteto da guerra
que elevou Israel a uma nova estatura. Para seus interlocutores no
governo Johnson, no Congresso dos Estados Unidos e na mídia, parecia ser uma figura
respeitável, sábia, um canal eficaz para acesso ao primeiro"-ministro e
seu gabinete e um destacado membro da elite política israelense, por
seus próprios méritos.

Rabin precisou de algum tempo para dominar sua nova posição.
Inicialmente, quis concentrar"-se nos mais altos escalões do executivo e
deixar que sua equipe se ocupasse do que considerava ``áreas
periféricas'', como o Congresso, as comunidades acadêmicas e os
\textit{think tanks}. Com o passar do tempo, entretanto, desenvolveria uma
relação forte e frutífera também com esses grupos.

A relação de Rabin com a comunidade judaica organizada dos Estados Unidos teve
altos e baixos. Johnson tinha um círculo de amigos e confidentes judeus
--- Abe Fortas, Abe Feinberg, Arthur Goldberg, Arthur e Mathilde Krim,
para nomear alguns --- e costumava discutir com eles questões israelenses,
além de contar com sua ajuda em arrecadações e na obtenção de apoio
para sua polêmica política vietnamita. Quando Rabin assumiu seu posto, o
vice"-chefe da missão do embaixador Avraham Harman era Ephraim (``Eppy'')
Evron, que havia exercido um papel preponderante na relação com o círculo
judaico do presidente. Ficou famosa também a relação estreita entre
Eshkol e Johnson; sua personalidades e estilos políticos eram,
\textit{mutatis mutandis}, parecidas.

Rabin dava grande importância à comunidade judaica norte"-americana. Ele
e Leah viajaram extensamente pelo país visitando comunidades judaicas e
criaram um círculo de amigos judeus em Washington. Mas Rabin não
apreciava a forma como as questões israelenses eram tratadas por Johnson
e seus amigos judeus, preferindo lidar diretamente com a Casa Branca.
Rapidamente substituiu Evron por Shlomo Argov como vice"-chefe da missão.
Johnson nunca se aproximou de Rabin e tratava"-o formalmente, como o
fazem a maioria dos presidentes com a maioria dos embaixadores. Rabin
teria que esperar até a eleição de Richard Nixon para obter livre acesso
à Casa Branca e uma forte relação pessoal com o presidente.

Rabin tampouco teve uma relação fluida com a American Israel Public
Affairs Committee (\textsc{aipac}) {[}Comitê de Relações Públicas Americano-Israelense{]}, a organização hoje conhecida como o \textit{lobby}
pró"-israelense. A \textsc{aipac} dos anos 1960, liderada por Isaiah Leo (``Si'')
Kenen, um ativista sionista de origem canadense, estava longe da grande
e poderosa \textsc{aipac} que, desde os anos 1980, desenvolveu"-se a partir de um
pequeno e discreto grupo para tornar"-se uma poderosa organização
política de base. Ele e seus sucessores queriam se tornar
influenciadores"-chave da política israelo"-americana no Congresso e
ressentiam"-se de um embaixador poderoso que desejava administrar suas
próprias relações com os congressistas. Rabin acreditava que poderia
conseguir muito mais lidando diretamente com o Congresso e o Executivo,
mas, com o tempo, descobriu que judeus influentes, como o rico Arthur Burns,
um empreendedor republicano de Detroit, ou Leonard Garment, presidente do Banco
Central e advogado de Nixon, poderiam ser muito úteis
em sua missão.

Durante o mandato de Rabin em Washington, a agenda das relações entre
Israel e Estados Unidos foi definida pelo resultado da Guerra dos
Seis Dias e pelos conflitos militares e diplomáticos sobre o futuro dos
territórios capturados por Israel em junho de 1967. Logo após a guerra,
o governo de Johnson apoiou a posição israelense segundo a qual Israel
não deveria se retirar dos territórios, a menos que obtivesse um acordo
de paz. Uma das claras lições aprendidas durante a crise de 1967 era a
de que acordos pós"-conflito, como aquele imposto pelo presidente Dwight
Eisenhower em 1957, eram inadequados, e que os territórios capturados em
1967 deveriam ser utilizados para encerrar o conflito 
ou, no mínimo, consolidar as relações árabe"-israelenses. Mas o governo
Johnson e seus sucessores eram categóricos ao exigir que um acordo
árabe"-israelense ``não refletisse o ônus da conquista'' ou seja,
poderiam ser contempladas pequenas alterações das fronteiras. Mas, em
troca da paz, Israel teria que se retirar dos territórios capturados em
junho de 1967.

Essa exigência estava somente parcialmente de acordo com a posição
israelense pois, em 19 de junho, o gabinete israelense havia decidido
que, em troca de um acordo de paz e de provisões adequadas sobre a
segurança, Israel se dispunha a retirar"-se do Sinai e das Colinas do
Golã; mas a resolução não abrangia nem a Cisjordânia nem a Faixa de
Gaza. Israel entendia que a questão da soberania sobre a Palestina a
oeste do Jordão estava em aberto. A resolução era secreta, mas seu âmago
foi transmitido por Eban, então ministro das Relações Exteriores, ao secretario
de Estado norte"-americano Dean Rusk logo após a sua votação. O próprio
Rabin não tinha ciência desses desdobramentos ao ser enviado a
Washington. Quatro meses mais tarde, sob o impacto das resoluções
negativas adotadas pela conferência de cúpula árabe em Khartoum, em
setembro, o gabinete israelense revogou a decisão de 19 de junho, dessa
vez sem informar os Estados Unidos.

As diferenças entre as posições israelense e norte"-americana emergiriam
esporadicamente ao longo dos anos seguintes, mas, na segunda metade de
1967, já havia consenso suficiente para permitir que os Estados Unidos
evitassem esforços dos blocos árabe, muçulmano e soviético na \textsc{onu}, para
aprovar uma resolução exigindo uma retirada israelense total, sem o
benefício de acordos de paz com os países árabes derrotados. Finalmente,
em novembro de 1967 um meio"-termo foi encontrado com uma fórmula na
melhor tradição da ``ambiguidade construtiva''. Redigida pelo diplomata
britânico Lorde Caradon, resultou ser aceitável para as duas partes e
tornou"-se a Resolução 242 do Conselho de Segurança.

Um dos desdobramentos desses eventos foi a nomeação do diplomata sueco
Gunnar Jarring como mediador entre Israel e seus antagonistas árabes.
Jarring começou a circular entre Israel e seus vizinhos, com exceção da
Síria, que naquele momento se recusava a aceitar a Resolução 242. Os
esforços de Jarring foram em vão e, eventualmente, foram complementados
por dois foros diplomáticos: as negociações bilaterais entre os Estados
Unidos e a União Soviética e as negociações que envolveram as duas
superpotências mais a França e a Grã"-Bretanha. Como embaixador
israelense em Washington, Rabin tinha que lidar tanto com a posição dos
Estados Unidos quanto com a missão de Jarring. A gestão do ângulo norte"-americano
produzia frequentes tensões com o embaixador israelense junto à \textsc{onu},
Yosef Tekoah, que alegava que a missão de Jarring fazia parte de seu
portfólio.

Quando Jarring iniciou seu ``pinga"-pinga'' diplomático, as hostilidades
já haviam recomeçado ao longo do Canal de Suez. O que se iniciou como
uma série de ataques e contra"-ataques isolados em outubro de 1967 havia
se transformado, em março de 1968, na ampla Guerra de Atrito que
durou até agosto de 1970. Foi travada primordialmente entre Israel e o
Egito, mas houve combates significativos ao longo da fronteira com a
Jordânia contra forças palestinas e jordanianas e, ocasionalmente, nas
Colinas do Golã e na fronteira israelo"-libanesa. Como denota o termo
``atrito'' o objetivo dos atores árabes não era obter uma vitória contra
Israel, mas desgastá"-la e gerar pressão politica e diplomática adicional
sobre o país e os Estados Unidos, seu principal apoiador.

Como embaixador israelense em Washington durante aqueles anos, a
principal missão de Rabin era manter o apoio de Washington ao
\textit{status quo} territorial, enquanto não estivesse disponível um
acordo diplomático aceitável para Israel; ele também ajudava a convencer
políticos e burocratas norte"-americanos relutantes a fornecer a Israel
sistemas sofisticados de armamentos. O período em que Rabin ocupou o cargo pode ser
dividida em três períodos distintos, definidos por sua eficácia e
estatura: uma fase introdutória, de sua chegada em março de 1968 até a
saída de Johnson da Casa Branca, em janeiro de 1969; de janeiro de 1969 a
setembro de 1970, quando sua proximidade com Kissinger e com a Casa
Branca de Nixon culminaram com a ação conjunta de Israel e dos Estados Unidos na
Jordânia; e de setembro de 1970 até seu retorno a Israel em março de
1973, quando esteve no auge de sua influência e eficácia.

Durante seus primeiros meses em Washington, Rabin teve que lidar com um
presidente que havia anunciado que não se candidataria à reeleição, e
cujo maior interesse era a maldição de seu governo: a guerra no Vietnã e
a crescente oposição a ela no país. Era limitado o seu interesse no
Oriente Médio e nas relações entre Israel e os países árabes. Em
setembro de 1968, com a ausência do foco presidencial, o apoio de
Washington ao \textit{status quo} o territorial em Israel começou a se
esvair. O secretário de Estado, Rusk, pediu a Rabin uma resposta
israelense às ofertas que os soviéticos haviam feito aos Estados Unidos para
promover uma solução diplomática. As propostas soviéticas refletiam o
apoio de Moscou ao Egito e eram claramente inaceitáveis para Israel.
Estava claro que o governo havia iniciado um diálogo com a União
Soviética que enfraqueceria sua posição original posterior a junho de
1967. No início de novembro de 1968, Rabin descobriu que, em um encontro
anterior em Nova York, Rusk havia dado a Mahmoud Riyad, ministro das
Relações Exteriores do Egito, um plano de sete pontos para solucionar
ambos os conflitos árabe"-israelense e egípcio"-israelense. De acordo com esse
plano, em troca de uma retirada israelense total, o Egito concordaria
com o encerramento formal do estado de guerra e assinaria com Israel um
documento conjunto (e mal"-definido). A questão dos refugiados palestinos
seria resolvida na base da livre escolha de cada refugiado. Rabin ficou
chocado e receoso ao receber as notícias sobre a proposta desse acordo.

No final, o documento norte"-americano foi rejeitado pelo Egito, mas os
Estados Unidos não desistiriam. Em 19 de setembro, Rusk convidou Rabin para uma
``conversa vigorosa''. Ele disse que ``Israel deveria deixar clara sua
posição, abandonar as banalidades e se concentrar nas especificidades''. Rabin
argumentou, exortando Rusk ``a rejeitar o plano russo. Afinal, faltavam
a ele os elementos que Israel e os Estados Unidos haviam definido como necessários
a qualquer acordo político no Oriente Médio. Ele não mencionava a paz e
não incluía nenhuma expressão concreta de reconhecimento de Israel ou de
aceitação de sua existência.'' Quando Rusk perguntou se seria suficiente
que ``Israel e os Estados árabes assinassem um documento multilateral
conjunto'', Rabin explicou que Israel ``queria um acordo de paz
bilateral, contratual, com cada um e com todos os países árabes
vizinhos''.\footnote{Rabin, \textit{The Rabin Memoirs}, \textit{op}. \textit{cit}., p. 139.} 
Rabin deixou com o secretário um
\textit{aide"-mémoire} por escrito, mas a distância entre o Departamento de
Estado e Israel era mais que evidente. A pressão continuou quando o vice
de Rusk, Nicholas Katzenbach, convidou Rabin para uma prolongada
discussão em 13 de novembro. A conversa revelou o crescente afastamento
entre as visões dos Estados Unidos e de Israel em relação às retificações de
fronteira exigidas por Israel em um acordo de paz e a melhor estratégia
diplomática para obtê"-las. O assessor de Segurança Nacional de Johnson,
Walt Rostow, resumiu a situação em um memorando enviado ao presidente em
15 de novembro de 1968: 

\begin{quote}
Rabin crê que mudamos nossa posição e
enfraquecemos a capacidade israelense de negociação. Na realidade, essa
tem sido nossa posição por mais de um ano, mas os israelenses deixaram
de nos escutar. Em relação a enfraquecer seu posicionamento, não podemos nos
permitir acompanhá"-los em sua prática de pechinchar, se quisermos dar
qualquer chance à paz.\footnote{Estados Unidos da América, \textit{Foreign Relations of the United States} 
{[}Relações Exteriores dos Estados Unidos{]}, 1964--1968, documento 322.}
\end{quote}

Grande parte da energia de Rabin durante esse período foi gasta tentando
impedir que a posição de Washington se aproximasse daquela da União
Soviética, das potências europeias e, obviamente, dos países árabes. Ele
foi muito melhor sucedido em suas conversas com Kissinger do que com o
Departamento de Estado.

Outro importante esforço de Rabin durante esse período foi feito para
garantir a venda de cinquenta jatos Phantom F4, que Johnson havia
prometido a Eshkol. Esse era um tema crucial. A venda permitiria uma melhoria
substancial da força aérea israelense e de sua capacidade militar como
um todo. Seria também a primeira vez que os Estados Unidos vendiam um
sistema de armas ofensivo a Israel, um marco importante na transformação
para que um país se tornasse a principal fonte de suprimento de equipamentos militares do outro.
A promessa de Johnson a Eshkol encontrou renhida resistência por
parte de poderosos elementos de seu governo, principalmente o secretário
de Estado Rusk e o ministro da Defesa Clark Clifford. Alguns acreditavam
que a venda ampliaria de forma perigosa a vantagem militar israelense
sobre os árabes. Outros viam nela uma oportunidade única para reavaliar
a opção nuclear israelense e buscavam criar um vínculo entre os dois
temas, forçando Israel a ampliar a transparência em troca dos jatos
Phantom. Em Israel, o embaixador norte"-americano Walworth Barbour
negociou com Eshkol e seus ministros, enquanto Eban negociava com seu
par, Rusk. Mas o principal esforço israelense foi investido em Rabin,
que iniciou sua campanha no Departamento de Estado e acabou em um ``cabo
de guerra'' com o vice"-ministro da Defesa, o formidável Paul Warnke.

O tema da capacidade nuclear israelense não era novidade para
Rabin. A comunidade de segurança nacional israelense estava dividida
sobre a questão nuclear nas décadas de 1950 e 1960 --- entre aqueles que
apoiavam a opção nuclear como a dissuasão definitiva, liderados por
Shimon Peres, e os que acreditavam na dissuasão convencional,
identificados com a facção do Partido Trabalhista do Achdut Haavodá e
seus especialistas em segurança, Alon e Galil. No início, Rabin apoiou a
dissuasão convencional, mas, já em 1963, havia mudado de opinião, chegando
à conclusão de que Israel não conseguiria, no longo prazo, bancar os custos
de uma corrida armamentista convencional.

A mudança de opinião de Rabin coincidiu com um período de enorme pressão
norte"-americana sobre Israel, para que declarasse seus planos e
capacidades nucleares, pressão que havia surgido no governo Kennedy e
continuou com Johnson. Comenta"-se que Eshkol dissipou as preocupações de
Johnson com a fórmula ``Israel não será o primeiro a introduzir armas
nucleares no Oriente Médio'', mas a pressão foi logo renovada. O próprio
Rabin foi exposto a ela como chefe do Estado"-Maior em 1965, em reuniões
com o imponente Averell Harriman, que na época detinha o posto de
embaixador itinerante do governo Johnson, e Robert Komer, funcionário do
Conselho de Segurança Nacional, ambos enviados a Israel por
Johnson. Harriman sutilmente, e Komer abertamente, expressaram sua
insatisfação com a fórmula de Eshkol. Komer disse a Rabin que ``se
Israel for nessa direção, poderá provocar uma das maiores crises em sua
relação com os Estados Unidos''.

Rabin e Warnke encontraram"-se diversas vezes em novembro de 1969 para
argumentar --- ou discutir --- sobre o draconiano memorando de
entendimento que o Pentágono tentava vincular à venda dos Phantoms a
Israel. As minutas desses encontros pareciam"-se com a descrição de um
elaborado minueto: Rabin rejeitando os esforços de Warnke para impor uma
proibição ou ao menos uma limitação às capacidades nucleares e
balísticas de Israel; Rabin defendendo"-se de Warnke, mas sem admitir que
Israel tivesse tal capacidade ou a intenção de adquiri"-la. Quando Warnke
pergunta: ``Qual o significado da palavra introduzir?'', Rabin lhe
pergunta qual é \textit{sua} definição de armas nucleares, ``já que
vocês estão mais familiarizados com elas do que nós'', e na sequência
perguntando a Wanke se ele ``considerava como arma um artefato nuclear
que nunca havia sido testado, e produzido por um país sem experiência
anterior?'' Rabin argumentou que todas as potências nucleares haviam
testado suas armas nucleares e perguntou: ``você realmente acredita que
a introdução {[}de armas nucleares{]} se dê antes do teste?''

Mas Rabin reconhecia que, sofismas aparte, o ``toma lá dá cá'' com os
secretários e assistentes do governo Johnson era inútil, e adotou outra
tática. Através de amigos democratas em comum, Rabin informou ao
presidente Johnson que o candidato republicano Nixon, se eleito, se
comprometeria a entregar os aviões a Israel. A tática pode não ter sido
elegante, mas foi muito efetiva. Em meados de janeiro de 1969, pouco
antes de deixar a Casa Branca, Johnson contradisse seus funcionários e
ordenou"-lhes que prosseguissem com a venda. Mas as negociações estavam
longe de terminar. Rabin teria de passar por uma segunda rodada de
diálogos estéreis em 1969, iniciada pelo vice"-secretário de Estado de
Nixon, e mais uma vez teria que levar a discussão para um nível
superior. O acordo só foi concluído em dezembro de 1969, em um encontro
organizado por Rabin entre Nixon e Golda Meir.

Com a chegada de Nixon à Casa Branca, em 20 de janeiro de 1969, iniciou"-se
um novo capítulo na política norte"-americana em relação ao conflito
árabe"-israelense e na estadia de Rabin em Washington. O secretário de
estado de Nixon era William Rogers, um ilustre advogado republicano. Não
levou muito tempo para que surgissem tensões entre Rogers e Henry
Kissinger, o brilhante assessor de Nixon para a Segurança Nacional. As
rivalidades burocráticas fazem parte do sistema político norte"-americano
e a tensão entre o secretário de Estado e o assessor de Segurança
Nacional é uma delas. Durante os primeiros anos do governo Nixon,
Kissinger foi mantido afastado das questões do Oriente Médio, dando a
Rogers e ao Departamento de Estado a primazia nessa área. Nixon tendia a
ver o Oriente Médio e o conflito árabe"-israelense fundamentalmente
através de um prisma global: como uma área de conflito intenso com a
União Soviética e, segundo Nixon, ``como um barril de pólvora''. Era
importante evitar uma colisão entre Estados Unidos e União Soviética no
Oriente Médio, mas era igualmente importante evitar que os soviéticos
ocupassem espaço na região. Era esse o ponto defendido por
Rogers e o Departamento de Estado --- que o apoio norte"-americano a
Israel minava a posição dos aliados árabes moderados dos Estados Unidos
e era, portanto, imperioso pressionar por uma rápida solução do conflito
árabe"-israelense, ainda que isso demandasse um distanciamento de Israel e
da própria política norte"-americana pós"-junho de 1967.

Kissinger era crítico dessa política. Sua rivalidade com Rogers
intensificou"-se devido a sua convicção de que a política de Rogers e do
Departamento de Estado não fazia sentido. Qual era a lógica em forçar
Israel a retirar"-se do Sinai, devolvendo"-o a Nasser, o cliente de
Moscou? Quando os egípcios e os outros árabes descobrissem que Moscou
não era capaz de recuperar os territórios perdidos e se aproximassem de
Washington para obtê"-los, a política norte"-americana poderia mudar.
Nixon continuou a ouvir os conselhos de Kissinger, mas, ao longo de 1969,
tendeu a apoiar Rogers e o Departamento de Estado. Ele lhes deu espaço
para negociar com a União Soviética, a Grã"-Bretanha e a França no marco
dos grupos das duas e das quatro potências e recusou"-se a assinar novos
acordos importantes de suprimentos de armas para Israel.

Rabin havia tido uma relação prévia com Nixon. Em 1966, Nixon visitou
Israel quando já era considerado uma força descartada, após ter perdido
tanto uma campanha presidencial quanto uma governamental, e foi
desdenhado pelo governo. Rabin, como chefe do Estado"-Maior, recebeu"-o com
honras e brindou"-lhe uma visita completa. Em 1968, às vésperas da
eleição presidencial, Rabin se encontrou com Nixon. Apesar de não
anunciá"-lo, ele expressou a seus confidentes sua preferência por
Nixon, frente a Hubert Humphrey. Rabin portanto tinha agora um
presidente amigável na Casa Branca, ainda que serem fossem raros os
encontros com ele. Seu contato regular na Casa Branca era com Kissinger
e, no Departamento de Estado, com Joseph Sisco, vice"-secretário de
Estado para o Oriente Médio.

Apesar de não lidar diretamente com a diplomacia do Oriente
Médio naquela época, Kissinger era um membro muito influente do governo e um guia de
primeira para a política em Washington, dos Estados Unidos e de temas internacionais.
Com o tempo, Rabin e Kissinger tornaram"-se amigos próximos,
desenvolvendo um mútuo apreço. Sisco também era um dos
participantes centrais desse grupo. Ele era o principal elemento da política
externa norte"-americana a cargo do Oriente Médio, lidando com Anatoli
Dobrynin, o influente embaixador soviético em Washington, e tratando também com os
europeus, os árabes e, obviamente, com os israelenses. Diferentemente da
maioria de seus colegas no departamento, ele não era um arabista, nem de
formação nem por inclinação. Sisco também era muito habilidoso na
manutenção de uma boa relação pessoal tanto com Kissinger quanto com
Rogers --- e o primeiro o descreveu com carinho, mas com um toque de ironia:

\begin{quote}
Intenso, sociável, ocasionalmente frenético, Joseph Sisco não era um
funcionário padrão da área de relações exteriores {[}\ldots{}{]} revelou"-se uma
prova viva daquilo que a liderança criativa poderia realizar no
Departamento de Estado {[}\ldots{}{]} muito engenhoso, com o talento para
estratagemas que são a alma da diplomacia do Oriente Médio e às vezes
oferecendo mais soluções do que os problemas existentes, Joseph Sisco
tomava a iniciativa burocrática e nunca a abandonava.\footnote{Henry Kissinger, 
\textit{White House Years}, \textit{op}. \textit{cit}., p. 348.}
\end{quote}

Com o passar do ano de 1969, o Departamento de Estado ampliou a pressão
para obter uma rápida solução, distante da política original de
Washington pós"-junho de 1967. Em outubro e dezembro, Nixon finalmente
aprovou o Plano Rogers, que representava tudo aquilo a que Israel se
opunha: uma retirada total israelense em troca de menos que um acordo de
paz contratual. Israel reagiu drasticamente, rejeitou o plano e lançou
uma maciça campanha contra ele junto à comunidade judaica e ao
Congresso. Nixon usou Kissinger e Rabin para enviar uma mensagem por
canais extraoficiais, informando que não apoiava totalmente o plano.
Kissinger também disse a Rabin, repetidas vezes, que o presidente era um
dos poucos amigos de Israel no Executivo e que seria calamitoso
antagonizá"-lo. Assim, a campanha orquestrada por Rabin mirou em Rogers e
não em Nixon. Rogers naturalmente se ofendeu e responsabilizou Rabin,
reclamando por ter sido transformado no bode expiatório e exposto como
anti"-israelense, quando simplesmente implementava a política
norte"-americana.\footnote{Rabin, \textit{The Rabin Memoirs}, 
\textit{op}. \textit{cit}., p. 164.} Rabin na verdade gostava de Rogers e o tinha
pessoalmente em alta estima, mas divergia profundamente de sua política
e entendia que seria um erro fatal criticar pessoalmente a Nixon.

Enquanto tentava repudiar as tendências políticas representadas pelo
Plano Rogers, Rabin realizou um esforço mais discreto direcionado a seu
próprio governo. Ele argumentou repetidamente em seus telegramas ao
primeiro"-ministro que a incapacidade israelense de encontrar uma solução
efetiva para a ampla guerra de atrito iniciada pelo Egito em março de
1969 estava minando a posição israelense nos Estados Unidos. Conforme
escreveu em suas memórias: 

\begin{quote}
Israel desapontou os Estados Unidos e não
tinha uma resposta apropriada para a Guerra de Atrito {[}\ldots{}{]} Os americanos
nunca o admitiram, e duvido que o admitiriam agora. Mas assumiram que
Israel era suficientemente poderosa para infligir ao Egito um golpe que
poria fim a sua vontade de continuar com a Guerra de Atrito. Com a
continuação da guerra, continuava a erosão da posição norte"-americana no
Oriente Médio, e com ela aumentava a disposição dos \textsc{eua} para chegar a um
acordo com a União Soviética.\footnote{Rabin, \textit{Pinkas Sherut}, 
\textit{op}. \textit{cit}., p. 248.}
\end{quote}

Rabin argumentou repetidas vezes que essa não era nem sua impressão nem
sua opinião pessoal, mas que lhe havia sido apontada por membros do alto
escalão do governo. Ele tendia a se referir a suas conversas com Sisco,
tentando disfarçar o fato de que Kissinger era seu principal
interlocutor (no sistema israelense, telegramas sensíveis tendiam a
vazar ou ser vazados). Kissinger era crítico de Rogers e da linha do
Departamento de Estado, inclinava"-se a ver o Oriente Médio em termos
geopolíticos, ou seja, basicamente como uma arena da rivalidade
soviético"-norte"-americana, e provavelmente não se opunha a ver
Rogers e sua política derrotados.

Rabin acreditava na necessidade de uma intensificação dos ataques contra
o Egito, ``bombardeios de profundidade'' contra alvos estratégicos, que
pusessem Nasser de joelhos. Na visão de Rabin, a capitulação ou colapso
de Nasser seria visto como um feito norte"-americano e um golpe para a
União Soviética, além de reduzirir a pressão tanto sobre Washington quanto
sobre Jerusalém para aceitar as imposições soviéticas e egípcias. Essa
abordagem tinha seus críticos no governo israelense, principalmente Abba
Eban. Ele e seus seguidores se opunham à escalada e duvidavam que Rabin
tivesse razão ao interpretar e reportar a posição norte"-americana.
Tinham dificuldade em acreditar que os Estados Unidos estavam
encorajando Israel a intensificar seus ataques em lugar de diminuí"-los.
O amargo debate entre o ministro das Relações Exteriores e seu
embaixador em Washington foi mais um elemento de uma relação que ia de
mal a pior. Conforme vimos, Rabin e Eban tiveram um profundo
desentendimento em maio de 1967, quando Eban argumentou no gabinete
israelense que os Estados Unidos não queriam que Israel iniciasse uma
guerra por causa do bloqueio do Estreito de Tiran. Mais tarde, quando
Rabin pediu a Eshkol que o enviasse a Washington, Eban se opôs. No
início do mandato de Rabin, Eban visitou o presidente Johnson na Casa
Branca. Ele esnobou Rabin ao excluí"-lo do encontro e ao pedir que se
retirasse de seu escritório na embaixada ao fazer uma ligação através da
linha segura que a conectava a Jerusalém. Rabin retaliou, abortando uma
tentativa de Eban de retornar a Washington durante uma visita a Nova
York. Era fácil para um embaixador da estatura de Rabin persuadir o
anfitrião norte"-americano de que o primeiro"-ministro não estava
interessado em ter o ministro das Relações Exteriores presente em
Washington naquele momento. Em fevereiro de 1969, Eshkol faleceu e foi
substituído por Golda Meir, que não tinha simpatia por Eban. A nova
primeira"-ministra queria que Rabin se reportasse diretamente a ela,
passando por cima do ministro das Relações Exteriores e agravando assim a
sua já frágil relação. Ela disse a Rabin que era de sua responsabilidade
atualizar o ministro das Relações Exteriores quando fosse necessário, o
que provavelmente não partiu seu coração. Passar por cima desse ministro e
de seu Ministério apresentava uma vantagem adicional,
além de não ter que lidar com Eban: era (e ainda é) um fato conhecido, e
por essa razão também provocativo, que os telegramas tendiam a ser
vazados do Ministério das Relações Exteriores. Vazamentos eram e são
usados por diplomatas e políticos por várias razões, com frequência para
cultivar relações com jornalistas, promover ou bloquear determinadas
políticas, algumas vezes para causar embaraço ao emissor. Ao longo dos
anos, diferentes embaixadores conceberam uma série de medidas para
resolver o problema. Rabin enviava telegramas regulares e relatórios ao
ministro e ao Ministério das Relações Exteriores, mas seus telegramas
mais importantes e sensíveis eram enviados à primeira"-ministra através
do canal de comunicação do Mossad. Assim, durante a visita de Golda Meir
a Nixon em setembro de 1969, tendo sido preparados por Rabin e
Kissinger, os dois líderes decidiram evitar o Departamento de Estado e o
Ministério das Relações Exteriores e comunicar"-se diretamente através do
assessor de Segurança Nacional e do embaixador.

As desavenças entre Rabin e Eban e a liderança do Ministério das Relações
Exteriores a respeito da recomendação para os bombardeios em
profundidade\footnote{A serem realizados contra o Egito. [\textsc{n.t.}]} eram particularmente agudas. Eban e seus associados não concordavam com a substância das recomendações de
Rabin e com o que entendiam serem generalizações sobre a política
norte"-americana baseadas em suas conversas com Sisco. Quando Eban o
criticou, Rabin respondeu diretamente, sugerindo que Eban, que havia
fracassado na interpretação da posição dos Estados Unidos em maio de 1967,
continuava a não entender a política norte"-americana.

Rabin em nenhum momento foi dissuadido pelas críticas. Continuou a
pressionar por táticas agressivas e chegou ao ponto de sugerir que
Israel ameaçassee ocupar o Cairo. Desnecessário dizer que Eban ficou
horrorizado. Ciente de que Dayan havia feito a Sisco uma pergunta
explícita durante a visita deste a Israel, a visão de Rabin foi
claramente articulada em um telegrama enviado a Eban em 17 de abril de
1970, com cópias para Dayan, Aharon Yariv (diretor da inteligência
militar e amigo de Rabin), e a Simcha Dinitz (diretor"-geral do
escritório do primeiro"-ministro):

\begin{quote}
A conduta da abordagem norte"-americana sobre as operações militares
israelenses se baseia em evitar que sejam levados a uma situação que
possa ser definida como conluio {[}\ldots{}{]} os Estados Unidos se recusarão
terminantemente a dizer a princípio e de forma clara e formal ``procedam
com os bombardeios em profundidade'', da mesma forma como, antes da
Guerra dos Seis Dias, recusaram"-se a nos dizer ``podem iniciar a
guerra''. Além disso, qualquer tentativa de nossa parte de formular tais
perguntas a um representante dos Estados Unidos, com o objetivo de obter
uma clara resposta, demonstra uma avaliação equivocada dessa questão
fundamental {[}\ldots{}{]} os Estados Unidos entendem que Israel tem o direito de adotar medidas
militares independentes, que veja como necessárias para a sua segurança.
Quando os Estados Unidos discordam dessas medidas, encontram os meios
para deixá"-lo claro a Israel {[}\ldots{}{]} Questionar diretamente a Sisco em
relação aos bombardeios em profundidade significa repetir, em pequena
escala, a nossa tentativa às vésperas da Guerra dos Seis Dias de obter
dos norte"-americanos uma aprovação formal para iniciar a guerra {[}\ldots{}{]} é
minha modesta opinião que apresentar essa questão levará os
norte"-americanos a repensarem sobre a capacidade de discutir esse tema
conosco no futuro. Tentaremos solucionar o problema através de nossos
contatos aqui.
\end{quote}

Kissinger, que conhecia bem a Rabin e Eban, e trabalhava com os dois,
via essa rivalidade como algo mais do que um confronto em relação à
política e a um lugar na hierarquia. Para ele, era um choque entre ``duas
pessoas diferentes: O ``don'' de Cambridge, urbano, complexo, educado, e
o soldado Sabra (israelense nativo) espinhoso, duro,

Rabin não tinha muitos amigos nem admiradores no Ministério das Relações
Exteriores, mas aqueles que o eram, como Moshe Bitan, vice"-diretor"-geral
a América do Norte, achavam que Rabin se comportava mais como
ministro que como diplomata. Ele mencionou em seus diários a
correspondência com o vice de Rabin na embaixada, Shlomo Argov, que
escreveu: ``suas críticas a alguns dos telegramas de Yitzhak são
pertinentes {[}\ldots{}{]} eu assumo que você não discorde de várias das coisas que
ele diz. O problema é que ele as expressa com um estilo e uma forma que
deve incomodar até a seus amigos. Tento aqui e acolá moderar suas
declarações, mas, como você pode ver, sem muito sucesso''.\footnote{Moshe Bitan, \textit{Diário Político, 1967--1970}. Tel Aviv: Olam Hadash, 2014 {[}em
hebraico{]}.}

Com o encorajamento de Rabin e principalmente por causa do crescente
custo da Guerra de Atrito, Israel adotaria a política de bombardeios em
profundidade tão veementemente advogada por Rabin, utilizando sua
superioridade aérea para atingir objetivos estratégicos no longínquo
interior do Egito. A nova política provocou uma rápida escalada. Nasser,
desamparado, viajou secretamente a Moscou em janeiro de 1970 e disse a
seus patronos soviéticos que, a não ser que viessem em seu auxílio, ele
se demitiria. Os soviéticos responderam, assumindo diretamente a
responsabilidade pela defesa do espaço aéreo egípcio, enviando
primeiramente várias baterias de misseis antiaéreos (\textsc{sam}) operados por
equipes soviéticas no Egito e, mais tarde, enviando pilotos e aviões
militares soviéticos para missões de combate ao longo do \textit{front}
israelo"-egípcio.

Iniciou"-se uma nova e perigosa fase da Guerra Fria no Oriente Médio: não
mais uma guerra por procuração, mas sim com o envolvimento direto da
União Soviética. A advertência de Nixon de que o conflito
árabe"-israelense era um barril de pólvora havia se confirmado. Os
pilotos israelenses saíram"-se bem no combate aéreo que travaram contra
os pilotos russos, mas os mísseis \textsc{sam}-6 demonstraram"-se mortais. Todas
as defesas eletrônicas dos jatos de combate israelenses eram
insuficientes contra eles. Os Estados Unidos pareciam não possuir (ou
não queriam compartilhar) equipamentos mais avançados. Israel perdeu
vários aviões, e sua liderança política e militar ficou abalada com o
confronto direto com a União Soviética. Rabin e seus colegas acreditavam
que podiam lidar com os adversários árabes, mas, uma vez que a União
Soviética havia decidido intervir, somente poderia e deveria ser contida
pelos Estados Unidos.

Foi nesse contexto que o secretário de Estado norte"-americano publicou a
Iniciativa Rogers. Era diferente do Plano Rogers, tendo seu foco em um
cessar"-fogo imediato entre o Egito e Israel, mas também semelhante, porque o
cessar"-fogo estava ligado à busca de uma solução diplomática abrangente,
conforme contemplava o Plano. A Iniciativa Rogers incluía,
intencionalmente, uma solução para todos: iria supostamente desarmar uma
perigosa crise internacional; evitaria que Israel se envolvesse em um
conflito militar direto com a União Soviética; e ofereceria ao Egito a
expectativa de um processo diplomático previsto no Plano Rogers
original.

Foi justamente esse último ponto que opôs Golda Meir à Iniciativa
Rogers. Ela imediatamente enviou a Rabin uma mensagem para o presidente,
rejeitando"-a. Para Rabin, Meir cometia um grave erro e ele relutou em
transmitir a mensagem ao presidente. Pediu a Meir que suspendesse a
mensagem e lhe permitisse voar a Israel para persuadi"-la, bem como ao gabinete,
a adotar uma abordagem distinta. Não era uma atitude usual para um
embaixador --- nem era comum que uma primeira"-ministra resoluta como Meir
o aceitasse --- mas Rabin não era um embaixador convencional e o momento
era crucial o suficiente para justificar uma reflexão mais profunda.
Essas foram semanas tensas nas relações entre Washington e Jerusalém.
Meir estava disposta a aceitar o princípio da Iniciativa Rogers, ou
seja, o cessar"-fogo imediato, mesmo que isso significasse a renúncia dos
membros da direita do governo de união nacional; ela não podia,
entretanto, aceitar a formulação que a vinculava ao Plano Rogers. Houve
telegramas e ligações telefônicas entre as duas capitais, frequentes e
tempestuosos. Meir ligou exasperada para Sisco, e Rabin foi chamado e
censurado por Kissinger. Conforme relatado por Rabin: ``milhares de
palavras, muitas delas enfurecidas. Golda me repreendeu furiosamente,
mas o problema não foi solucionado.''\footnote{Rabin, \textit{The Rabin Memoirs}, \textit{op}. \textit{cit}., p. 182.}

Apesar da ausência de uma nítida aceitação da iniciativa por parte de
Israel, um cessar"-fogo por noventa dias entrou em vigor em 7 de agosto
de 1970, mas o final dos combates logo seria eclipsado. Israel descobriu,
através do reconhecimento aéreo, que os russos e os egípcios haviam
violado os termos do cessar"-fogo imediato (ou seja, baseado na
manutenção do \textit{status quo}) e haviam deslocado várias baterias de
mísseis \textsc{sam}-6 em direção ao Canal de Suez. Os Estados Unidos reagiram
com ceticismo e levou algum tempo até que Israel pudesse convencê"-los de
que houvera uma violação. Houve desavenças genuínas entre os analistas
de imagens israelenses e norte"-americanos, mas era evidente a relutância
dos norte"-americanos em aceitar que os soviéticos haviam trapaceado,
trazendo sérias implicações às relações entre as superpotências.

Entretanto, uma vez que os Estados Unidos endossaram a posição
israelense, Washington e Jerusalém decidiram encampar a Iniciativa
Rogers. Os Estados Unidos engajaram"-se para limitar o dano potencial a
Israel, fornecendo ao país novos equipamentos sofisticados; mas o dano
causado pela transferência de baterias de mísseis antiaéreos às
proximidades do Canal de Suez não podia ser revertido, e a força aérea
israelense pagaria um alto preço durante os primeiros dias da Guerra de
Outubro em 1973. O cabo"-de"-guerra sobre o cessar"-fogo ao longo do Canal
de Suez encerrou"-se finalmente em 8 de agosto, quando o cessar"-fogo
entre Israel e o Egito entrou em vigor. Mas a calmaria de agosto logo
seria ofuscada por dois importantes eventos em setembro.

O primeiro deles ficou mais tarde conhecido como Setembro Negro: a
campanha militar lançada pelo rei Hussein contra a \textsc{olp} em 7 de setembro,
com o objetivo de reafirmar sua soberania e recuperar o controle sobre o
território jordaniano. Hussein foi motivado por flagrantes provocações:
um atentado contra ele, perpetrado em primeiro de setembro, e o
sequestro de três jatos ocidentais pela Frente Popular para a Libertação
da Palestina. Os aviões pousaram, foram estacionados e em seguida
explodidos em Zarqa, na área central da Jordânia. Com determinação e
brutalidade o exército de Hussein derrotou a \textsc{olp}. A ação levou a Síria,
aliada e protetora da \textsc{olp}, a intervir a seu favor, enviando tanques ao
território jordaniano em 8 de setembro.

A ação de um cliente soviético, ao invadir o território de um protegido
norte"-americano, era um movimento no tabuleiro de xadrez internacional
e confrontava o governo Nixon com um agudo dilema: no auge da Guerra do
Vietnã, a última coisa necessária era um novo envolvimento militar no
Oriente Médio, mas ver a Jordânia ser invadida e derrotada pela Síria
era igualmente inaceitável. A opção escolhida por Nixon e Kissinger foi
solicitar a Israel que salvasse o governo hachemita. Por coincidência,
Meir estava em Nova York, prestes a voltar para Israel. Rabin estava
com ela e foi contatado telefonicamente por um Kissinger ansioso. O
presidente e seu assessor de Segurança Nacional queriam saber se Israel
estaria disposta a utilizar seu exército para salvar o rei.

Estava longe de ser uma decisão fácil. Israel necessitava
esclarecimentos e garantias dos Estados Unidos. A mais importante delas era uma
redoma de proteção contra uma intervenção militar soviética. As minutas
da conversa telefônica de Rabin com Kissinger em 21 de setembro ainda
transmitem o clima do drama daqueles dias:

\begin{quote}
\textsc{kissinger}: Direi a Sisco que fale com você em alguns minutos {[}\ldots{}{]} ele vai
dar uma resposta que em princípio é sim, mas gostaria de fazer a
seguinte sugestão: quanto menos você disser e responder, melhor. Somente
se comunique com seu governo e então venha me ver. É extremamente
importante que saibamos quem diz o que a quem, e te darei instruções a
respeito.

\textsc{rabin}: Enquanto isso, também tenho instruções. Mais
detalhadas.\footnote{Estados Unidos da América, \textit{Foreign Relations of the United States}, vol. 24, documento 301.}
\end{quote}

A Casa Branca levou Rabin de avião de Nova York para Washington, para
que este estivesse perto de Kissinger. Após muita análise, a resposta
israelense foi positiva. No final, nenhuma ação militar foi necessária,
pois as movimentações israelenses em terra e no ar foram suficientes. O
comandante da força aérea síria, Hafez al"-Assad, recusou"-se a engajar
seus jatos e, sem o apoio aéreo, os tanques sírios transformaram"-se em
presas fáceis. Os aviões e tanques da própria Jordânia fizeram
retroceder as colunas blindadas sírias.

Este seria um dos pontos altos da carreira de Rabin como embaixador, uma
demonstração de sua habilidade como negociador político e diplomático.
Não somente o governo Nixon não precisou intervir como o presidente pôde
argumentar que a Doutrina Nixon, que pregava a dependência em aliados
locais em lugar de tropas norte"-americanas, estava funcionando. Em 25 de
setembro, Kissinger transmitiu a Rabin uma mensagem de agradecimento de
Nixon, endereçada a Meir:

\begin{quote}
O presidente nunca esquecerá o papel de Israel ao impedir a
deterioração na Jordânia, impedindo a tentativa de derrubar o governo {[}\ldots{}{]}
Os Estados Unidos são afortunados ao contar com Israel como aliado no
Oriente Médio; o que ocorreu será levado em conta em qualquer evento
futuro.\footnote{Rabin, \textit{The Rabin Memoirs}, \textit{op.\,cit}., p.\,189.}
\end{quote}

O risco assumido por Israel pagaria amplos dividendos ao longo dos três
anos seguintes. Definiria o contexto de um período de proximidade e
intimidade inéditos na relação entre Israel e os Estados Unidos. Mas, em
retrospecto, representou uma faca de dois gumes, pois reforçou a
política de Meir de tentar manter o \textit{status quo} e contribuiu para
o impasse que levaria à Guerra de Outubro.

O outro evento marcante que ocorreu em setembro de 1970 foi a morte de
Nasser, vitimado por um ataque cardíaco no dia 28. Pode muito bem ter
sido acelerado pela tensão e irritação causados pelos eventos do
Setembro Negro. A morte de Nasser marcou o fim de uma era na política
árabe. Também levou Anwar Sadat ao poder e, com ele, teve início a Guerra de Outubro e
o primeiro acordo de paz de Israel com um país árabe.

A intimidade e a proximidade com a Casa Branca não significavam que
Israel pudesse repousar sobre suas glórias e perpetuar o \textit{status
quo}. A Iniciativa Rogers estava ligada à retomada da busca por um
acordo e, mais especificamente, à revitalização da missão Jarring. Em
dezembro de 1970, tornou"-se mais palpável a pressão do Departamento de
Estado e da Casa Branca para avançar em direção a um acordo. Impelidos
por essa pressão, Meir e Dayan sugeriram vigorosamente a Kissinger a
noção de um acordo provisório, baseado em uma retirada israelense da
margem oriental do Canal de Suez.

A ideia israelense do acordo provisório recebeu um ímpeto adicional
quando o Egito enviou aos Estados Unidos uma mensagem, em janeiro de
1971, expressando interesse na ideia e informando que preferia fazê"-lo
através dos Estados Unidos e não dos esforços de Jarring. Sadat não foi
seriamente considerado como sucessor de Nasser quando foi escolhido em
setembro de 1970. Ele era o vice de Nasser, mas era visto pelos
poderosos no Egito como uma figura transitória, inofensiva, que seria
substituída assim que as diferenças entre eles fossem acertadas. Sadat,
entretanto, demonstrou ser ambicioso e ardiloso e, em pouco tempo,
consolidou seu poder e domínio sobre a máquina do Estado egípcio. Ele
tinha um programa completo para a transformação das políticas interna e
externa do Egito e estava determinado a migrar da órbita soviética para
a norte"-americana e desengajar"-se do conflito com Israel. Isso se
demonstrou em fevereiro de 1971, através da resposta egípcia ao
questionamento posto por Jarring a Israel e ao Egito: uma declaração
dramática ``para firmar um acordo de paz com Israel'' em troca de uma
retirada completa. A agenda diplomática israelense, assim como a de
Rabin em Washington, era agora determinada pela necessidade de responder
à nova ousadia diplomática de Sadat, pelo desejo de Washington de
explorar a nova oportunidade oferecida pela mudança no Egito e pela
compreensível ânsia de Jarring e do Departamento de Estado em iniciar
negociações para um acordo abrangente entre Israel e Egito.

Os Estados Unidos desapontaram"-se com a resposta israelense evasiva a
Jarring e não se entusiasmou com o conceito de Meir de um acordo
provisório. As negociações sobre um acordo provisório envolviam três
temas:

\begin{enumerate}
\def\labelenumi{\arabic{enumi}.}
\item
  Se era interessante para os Estados Unidos reabrir o Canal de Suez. Assumia"-se
  que o Pentágono preferia que ele continuasse fechado, dificultando
  assim os embarques soviéticos para o Vietcong. A posição dos \textsc{eua} em
  1971 era de que seria melhor manter o canal fechado mas, no interesse
  da paz no Oriente Médio, estava disposto a auxiliar na sua reabertura.
\item
  A insistência do Egito de que o acordo provisório fosse explicitamente
  vinculado ao progresso em direção a um acordo final. Sadat estava
  claramente preocupado que, ao concordar com um acordo provisório,
  colaboraria com o adiamento de um acordo abrangente para uma data
  indefinida no futuro.
\item
  Os termos do acordo: a profundidade da retirada israelense do Canal e
  o tamanho e a natureza da presença egípcia na sua margem oriental.
\end{enumerate}

Como condição para concordar com a ideia de um acordo provisório, Meir
queria que Washington abandonasse formalmente o Plano Rogers, mas seu
próprio conceito do acordo provisório que oferecia era bastante
limitado. Nos meses seguintes, Rabin tornou"-se um defensor consistente e
persistente da flexibilidade israelense. Tentando manter a credibilidade
de Israel e a boa vontade do governo Nixon, foi taxado de moderado
junto ao governo israelense. Em suas conversas com interlocutores
norte"-americanos, Rabin se referia a sua imagem de moderado e lhes pedia
que não o envergonhassem, complicando sua relação com a primeira"-ministra
ao mencionar as ideias flexíveis que ocasionalmente sugeria
por conta própria. Mas não foi bem"-sucedido e sua relação com Meir
tornou"-se tensa. Ao reportar uma de suas conversas com Kissinger, Rabin
tentava ``evitar a ira'' da primeira"-ministra. Ele admitiu em seu
telegrama que ``posso ter ido além de meus poderes ao sugerir propostas
de um acordo parcial mas, tendo em vista meu relacionamento com
Kissinger e minha confiança nele --- e considerando as atuais
circunstâncias políticas --- acreditava que Israel deveria fazer
importantes contribuições para avançar o processo político. Temos um
interesse vital em induzir o presidente a abandonar o Plano Rogers e não
faz sentido fazê"-lo, a não ser que injetemos algum ímpeto no processo
político, no contexto de um acordo parcial.''

A resposta da primeira"-ministra esteve de acordo com os temores de
Rabin. Após agradecer brevemente por seu relatório, ela lhe telefonou após
uma semana, ``notificando"-me'', relatou Rabin, ``que as propostas que
discuti com Kissinger eram inaceitáveis para ela. Ela lamentou que as
houvesse feito, mesmo de forma privada, sem antes solicitar permissão.
Ela também expressou sua preocupação de que Kissinger possa ter
transmitido ao presidente os principais pontos de minha proposta,
enfraquecendo assim a posição de Israel em nossa discussão com o
Departamento de Estado sobre os termos de nosso acordo parcial.
Finalmente, a primeira"-ministra instruiu"-me a notificar Kissinger sobre
sua dura reação, pedir"-lhe que ignore nossa conversa particular e que
considere nulas as minhas propostas''.\footnote{\textit{Ibid}., p. 203.}

As negociações entre Israel e os Estados Unidos sobre a ideia de um
acordo provisório continuaram até 1973, mas então já haviam se tornado
estéreis. Nixon e Kissinger queriam que houvesse avanços entre Israel e
o Egito, mas não havia um senso de urgência; nem queriam um confronto
com Meir às vésperas da eleição presidencial de 1972. Enquanto isso,
Sadat movia"-se na direção certa: em julho de 1972, expulsou do Egito os
assessores soviéticos e abriu um discreto canal de comunicação com
Kissinger através de seu assessor de Segurança Nacional, Hafez Ismail. A
insistência de Kissinger em 1969 de que o Egito eventualmente
``acordaria'' e se alinharia com Washington, afastando"-se de Moscou para
recuperar o Sinai, finalmente dava frutos. Apesar de Sadat não romper
totalmente com a União Soviética, o processo avançava e Kissinger estava
disposto a esperar.

Meir, por seu lado, aguardava. Ela não tinha interesse em um acordo
parcial ou geral, mas sabia que, para manter Kissinger e Nixon a seu
favor, tinha que propor ideias positivas. Em dezembro de 1971, ela teve
mais um encontro com o presidente, com novas ideias sobre um acordo
provisório. Ela concordaria com o reconhecimento, em princípio, do
direito de Israel de utilizar o Canal de Suez quando este fosse
reaberto, deixando a implementação para uma fase posterior. Ela propôs
que o exército israelense retrocedesse até o lado ocidental dos passos
de Mitla e Giddi, e que policiais e técnicos egípcios fossem estacionados
na área evacuada por Israel após a reabertura do canal. A proposta não
era aceitável nem para o Egito nem para os Estados Unidos, e nem mesmo Meir a
defendia com afinco. Rabin, a princípio, havia se demonstrado cético sobre
o acordo, mas gradualmente passou a apoiá"-lo. Meir informou aos seus
interlocutores norte"-americanos que aceitava a ideia, sujeita a certas
condições, mas é pouco provável que a tenha levado a sério. Dayan,
um dos primeiros apoiadores do acordo, abandonou"-o para não ter que
enfrentar a primeira"-ministra.

Não houve grandes mudanças entre dezembro de 1971 e o final da missão de
Rabin em Washington, em março de 1973. Nixon e Kissinger tinham plena
consciência da relutância da primeira"-ministra em avançar rumo a um
acordo total ou parcial, apesar de suas declarações. Seu argumento era
de que não havia muito a ser feito antes das eleições parlamentares
israelenses, planejadas para outubro de 1973, e Nixon e Kissinger estavam
dispostos a aguardar antes de pressionarem mais vigorosamente por algum
movimento. O último encontro de Meir com Nixon e Kissinger durante o
período em que Rabin atuou como embaixador ocorreu entre 28 de fevereiro e
1\textsuperscript{o} de março de 1973. As minutas das conversas de
Rabin com Kissinger às vésperas da visita refletem a intimidade incomum
e a franqueza que vieram a caracterizar suas relações. Kissinger
preparou Rabin detalhadamente para os diálogos da primeira"-ministra com
o presidente, oferecendo a ele conselhos táticos sobre como maximizar os
efeitos do encontro. Rabin, por sua vez, quando perguntado por Kissinger
sobre a racionalidade de uma ação militar israelense no Líbano,
respondeu que Dayan pode ter agido para obstruir o sucesso da visita da
primeira"-ministra. Nas palavras de Rabin, ``não sei porque a primeira"-ministra
aprovou o ataque ao Líbano. Quando fui chefe do Estado"-Maior,
sempre que tentávamos obter algo dos Estados Unidos na arena política,
mantinha tudo calmo [{]\ldots{}[}]''. E quando Kissinger lhe perguntou ``Qual você
acha que é a razão para essas ações?'', Rabin respondeu:
``Sinceramente, acho que o único motivo é o desejo de Dayan de evitar
uma visita bem"-sucedida da primeira"-ministra, porque suas chances
melhorariam se ela se aposentasse antes das eleições. É um ano
político''. Kissinger retribuiu a franqueza incomum de Rabin com uma
descrição detalhada da conversa que teria com Ismail.\footnote{Documento \textsc{nara}, tornado público em 16 de abril de 2009.}

Meir não era a única em compasso de espera em 1972. Rabin sentia que sua
missão em Washington havia se encerrado, que havia chegado ao ápice de
sua carreira como embaixador e que, para avançar em sua carreira política,
teria d estar em Israel e não no exterior. Sua relação com Meir
deteriorou"-se. Ela já há algum tempo estava descontente com a crescente
independência de Rabin e sua tendência de dar sermões em seus telegramas
ao governo (inclusive a ela) sobre como entender Washington da maneira
adequada. Rabin estava totalmente ciente das mudanças ocorridas na
posição de Washington, em função das transformações no Egito
representadas pelo governo de Sadat, e da determinação em obter algum
avanço; por isso, pressionava por um acordo provisório. Meir estava
descontente com a linha adotada por Rabin; ela acreditava que Rabin
havia se ``tornado um nativo'', como tende a acontecer com os
embaixadores, e que ele era demasiadamente influenciado por Sisco.
Durante uma de suas visitas, ela até perguntou para Amos Eran, o auxiliar
e confidente de Rabin na embaixada, ``diga"-me, Sisco fez uma lavagem
cerebral em Rabin? O que ocorre, ele aceita todas as suas posições? É
esse o tipo de influencia que ele exerce?''\footnote{Testemunho de Eran, Arquivo do Centro Rabin, 186.}

Rabin estava ciente da mudança no relacionamento com a primeira"-ministra.
Mas também estava angustiado com duas outras questões: uma era
a campanha incessante movida contra ele por Eban, seus subordinados no
Ministério das Relações Exteriores e o grupo de jornalistas que recebiam
deixas de Eban e seus apoiadores. E, mais importante, Rabin sentia que sua
luta por uma posição no governo não avançava. Até então ele já havia
sido menosprezado três vezes por Meir e pelo Partido Trabalhista
durante sua estada em Washington. O primeiro episódio ocorreu em
setembro de 1969: mais cedo naquele ano, a liderança do partido havia
sugerido que ele se juntasse à disputa política para concorrer ao
Knesset, o parlamento israelense, e tornar"-se um membro do governo. Isso
ocorreu no início de sua estada em Washington e ele recusou. Mas, em
setembro, Meir lhe ofereceu um atalho: durante uma das visitas de Rabin
a Israel naquele ano, para prepará"-la para um encontro com Nixon, Meir
lhe disse que, após as próximas eleições parlamentares (previstas para
outubro daquele ano e assumindo que ela formaria o novo governo), ela
queria que ele se juntasse ao governo como Ministro da Educação. Rabin
aceitou com satisfação e, galantemente, disse à primeira"-ministra que
ela não deveria sentir"-se presa a sua promessa, e que ele entenderia
perfeitamente se, após as eleições, ela não pudesse se cumprir. E
assim foi. Rabin foi duramente usado, não pela primeira nem pela
última vez, como um contrapeso a Dayan. O Partido Trabalhista havia
temido que Dayan se separasse e concorresse nas eleições por conta
própria.

Rabin expressou sua indignação ao relatar o episódio em uma carta
endereçada a seu pai e sua irmã:

\begin{quote}
Não sei exatamente o que aconteceu após as eleições. Quando estive em
Israel há duas semanas, ela {[}Meir{]} me chamou e me explicou que, devido à
importância de Washington e a ausência de alguém apropriado (para
me substituir), ela havia decidido que, naquele momento, eu deveria
continuar nos Estados Unidos. Respondi dizendo que, mesmo tendo me
perguntado em setembro, nada exigia dela. Muito pelo contrário,
deixei"-lhe uma saída. Acrescentei que em setembro tive dúvidas se ela
realmente queria que eu viesse. Além disso, todo o circo da mídia para comigo e
com o governo era indecoroso. Ainda assim, ela não me devia nada e não
me deve nenhuma explicação. Independentemente das qualidades que fazem
dela a primeira"-ministra mais apropriada para o momento, me resulta
difícil dizer que seu comportamento nesse episódio foi elogiável {[}\ldots{}{]} De
qualquer forma, não lamento ter ficado aqui.\footnote{Carta de Rabin a seu pai e sua irmã. Sem data, provavelmente início da década de 1970. Coleção privada.}
\end{quote}

O episódio ocorreu no início de 1970. Em 1971, a questão da entrada de
Rabin no governo foi discutida várias vezes. Durante uma visita a
Washington, o Ministro da Fazenda, Pinchas Sapir, um aliado próximo de
Meir, consultou Rabin a respeito de um convite a Haim Bar"-Lev, que o
havia sucedido como chefe do Estado"-Maior, para integrar o governo.
Rabin ficou constrangido, mas elogiou Bar"-Lev. A liderança trabalhista
via um maior valor político no recrutamento do ``calouro'' Bar"-Lev do que
em Rabin. Entretanto, em julho de 1971, durante a visita de Rabin a
Israel, Sapir tentou acertar as coisas e disse a Rabin que queria que
ele se juntasse ao governo antes de Bar"-Lev, que deveria encerrar sua
carreira militar em 1972. Mas havia oponentes, principalmente Eban.
Havia também o problema de encontrar um substituto para o posto de
embaixador em Washington. A posição foi oferecida ao general Yariv,
diretor da inteligência militar, mas ele a recusou. Meir disse a Rabin
que ele teria de ficar mais um ano em Washington e se juntaria ao
governo ao retornar em 1972. E então, em marco de 1972, Bar"-Lev foi
incorporado ao governo.

Desalentado e desapontado, Rabin escreveu uma dura carta a Meir. A carta
não está datada e pode ou não ter sido enviada, mas o texto demonstra o
quanto Rabin estava desanimado e frustrado:

\begin{quote}
Prezada Golda,

Hesitei a respeito de lhe escrever essa carta. Finalmente, decidi
escrevê"-la esperando que seja recebida com amizade e compreensão. Quando
estive em Israel e você decidiu, contra minha vontade, que eu ficasse
aqui por mais um ano, aceitei com pesar aquela decisão. Todas as minhas
preocupações se materializaram; uma situação pessoal e geral
desenvolveu"-se e tornou"-se difícil e quase intolerável. Tudo o que você
me disse, duas vezes, e não foi implementado (ou seja, a minha entrada
no governo) colocou"-me em uma posição ridícula. Parece que se tornou
rotina não respeitar aquilo que me dizem {[}\ldots{}{]} obviamente, sou instruído por
você para calar"-me e não posso me defender {[}\ldots{}{]} e você nunca se posicionou
para fazê"-lo. O que não seria necessário se eu estivesse em casa. Saberia como
fazê"-lo por minha própria conta.\footnote{Coleção privada.}
\end{quote}

A relação entre Rabin e Meir não melhorou quando ele se encrencou
durante a campanha presidencial norte"-americana, ao tomar partido
publicamente a favor de Nixon, em oposição ao candidato democrata George
McGovern. Rabin via Nixon como um amigo genuíno de Israel, sentia"-se em
dívida com ele e estava preocupado com o liberalismo de McGovern. Em uma
entrevista a uma rádio israelense em junho de 1972, Rabin declarou:
``apesar de apreciarmos o apoio através de palavras que recebemos de uma
das partes, devemos preferir o apoio através de ações que recebemos da
outra parte''. Os comentários de Rabin provocaram uma reação irada,
tanto nos Estados Unidos quanto em Israel. O \textit{Washington Post}
publicou um editorial criticando"-o, intitulado ``O diplomata israelense
não diplomático'', mas Rabin não se arrependeu e defendeu sua conduta em
suas memórias publicadas vários anos depois: ``naquela entrevista
mencionei que, nunca na história dos Estado Unidos, um presidente havia
expressado tamanho comprometimento norte"-americano com a segurança
israelense ou em suas declarações pró"-Israel, como o fez o presidente
Nixon em seu discurso no Congresso ao voltar de Moscou. Era um fato, e
eu, no máximo, apresentava"-o ao público israelense, não ao
norte"-americano. Realmente não sou capaz de entender como minhas
palavras puderam ser interpretadas como um `discurso de campanha' se a
campanha ainda não começou e eu me dirigia a um público que não
votaria''.\footnote{Rabin, \textit{The Rabin Memoirs}, edição expandida, \textit{op}. 
\textit{cit}., p. 142.}

O incidente não teve consequências graves. Nem McGovern nem o líder do
partido democrata, Lawrence O'Brien, viram razão para amplificá"-lo.
Henry Jackson e Stuart Symington, amigos de Rabin e importantes aliados
de Israel no Senado, escreveram a Meir defendendo Rabin. O incidente foi
esquecido pouco depois.

Rabin completou sua missão em Washington e voltou a Israel em março de
1973. Ele recebeu Meir uma última vez, pouco antes de sua partida em
fevereiro. Nixon, que apreciava Rabin, expressou sua esperança de que
ele fosse promovido ao retornar a Israel. Meir respondeu que dependeria
de seu comportamento. Era um óbvio reflexo da tensão que veio a
caracterizar a relação entre Rabin e a primeira"-ministra, bem como o difícil
início do capítulo político seguinte, na primavera de 1973.

O período de Rabin em Washington foi uma fase importante, tanto de sua
vida pessoal quanto de sua carreira. Ajudou"-o em sua transição de
oficial de carreira do exército para um político com uma rica e extensa
experiência diplomática, um profundo conhecimento do sistema político
norte"-americano e da política internacional, e um valioso \textit{network}
nos Estados Unidos. Apesar de sua relação tumultuada com Meir ao final,
a missão de Rabin em Washington foi considerada um sucesso e um
trampolim. Mas, igualmente importante foi o impacto dos Estados Unidos
na personalidade de Rabin. Conforme ele mesmo escreveu, ele estava
predisposto a se apaixonar pelos Estados Unidos baseado nas histórias e
memórias de seu pai. Ao voltar para Israel, admirava o estilo de vida e o
sistema político daquele país estrangeiro. Rabin adotou noções norte"-americanas da
economia de mercado e mudou seu estilo de vida. Desenvolveu o gosto por
uísque e tornou"-se um entusiástico jogador de tênis. A intensa vida
social do casal Rabin em Washington transformou Leah em uma anfitriã de
sucesso, e seu marido em um entusiasmado participante, em lugar do típico
solitário que havia sido. O filho de Rosa Cohen havia trilhado um longo
caminho, distanciando"-se dos valores socialistas radicais de sua mãe.

\chapter[O primeiro mandato, 1974--77]{O primeiro mandato, 1974--77}
\markboth{O primeiro mandato}{}

Em três de junho de 1974, Yitzhak Rabin foi confirmado pelo Knesset como
o quinto primeiro"-ministro de Israel. Ele estava então com 52
anos, jovem para um político israelense, e era o primeiro israelense
nato a se tornar líder do país. Quando voltou de Washington em março de
1973, Rabin não esperava se tornar o próximo líder do governo e, na
verdade, não estava preparado para tanto. Ele entrou na política na
primavera de 1973, esperando ser eleito para o Knesset pelo Partido
Trabalhista para, finalmente, incorporar"-se ao governo após as eleições
planejadas para outubro de 1973. Dov Tzamir, um membro do \textit{kibbutz} Bror
Hail, localizado no sul do país e representando o Movimento Kibbutziano Unificado
(\textsc{takam}, na sigla em hebraico) na sede do Partido
Trabalhista, propôs"-se a ajudar Rabin, que não tinha nenhuma
familiaridade com as realidades da vida partidária, a fazer a transição
da teoria para a prática. Yoram Peri, o porta"-voz do partido, também se
juntou ao pequeno time. Tanto Tzamir quanto Peri já estavam na sede do
partido quando Rabin chegou e decidiram introduzir o novo recruta no
Partido Trabalhista e na política israelense. Rabin recebeu um pequeno
escritório na sede e ele e seus dois assistentes começaram a circular
pelo país para que Rabin se familiarizasse com os costumes e a estrutura
do partido.

A transição de soldado e diplomata para político e líder
político não foi nem fácil nem serena para Rabin. Sua incipiente carreira política
havia sido transformada pela Guerra de Outubro. Durante os meses
anteriores à guerra, ele participou da campanha do Partido Trabalhista
para as eleições parlamentares. Viajou por todo o país ora com Tzamir, ora com
Peri, em um carro modesto, discursando nas filiais do partido. Tanto
Tzamir quanto Peri recordam um Rabin dedicado e disciplinado,
conversando seriamente com membros do partido e simpatizantes,
ensinando"-lhes sobre geopolítica israelense e a situação internacional.
Ele evitou o contato mais íntimo ou envolvimento com ``papo furado'', mas
impressionava a audiência com sua disposição para compartilhar suas
experiências com o baixo escalão do partido. Rabin não tinha nenhum
cargo quando a guerra eclodiu. Quando ficou claro que o comandante
responsável pela frente sul não era capaz de lidar com o desafio,
decidiu"-se enviar um general do alto escalão para restaurar a calma e a
ordem no \textit{front}. Dayan rapidamente rejeitou a ideia de nomear Rabin e em
lugar dele foi escolhido Haim Bar"-Lev. O Ministro da Fazenda, Pinchas
Sapir, pediu a Rabin que o ajudasse na coleta de fundos de emergência;
não era a sua preferência, mas Rabin concordou.

Após a guerra havia chegado a hora do acerto de contas político. A
guerra terminou com ganhos israelenses impressionantes tanto frente à
Síria quanto frente ao Egito, mas as baixas israelenses foram terríveis.
O fracasso da inteligência às vésperas da guerra e os reveses
operacionais durante os primeiros dias tinham de ser investigados e os
responsáveis apontados. Uma comissão judicial de inquérito foi criada
para investigar os reveses, a Comissão Agranat, presidida pelo
presidente da Suprema Corte. Seu relatório estava previsto para ser
publicado em abril de 1974.

Enquanto o relatório era preparado, as idas e vindas da vida
política continuaram de forma aparentemente normal. As eleições de
outubro de 1973 foram adiadas por Meir para 31 de dezembro do mesmo ano.
Em uma astuta manobra política, o governo de Meir participou da
conferência de paz árabe"-israelense organizada por Kissinger em Genebra,
às vésperas das eleições. Essa conferência era principalmente
um evento cerimonial destinado a dar à União Soviética algum papel no
processo de paz do pós"-guerra e a ajudar o Egito a avançar naquele
processo sob o manto da busca por um acordo mais amplo. Como Kissinger
bem sabia, a conferência representava um meio para que Meir apresentasse
seu governo como aquele que se movia em direção à paz que tantos
israelenses almejavam após uma guerra atroz. No final, a tática
funcionou.

Novos rostos foram adicionados ao novo governo de Meir: Rabin como
ministro do Trabalho e Aharon Yariv como ministro dos Transportes.
Yariv era o ex"-chefe da inteligência militar e tornou"-se uma figura
popular e respeitada depois do sucesso de suas negociações com o general
egípcio Abd al"-Ghani Gamasi, ao final da guerra. Rabin e Yariv eram
amigos e aliados políticos desde o início da década de 1950 e, durante o
processo de formação do governo, a história se repetiu. Rabin foi mais
uma vez cogitado como uma alternativa a Dayan, mas logo se desiludiu.
Dayan e seu aliado, Peres, recusaram"-se inicialmente a fazer parte do
novo governo pós"-guerra de Meir. Meir ofereceu a Rabin o Ministério da
Defesa e este estava a ponto de assumir o cargo com Yariv como seu vice
quando Dayan e Peres mudaram de ideia. Um relatório da inteligência
emergiu, alertando que a Síria estava a ponto de reiniciar a guerra, e
Dayan e Peres aproveitaram a oportunidade, decidindo juntar"-se ao
governo. Rabin foi então nomeado para o menos relevante Ministério do
Trabalho e, ao lado de Yariv, não estava satisfeito no governo de Meir,
sentindo que a pretensa normalidade após a Guerra de Outubro não era uma
resposta apropriada para a profundidade da crise. Eles participavam dos
encontros do Círculo Etgar, um grupo composto principalmente de altos
oficiais aposentados, descontentes com a situação do país e do Partido
Trabalhista. Seus encontros não eram secretos e eram apelidados de Noite
dos Generais. Meir via tais encontros como um ato desleal, mais um
elemento de tensão acrescentado à sua espinhosa relação com Rabin.

Meir também ficou incomodada com uma declaração de Rabin a um grupo de
jovens sionistas ortodoxos que foi vazada para a imprensa israelense em
23 de abril: ``Sou guiado por um princípio importante --- o povo de
Israel precisa saber que, para alcançar a paz entre as nações, temos de
estabelecer contatos que nos levarão a um acordo político. Sobre
Jerusalém não haverá concessão; esse é meu ponto central''. O ministro do Partido Nacional Religioso, Michael Hasani, questionou"-o:
``E Ramallah?''. Sua resposta foi a seguinte: ``Não é uma questão de vida e morte para
mim''. Um colono do Bloco Etzion perguntou: ``E eu?'', ao que Rabin
respondeu: ``Para mim, a Bíblia não é um registro de imóveis do Oriente
Médio. É um livro que fornece educação sobre valores, e seus propósitos
são diferentes''.\footnote{\textit{Yedi'ot Achronot}, 23 abr. 1974.} Esse era Rabin em sua forma mais direta. Em
uma Israel que ainda sofria os efeitos da Guerra de Outubro, uma
declaração de tal envergadura deixou de causar a comoção que teria
causado em outras circunstâncias. Mas não se deve tirar conclusões
apressadas. Não foi um prenúncio dos Acordos de Oslo de quase vinte anos
depois, mas sim um reflexo da abordagem básica de Rabin, ou seja, que a
principal área da Cisjordânia não deveria ser retida por Israel e
certamente não deveria ser colonizada.

De qualquer forma, o tempo de Rabin como ministro do Trabalho seria
curto. Em 1\textsuperscript{o} de abril de 1974, a Comissão Agranat
publicou seu relatório interino. A comissão decidiu excluir os políticos
de seu mandato e limitou suas conclusões e recomendações à esfera
militar. O chefe do Estado"-Maior das \textsc{idf} e vários altos oficiais da
inteligência foram demitidos pela comissão, e Meir e Dayan foram por ela
exonerados. Mas não pelo público. Demonstrações maciças realizadas por
soldados da reserva que haviam sido desmobilizados forçaram Meir a
renunciar em 11 de abril de 1974.

Na esteira da renúncia de Meir, a liderança do Partido Trabalhista optou
por aproveitar a sua maioria parlamentar para formar um novo governo.
Mas quem deveria ser escolhido pelo partido como o novo primeiro"-ministro?
A figura"-chave da velha guarda dos trabalhistas era Sapir, mas
ele não queria ser primeiro"-ministro. Outro candidato potencial,
representante da velha guarda do Mapai, era Hamim Tzadok, um proeminente
membro do partido, membro do Knesset e ex"-membro do governo
--- mas Tzadok concorreria somente se não houvesse oposição. Peres
anunciou que concorreria e assim Tzadok desistiu da candidatura. Peres
mesmo não era uma escolha natural, pois não era apreciado pelos
veteranos das facções do Mapai e da Achdut Haavodá. Alon, como membro da
Achdut Haavodá, não era bem aceito e, para piorar, estava fortemente
identificado com as políticas de Meir. O candidato preferido de Sapir
era Rabin, o chefe do Estado"-Maior das \textsc{idf} responsável pela vitória de
1967, um importante trunfo em um país ainda chocado pela debacle de
1973, um embaixador bem"-sucedido em Washington e intocado pelo revés de
outubro. Rabin também era próximo, mas não um membro, do Achdut Haavodá,
o rival tradicional da facção Rafi e, para os veteranos do Mapai isso
representava uma vantagem. Rabin foi recrutado por Sapir após
convencê"-lo de que, em circunstancia alguma, se candidataria. Rabin se
surpreendeu com essa reviravolta dos eventos mas, em pouco tempo,
comprou a ideia de competir pelo posto de primeiro"-ministro.

A sorte estava lançada e iniciou"-se uma corrida frenética. O candidato
teria que ser eleito pela central do partido, que contava com
aproximadamente seiscentos membros. Peres tinha o apoio da facção Rafi e
de parte do Mapai; Rabin tinha o apoio da facção do Ahdut Haavodá e dos
seguidores de Sapir no Mapai, mas Peres era um político nato e um
ativista capaz de fazer uma campanha muito melhor do que Rabin. Este não
apreciava a realidade da vida partidária e da campanha política; ele era
extremamente tímido e introvertido, desconfortável na presença de
desconhecidos, conversa fiada e cordialidade fingida. Seu aperto de mão
era surpreendentemente flácido, nada do que se esperava de uma figura
militar, uma autoridade. Ele relutava em se expor a um completo estranho
e não lhe dava prazer, nem se saía bem, tentando cativar membros e
ativistas do partido. Levantar o fone e pedir a um completo estranho que
votasse nele representava para Rabin um esforço hercúleo.

Em suas memórias, os relatos de Rabin sobre os eventos que antecederam
sua eleição foram escritos durante um período particularmente difícil
tanto de sua carreira quanto de sua relação com Peres. Elas foram
publicadas em 1979, após a sua renúncia e a derrota do partido nas
eleições de 1977. Ele era então um simples deputado em um partido de
oposição, que via como Begin celebrava seu grande feito: a paz com o
Egito. Rabin considerava Peres responsável por sua derrota e pela perda
do poder pelo partido e expôs no texto sua raiva e sua amargura. De
acordo com o relato de Rabin em suas memórias, Peres o convidou para
almoçar em um restaurante de Jerusalém e ``tentou me vender sua conversa
mole: `Vamos aprender da experiência de nossos velhos companheiros. Alon
e Dayan lutaram um contra o outro, minaram mutuamente as suas forças e
nenhum deles chegou a primeiro"-ministro. Façamos um acordo de
cavalheiros para travar uma luta limpa. Aquele que perder aceitará a
decisão de bom grado e será leal ao vencedor.' Eu desconfiei e minha
tendência era de não acreditar em nenhuma palavra do que ele havia dito.
Além do mais, estava determinado a não me aproximar do governo se ele se
tornasse primeiro"-ministro. Mas, certamente, não me opunha aos termos
que ele sugeriu. Assim que respondi, secamente, Combinado'''.\footnote{Rabin, \textit{The Rabin Memoirs}, \textit{op}. \textit{cit}., p. 239.}

Essa tensa trégua não seria mantida --- ou, segundo Rabin, respeitada. Às
vésperas da eleição na central do partido em maio de 1974, Weizman
lançou uma bomba, com sua história sobre a crise pessoal de Rabin em maio
de 1967. A intenção da mensagem era clara: um chefe do Estado"-Maior que
não conseguiu enfrentar a pressão de uma crise de segurança nacional em
1967, certamente não tinha condições de ser primeiro"-ministro em 1974.
Weizman estava acertando as contas com aquele que o havia preterido em
favor de Bar"-Lev para chefe do Estado"-Maior em 1968, mas também era um
amigo e apoiador de Peres. Na visão de Rabin, a ação era prova
suficiente de que Peres não cumpria o acordo. De qualquer forma, com o
apoio do formidável Sapir e de sua máquina partidária, Rabin ganhou a
eleição, ainda que por pouco: 298 votos contra os 254 de Peres.

Desgostoso com Peres e com a tensão entre eles, Rabin teria ficado feliz em
excluí"-lo de seu governo, mas logo descobriu que não seria possível. A
pequena maioria que obteve indicava que Peres tinha amplo apoio e que não
podia ser ignorado. Nem Golda Meir o apoiava totalmente ---
como ministra das Relações Exteriores, havia se ressentido do ``ministério
alternativo'' que Peres havia criado no Ministério da Defesa. A relação
dela com Rabin havia se tornado tensa e, assim como com outros líderes
aposentados, ela era, no máximo, indiferente a seu sucessor. A facção do
Rafi não votaria a favor de um governo que não incluísse Peres em uma
posição central; sem esses votos, Rabin não contaria com uma
maioria no Knesset.

Rabin terminou oferecendo a Peres o poderoso Ministério da Defesa. Era a
primeira rodada de uma jornada mútua de dois gêmeos siameses políticos,
que duraria vinte e um anos. Eram gêmeos que se apreciavam e
antipatizavam, competiam e se aliavam, eventualmente percebendo que
estavam unidos umbilicalmente e destinados a colaborar um com o outro.
Eram pessoas dramaticamente diferentes, com diversas qualidades e
talentos. Peres tinha imaginação, era criativo, sempre inquieto, testando
novas ideias, um político nato e experiente. Rabin era mais ``cabeça'',
um excelente analista do ambiente estratégico, com os pés firmados
no chão. Peres era um leitor voraz, amigo de escritores, poetas e
artistas, enquanto Rabin preferia concentrar"-se em questões práticas e
nunca fingiu ter objetivos artísticos ou intelectuais. Rabin era o Sabra
por excelência, enquanto Peres nunca conseguiu perder o traço do
israelense nascido na diáspora. Quando conseguiam cooperar, formavam um
time incrível, poderoso.

Enquanto isso, tornava"-se difícil a criação de uma nova coalizão sob o
comando de Rabin em 1974. O Partido Nacional Religioso, parceiro
tradicional nas coalizões com os trabalhistas, estava em transformação
de um partido moderado em um radical; na realidade, o novo braço
político do movimento dos colonos e seus candidatos mais jovens à
liderança tinham demandas e condições que, inicialmente, os mantiveram
fora da coalizão. A ala jovem do partido opunha"-se a concessões
territoriais e adotou uma posição radical na questão conhecida pelo
publico israelense e no ambiente político como ``Quem é judeu?''. Eles
insistiam na definição religiosa estritamente ortodoxa do termo
``judeu''. Rabin iniciou seu mandato com uma frágil coalizão de 61 dos
120 membros do Knesset. Meses depois, no final de outubro de 1974, o
Partido Nacional Religioso juntou"-se à coalizão. Para agregá"-los, Rabin
aceitou que qualquer concessão territorial na Cisjordânia demandaria um
referendo ou uma nova eleição geral. A distribuição dos ministérios
também teve contratempos. Rabin não queria a Eban como seu ministro das
Relações Exteriores e lhe ofereceu o simplório Ministério da Informação.
Eban recusou e ficou de fora. Rabin então ofereceu a posição para Alon,
seu velho amigo e antigo mentor. Seus candidatos originais para o
Ministério das Finanças também declinaram, talvez por acreditarem que
seu governo não duraria muito. Sapir pressionou Yehoshua Rabinovich, o
ex"-prefeito de Tel Aviv e este, relutantemente, aceitou. Ele se tornaria
um aliado muito efetivo, leal e útil a Rabin.

Eram enormes os desafios e as missões à espera de Rabin. Ele precisava
reconstruir o moral e a confiança pública no governo após a derrota de
outubro; reabilitar e reformar as Forças Armadas e a economia; e lidar
com as expectativas de Washington e do mundo árabe de que o processo
diplomático iniciado após a guerra teria continuidade. A Guerra de
Outubro, o embargo do petróleo e a quadruplicação dos preços do petróleo
deram início à assim chamada década árabe, marcada pela riqueza e pela
influência árabes na arena global. Em Israel, a guerra havia acentuado
as divisões. Os advogados da paz (e das concessões que ela exigia)
sentiam que a guerra havia demonstrado que o \textit{status quo} era
insustentável. A ala direita, por sua vez, argumentava que ela na
verdade havia realçado a crucial importância dos ativos territoriais.
Uma das consequências da guerra foi o surgimento do Gush Emunim {[}Bloco
dos Fiéis{]}, o movimento de colonos ortodoxos sionistas que buscava
assentar"-se na Cisjordânia. O Gush Emunim opunha"-se a qualquer concessão
territorial e alegava representar uma nova fase na revolução sionista,
frente ao sombrio cenário da Guerra de Outubro. Como um movimento
messiânico liderado por astutos manipuladores do sistema político
israelense, ele teria um enorme impacto na vida e na política de Israel.

Esses difíceis desafios tinham que ser enfrentados pelo que era,
inerentemente, um governo fraco. Rabin não era um primeiro"-ministro
eleito e carecia da autoridade embasada em um mandato popular. Sua
coalizão era estreita e frágil; apesar de haver"-se ampliado de 61 para
71, ela voltou a se contrair quando a facção do Ratz abandonou a
coalizão. E o Partido Nacional Religioso, abalado por seus conflitos
internos e cada vez mais radical, não era um aliado confortável. Rabin
tampouco controlava firmemente seu próprio partido, havendo derrotado
Peres por estreita margem na disputa pela liderança. Ele não tinha seu
próprio ``campo'' e dependia da velha guarda do Mapai e do
Achdut Haavodá. Rabin havia conseguido a nomeação em grande parte graças
ao apoio de Sapir, que havia declinado a posição de primeiro"-ministro e
de ministro das Finanças e, fora do governo, pouco a pouco perdeu sua
influência. De qualquer forma, distanciou"-se de Rabin e de
Rabinovich, seu antigo protegido e sucessor no Ministério das Finanças.
Um volume de trabalho atroz deixava a Rabin pouco tempo para cultivar e
reforçar suas relações com o aparelho partidário e suas bases, algo que
o desagradava e que fazia muito mal.

Teoricamente, Rabin poderia ter decidido encarar de frente essas
dificuldades, tomando decisões drásticas e espalhando pelo país a imagem
de um jovem primeiro"-ministro que representava uma nova era na política
israelense. Mas não era isso que convinha ao Rabin de 1974. Ele era
cauteloso por natureza e procedia passo a passo. Como líder político,
carecia de autoconfiança e de experiência. Tinha ciência da presença e
das sombras dos líderes da geração anterior, especialmente Golda Meir, e
do antagonismo de Dayan e Eban. Rabin não estava pronto para lançar"-se
adiante, preferindo se mover cautelosa e deliberadamente. Rabin optou
por apresentar seu governo como um ``governo de continuidade e de
mudança''. Foi um erro. O povo, conforme se comprovou mais tarde,
estava farto do velho Partido Trabalhista, de sua liderança e de seu
legado e queria novidades e mudanças, não mais do mesmo. O sociólogo
político israelense Jonathan Shapiro afirmou que a poderosa geração dos
pais fundadores do movimento trabalhista conseguiu apequenar e suprimir
seus sucessores --- e para Rabin, naquela época, parecia ser exatamente o
que acontecia.\footnote{Jonathan Shapiro, \textit{Uma elite sem sucessores: gerações de líderes políticos em Israel}. Tel Aviv: Poalim, 1984 {[}em hebraico{]}.}

Os dois principais aliados de Rabin no governo eram o Ministro da
Fazenda Yehoshua Rabinovich, um insignificante ativista do Partido
Trabalhista que se superava em sua função ministerial, e o ministro sem
pasta Yisrael Galil, sábio e experiente, um político sagaz e hábil
comunicador. Galil, ex"-chefe do Estado"-Maior da Haganá que havia sido
deposto por Ben Gurion, era um radical na tradição da ala de Yitzhak
Tabenkin do Achdut Haavodá. Rabin também tendia a aconselhar"-se com o
ministro da Justiça Haim Zadok, também sábio e experiente. A relação de
Rabin com Alon, seu ministro das Relações Exteriores, era estranha.
Conforme descrito anteriormente, Alon havia sido seu amigo e mentor e
foi primordialmente responsável por sua rápida ascensão no Palmach e nas
\textsc{idf}. A inversão de posições era desconfortável para ambos. Para
complicar, Rabin havia perdido o respeito por seu antigo líder. Algo
havia se rompido em Alon em 1949, quando foi exonerado por Ben Gurion.
Ele não havia enfrentado Ben Gurion em 1949 e nunca se recuperou daquele
revés, perdendo seu carisma e a aura de grande comandante na guerra. O
Alon de meados da década de 1970 também tendia a estender"-se e Rabin,
que não tinha paciência para pessoas que não lhe agradavam, costumava
interrompê"-lo nas reuniões governamentais. Esse tratamento não agradava
aos admiradores de Alon, que se recordavam de sua gloriosa época no Palmach.

Mas o principal problema político de Rabin era sua relação com Peres.
Sua rivalidade tinha raízes profundas, que remontava à década de 1950.
Houve um choque quase inevitável entre o poderoso e ambicioso diretor"-geral
(e depois vice"-ministro) do Ministério da Defesa e aquele que fora um incisivo
vice e mais tarde chefe do Estado"-Maior das \textsc{idf}, um caso clássico da
tensão entre a liderança politica civil do \textit{establishment} de
defesa e o militar de mais alta posição. Rabin também sabia que Peres,
protegido de Ben Gurion, era de certa forma responsável pela nomeação do
amigo Tsur, preterindo"-o para o cargo de chefe do Estado"-Maior em
janeiro de 1961; e Rabin suspeitava que Peres havia continuado a se opor
a sua candidatura em 1964. A disputa pela liderança do partido
contaminou ainda mais uma relação que já era ruim.

Uma coisa era ter Peres como concorrente em seu governo e outra,
completamente diferente, tê"-lo como ministro da Defesa. Esta é uma
posição especialmente poderosa em um país preocupado com sua segurança.
O Ministério da Defesa supervisiona as \textsc{idf}, as indústrias de defesa e a
administração dos territórios ocupados em 1967. No sistema israelense
não existe comandante em chefe; o chefe do Estado"-Maior das \textsc{idf} reporta
diretamente ao governo através do ministro da Defesa. Ben Gurion
entendia bem essa dinâmica e manteve em suas mãos a pasta.
Assim o fez seu sucessor Eshkol e vários outros primeiro"-ministros,
inclusive Rabin em seu segundo mandato. Entretanto, durante sua primeira gestão,
Rabin sentiu"-se frequentemente frustrado ao se ver incapaz de aplicar
seu profundo conhecimento e compreensão das questões militares e temas
de segurança nacional israelense através de um contato direto com Motta
Gur, então chefe do Estado"-Maior. Peres via qualquer tentativa desse
tipo como uma grave violação de sua autoridade, e Rabin identificava Gur
como aliado de Peres. Em junho, nomeou Ariel Sharon como seu
conselheiro. Sharon não foi indicado para a área militar ou de segurança,
mas a segurança era sua especialidade, e tanto Peres quanto Gur se
ressentiram de sua nomeação.

Rabin e Peres poderiam eventualmente ter superado esses enormes
obstáculos e trabalhado conjuntamente mais do que o fizeram. Quando
podiam fazê"-lo, esses dois homens tão diferentes complementavam"-se e
constituíam um time muito eficiente. Mas, na maior parte do tempo, essa
conexão não se concretizou e a relação foi tóxica desde o princípio.
Segundo o ponto de vista de Rabin, Peres nunca aceitou seu fracasso em
1974 e estava determinado a derrotá"-lo e depô"-lo. Para Rabin (e outros)
era considerado inaceitável que um ministro importante, ao mesmo tempo
que participava do governo do primeiro"-ministro, almejava e empenhava"-se
em substituí"-lo. Rabin sabia que Peres se mantinha em contato com Dayan
e via"-o como um Cavalo de Troia. Onde Rabin via subversão e intriga,
Peres via paranoia e perseguição. O temperamento de Rabin (conhecido por
ter um pavio curto) levou a varias explosões contra Peres, à vista de
todos; Peres, por sua vez, controlava muito melhor suas emoções. Essa
relação doentia entre os líderes era agravada pelas ações de seus
assistentes e porta"-vozes, ansiosos e ultrazelosos, que alimentavam a
fogueira com fofocas e vazamentos para a mídia. O fato é que Peres se
saía bem em sua competição com Rabin: ele tinha mais apoio na imprensa
israelense e era popular na base do partido. Quando competiu com Rabin
pela segunda vez, pela liderança do partido em fevereiro de 1977, quase
o derrotou. Apesar de os dois obviamente terem trabalhado juntos a maior
parte do tempo, e de o governo ter funcionado, a rivalidade entre eles
lançou uma sombra permanente sobre o primeiro mandato de Rabin.

\section{O caminho para Sinai \textsc{ii}}

A negociação e assinatura do acordo provisório com o Egito e do
memorando de entendimento com os Estados Unidos, que dele fazia parte,
foi uma das principais realizações do primeiro mandato de Rabin. O
acordo, também conhecido como Sinai \textsc{ii}, consolidou as disposições
adotadas imediatamente após a guerra, eliminou o perigo de uma retomada
das hostilidades com o Egito, criou as bases para a busca de um acordo
de paz amplo no final da década e elevou a um novo nível as relações
entre Israel e os Estados Unidos. O feito não foi simples e exigiu um
esforço diplomático intenso de mais de um ano. E alcançá"-lo geraria uma
séria, ainda que temporária, crise nas relações entre Israel e
Washington, e nas relações pessoais entre Rabin, o presidente norte"-americano Gerald Ford e Henry Kissinger.

Em junho de 1974, Nixon visitou Israel durante um giro pelo Oriente Médio
que foi, essencialmente, uma viagem de despedida por parte de um
presidente acossado que em breve abandonaria a Casa Branca. As
discussões de Rabin com ele durante a visita não trataram da fase
seguinte do processo de paz. Ficou acordado que o tema seria tratado no
final do verão, em agosto ou setembro. Mas um importante aspecto da
questão foi discutida entre Kissinger e o ministro das Relações
Exteriores Alon durante sua visita, em julho, a Washington,
a respeito de um possível acordo provisório entre Israel e a
Jordânia. Enquanto isso, o rei Hussein da Jordânia visitou Sadat no
Cairo para discutir as possibilidades de uma cooperação na fase seguinte
do processo de paz. Kissinger apoiava a ideia de um acordo modesto, que
desse à Jordânia um acesso à área de Jericó, por duas razões: a primeira
era a necessidade de encontrar outro parceiro árabe para a rodada
seguinte da diplomacia árabe"-israelense. O Egito seria o principal
parceiro árabe, mas Sadat não queria se expor à acusação de que faria,
sozinho, um acordo de paz com Israel. Após a Guerra de Outubro, a Síria,
sua aliada, alinhou"-se ao Egito e assinou seu próprio acordo
de separação de forças com Israel. Mas não havia grande entusiasmo em
nenhuma parte para mais um esforço conjunto com a Síria. Assad era
considerado um parceiro difícil e problemático para negociar, ainda
fortemente ligado à União Soviética. E era difícil arquitetar outro
acordo parcial na relativamente exígua área das Colinas do Golã. A
segunda consideração tinha a ver com a crescente influência e exposição
da \textsc{olp}. Dar à Jordânia uma posição na Cisjordânia reforçaria seu papel
como o principal defensor da causa palestina e sua reivindicação sobre o
território.

Rabin avaliou o Plano Jericó mas, no final, rejeitou"-o. Ele apoiava a
noção de um acordo territorial com a Jordânia na Cisjordânia, contudo,
naquele momento, acreditava não ter poder político suficiente para
negociar nem um acordo parcial sobre parte do território. A direita
israelense se opunha a qualquer concessão territorial, mas seu vínculo
com a Cisjordânia, ou a Judeia e a Samaria, como a chamavam, era ainda
mais forte que com o Sinai ou o Golã. Essa era a Terra de Israel e
qualquer concessão era um anátema. Rabin estava nas primeiras semanas de
sua gestão e seu governo apoiava"-se em uma pequena minoria; havia
membros de seu próprio partido e da coalizão que se oporiam à ideia.
Meir e Galil eram dois falcões, líderes trabalhistas cujas ideias eram
especialmente importantes para Rabin. E se Rabin quisesse agregar o
Partido Nacional Religioso à sua coalizão (como eventualmente fez),
eles nunca aceitariam tal concessão. A ideia foi discutida pelo gabinete
de Rabin e a maioria dos ministros não a apoiou. E assim Rabin rejeitou
a noção do Plano Jericó. Levando em conta o subsequente curso dos
eventos, foi uma decisão compreensível, mas lamentável. Também indicava
a insegurança de Rabin naquela fase inicial de seu mandato. Outro líder,
e eventualmente até Rabin em uma fase mais tardia de sua carreira, talvez
tivesse aproveitado a oportunidade para dar um passo ousado. Mas esse
não era o Rabin do verão de 1974.

Kissinger, por sua vez, manteve a bola no ar. Em 18 de agosto, ao final
de uma visita do rei Hussein a Washington, foi emitido um comunicado
conjunto que dizia: ``As discussões entre Sua Majestade, o presidente e
o secretário de Estado representam uma colaboração construtiva para as
consultas em andamento, em busca da próxima fase das negociações para
uma paz justa e duradoura no Oriente Médio. Ficou acordado que essas
consultas continuarão para, em uma data próxima e apropriada, tratar das
questões que interessam particularmente à Jordânia, incluindo um acordo
de separação entre a Jordânia e Israel'.'\footnote{Israel. Ministério das Relações Exteriores. 
``22\textsuperscript{o} Comunicado Estados Unidos"-Jordânia e a reação de Israel'', 18 ago. 1974. Disponível em: http://mfa.gov.il/MFA/ForeignPolicy/MFADocuments/Yearbook2/Pages/22\%20Joint\%20statement\%20US-Jordan\%20and\%20Israel-s\%20reaction.aspx
{[}em inglês{]}.}

Em resposta, o porta"-voz do Ministério das Relações Exteriores israelense
emitiu uma declaração dizendo que ``o governo de Israel está preparado,
conforme repetiu reiteradamente, para se empenhar em obter um acordo de
paz com a Jordânia. Israel, entretanto, rejeitou e irá rejeitar a
demanda jordaniana por uma retirada israelense ao longo do rio Jordão
como parte do que a Jordânia define como `separação de
forças'''.\footnote{\textit{Ibid}.}

Em 29 de agosto de 1974, Rabin teve seu primeiro encontro secreto com o
rei Hussein; eles se encontraram oito vezes durante o primeiro mandato
de Rabin. Sua relação se baseava na inimizade comum com a \textsc{olp} e a Síria
e no interesse israelense na sobrevivência do regime Hachemita, mas
mesmo assim não conseguiram chegar a um acordo parcial ou final. Hussein
não abandonaria sua posição em relação a um acordo definitivo: estava
disposto a oferecer uma paz total a Israel, mas somente em troca de uma
retirada completa, incluindo Jerusalém Oriental. Se Israel quisesse
manter parte da Cisjordânia, deveria falar com a \textsc{olp}. Em relação a um
acordo provisório, ele insistia em receber uma faixa que tivesse entre
oito e dez quilômetros, ao longo do Vale do Jordão. Isso protegeria o
reino de uma possível presença palestina no Jordão, mas era inaceitável
para Rabin, tendo em vista a oposição, dentro do Partido Trabalhista e
do Partido Nacional Religioso, a qualquer concessão territorial na
Cisjordânia.

Os primeiros passos concretos nas negociações para um novo acordo entre
o Egito e Israel foram dados em setembro de 1974. Rabin realizou a sua
primeira visita aos Estados Unidos como primeiro"-ministro de Israel e
seu anfitrião foi Gerald Ford, que havia substituído Nixon. Os dois
principais itens na agenda de Rabin com Ford e Kissinger eram a
solicitação israelense de novos fornecimentos de armas e a diplomacia no
Oriente Médio. Naquele momento, Rabin já havia chegado à conclusão de que o
novo acordo teria de ser negociado somente com o Egito que, em
princípio, estava disposto a fazê"-lo sozinho. Sadat não queria ser
retido por Assad e em setembro distanciou"-se da parceria com a Jordânia
que havia previsto em julho. Mas haveria um alto preço a pagar por um
acordo exclusivo entre Israel e o Egito. Sadat explicou a seus
interlocutores norte"-americanos que, para proteger seu flanco ao fazer um
acordo em separado, necessitaria melhores condições de Israel, que
justificassem seu rompimento com a solidariedade árabe.

O governo Ford buscava reiniciar o processo de paz, negociando novos
acordos entre Israel e o Egito, de preferência reforçados por um acordo
israelo"-jordaniano. Rabin iniciou sua conversa com Ford e Kissinger,
insistindo que o acordo seguinte com o Egito deveria incluir uma
dimensão política, para torná"-lo distinto dos acordos anteriores,
primordialmente militares, negociados em 1973--1974. Rabin preferia a
abordagem passo a passo à busca por um acordo abrangente. Mas sabia --- e
foi por isso fustigado por seus críticos domésticos --- que havia nisso
uma desvantagem significativa: poderia terminar sendo um evento da
``tática do salame'', o que significaria a perda de recursos territoriais de Israel 
em troca de acordos limitados, terminando com as mãos vazias ao final do
processo. Para conter essa possibilidade, Rabin exigiu que o acordo
seguinte com o Egito oferecesse a Israel o fim do estado de guerra, em
troca de uma retirada parcial israelense.

Os interlocutores norte"-americanos de Rabin disseram a ele que Sadat
concordaria com o fim do estado de guerra em troca de uma retirada
completa do Sinai, o que era inaceitável para Rabin. Ele então propôs um
recuo de trinta a quarenta quilômetros em troca de um acordo de
não beligerância. Esses números, ele lembrava, já haviam sido
mencionados durante o mandato de Meir. Os territórios a serem incluídos
em uma retirada dessa magnitude incluíam as três localidades que
estariam no foco das negociações, dos acordos e das desavenças durante o
ano seguinte: (1) o campo petrolífero de Abu Rodes; (2) os passos de
Gidi e Mitla, considerados cruciais para qualquer manobra ou ação,
defensiva ou ofensiva; e (3) a estação israelense de monitoramento
eletrônico em Umm Hashiba, um posto avançado extremamente valioso para a
inteligência. Sadat recusou a proposta de não beligerância em troca de
uma retirada parcial e insistiu na recuperação dos três locais em seu
acordo seguinte com Israel.

Esse vai e vem inicial se encerrou em setembro"-outubro de 1974, após a
cúpula árabe de Rabat, dando inicio a um hiato de vários meses. Por um
lado, a cúpula representava um revés para a Jordânia, e uma vitória para
a Síria e a \textsc{olp}. Suas resoluções reforçaram o reconhecimento da \textsc{olp} como
``o único representante legítimo do povo palestino'' e o único
demandante legítimo sobre a Cisjordânia. Sadat completou a reviravolta
iniciada em setembro ao juntar"-se à Síria, à \textsc{olp} e ao consenso árabe.
Rabin, preocupado, perguntou"-se se Sadat poderia ser considerado um
parceiro confiável. Israel e os Estados Unidos também viam com
preocupação a visita ao Cairo, planejada pelo líder soviético Leonid
Brezhnev, sem dúvida com o objetivo de levar o Egito de volta à esfera
soviética. Washington e Jerusalém decidiram aguardar o desfecho da
visita que, no final, foi cancelada. Kissinger viajou em 10 de fevereiro
de 1975 para o que foi chamado de ``uma viagem exploratória'' pela
região, visitando Israel, Egito, Jordânia, Síria e Arábia Saudita.

Rabin envolveu"-se em uma confusão antes da chegada de Kissinger, ao
conversar francamente com Yoel Marcus, um influente colunista do jornal
israelense \textit{Haaretz} --- e não seria a primeira nem a última
vez que Rabin se via em apuros por falar francamente com a mídia.
Nesse caso, seu acordo com Marcus era de que ele reportaria a substância
das declarações de Rabin, mas sem citá"-las diretamente. O resultado foi
uma exposição excepcionalmente franca das ideias de Rabin sobre a
posição e a estratégia israelenses, atribuídas ao próprio Rabin. Este
disse a Marcus que Israel tinha de passar por ``sete anos magros''(uma
alusão ao sonho bíblico do faraó, decifrado por José) e deveria,
portanto, ganhar tempo. Como resultado da utilização por parte dos
árabes da ``arma do petróleo'' e da quadruplicação dos preços desse
produto após a Guerra de Outubro, a influência árabe havia chegado ao
ápice. Israel não poderia almejar grandes sucessos diplomáticos e
deveria, portanto, tentar obter acordos limitados que lhe permitissem
passar por esse período difícil. Enquanto isso, seu objetivo deveria ser
``fincar uma cunha'' entre o Egito e a Síria e entre o Egito e a União
Soviética.\footnote{\textit{Haaretz}, 3 dez. 1974.}

A exposição franca gerou uma tormenta de protestos na mídia e por parte
da oposição. Begin, líder da oposição, insistindo que o Knesset deveria
discutir o problema, denegriu Rabin, tanto em substância quanto em
estilo; a oposição parlamentar e vários jornalistas repreenderam Rabin
por expor tão abertamente sua estratégia. Algumas semanas mais tarde,
John Lindsay, um ex"-prefeito de Nova York, viajou a Israel a mando da
\textsc{abc} News e entrevistou Rabin. Lindsay perguntou"-lhe o que Israel estava
disposta a oferecer ao Egito em troca da não beligerância. A resposta de
Rabin foi que o Egito poderia receber os passos e o campo petrolífero,
com algumas ressalvas. Houve outra leva de protestos. Estas foram as
``dores do parto'' de um político sem experiência, demasiadamente franco
em suas opiniões e comentários e sem o conhecimento de como lidar com a
mídia e como manipulá"-la.

Kissinger voltou brevemente em fevereiro e depois em março de 1975
para visitas no Cairo e a Jerusalém. Houve algum progresso, mas as
diferenças em três questões simplesmente não podiam ser superadas: o
alcance da retirada israelense e do avanço egípcio; a estação de
monitoramento israelense em Umm Hashiba; e a duração do acordo --- Sadat
insistia em dois anos, enquanto Israel queria um período mais longo.

Em 22 de março, tornou"-se óbvio que os esforços de mediação de Kissinger
haviam fracassado. Rabin recusou"-se a aceitar as exigências egípcias,
especialmente sua insistência em incluir os estratégicos passos de Mitla
e Giddi na área da retirada israelense, e sua recusa em acrescentar uma
dimensão política ao acordo. Kissinger, que apoiava a postura egípcia,
deixou a nação de Israel emotiva e irritada. Ele responsabilizava Israel, e Rabin
especificamente, por seu fracasso. Quando Rabin e Kissinger trabalharam,
íntima e eficientemente, durante o mandato do primeiro como embaixador,
Kissinger era claramente aquele que tinha o posto mais elevado. Em 1974
e 1975, ele era o secretário de Estado de uma superpotência, e Rabin era
o primeiro"-ministro --- de um pequeno país mas, ainda
assim, primeiro"-ministro. Quando Rabin foi a Washington para sua
primeira visita no cargo, registrou ironicamente em suas
memórias que já não visitava Kissinger em seu escritório de secretário
de Estado; era Kissinger que ia, conforme exigia o protocolo,
encontrá"-lo em Blair House, a casa de hóspedes presidencial. Os dois
necessitavam de tempo --- e de uma crise --- para ajustar suas relações às
novas circunstâncias. Segundo a opinião de Kissinger, quando Rabin o
convidou para sua maratona de mediação em março, ele havia se
comprometido, implicitamente, com o sucesso da missão. Não se convida o
secretário de Estado dos Estados Unidos para engajar"-se numa missão mal"-sucedida.

A famosa ``fonte de alto nível que viajava com o secretario de Estado'' no Air Force 3 
(uma referência levemente velada ao próprio Kissinger)
não perdeu tempo em reportar à imprensa norte"-americana, culpando Rabin
e Israel pelo fracasso. A isso se seguiu uma mensagem do presidente Ford
endereçada a Rabin, informando"-o de que, tendo em vista o fracasso, os
Estados Unidos teriam de rever sua política para a região, ou seja, sua
relação com Israel. Assim iniciou"-se um difícil período de seis meses,
conhecido como ``a reavaliação''. Durante esse período não foram
assinados novos acordos de fornecimento de armas a Israel, mas os
acordos antigos foram mantidos. Uma disputa explícita e uma persistente
campanha negativa promovida pelo governo Ford minou de forma
significativa a posição regional e internacional de Israel e a situação
do próprio governo. Rabin revidou, mobilizando a comunidade judaica e os
amigos de Israel no Congresso. Em junho foi lançada uma nova iniciativa
para tentar solucionar a crise. Rabin viajou a Washington, encontrou"-se
com Ford e Kissinger e juntos reexaminaram soluções criativas para o
impasse. Ford e Kissinger utilizaram uma ameaça: que a continuação do
impasse levaria à convocação da Conferência de Genebra, onde a busca por
uma solução mais ampla seria retomada. Eles sabiam muito bem que Israel
se opunha ferozmente à ideia de convocar um foro que incluísse a União
Soviética e a Síria, e renovasse a pressão para incluir a questão
palestina na agenda.

No início de julho, Rabin já havia decidido buscar uma rápida solução
para a crise. Ele enviou o embaixador Dinitz para um encontro com
Kissinger nas Ilhas Virgens, com a clara mensagem de que queria
encontrar um meio termo. Nas semanas seguintes a missão foi facilitada
por dois avanços: primeiramente, Rabin concordou com uma linha a leste
dos passos de Mitla e Giddi, o que permitiria que o Egito assumisse o
controle daquele local, mas deixando as forças israelenses estacionadas em
uma posição defensiva satisfatória; e surgiu a ideia de uma presença
militar dos Estados Unidos, mais tarde transformada na operação da estação
de monitoramento norte"-americana.

Levou algum tempo para que os ânimos esfriassem e para o retorno da
confiança entre os governos de Rabin e Ford. O presidente
norte"-americano alertou o primeiro"-ministro israelense que sua conduta
colocava em risco a futura cooperação entre os dois países. Mas essas
dificuldades foram superadas quando Rabin concordou em fazer as
concessões que recusara em março. Rabin convidou Kissinger a
retomar sua maratona entre o Cairo e Jerusalém em 21 de agosto. Dessa
vez estava absolutamente claro que Kissinger fora convidado com o
intuito de completar sua missão.

Mas uma sombra, escura e ameaçadora, pairava sobre as visitas de
Kissinger a Israel nesse período: as violentas e horríveis manifestações
do Gush Emunim. Seus líderes viam o Sinai como parte da Terra de Israel
e se opunham a qualquer concessão territorial. Eles difamaram Kissinger,
denominando"-o ``garoto judeu'' e ``marido de uma mulher gentia'',
provocando distúrbios em Jerusalém e outros locais. Uma das facetas do
caráter de Rabin era a de que a oposição a suas políticas, especialmente
oposições raivosas, somente reforçavam sua determinação. Utilizando
termos duros, Rabin ordenou à polícia que dispersasse os manifestantes
do Gush Emunim e seus líderes, se necessário com o uso da força. Em uma
entrevista para a mídia, Rabin declarou: ``Os colonos {[}\ldots{}{]} são como um
câncer no tecido social e democrático do Estado de Israel, um grupo que
faz as leis com as próprias mãos {[}\ldots{}{]} com uma perspectiva histórica, as
pessoas se questionarão o que Israel fazia em 1976 (\textit{sic}), em um
lugar ruim e sem importância. Um debate místico focava o problema
existencial de Israel. É inacreditável {[}\ldots{}{]} o que é, afinal, a
colonização? Que tipo de luta é essa? O que significa?''.\footnote{Essas citações foram incluídas no documentário \textit{Rabin em suas próprias palavras},
exibido em novembro de 2015 como parte da celebração do
20\textsuperscript{o} aniversário do assassinato de Rabin.}

Vinte anos mais tarde, as manifestações e o incitamento oriundo dos
mesmos grupos seriam dirigidos contra o próprio Rabin.

Em 31 de agosto, a parcela do acordo relativa a Israel e ao Egito estava
pronta. Foi necessária mais uma noite para completar o memorando de
entendimento israelo"-norte"-americano e assim terminar a odisseia para
chegar a um acordo em 1\textsuperscript{o} de setembro de 1975.

O acordo provisório baseou"-se no acordo de desengajamento de forças
acordado entre Israel e o Egito em 1974. Rabin foi incapaz de obter a
cláusula de não beligerância que desejava, mas os dois lados
comprometeram"-se a resolver o conflito entre eles de forma pacífica e a
não ameaçar ou recorrer ao uso da força. A força da \textsc{onu} continuaria a
cumprir a sua função. O Egito concordou em permitir que cargas
não militares originadas ou destinadas a Israel trafegassem pelo Canal
de Suez, e o acordo ficaria em vigor até ser substituído por uma nova
tratativa. As disposições para o manejo e supervisão das estações de
monitoramento por parte de técnicos norte"-americanos foram especificadas
em detalhes. Do ponto de vista de Rabin, o memorando de entendimento
assinado com os Estados Unidos era tão importante quanto o acordo com os
egípcios. O memorando lidava com a assistência militar, o suprimento de
petróleo, a assistência financeira e questões diplomáticas. Os dois
países acordaram que o passo seguinte com o Egito seria a assinatura de
um acordo de paz definitivo. O mesmo deveria ocorrer com a Jordânia.
Washington concordou com consultas imediatas com Israel caso o país
fosse ameaçado por uma ``potência mundial''. O congelamento no
suprimento de armas, em vigor desde abril, foi suspenso e foi assinado
um memorando especial que lidava com a Conferência de Genebra.
Washington comprometeu"-se a não negociar e nem reconhecer a \textsc{olp} até que
esta confirmasse o direito à existência de Israel e aceitasse as
resoluções 242 e 338 do Conselho de Segurança. Também comprometeu"-se a
coordenar sua estratégia cuidadosamente com Israel caso a Conferência de
Genebra fosse convocada e concordou em continuar as negociações de forma
bilateral. Ford também escreveu uma carta endereçada a Rabin declarando
que ``Os Estados Unidos não haviam definido uma posição final sobre a
fronteira {[}entre Israel e a Síria{]}. Quando isso acontecesse, dariam grande
importância à posição israelense de que qualquer acordo de paz com a
Síria deve ser baseado na permanência de Israel nas Colinas de
Golã''.\footnote{William B. Quandt, \textit{Peace Process: American Diplomacy and the
Arab"-Israeli Conflict Since 1967} {[}Processo de paz: a diplomacia americana e o conflito árabe-israelense desde 1967{]}. Berkeley: University of California
Press, 2005, p. 242.} Ao mesmo tempo, nas cartas enviadas em paralelo ao
Egito, os Estados Unidos se comprometeram a promover negociações
adicionais entre a Síria e Israel.

Tendo assinado o acordo provisório com o Egito em 1\textsuperscript{o}
de setembro de 1975, Rabin esperava poder repousar sobre seus louros por
algum tempo. Ele bem sabia que teria de avançar com o processo de paz
e assumia que a fase seguinte seria outro acordo israelo"-egípcio, um
acordo de longo prazo, fundamentado em uma retirada israelense de
aproximadamente metade do Sinai --- para a linha que vai de El"-Arish a
Ras Muhamad --- em troca do final do estado de guerra entre o Egito e
Israel. Mas logo ficou claro que não poderia haver uma pausa no
\textit{status quo}. Havia uma desaprovação crescente em Washington referente à
diplomacia do ``passo a passo'' de Kissinger. Seus críticos argumentavam
que havia chegado a hora de mudar para a busca de um acordo geral e
incluir o tema palestino na agenda. Em 29 de setembro de 1975, Kissinger
disse aos representantes árabes na \textsc{onu} que ``Os Estados Unidos
considerarão maneiras de trabalhar para obter um acordo global'' e que
começaria a refinar sua reflexão sobre como atender os legítimos
interesses dos palestinos.\footnote{William B. Quandt, \textit{Decade of Decisions: American Policy Toward the
Arab"-Israeli Conflict, 1967--1976} {[}Década de decisões: a politica americana a respeito do conflito árabe-israelense, 1967--1976{]}. Berkeley: University of California
Press, p. 276.} Outro passo na mesma direção foi
dado quando o vice"-assistente do secretário de Estado para o Oriente
Médio, Harold (Hal) Saunders, testemunhou junto ao Comitê de Relações
Internacionais da Câmara em 12 de novembro de 1975. O comitê, liderado
pelo deputado Lee Hamilton, conduzia audiências sobre a questão
palestina, uma evidência de que o tema ganhava relevância em Washington.
O testemunho de Saunders, apresentado também em um texto, definia a
dimensão palestina das hostilidades árabe"-israelenses como ``o cerne do
conflito''. Era uma escolha de palavras extremamente significativa. Se a
questão palestina era o cerne do conflito, qualquer tentativa de obter
uma solução que não a contemplasse seria limitada e temporária. Saunders
também declarou que ``o interesse legítimo dos palestinos tem que ser
levado em conta nas negociações de paz entre árabes e
israelenses.''\footnote{Quandt, \textit{Peace Process}, \textit{op}. \textit{cit}., p. 244.}

\section{Israel, Síria e Líbano}

Rabin se preocupava com esses eventos assim como com a tentativa de
Kissinger de apaziguar Assad. O acordo israelo"-sírio de separação de
forças de 1974 estabelecia que o mandato das forças mantenedoras da paz
da \textsc{onu} no Golã (a Força das Nações Unidas de Observação da Separação, \textsc{undof}), deveria ser renovado a cada seis meses. Isso
dava à Síria excelentes meios para pressionar Israel. Quando se
aproximou o momento da renovação, Assad criou, habilmente, a impressão
de que a Síria se recusaria, o que causou uma crise. A reação síria
aos acordos intermediários entre Israel e o Egito, em 1975, havia sido
dura. Denunciava o Egito, acusando"-o de se vender e fazer um acordo
separado com Israel. Enquanto era mantida isolada da diplomacia
árabe"-israelense promovida por Kissinger, a Síria construía uma posição
regional. Após anos de instabilidade, Assad, que tomou completamente o
poder em novembro de 1970, conseguiu estabelecer um Estado estável e
cada vez mais poderoso na Síria, engajando"-se em uma política ambiciosa
de extensão de sua hegemonia sobre seus vizinhos árabes mais fracos, o
Líbano, a Jordânia e os palestinos. O que quer que Kissinger pudesse
oferecer a Assad através de um modesto acordo adicional no Golã era
insatisfatório, e Kissinger tentou compensá"-lo de outras formas. Uma
delas era sua disposição em aceitar uma proposta sírio"-soviética para
convidar a \textsc{olp} para uma discussão do Conselho de Segurança em janeiro de
1976. Rabin reagiu com irritação. Segundo ele, os Estados Unidos, Israel
e o Egito haviam acabado de concluir uma importante etapa de sua
negociação e Washington, ao atender a uma reação russo"-síria, estava
minando sua própria política e a posição israelense. Isso causou mais um
(breve) período de tensão nas relações entre Washington e Jerusalém, e
certamente nas relações entre Rabin e Kissinger.

Contudo, no início de 1976, as relações triangulares entre os Estados Unidos,
Israel e a Síria seriam transformadas pela guerra civil no Líbano. A
guerra eclodiu em abril de 1975 e era travada entre uma coalizão
liderada pelos cristãos, que tentava preservar o \textit{status quo} na
única democracia parlamentar do mundo árabe, e uma coalizão revisionista
composta de muçulmanos, palestinos e libaneses de esquerda que tentavam
revertê"-la. A Síria e Israel se encontravam, previsivelmente, em lados
opostos nesse conflito. A Síria tinha demandas irredentistas em todo ---
ou no mínimo em parte --- do Líbano e apoiava o campo revisionista.
Israel preocupava"-se com a potencial transformação de seu vizinho árabe
mais amigável; com a possibilidade de o país cair nas mãos de elementos
radicais; com um maior empoderamento da \textsc{olp} em sua fronteira norte; e
com o aumento da influência síria. Rabin tinha receio de ser arrastado
para a guerra civil e limitou o envolvimento direto israelense ao sul do
Líbano. Ele endossou a política de Peres de oferecer ajuda humanitária
(conhecida como a política de ``boa vizinhança'') e autorizou a promoção de
uma pequena milícia local no sul de Israel, com o objetivo de ajudar a população
local a se defender da penetração de elementos hostis nas vizinhanças da
fronteira. Mas rejeitou os pedidos da alta cúpula da liderança maronita
cristã por uma intervenção direta. Rabin se encontrou com os dois
principais líderes maronitas, Camille Chamoun e Pierre Gemayel, a bordo
de um barco da marinha israelense. Sua mensagem foi de que Israel não
faria mais que ``ajudá"-los a se ajudarem''. Essa ajuda consistia em
prover equipamento e treino militar.

Mas, na primavera de 1976, uma mudança importante foi detectada na
política Síria. Assad modificou seus cálculos, passando a acreditar que
uma vitória dos radicais poderia ameaçar seu próprio país, levando"-o a
uma guerra indesejada contra Israel. Ele modificou sua política passando
a apoiar os promotores do \textit{status quo} e, ao enfrentar a oposição
de seus antigos aliados, dispôs"-se a enviar seu exército para além da
fronteira. Voltou"-se então para os Estados Unidos, para garantir que sua
intervenção militar no Líbano não provocaria uma reação israelense, e
Kissinger ficou feliz em mediar. Rabin podia ver a vantagem de um papel
estabilizador por parte da Síria no Líbano, mas não queria os sírios
demasiado próximos da fronteira israelo"-libanesa, nem livres o
suficiente para exercer seu poderio militar.

Ele insistiu em quatro condições:

\begin{enumerate}
\def\labelenumi{\arabic{enumi}.}
\item
  que o exército sírio concordasse em não cruzar um limite estabelecido
  (uma linha vermelha) quarenta quilômetros ao norte da fronteira
  israelense;
\item
  que não introduzissem no Líbano mísseis antiaéreos;
\item
  que não utilizassem sua força aérea para atacar alvos em solo
  libanês;
\item
  que a força aérea israelense mantivesse sua liberdade de ação no
  Líbano.
\end{enumerate}

Assad concordou. O episódio, conhecido desde então com o Acordo da Linha
Vermelha,
levaria à intervenção militar síria no Líbano e ao estabelecimento da
hegemonia síria no país.

Para Israel, essa era uma bênção controversa. Por vários anos o exército sírio
esteve dividido entre os dois países, reduzindo assim a ameaça que
representava para Israel. A Síria causou duras baixas à \textsc{olp} e sua
preocupação com o Líbano amenizou sua campanha contra o acordo Sinai \textsc{ii}.
No âmbito mais amplo, a guerra no Líbano capturou a atenção árabe e a
desviou, por um tempo, da questão palestina. Mas, eventualmente, não
haveria como dela escapar. A \textsc{olp} capturou o sul do Líbano e o
transformou em uma nova base operacional, em substituição àquela que
havia perdido na Jordânia. Esses eventos ocorreriam na sua totalidade
após a ascensão de Begin ao governo em Israel e fariam com que ele
adotasse uma política completamente diferente daquela adotada por Rabin.
Por dado período, a Síria respeitou o Acordo da Linha Vermelha. Entretanto, no
final de 1976, após a vitória de Jimmy Carter na eleição presidencial
norte"-americana, os sírios fizeram uma tentativa de testar a disposição
de Israel e de Washington, enviando tropas que cruzaram a linha
vermelha. A resposta de Rabin foi rápida: ele aumentou a concentração de
tropas na fronteira libanesa, convencendo os sírios a se
retirarem.

\section{Rabin, a questão palestina\break e o movimento dos colonos}

Como primeiro"-ministro israelense, Rabin tinha uma opinião absolutamente
clara sobre a questão palestina. Ele acreditava que a Jordânia era o
parceiro israelense para solucionar, ou ao menos lidar, com o problema
palestino e que a solução deveria basear"-se em um acordo territorial com
a Jordânia na Cisjordânia. Ele se opunha veementemente a lidar com a \textsc{olp},
mas aceitava que outros israelenses (como o general aposentado Matti
Peled ou o jornalista Uri Avneri), se encontrassem, não oficialmente,
com representantes da \textsc{olp}, ainda que contrariando a lei. Durante os
primeiros meses do mandato de Rabin, entretanto, ficou claro que não
seria viável assinar um acordo com a Jordânia, nem provisório nem
definitivo. Rabin sentia não possuir o poder político para oferecer
concessões na Cisjordânia, e Israel teria que continuar a administrar a
Cisjordânia e a Faixa de Gaza.

Um dos principais desafios decorrentes da manutenção do \textit{status
quo} na Cisjordânia e em Gaza era a questão dos assentamentos
israelenses. Rabin apoiava a criação de assentamentos nas Colinas do
Golã, em áreas pouco habitadas da Cisjordânia e em áreas no entorno de
Jerusalém que estivessem destinadas a fazer parte de Israel quando fosse
possível chegar a um acordo definitivo. Mas essa política era desafiada
em seu próprio partido e coalizão, assim como pelo recém"-criado Gush
Emunim. Alguns de seus aliados mais próximos, como Galil, líder da
facção Achdut Haavodá, eram bem radicais. Peres, naquela época, apoiava
o conceito de Dayan de um ``acordo funcional'' na Cisjordânia. O
conceito previa uma divisão de trabalho entre Israel, que controlaria o
território, sendo responsável pela segurança, e um parceiro árabe, que
seria responsável pela administração civil.

O Gush Emunim persistia em seus esforços para criar assentamentos na
Samaria, a região montanhosa e populosa da Cisjordânia. Seu primeiro
sucesso foi em Ein Yabrud, uma antiga base militar jordaniana que,
eventual e sutilmente, transformou"-se no assentamento de Ofra. Os
líderes do movimento dos colonos tornaram"-se hábeis na manipulação do
sistema político israelense. Nesse caso, persuadiram o ministro da
Defesa, Peres, responsável pela administração da Cisjordânia, a adotar
uma política de aceitação passiva das instalações em Ein Yabrud como se
fosse um campo de trabalho. Foi o primeiro assentamento construído na
área montanhosa da Cisjordânia, uma região supostamente reservada para um
futuro acordo com a Jordânia, ou com uma entidade palestina.

Ainda mais proeminentes foram os repetidos esforços do Gush Emunim para
assentar"-se em Sebastia, próximo a Nablus, finalmente bem"-sucedido em
dezembro de 1975. O momento foi bem escolhido. Em Jerusalém houve um
grande encontro judaico em resposta à resolução aprovada na Assembleia
Geral da \textsc{onu} denunciando o sionismo como uma forma de racismo, resultado
de uma cooperação árabe"-soviética na \textsc{onu}, com o objetivo de minar a
legitimidade de Israel. Quando Rabin foi informado da tentativa do Gush
Emunim de se assentar em Sebastia, enfrentou uma situação política
complexa. Seu ministro da Defesa, Peres, era favorável à ideia, talvez
até aos colonos. Outros aliados membros do partido de Rabin também
simpatizavam com os colonos que, habilmente, apresentavam"-se como um
movimento que reproduzia os primeiros movimentos dos pioneiros do
sionismo trabalhista. Motta Gur, chefe do Estado"-Maior das \textsc{idf}, disse
que necessitaria três dias e cinco mil soldados para evacuar os colonos.
Os membros do Partido Nacional Religioso, no governo e na coalizão, eram
abertamente favoráveis aos colonos. A possibilidade de um enorme
confronto, violento e eventualmente sangrento, entre as \textsc{idf} e os
colonos, enquanto em Jerusalém acontecia uma grande demonstração de
solidariedade judaica era, no mínimo, problemática. Rabin convocou seu
assessor, Sharon, ele mesmo um patrono dos colonos, para buscar uma
solução. O poeta e jornalista Haim Guri, amigo de infância de Rabin,
também foi designado como mediador. O resultado foi uma
permissão para que trinta colonos permanecessem em uma base das \textsc{idf} em
Qaddum. Somente mais tarde Rabin descobriu que os trinta colonos
transformaram"-se em trinta \textit{famílias} de colonos. Estas
eventualmente criaram um novo assentamento denominado Qdumim. Em
dezembro de 1975, o governo decidiu que analisaria o tema após seis meses
e, efetivamente discutiu"-o em maio de 1976. Mais uma vez Rabin não pôde
criar um consenso e a forma de resolver a questão foi decidindo não
decidir.

O evento de Sebastia foi um ponto de inflexão em vários aspectos. Foi um
marco formador na história do Gush Emunim e do movimento dos colonos,
levando à criação de novos assentamentos na Samaria. Também expôs a
fraqueza de Rabin e de seu governo. Foi um momento que demandou a
coragem política e a determinação que Rabin ainda não possuía em 1975.
Foi também mais um ponto importante na deterioração das relações entre
Rabin e Peres. Rabin era crítico do apoio de Peres aos colonos e o via
como um ato de subversão destinado a minar sua própria posição. A partir
de então, foi difícil deter o desgaste de suas relações.

A irritação de Rabin com Peres explodiria poucas semanas depois, em
janeiro de 1976, durante a viagem de Rabin aos Estados Unidos. Rabin
viajou para participar dos festejos do bicentenário da independência
norte"-americana. Ele foi calorosamente recebido pelo presidente Ford,
que havia ficado satisfeito com a assinatura do acordo provisório
em 1\textsuperscript{o} de setembro de 1975. Mas, quando Rabin
visitou o Congresso, foi constrangido pelas perguntas relativas à
solicitação, por parte de Israel, de armamentos sofisticados, inclusive
de mísseis balísticos capazes de transportar armas nucleares. As
requisições haviam sido apresentadas por Peres em dezembro e então
modificadas, quando Peres e Rabin perceberam, graças aos questionamentos
do Congresso, que a lista israelense havia sido exagerada. Quando Rabin
se encontrou com um grupo de jornalistas israelenses na Blair House,
permitiu"-se criticar e desdenhar abertamente tanto Peres quanto o
professor Yuval Neeman, assessor de Peres considerado responsável pela
preparação da lista. Rabin declarou que a lista ``continha alguns itens
mais que supérfluos, quase ridículos''.\footnote{Rabin,
\textit{The Rabin Memoirs}, \textit{op}. \textit{cit}., p. 277.} Seus comentários
foram amplamente divulgados em Israel e receberam críticas e reações
exaltadas. Em suas memórias, Rabin mais uma vez lamenta seu estilo, mas
não o conteúdo de seus comentários.

A tensão pessoal entre Rabin e Peres se intensificou e, na primavera de
1976, Rabin decidiu que teria de lidar com ela, assim como com os
aspectos negativos do caso Sebastia. Rabin sabia que sua posição e sua
autoridade vinham sendo duramente desafiadas e decidiu explicar"-se, dando
outra entrevista a Yoel Marcus. Dessa vez foram aplicadas as lições
aprendidas na entrevista de 1974, e o artigo de Marcus não foi atribuído
a Rabin, mas sim ``a uma fonte interna e bem"-informada''. Seu título:
``Uma fonte interna e bem"-informada avalia as visões, a autoridade, os
sucessos e os fracassos do primeiro"-ministro''.\footnote{\textit{Haaretz}, 12 maio 1976.}

Na entrevista, Rabin (``a fonte interna'') tentou minimizar o evento
de Sebastia. Lugares como Qaddum ``não eram realmente assentamentos'' e
não determinariam as futuras fronteiras de Israel. Não era, portanto, um
tema que justificasse uma crise partidária ou governamental. Em relação
à preocupação com a autoridade de seu governo, ele argumentava não haver
razão para preocupação. Rabin estava em uma boa situação se
comparado ao grande Ben Gurion, que várias vezes teve de ameaçar
demitir"-se para se livrar de Moshe Sharett, seu ministro das Relações
Exteriores; ou à poderosa Golda, que por cinco anos teve de tolerar a
Eban, um ministro das Relações Exteriores com quem não se importava, e
cuja política era ditada por Dayan. Rabin disse a Marcus que não precisou
se render aos ditames de seu ministro da Defesa mas, mesmo assim,
passou boa parte da entrevista reclamando dele. Ele lamentou que o
Knesset não houvesse aprovado a proposta de lei autorizando o primeiro"-ministro
a demitir membros de seu governo, e não escondeu o fato de que
se referia ao ministro da Defesa.

A crítica de Rabin ao Gush Emunim na entrevista foi relativamente amena,
quando equiparada ao que escreveu sobre o tema em suas memórias, ou às
suas declarações em várias entrevistas \textit{off"-the"-record} em 1976, 
publicadas em 2015.\footnote{Do documentário \textit{Rabin em suas próprias palavras}.} ``Eu vejo no Gush Emunim'', disse
ele, ``uma das maiores ameaças ao Estado de Israel {[}\ldots{}{]} não é um movimento
de colonos, é um câncer no tecido social e democrático israelense, a
manifestação de uma entidade que faz a justiça pelas próprias mãos.'' Em
outra entrevista não oficial, Rabin descreveu a aflição pelas
circunstâncias provocadas pelo Gush Emunim: ``Não acredito que se possa
sobreviver se não quisermos impor o \textit{apartheid} a um milhão e meio
de árabes adicionais, dentro do Estado judeu. Por essa questão, estou
disposto a enfrentar eleições {[}\ldots{}{]} a partir de uma perspectiva histórica,
perguntar"-se"-á com o que Israel lidou em 1976. Foi com um lugar
miserável que não tem importância alguma; em um debate místico em torno
do qual eles concentram os problemas existenciais de Israel {[}\ldots{}{]} O que é um
assentamento? Que tipo de luta é essa? Qaddum é um saco
vazio''.\footnote{\textit{Ibid}.}

Era típico de Rabin, conforme mencionei, que o confronto e o antagonismo
faziam com que se retraísse e se tornasse contundente. Seu argumento
sobre o Gush Emunim baseava"-se em três elementos: eles representavam
uma ameaça à democracia israelense devido a sua ideologia e ao seu uso
da violência; era falsa sua alegação de que eram a nova encarnação
dos pioneiros que povoaram o país e construíram a Israel de 1948; e, por
último, que o verdadeiro problema de Israel era a ameaça demográfica
resultante da manutenção do controle sobre um milhão e meio de
palestinos. A saída para o problema era um acordo político e, para
consegui"-lo, Rabin estava disposto a enfrentar novas eleições. Mas essa
não era uma opção viável em 1976. Ele já estava em uma rota de colisão
com o Gush Emunim, que teria um desfecho trágico em 1995.

Como era previsto, a entrevista de maio com Marcus trouxe as
questões à tona no Partido Trabalhista. Peres respondeu irritado às
críticas e ofereceu a sua demissão. Mas nem Rabin nem Peres queriam
levar seu conflito ao extremo. Eles se encontraram e combinaram uma
trégua. Rabin concedeu a Marcus outra entrevista (na qual mais uma vez
foi descrito como uma ``fonte interna'') e colocou seus comentários
anteriores em um contexto mais positivo e sob uma luz mais
favorável.\footnote{\textit{Haaretz}, 16 maio 1976.} A crise foi evitada, mas a tensão entre Rabin e
Peres não arrefeceu, continuando a minar a popularidade e a reputação do
governo.

\section{Entebbe}

Entebbe, o aeroporto internacional de Uganda, é o termo normalmente
utilizado para referir"-se a uma das maiores crises --- e uma das maiores
realizações --- do primeiro mandato de Rabin. No sábado, 26 de junho de
1976, quatro terroristas, dois palestinos e dois alemães, sequestraram
um avião de passageiros da Air France num voo de Tel Aviv para Paris. O
ato foi praticado pela ala de Operações Externas da Frente Popular para a Libertação da Palestina, liderada por Wadie Haddad. Era um grupo radical
palestino sediado em Bagdá e operando em cooperação com grupos
terroristas europeus como o Bader"-Meinhof. O avião foi primeiramente
desviado para a Líbia e finalmente aterrissou em Entebbe, onde os
sequestradores podiam contar com a colaboração do ditador ugandense Idi
Amin. A bordo havia 246 passageiros, entre eles 105 israelenses e doze
membros da tripulação. Os sequestradores exigiram a libertação de 53
prisioneiros, a maioria palestinos presos em Israel e alguns terroristas
europeus presos em seus países.

Quase uma década depois de junho de 1967 e do início do controle
israelense sobre a Cisjordânia e a Faixa de Gaza, Israel havia adquirido
ampla experiência para lidar com uma variedade de atos terroristas.
Havia registrado sucessos espetaculares, assim como dolorosos fracassos.
Em casos anteriores envolvendo a tomada de reféns em troca da libertação
de prisioneiros, a política havia sido a de dominar os sequestradores,
sem negociar um acordo. Em maio de 1974, durante as últimas semanas do
governo Meir, na cidade de Maalot, próxima à fronteira do Líbano,
22 crianças foram assassinadas por seus sequestradores quando
as \textsc{idf} atacaram a escola que ocupavam. Já no governo Rabin, em março de
1975, oito civis e três soldados, incluindo um oficial do alto escalão,
foram mortos quando um grupo do Fatah ocupou o decadente hotel Savoy, na
costa de Tel Aviv. Negociações não eram uma opção e o hotel foi atacado
em uma operação que teve um alto custo. Entretanto, em junho de 1976, uma
operação militar a 3.800 quilômetros de Israel, a princípio, não parecia
viável. Rabin criou um pequeno time de membros do governo para ajudá"-lo
a administrar a crise. Todas as esperanças de encontrar uma solução
diplomática esvaíram"-se rapidamente; anos antes, Idi Amin havia tido uma
relação íntima com Israel, mas havia se afastado e era claramente a
favor dos sequestradores. A pressão sobre o governo francês controlador
da Air France, para influenciar os sequestradores, não deu resultado.
Rabin, relutante, autorizou o início de negociações baseadas na premissa
de libertação dos terroristas presos. Levou três dias para que as \textsc{idf}
começassem a avaliar seriamente a ideia de planejar uma operação
militar, com o objetivo de liberar os reféns.

As dificuldades de planejamento de tal operação eram, no mínimo,
desanimadoras. Os reféns estavam presos no antigo terminal de Entebbe,
vigiados por doze terroristas e cercados por soldados ugandenses. A
força aérea israelense tinha aviões de carga que podiam voar até lá, mas
teriam que aterrissar sem alertar os guardas e, assumindo que a operação
fosse bem"-sucedida, seria necessário reabastecer para voar de volta a
Israel. Havia, contudo, alguns aspectos favoráveis: uma empresa
israelense havia construído o aeroporto e estava familiarizada com ele;
os passageiros não israelenses haviam sido libertados e alguns deles
poderiam fornecer detalhes valiosos; e Israel tinha uma relação cordial
com o Quênia, vizinho de Uganda.

No meio da semana seguinte o planejamento se intensificou. Rabin adotou
uma política dual: negociar com os sequestradores e explorar as
possibilidades de uma operação de resgate. A negociação foi conduzida
com toda seriedade. Era outra manifestação da cautela característica de
Rabin. Ele não autorizaria a operação de resgate antes que ficasse
absolutamente claro que ela poderia ser bem"-sucedida. Sua ampla
experiência militar havia lhe ensinado que muita coisa poderia dar
errado, que o inesperado poderia acontecer, e aconteceria. A aposta era
muito alta: se os atacantes não conseguissem surpreender os
sequestradores, um grande número de reféns poderia ser executado. Se
ocorresse alguma falha fundamental, os melhores soldados das forças
especiais israelenses estariam em perigo a milhares de quilômetros de
distância. Rabin continuou a pressionar os militares, verificando cada
detalhe. Seu porta"-voz, Dan Pattir, registrou em seu diário: ``Yitzhak
navega a crise em seu sentido mais amplo. Ele sente que, à sua revelia,
o sistema de defesa reclama de sua hesitação em autorizar a operação
militar. Mas apesar dos danos à sua imagem ele mantém sua posição:
tragam"-se somente propostas sérias, práticas para uma operação militar
abrangente, e não ideias `farmacêuticas' (\textit{sic})\,''\footnote{Diário de Pattir, citado em Amos Shifris, \textit{O primeiro governo de Rabin, 1974--1977}.
Tzur Yigal: {[}s.n.{]}, 2013), p. 70 {[}em hebraico{]}.}

Foi somente na sexta"-feira, 2 de julho, em um encontro no escritório do
primeiro"-ministro em Tel Aviv, que Rabin percebeu que as \textsc{idf} tinham um
plano que ele poderia aprovar. Naquela noite, um simulacro
havia sido praticado com sucesso em Sharm el"-Sheikh, na extremidade sul
da Península do Sinai. Quatro Hercules da força aérea decolaram de lá na
manhã de sábado e os ministros se reuniram à tarde para aprovar a
operação. Havia tempo para ordenar aos aviões que voltassem, se esta não
fosse autorizada. Rabin redigiu, com seus assessores, uma carta de
demissão. Ele planejava assumir toda a responsabilidade e pagar o preço
no caso de um fracasso. Afortunadamente, foi um gesto galante, mas
desnecessário, pois a operação foi um sucesso retumbante. Os aviões
israelenses conseguiram pousar entre os voos comerciais, sem serem
detectados pela vigilância ugandense, surpreenderam e mataram os
sequestradores. Os reféns foram embarcados nos aviões e, após o
reabastecimento no Quênia, chegaram em segurança a Israel. Um oficial
israelense, Yonatan (Yoni) Netanyahu foi abatido por tiros dos
ugandenses. Era o líder de uma das unidades de comando mais prestigiadas
das \textsc{idf} e comandava um dos grupos de ataque sob o comando geral de Dan
Shomron, que depois viria a ser chefe do Estado"-Maior das \textsc{idf}.

Em resumo, a operação Entebbe foi um grande sucesso. Uma operação
complexa, executada com precisão, baseada em planejamento meticuloso e
imaginativo, e demandando a cooperação de diversas áreas do segmento de
defesa. Ela não somente proporcionou um desfecho de sucesso para uma
complicada crise mas também restaurou muito da confiança e prestígio
que Israel e as \textsc{idf} haviam perdido na Guerra de Outubro. Rabin
orquestrou a operação e obteve a colaboração tácita da oposição e da
mídia, um feito nada desprezível. Mas, como tantas outras realizações de
seu primeiro mandato, também esta foi manchada pelas péssimas relações
entre o primeiro"-ministro e o ministro da Defesa. Uma vez que Peres se
convenceu de que havia uma opção militar, ele a promoveu com sua
habitual tenacidade, o que gerou conflito com a cautela metódica e
deliberada de Rabin. Peres tampouco apreciava o fato de que Rabin
lidasse diretamente com os militares. O chefe do Estado"-Maior durante
aqueles anos, Motta Gur, era próximo de Peres, e ele e Peres, seu chefe,
ressentiam"-se da conduta de Rabin. Rabin, por sua vez, sabia que era o
principal responsável e que uma crise dessas não permitia formalidades e
amenidades.

O sucesso da operação também provocou uma ``guerra suja'' entre aqueles
que demandavam o crédito. Peres via a operação como sua e assim a
apresentou. Ele alegou ter sido quem a planejou e a promoveu,
enfrentando o ceticismo de Rabin. Essa versão dos eventos foi propagada
com eficácia por seus apoiadores na mídia. Em 1991 ele até publicou um
livro chamado \textit{Diário de Entebbe}. Um vínculo especial foi formado
entre ele e a família Netanyahu, que perdeu seu filho mais velho na
operação e a incorporou em seu legado (a reputação do pai como grande
historiador, a morte de Yoni em Entebbe e, logicamente, o longo período
de Benjamin Netanyahu como primeiro"-ministro). Rabin respondeu irado a
essa versão dos eventos. Em suas memórias ele minimiza o papel de Peres
durante a crise. Rabin afirma que foi ele mesmo quem obrigou Peres a levar o
chefe do Estado"-Maior a um encontro dos ministros, realizado três dias
depois do sequestro. ``O fato deplorável'', descreve, ``é que 53
horas após ficarmos sabendo do sequestro, Peres ainda tinha de
consultar o chefe do Estado"-Maior sobre possíveis meios militares de
libertar os reféns''.\footnote{Rabin, \textit{The Rabin Memoirs},
\textit{op}. \textit{cit}., p. 284.} Em condições normais, as duas
abordagens e as duas narrativas poderiam ser conciliadas. Era o ministro
da Defesa que deveria insistir em uma operação militar, e o primeiro"-ministro,
como detentor da responsabilidade final, garantiria a alta
probabilidade de sucesso antes de autorizá"-la. Em condições normais,
haveria suficiente crédito para ser compartilhado por ambos. Mas, no
verão de 1976, a relação entre Rabin e Peres já estava tão envenenada que
uma tal divisão das funções e do crédito já não era possível. É preciso
reconhecer, entretanto, que ambos perceberam quais eram os limites ---
apesar da atmosfera envenenada, a conduta da operação em si não foi
perturbada.

\section{O fim do primeiro mandato de Rabin}

As eleições parlamentares de maio de 1977 definiram um ponto de inflexão
histórico. Após 29 anos de hegemonia dos trabalhistas, o poder
trocou de mãos em Israel, e Menachem Begin e seu partido, o Likud,
formaram um governo de direita. A mudança representou o ápice do
declínio dos trabalhistas após tempo demais no poder. Era também, mais
especificamente, uma reação atrasada à debacle de outubro de 1973, ou
seja, o fracasso da inteligência e os reveses militares dos primeiros
dias da guerra.

Em retrospecto, é fácil enxergar os quatro principais eventos que
facilitaram a transição de maio de 1977, durante os últimos meses do
governo Rabin. O primeiro, e talvez o mais importante, teria sido uma
série de escândalos políticos e financeiros que retratavam o Partido
Trabalhista como uma entidade decadente, corrompida por décadas de
dominação.

Em fevereiro de 1975, Michael Tzur, ex"-diretor"-geral do Ministério da
Indústria e Comércio e presidente das refinarias, propriedade da
Corporação Israelense, foi preso, indiciado e condenado pelo roubo de
catorze milhões de dólares dos cofres da companhia. Tzur era próximo de
Pinchas Sapir, o arquiteto da industrialização e do rápido crescimento
da economia na década de 1960. O sistema de Sapir, conforme era
conhecido, dava prioridade ao desenvolvimento econômico do país, muitas
vezes desprezando o devido processo. Transferir dinheiro para os cofres
do partido era ilegal, mas tolerado. Sapir e seus subalternos eram
pessoas honestas, mas a cultura que criaram facilitava a corrupção
pessoal. Esse caso foi a primeira indicação de que o sistema já não
funcionava da maneira apropriada. E havia mais por vir.

Em novembro de 1976, Asher Yadlin, presidente da seguradora de saúde da
Federação dos Trabalhadores, que havia sido nomeado pelo governo como
novo presidente do Banco Central, foi preso por suborno e sentenciado,
em fevereiro do ano seguinte, a cinco anos de prisão. Durante o
mesmo período, Avraham Ofer, ministro da Habitação, também levantou
suspeitas de ter aceitado suborno. Ofer insistiu ser inocente, lutou
para que a investigação fosse acelerada e, finalmente, cometeu suicídio
em 3 de janeiro de 1977.

Foi nesse contexto que Dan Margalit, o correspondente em Washington do
jornal \textit{Haaretz}, publicou, em março de 1977, que
Leah Rabin, esposa do primeiro"-ministro, mantinha uma conta bancária
ativa em Washington. Era uma violação das leis que regiam a posse de
moeda estrangeira na época. Não era a primeira vez que as questões
financeiras de Rabin em Washington se tornavam públicas. Já em maio de
1974 questionou"-se no Knesset o recebimento de pagamento por palestras
que ele havia dado enquanto embaixador. Agora, circulavam rumores sobre
a conta bancária. Rabin foi alertado diversas vezes que jornalistas e
inimigos políticos tentavam explorar o tema. O procurador"-geral, Aharon
Barak, lidou com o casal Rabin de forma estrita: Leah foi processada e
pagou uma pesada multa. Rabin acho impróprio alegar que a conta era de
sua esposa e, em 7 de abril, renunciou à nomeação como candidato do
Partido Trabalhista às eleições de 17 de maio. Ele também 
suspendeu a si mesmo do cargo de primeiro"-ministro. Peres se tornou o
primeiro"-ministro em exercício e também o candidato do Partido Trabalhista às eleições
para o Knesset.

O segundo evento importante na derrocada do Partido Trabalhista foi a
emergência de um sério adversário nas fileiras do antigo
\textit{establishment} israelense. Em 1974, um grupo de intelectuais
liderado por Amnon Rubinstein, reitor da faculdade de direito da
Universidade de Tel Aviv, lançou um novo movimento denominado Shinui
(Mudança). Sua plataforma era simples: a evidente decadência do Partido
Trabalhista e o fracasso de 1973 exigiam uma mudança. Ele e seus
colegas estavam perfeitamente cientes de que um grupo de intelectuais não
poderia vencer as eleições; necessitavam de um líder que pudesse
galvanizar o povo. Sua escolha foi Yigael Yadin, o segundo chefe do
Estado"-Maior das \textsc{idf}, e um conhecido arqueólogo da Universidade
Hebraica. Yadin foi visto por anos como o De Gaulle israelense,
aguardando o chamado para voltar à arena pública e liderar o país para
sair da crise. Após vários meses de negociação, na segunda metade de
1976, Rubinstein e Yadin acordaram uma colaboração e, em novembro de
1976, criaram o Movimento Democrático pela Mudança (\textsc{dmc}). A eles se
uniram dois apoiadores: Meir Amit, um general aposentado das \textsc{idf} e
ex"-chefe do Mossad, presidente de um grande conglomerado; e Shmuel
Tamir, um conhecido advogado que havia se rebelado contra Begin no
Likud. O \textsc{dmc} oferecia uma alternativa para os eleitores tradicionais dos
trabalhistas que haviam se desiludido, mas que não eram capazes de votar
no Likud de Begin.

Ao mesmo tempo, esse afastamento dos trabalhistas era reforçado pela
crescente aceitação de Begin e do Likud como competidores legítimos. Por
décadas Ben Gurion havia conseguido apresentar Begin como um pária,
uma ameaça à democracia israelense. Mas essa percepção estava
desaparecendo. Eshkol havia tratado Begin de outra maneira; por três
anos Begin e seu partido fizeram parte de um governo de união nacional
estabelecido às vésperas da Guerra dos Seis Dias, em junho de 1967, e
passaram a ser vistos pelo público com parte do sistema. O partido havia
se expandido através da união com os burgueses Sionistas Gerais e havia
perdido um pouco de sua radicalidade. Tornou"-se também um ímã para os
imigrantes insatisfeitos dos países do Oriente Médio, que viam em Begin,
ele mesmo um produto característico do judaísmo da Europa oriental, como
seu defensor. Assim como eles, Begin era a antítese do
\textit{establishment} trabalhista que os havia absorvido, integrado e, a
seu ver, humilhado na década de 1950. Muitos desses imigrantes e seus
descendentes tinham ressentimento dos governos trabalhistas que, quando
de sua chegada a Israel no início da década de 1950, colocou"-os em
barracas, mandou"-os para cidades periféricas remotas e, para piorar,
pulverizou"-os com \textsc{ddt}.

A formação de um grande bloco de direita ocorreu paralelamente ao que
podia ser visto como um terceiro evento, um intenso declínio da
popularidade e da reputação dos trabalhistas na percepção do público
israelense. As pesquisas de opinião pública mostravam insatisfação tanto
com Rabin quanto com Peres, e com o governo em geral. A imprensa
israelense, especialmente o influente \textit{Haaretz}, era crítico de
Rabin. A televisão começava a impactar a opinião pública. Um programa
humorístico popular, \textit{Nikkuy Rosh} (do hebraico ``retífica completa do
motor''), fazia sátiras impiedosas do primeiro"-ministro e de seu
governo. O povo estava insatisfeito com o débil impacto das reformas
econômicas, mas o principal dano causado ao funcionamento do governo
devia"-se à disputa infinita entre Rabin e Peres. Peres havia contestado
a liderança de Rabin no partido na votação realizada em fevereiro de
1977, quando a central do partido se reuniu para escolher o candidato
para as eleições gerais de maio de 1977. Rabin ganhou por pouco, com uma
ínfima maioria de quarenta votos. O que Peres viu como prova de seu
direito à liderança do partido, Rabin viu como a prova definitiva de que
Peres nunca havia aceitado a decisão do partido de 1974 e havia, desde
então, tentado minar sua posição como membro de seu governo.

O quarto e último evento que levou à era Begin foi o fim da íntima
cooperação entre os governos de Rabin e de Washington. Essa ruptura
deveu"-se principalmente à vitória de Carter em novembro de 1976, o que
encerrou o domínio de Kissinger e substituiu sua diplomacia gradual pela
busca de uma solução definitiva para o conflito árabe"-israelense. A
divergência entre Washington e Jerusalém tornou"-se aparente durante a
primeira visita de Rabin ao novo presidente em março de 1977, mas
indícios da mudança já haviam aparecido anteriormente.

Na segunda metade de 1975, após a assinatura do acordo provisório, a
atenção dos Estados Unidos começou a se mover na direção da questão
palestina. Além da crescente oposição à diplomacia gradual de Kissinger
e das sessões do Comitê de Assuntos Internacionais da Câmara sobre a
questão palestina, a Brookings Institution criou um grupo de estudo para
lidar com o conflito árabe"-israelense. Era composto de especialistas em
política externa, especialistas em Oriente Médio, um líder
árabe"-americano e um líder americano judeu, e comandado por Charles
Yost, um diplomata aposentado. O relatório que o grupo publicou em
dezembro de 1975 recomendava que os Estados Unidos buscassem uma solução
ampla para o conflito, e que a solução fosse baseada na retirada
israelense para a fronteira de junho de 1967, assim como a criação de um
Estado palestino em troca de uma paz contratual dada pelos árabes a
Israel.

O relatório Brookings poderia ter sido simplesmente mais um relatório
sobre o assunto. Mas Carter venceu as eleições de novembro de 1976 e
dois dos autores do relatório tornaram"-se membros importantes de seu
time de política externa: Zbigniew Brzezinski tornou"-se assessor de
Segurança Nacional, e William B. Quandt tornou"-se responsável pelo Oriente
Médio no Conselho de Segurança Nacional. O documento foi, na prática,
adotado como um roteiro para a política de Carter no Oriente Médio. Com
o encorajamento de Brzezinski, Carter decidiu lançar sua nova política
para o conflito árabe"-israelense logo no início de seu mandato. Seu novo
secretário de Estado, Cyrus Vance, foi enviado ao Oriente Médio em uma
viagem prospectiva em fevereiro; em março, Rabin aterrisou em Washington
como o primeiro de uma série de líderes do Oriente Médio que visitaram
a capital dos Estados Unidos para encontros com o novo presidente afim de discutir
seus planos para avançar em um novo e ambicioso processo de paz.

Em 6 de março de 1977, Rabin reuniu"-se com o presidente, um encontro
que seria muito diferente daqueles que tivera como embaixador e como
primeiro"-ministro durante os mandatos de Nixon e Ford. Carter
pressionava sistematicamente por avanços, baseado no relatório
Brookings: uma conferência com os países árabes e os palestinos; uma
retirada total israelense; e a criação de um Estado palestino. Rabin
expressou sua visão de um acordo: retirada da maior parte do Sinai, com
Israel mantendo Sharm el"-Sheikh e um corredor até ela; um acordo
territorial com a Jordânia relativo à Cisjordânia; nenhuma negociação
com a \textsc{olp}; não a um Estado palestino; nenhuma negociação com a Síria, que
não estava, segundo Rabin, preparada para um acordo de paz. Conforme a
discussão continuava, escreveu Rabin mais tarde, ``um alerta soou em
minha mente, essa era a essência do relatório Brookings {[}\ldots{}{]} parecia que
Carter estava convencido pelo relatório Brookings e decidido a me
vendê"-lo aos poucos''.\footnote{\textit{Ibid}., p. 293--4.}

O encontro terminou em um impasse, e Carter optou por outra abordagem,
convidando Rabin para uma discussão privada. Carter perguntou a Rabin
qual era sua ``real posição''. Rabin, como era característico,
explicou"-lhe que não tinha duas posições: o que havia apresentado na
reunião mais ampla era sua verdadeira posição. Carter tentou então outra
artimanha e convidou Rabin para acompanhá"-lo à área residencial da Casa
Branca, onde iria colocar para dormir sua filha Amy. Carter deve ter
assumido que tal familiaridade abrandaria Rabin e o tornaria mais
flexível. Rabin não se deixou impressionar pelo gesto e a visita
encerrou"-se em uma atmosfera bastante negativa.

Carter, Rabin e ambos os times reuniram"-se por mais uma hora na manhã
seguinte, 8 de março. Foi mais uma sessão hostil, um diálogo estéril.
Carter continuou a pressionar por uma conferência em Genebra, com a
participação da \textsc{olp}. Ele repreendeu Rabin por deixar de agir ``mais
agressivamente'', para tirar vantagem da ``chance de paz'' que, em sua
opinião, havia ``chegado'', e ``esquecer o passado e a história'',
adotando uma nova perspectiva. Os assentamentos nos territórios ocupados
``são ilegais'' e a quantidade de território que vocês manterão no final
envolverá somente pequenas mudanças''. Carter também lamentava o fato de
que ``sua posição é agora mais inflexível do que quando o secretário Vance
conversou com você''. Rabin obviamente contestou: ``Foi um erro os
Estados Unidos terem apresentado sua posição no início. Tendo"-o feito,
não poderia atuar como um mediador eficaz. Isso já havia sido
demonstrado em 1969 {[}quando o governo Nixon promulgou o Plano Rogers{]}''.
A posição do governo relativa à \textsc{olp} era contraproducente e minava a
tendência do mundo árabe de dar à Jordânia um papel mais preponderante.
Sobre lidar com a \textsc{olp}, Rabin lhe disse: ``você tem sua posição e nós
temos a nossa''.

Carter, irritado, resumiu: ``Mas você evita ser especifico sobre
fronteiras e a questão palestina por suas próprias razões. Você também
evita ser específico sobre a representação palestina em Genebra. Bem,
creio que nos entendemos. Podemos avançar.''\footnote{Estados Unidos da América. Relações Exteriores dos Estados Unidos, 1977--1980, vol. 8: a disputa árabe-israelita (jan. 1977--ago. 1978). Memorando de conversação. Disponível em: https://history.state.gov/historicaldocuments/frus1977-80v08/d20. {[}em inglês{]}.}

E certamente avançou. Em 16 de março, enquanto Rabin ainda estava nos
Estados Unidos, Carter declarou em um discurso em Clinton,
Massachusetts, que apoiava a criação de um lar palestino --- um Estado
com suas próprias fronteiras.

A visita de Rabin aos Estados Unidos ocorreu enquanto ele era o primeiro"-ministro
de um governo de transição. Em dezembro de 1976, Rabin havia
dissolvido seu governo e a coalizão e convocado eleições para maio de
1977. O episodio é conhecido no folclore político israelense como ``a
manobra brilhante''. Em 10 de dezembro de 1976, os três primeiros jatos
F"-15 fornecidos pelos Estados Unidos chegaram a Israel. Era uma sexta"-feira e os
aviões pousaram ao entardecer. Na lei religiosa judaica, o entardecer é
considerado o início do sábado, o dia do descanso. Rabin deu
continuidade à cerimônia, e por isso seus parceiros na coalizão, do
Partido Nacional Religioso, se abstiveram quando uma moção de
não confiança foi apresentada no Knesset. O governo ganhou a votação e
Rabin poderia ter avançado, mas decidiu demitir os ministros do Partido
Nacional Religioso, provocando assim uma crise governamental e chamando
uma eleição, marcada para maio de 1977. Rabin seguiu os conselhos de
seus confidentes políticos no Partido Trabalhista. Especulou"-se que era
uma tentativa de interromper o declínio político de seu governo e
realizar as novas eleições antes que o \textsc{dmc} tivesse tempo de se
organizar. Qualquer que tenha sido o cálculo, a ``brilhante manobra''
revelou"-se um tiro pela culatra. O \textsc{dmc} se saiu bem nas eleições,
e o Partido Nacional Religioso,
já a caminho de abandonar a sua histórica aliança com os trabalhistas,
obteve motivação adicional para se juntar ao Likud. Eventos
subsequentes, como a renúncia de Rabin e o conflito com o governo
Carter, minaram a posição do Partido Trabalhista às vésperas da eleição
de maio.

No final, Rabin precisou renunciar em abril de 1977 e não esteve
presente para absorver o dano causado pela ruptura com o governo Carter.
Mas a renúncia representou mais um golpe no Partido Trabalhista às
vésperas da eleição. Um partido, orgulhoso de sua conduta na política
externa israelense em seu momento mais importante, teria que enfrentar o
eleitorado em um contexto de tensão com Washington e uma guinada na
política dos Estados Unidos em relação a Israel e ao Oriente Médio.

Na eleição de maio de 1977 o Partido Trabalhista perdeu dezenove
mandatos e caiu de 51 para 32 membros no Knesset,
enquanto o Likud subia de 39 para 43. A vitória
do Likud foi amplificada pelo sucesso do \textsc{dmc}, que obteve quinze
mandatos. Os trabalhistas ficaram claramente enfraquecidos pela
transferência de parte de seu eleitorado tradicional para o \textsc{dmc}. Begin
conseguiu formar inicialmente um governo limitado com o Partido Nacional Religioso (que havia
obtido doze mandatos) e outros pequenos partidos, para depois incluir o
\textsc{dmc} na coalizão. Foi a primeira real transferência de poder na política
israelense após a hegemonia dos trabalhistas, que havia durado desde a
década de 1930, ainda antes da criação do Estado de Israel, e desde a
primeira eleição do país, realizada em 1949.

\chapter[Queda e ascensão, 1977--92]{Queda e ascensão, 1977--92}
\markboth{Queda e ascensão}{}

Quando os líderes políticos fracassam, raramente conseguem uma segunda
chance. Rabin se tornaria uma exceção. Ao responder às adversidades com
persistência, aproveitando ao máximo suas habilidades para adquirir o
arsenal político que lhe faltava no primeiro mandato como
primeiro"-ministro, ele a princípio retomaria uma posição de liderança
dentro do Partido Trabalhista, para depois construir um sólido
personagem junto à opinião pública israelense, como ``Sr. Segurança''. A
falta de carisma seria compensada com autoridade, objetividade e
integridade. O mandato de seis anos como um ministro da Defesa popular e
confiável funcionaria como plataforma perfeita para reconquistar a
liderança dentro do partido e o cargo de premiê.

Porém, tudo isso parecia muito distante em maio de 1977. Considerando os
54 anos de carreira pública de Rabin --- primeiro como
soldado, depois diplomata e líder político ---, o período entre 1977 e
1980 representou o momento mais difícil de todos. Ele terminou o mandato
de primeiro"-ministro completamente desmoralizado, foi substituído pelo
rival Peres, seu partido perdeu o poder e muitos o culparam pela derrota
eleitoral de maio de 1977.

Na sequência dessa derrota, Rabin tomou duas decisões importantes:
permanecer na política, como membro do Knesset, e acertar as contas com
Peres, a quem atribuía responsabilidade parcial por sua queda, porém
ampla responsabilidade pelas dificuldades que teve de enfrentar no
mandato como primeiro"-ministro. Nas eleições de maio de 1977, Rabin
pediu para receber o vigésimo número na lista de 120 nomes do Partido
Trabalhista --- o mesmo número que recebera nas eleições de 1973. Foi um
sinal claro: estava se distanciando da liderança do partido, mas queria
permanecer no jogo. Não ficou muito feliz com o papel de membro do
Knesset, mas logo acabou se envolvendo em atividades mais
significativas. Para surpresa de alguns, pesquisas de opinião pública
revelaram que ele era bastante popular entre o povo israelense. Sua
imagem não ficou muito abalada pelo caso da ``conta em dólar''; na
verdade, ele ganhou crédito por ter dividido a responsabilidade com a
esposa. Contudo, entre os colegas de liderança no Partido Trabalhista,
sua popularidade era menor, pois o responsabilizavam, pelo menos em
parte, pela derrota de 1977. Ficou, então, estabelecido um padrão que
continuaria valendo na década seguinte: Rabin tinha popularidade entre o
povo, mais do que Peres, enquanto Peres dominava as instituições
partidárias.

Esse padrão foi temporariamente rompido em agosto de 1979, com a
publicação das memórias de Rabin, em hebraico. A maior parte do texto
cumpria o esperado para livros desse gênero, com minúcia de detalhes,
mas os capítulos que tratavam do primeiro mandato dele e de sua relação
com Peres eram uma grande diatribe. As acusações eram de que Peres o
sabotava, vazava informações sensíveis e incitava a imprensa e a opinião
pública contra ele. Com uma linguagem dura, o autor acusou Peres de
fazer o jogo dos colonos radicais, ao apoiar o projeto deles, e o chamou
de ``conspirador incansável''. Rabin não era tido como uma pessoa
especialmente eloquente, mas ao longo dos anos cunhou um certo número de
frases memoráveis, e essa acabou sendo uma delas. Na sequência, houve
uma onda de protestos. Muitos membros e apoiadores do Partido
Trabalhista que não eram necessariamente hostis a Rabin nem favoráveis a
Peres acharam que Rabin tinha causado sérios prejuízos ao partido com os
virulentos ataques contra seu líder. No dia 12 de agosto, a cúpula do
partido se reuniu para discutir a questão. Alguns participantes
defendiam que Rabin merecia uma reprimenda, mas Peres, receoso dos
confrontos diretos, foi contra. A retaliação veio de outra forma. Em
setembro, foram enviadas diversas ``cartas ao editor'' para dois jornais
israelenses: o \textit{Haaretz} e o \textit{Davar}. As cartas faziam
críticas a Rabin, apoiavam Peres e, para difamar Rabin, incluíam uma
citação falsa dos diários de um antigo primeiro"-ministro, Sharett,
manchando sua reputação. Na verdade, as palavras difamatórias de Sharett
eram direcionadas ao jovem Sharon, mas alguém forjou a citação, trocando
os nomes. Foi um ato grosseiro e desastrado; bastou um mínimo esforço
para provar que a passagem do diário de Sharett era falsa e que os
signatários eram todos funcionários da Fogel, agência de publicidade
contratada pelo Partido Trabalhista. Rabin apresentou uma denúncia ao
auditor do partido, que conduziu uma investigação minuciosa e recomendou
que fosse encerrado o contrato com a Fogel. O auditor também repreendeu
o porta"-voz do partido, Yossi Beilin.

Em novembro de 1980, Rabin enfrentou outro momento difícil. Em 1979,
Alon tinha anunciado que disputaria contra Peres as primárias do Partido
Trabalhista, em dezembro de 1980. Em fevereiro, Alon morreu de forma
súbita, em decorrência de um infarto. Rabin foi convencido por amigos de
Alon e pela viúva a assumir seu lugar e disputar contra Peres, em nome
de sua ala do partido. A decisão foi equivocada, prematura, uma vez que
ele ainda precisava se recuperar dos eventos de 1977 e da repercussão
causada pela publicação de suas memórias. Mas seus rivais não queriam
arriscar. Em novembro de 1980, um mês antes das primárias, plantaram uma
história no semanário francês \textit{L'Express}, dizendo que Betzalel
Mizrahi, suposto líder do crime organizado em Israel, teria pagado a
multa imposta a Leah Rabin em 1977. Foi mais uma atitude grosseira,
repulsiva e notoriamente falsa. Rabin entrou com um processo contra o
jornal francês, que logo publicou um pedido de desculpas e pagou pelos
danos causados. Mesmo assim, acabou sendo derrotado por Peres, de forma
humilhante, contabilizando menos de 30\% dos votos.

Essa fase na carreira de Rabin coincidiu com um momento difícil na vida
de sua filha Dalia, cujo marido fora ferido durante o serviço
militar no Sinai, e os dois acabaram se divorciando. Rabin e Leah
passaram muito tempo ao longo desses anos ajudando Dalia a cuidar dos
dois filhos. Nesse período, ficou evidente o lado doce de Rabin como
homem de família, pai e avô presente, muito ligado aos filhos e netos. A
relação afetuosa entre ele e os netos ganhou um triste destaque por
ocasião de seu funeral, em 1995, quando sua neta Noa prestou a homenagem
mais comovente de toda a cerimônia.

Foram necessários mais alguns meses até que a relação entre Rabin e
Peres retomasse um rumo razoável. Antecipando as eleições parlamentares
de 1981, Peres anunciou que se vencesse, seu candidato para o Ministério
da Defesa seria Haim Bar"-Lev. No final de 1980 e início de 1981, Begin
foi tomado por uma depressão e estava ficando para trás nas pesquisas.
Porém, com a proximidade do pleito, acabou se recuperando. O fato de
ter comandado a destruição da instalação nuclear iraquiana, em maio de
1981, alavancou sua campanha. Peres, por sua vez, começou a prestar mais
atenção às pesquisas de opinião pública que indicavam a nomeação de
Rabin para o Ministério da Defesa como possível fator de diferença.
Diante disso, substituiu Bar"-Lev por Rabin, mas o cenário não se
alterou. Begin venceu por uma pequena margem e acabou eleito.

Peres e Rabin perceberam que estavam vinculados um ao outro e que era do
interesse de ambas as partes superar o antagonismo pessoal e colaborar
na liderança do Partido Trabalhista, com Rabin aceitando a posição de
número dois. Não passaram a se adorar mutuamente, mas começaram aos
poucos a contornar as diferenças e trilhar um caminho para o trabalho em
conjunto. Em 1982, abriu"-se um novo canal de comunicação entre eles,
graças a Giora Eini. Eini era advogado na Histadrut (Organização Geral
dos Trabalhadores de Israel) e tinha uma boa relação com Peres; os dois
haviam se conhecido por intermédio do partido. Em 1982, Rabin fez uma
queixa de que seus apoiadores na Histadrut estavam sofrendo
discriminação. Eini conseguiu garantir a Rabin que isso não estava
acontecendo e então estabeleceu com ele uma relação amigável. Ao
vislumbrar um possível caminho para ajudar os dois cabeças"-duras a
trabalharem juntos, Eini marcou uma reunião entre eles no gabinete de
Peres. Nos treze anos seguintes, manteve aberto esse canal de
comunicação, servindo de mediador e mensageiro. Era confiável, discreto
e não queria extrair benefícios pessoais daquela situação. Rabin e Peres
moravam perto um do outro; naquela era anterior à telefonia celular, Eini podia ser visto
telefonando de um orelhão no meio da rua, falando com Peres ou Rabin
depois de ter encontrado um deles. Sua atuação como intermediário se
prolongou até o último dia de vida de Rabin, em novembro de 1995.

Como membro sênior do partido de oposição, Rabin precisava responder às
políticas e decisões do primeiro"-ministro Begin; não era seu papel
formulá"-las. Entre 1977 e 1979, teve de assistir passivamente à
assinatura dos acordos de paz entre Begin e Sadat. Rabin acreditava, com
razão, que muito do crédito era devido a ele, graças ao acordo interino
de 1975, um passo fundamental. Mais tarde, em 1981, quando as
negociações sobre autonomia --- fruto do acordo de paz com o Egito ---
fracassaram e a crise do Líbano ganhou proeminência, os comentários e as
críticas de Rabin se tornaram mais relevantes. Ele conquistou
popularidade como analista e comentarista das questões israelenses e do
Oriente Médio. Esse se mostrou um excelente caminho para manter seu nome
no noticiário e reconstruir sua reputação e autoridade. Como exemplo,
vale lembrar que Rabin era enviado pelo importante jornal israelense
\textit{Yedi'ot Achronot} para se encontrar com figuras internacionais de
destaque. O jornalista Eitan Haber, que mais adiante se tornaria seu
assistente e redator de seus discursos, viajava com ele e publicava no
jornal o texto das conversas com esses líderes, o que depois se
transformou em livro.

A política adotada por Rabin na crise do Líbano de 1976
--- de oferecer ao lado cristão ajuda humanitária e apoio militar
limitado, mas sem se envolver na guerra civil libanesa --- foi alterada
pelo governo Begin. O desafio crescente imposto pela \textsc{olp} no sul do
Líbano, a ambição romântica de Begin de transformar o Estado judeu em
salvador dos cristãos sitiados do país vizinho, o fascínio do Mossad
pela possibilidade de uma aliança estratégica com a comunidade maronita
e, a partir de 1981, a influência de Sharon como ministro da Defesa de
Begin foram elementos que levaram Israel a intensificar o envolvimento
no Líbano. Em 1981, dois confrontos --- o primeiro contra a Síria, em
parceria com os aliados maronitas de Israel, envolvendo a cidade de
Zahlé; o segundo, contra a \textsc{olp}, num duelo de artilharia --- acabaram sem
vencedores. No confronto em torno de Zahlé, os sírios avançaram com seus
mísseis antiaéreos soviéticos, e no duelo de artilharia contra a \textsc{olp} ao
longo da fronteira entre Israel e Líbano, Israel não conseguiu chegar a
uma vitória decisiva. Rabin estava apreensivo com a mudança de rumo das
ações israelenses; em termos específicos, sua preocupação era que, sob o
governo Begin, o país estava aos poucos sendo levado a um envolvimento
massivo na guerra civil libanesa e a um confronto de grandes proporções
contra a Síria. Em agosto de 1981, ele fez um alerta importante:

\begin{quote}
A meu ver, no que diz respeito ao Líbano, os objetivos israelenses
deveriam ser {[}\ldots{}{]} um governo central que se fortalecesse gradualmente,
tornando"-se apto para governar, e que num estágio posterior levasse a Síria
a deixar o país. Esses objetivos só serão alcançados se o governo
central no Líbano for amparado pela implementação dos Acordos de Chtaura
{[}mais uma tentativa de encerrar a guerra civil libanesa{]} {[}\ldots{}{]}
Qualquer iniciativa de estabelecer objetivos mais radicais poderia levar
a um fracasso israelense, o que já ficou provado com o confronto em
torno de Zahlé {[}\ldots{}{]} Se Israel tivesse dado maior apoio ao braço cristão
radical, acabaria se envolvendo numa guerra contra a Síria, dentro do
Líbano. Israel não tem interesse nessa guerra e não teria como emergir
desse cenário com ganhos políticos. Portanto, no que concerne ao futuro
do Líbano, deveríamos continuar fortalecendo os cristãos, com base no
princípio definido pelo governo que tive a honra de liderar: ajudando"-os
a se ajudar, sem perder de vista a solução (essencialmente política)
fundamentada na sustentabilidade militar e econômica dos cristãos e no
apoio à unidade do estado libanês.\footnote{Yitzhak Rabin, \textit{A guerra no
Líbano}. Tel Aviv: Am Oved, 1983, p. 10 {[}em hebraico{]}.}
\end{quote}

Em 5 de junho de 1982, o governo Begin fez justamente o contrário do
que Rabin defendia. Begin foi eleito para um segundo
mandato em 1981 e entregou a pasta da Defesa a Sharon, que imaginara
um plano ambicioso para transformar o posicionamento de Israel
no Oriente Médio: invadir o Líbano, em colaboração com as milícias
falangistas libanesas, destruir a infraestrutura da \textsc{olp} no sul do país e
em Beirute e definir como novo presidente do país Bashir Gemayel, aliado
israelense. Bashir era filho de Pierre Gemayel, fundador do Partido
Falangista Maronita"-Cristão, que criou a milícia Forças Libanesas. O
objetivo era que Israel fizesse, então, um acordo de paz e uma aliança
estratégica com seu vizinho ao norte. Além disso, Gemayel e seus
correligionários planejavam forçar uma ampla parcela da comunidade
palestina do Líbano a se deslocar para o lado sírio da fronteira. Dali,
presumia"-se, eles seriam forçados a ir para a Jordânia, onde o influxo
ajudaria a transformar o reino hachemita da Jordânia num Estado
palestino.

No final de 1981, Begin percebeu que tinha se equivocado ao presumir que
Sadat estaria disposto a fazer a paz em separado com ele e a ajudá"-lo a
tirar a questão palestina do centro dos debates. As conversas sobre o
plano de autonomia para os palestinos, previsto pelo acordo de Camp
David (prelúdio de 1978 ao acordo de paz de 1979 entre Israel e o
Egito), não estavam avançando, e o duelo de artilharia contra a \textsc{olp} ao
longo da fronteira com o Líbano, em 1981, acabou sem vencedores, como já
mencionado. Diante disso, Begin estava pronto para adotar o ambicioso
plano de Sharon. A tentativa de assassinato do embaixador israelense em
Londres, Shlomo Argov, detonou a Operação Paz para a Galileia, lançada
no dia 5 de junho de 1982. Foi o início de uma aventura trágica e
dispendiosa, que só terminaria no ano 2000 e cujas consequências de
longo prazo ainda são sentidas no Líbano e em Israel.

A Primeira Guerra do Líbano, como ficou conhecida, era descrita pelo
governo como uma operação militar limitada, cujo objetivo era destruir e
fazer recuar os lançadores de foguetes Katyusha da \textsc{olp}, para além de seu
alcance de quarenta quilômetros. A descrição, no entanto, era falaciosa,
e os objetivos mais ambiciosos só foram revelados depois, quando o
público israelense descobriu que já estava envolvido inadvertidamente
numa guerra de grandes proporções. Os aliados libaneses de Israel não
implementaram sua parte do plano conjunto de atacar Beirute. Como
consequência, as \textsc{idf} tiveram de abrir caminho para chegar aos arredores
da cidade e impor um cerco. Sharon conseguiu fazer com que o parlamento
libanês elegesse seu aliado Bashir Gemayel para a presidência do Líbano,
em 31 de agosto de 1982, e por um momento parecia que o plano inicial
estava funcionando, mas, num curto espaço de tempo, seu castelo de cartas
desmoronou. No dia 14 de setembro, Gemayel foi assassinado,
provavelmente por emissários sírios, e dois dias depois as Forças
Libanesas perpetraram um massacre nos campos de refugiados palestinos de
Sabra e Shatila. O episódio desencadeou em Tel Aviv a maior manifestação de protesto
na história de Israel. Criou"-se uma comissão judicial de
inquérito, que expulsou Sharon do Ministério da Defesa. Yasser Arafat,
então presidente da \textsc{olp}, foi mandado do Líbano para a Tunísia com suas
tropas, mas, depois do assassinato de Gemayel, Israel ficou sem um aliado
libanês. Begin acabou atropelado pelo rumo dos acontecimentos e,
incapacitado por uma severa depressão, decidiu por fim renunciar, sendo
substituído por Yitzhak Shamir.

Ainda que Rabin e o Partido Trabalhista fossem críticos em relação à
guerra de Begin e Sharon, a partir do momento em que ela se tornou
realidade, Rabin se viu diante de um duplo dilema: em termos políticos,
ele era moderado, mas em termos militares, era radical. Como ex"-chefe do
Estado"-Maior, acreditava que, uma vez que as \textsc{idf} entravam num conflito
--- que dirá uma guerra ---, era imperativo vencê"-la. Tampouco seria
apropriado que o principal partido de oposição criticasse uma guerra ou
uma importante operação militar que ainda estivesse em curso. E mais: em
algum momento parecia que a aventura libanesa tinha tudo para acabar de
forma vitoriosa. Rabin precisava conciliar suas declarações de acordo
com as oscilações da guerra. Ele falhou uma vez. No dia 4 de julho de
1982, Sharon, seu ex"-assessor e protegido, propôs a ele um convite para
visitar as tropas que cercavam Beirute. O conselho de Rabin foi
``apertar o cerco'' à cidade. A declaração, por si só equivocada, foi
feita na presença de jornalistas e acabou sendo muito citada e criticada
em Israel e no exterior.

\section{Ministro da Defesa, 1984--1990}

Em 1983, já estava claro que a aventura israelense no Líbano era um
fracasso, maculada ainda mais pelo massacre perpetrado em Sabra e
Shatila por aliados libaneses. Amin Gemayel, irmão mais velho de Bashir,
então eleito à presidência, se distanciou de Israel. O que ainda restava
de esperança do lado israelense quanto à aliança com as Forças
Libanesas e com a comunidade maronita estava se desfazendo. Terroristas
xiitas aliados ao Irã e à Síria perpetraram uma série de ataques
suicidas contra forças americanas e francesas que tinham sido enviadas
ao Líbano num esforço de estabilizar a situação. Outro ataque suicida
destruiu a embaixada norte"-americana em Beirute. Israel chegou a assinar
uma espécie de tratado de paz com o Líbano, que se provou um pedaço de
papel sem qualquer utilidade. Moshe Arens substituiu Sharon como
ministro da Defesa e supervisionou a retirada das \textsc{idf} dos arredores de
Beirute rumo à linha do rio Awali, ao sul da cidade. A única conquista
que o governo de Begin poderia contabilizar era que Arafat e suas
tropas saíam do Líbano e partiam rumo à Tunísia. Além de se provar um fracasso, a
imprudente aventura em terras libanesas também acabou prejudicando a
aura de paz com o Egito.

As eleições de julho de 1984 foram conduzidas à sombra da Guerra do
Líbano. Porém, por mais que o público israelense estivesse revoltado
contra os arquitetos da guerra, os trabalhistas, sob o comando de Peres,
não conseguiram maioria suficiente para formar um novo governo. Mesmo
com o pano de fundo do fiasco do Likud no Líbano, a inflação
descontrolada e o nada carismático Shamir na liderança, o partido
conseguiu, para surpresa geral, 41 assentos no Knesset,
enquanto o Partido Trabalhista chegou a 44. A estreita
vantagem não foi suficiente para que Peres formasse um governo, porque
os partidos ortodoxos e algumas pequenas facções de direita tendiam a
apoiar o Likud. Vieram à tona novos padrões na política israelense: os
eleitores tinham feito uma guinada à direita, e os trabalhistas não
conseguiam formar uma coalizão dominante, mesmo com a maioria dos votos.
Um dos principais desafios do Partido Trabalhista, evidenciado nas
eleições de 1977, continuava sendo a hostilidade do eleitorado
mizrahim, os judeus de países do Oriente Médio e seus descendentes. Para
eles, os trabalhistas representavam o \textit{establishment} asquenaze do
Leste Europeu que os tinha absorvido e humilhado na década de 1950 e
continuavam sendo considerados o \textit{establishment}, mesmo muito tempo depois
de terem perdido o poder político. O Likud, ao contrário, representava o
\textit{antiestablishment} (ainda que estivesse no poder), a segurança nacional e uma
postura firme em relação aos árabes. Nesse contexto, Peres era visto
como epítome do Partido Trabalhista.

O terceiro fracasso na tentativa de vencer uma eleição deixou claro que
a liderança de Peres era problemática. Ainda assim, em 1984 a supremacia
dele não foi contestada por Rabin. A relação entre os dois passou então
por uma transformação curiosa; Yitzhak Navon, popular ex"-presidente de
Israel, estava considerando concorrer à liderança contra o amigo Peres.
Depois de alguma hesitação, Rabin decidiu manter a parceira com Peres,
contra o possível novo candidato, e Navon acabou desistindo. Como conta
em suas memórias, Navon chamou Rabin para uma reunião, pois queria
persuadi"-lo a ser seu número dois. Perguntou a Rabin se ele apoiaria
Peres, e Rabin respondeu: ``Sim. O partido não quer uma disputa''. Navon
então retrucou: ``Yitzhak, mas você sabe muito bem que, segundo as
pesquisas, nós não venceremos com Peres, mas se eu encabeçasse a lista,
poderíamos ganhar. Vamos juntos, precisamos pensar no bem do Estado e em
como derrubar o Likud''. Rabin respondeu de forma assertiva e educada:
``Veja bem, o partido não quer mais nenhuma disputa. Cheguei a um acordo
com Shimon, e o partido está apaziguado''.\footnote{Yitzhak Navon,
  \textit{Todo o caminho: uma autobiografia}. Jerusalém: Keter, 2015, p. 384
  {[}em hebraico{]}.}

Para encerrar o impasse entre o Likud e o Partido Trabalhista produzido
pelas eleições de 1984, introduziu"-se um novo mecanismo na política
israelense: um governo de unidade nacional baseado em alternância. O
acordo previa que Peres seria o primeiro"-ministro pelos dois primeiros anos
do mandato de quatro anos, enquanto Shamir ocuparia o cargo de ministro
das Relações Exteriores. Rabin ficaria com o Ministério da Defesa nos
quatro anos. O acordo ainda fazia uma divisão detalhada de cargos,
incluindo as nomeações para embaixador em Washington e na \textsc{onu}. Ao
contrário do que muitos imaginavam, o governo sobreviveu aos quatro anos
e funcionou bastante bem. Conseguiu dar conta das consequências
imediatas e do rescaldo da Guerra do Líbano --- em primeiro lugar, a
necessidade de recuar as \textsc{idf} mais para o sul --- e conseguiu reduzir a
inflação crescente que ameaçava destruir a economia israelense. Na
prática, o governo não estava de fato dividido em dois, mas em três.
Como ministro da Defesa, Rabin tinha seu próprio feudo. Em seu mandato
como primeiro"-ministro, ele se incomodara com o enorme poder investido
no Ministério da Defesa, comandado então por Peres, mas, dessa vez,
ao assumir a pasta, passou a desfrutar desse poder.
A verdade é que Rabin desabrochou na nova função. Na Defesa, sentia"-se
em casa: gostava do ambiente e do trabalho e assim se mantinha
relativamente afastado da política. Não precisava mais atuar como
analista ou comentarista político; a mídia estava sempre querendo
entrevistá"-lo e citá"-lo oficialmente nessa função. Fato é que, durante os
dois governos de unidade nacional, referiam"-se de brincadeira ao
Ministério da Defesa como ``o governo de Tel Aviv''. Rabin tinha
controle total sobre o imenso \textit{establishment} da Defesa e, como
membro do trilateral ``clube dos primeiros"-ministros'' (junto com Shamir
e Peres), seu \textit{status} e influência se equiparavam aos dos outros dois. Na
época, sua relação com Peres era razoável, e a relação com Shamir,
bastante boa.

O clube dos primeiros"-ministros teve de enfrentar dois escândalos que
ameaçaram a estabilidade do governo de unidade. Um deles foi o escândalo
da ``Linha de ônibus 300'', no final do período de governo de Shamir, em
abril de 1984. Dois palestinos sequestraram um ônibus israelense e
acabaram mortos por oficiais do Serviço de Segurança Geral de Israel (\textsc{gss}, no
acrônimo em inglês). Os assassinatos, em si, mas principalmente os esforços
para encobri"-los e os danos provocados ao \textsc{gss} e ao sistema legal
israelense se transformaram numa prolongada e sórdida questão política.
Embora o episódio tenha acontecido sob o governo de Shamir, Peres e
Rabin tiveram de lidar com seu potencial destrutivo: estavam sob ameaça
as próprias fundações da democracia e do sistema legal israelense, além
da estabilidade do governo como um todo. Em 1985, veio à tona outro
escândalo, quando Jonathan Pollard, funcionário da Inteligência Naval
dos Estados Unidos, foi preso pelo Departamento Federal de Investigação (\textsc{fbi}) por fazer espionagens para
Israel. Pollard tinha sido recrutado para o serviço por um braço militar
da comunidade de inteligência israelense, durante o mandato de Sharon
como ministro da Defesa. Foi um tremendo equívoco, que abalou as
relações entre Estados Unidos e Israel e, novamente, a estabilidade do
governo. O governo de unidade nacional respondeu às demandas
apresentadas pela administração do presidente Ronald Reagan e acabou
conseguindo controlar a crise na relação com Washington. Pollard foi
condenado a uma longa pena de prisão e, até sua soltura, em 2015, o caso
continuava perturbando as relações entre os dois países. Nesse exemplo,
também, a cooperação entre os três membros do clube se provou muito
efetiva.

Além de cumprir uma agenda ocupada, típica de um ministro da Defesa,
Rabin precisou encarar quatro questões importantes durante esse período:
a retirada parcial do Líbano, a Primeira Intifada, o cancelamento do
Projeto Lavi e o esforço para restabelecer o plano de autonomia para a
Cisjordânia e a Faixa de Gaza.

\section{A retirada parcial do Líbano}

Como ministro da Defesa no governo de unidade nacional, Rabin era o
responsável pela política israelense no Líbano. A estratégia do Likud
após a guerra não fora coerente nem efetiva. Israel se viu diante
do seguinte cenário: o presidente libanês Amin Gemayel estava
enfraquecido e hostil; o regime sírio, por sua vez, parecia cada vez
mais confiante e tentava apagar qualquer vestígio da guerra de 1982; os
Estados Unidos ainda estavam sob o efeito dos devastadores ataques à sua
embaixada em Beirute e aos quartéis de fuzileiros navais; e no sul do
Líbano, ao norte da fronteira israelense, grupos locais rivais se
enfrentavam mutuamente. Em termos oficiais, a situação de Israel no
Líbano e sua relação com o governo do país eram reguladas pelo assim
chamado acordo de paz de maio de 1983, mas esse acordo não passava de
letra morta, e Hafez al"-Assad estava determinado a anulá"-lo. Depois da
saída forçada de Sharon do Ministério da Defesa, seu sucessor, Moshe
Arens, retirou as \textsc{idf} do entorno de Beirute, rumo ao sul, para uma linha
que passava mais ou menos ao longo do rio Awali. Não fazia sentido se
ater àquela linha. As \textsc{idf} estavam sendo fustigadas e atacadas, e Arens e
a liderança das \textsc{idf} não viam sentido em ficar. Porém, Arens não
conseguia fazer com que o primeiro"-ministro Shamir e uma maioria dos
ministros do Likud apoiassem uma retirada mais ampla. Sharon ainda era
influente na época e estava se apegando a qualquer fragmento do que
poderia ser interpretado como conquista produzida pela guerra. Israel
insistia que todas as forças estrangeiras --- ou seja, o exército sírio
--- deveriam se retirar do Líbano, como condição para sua própria
retirada, enquanto tentava ao mesmo tempo chegar a um acordo com o
governo Gemayel. Nenhuma das duas políticas tinha qualquer perspectiva
de ser implementada. A Síria não tinha o menor interesse em retirar suas
tropas e, ainda por cima, seus aliados iranianos estavam consolidando a
presença e influência na comunidade xiita.

O esforço inicial de Rabin para arrumar a casa tinha por objetivo chegar
a um entendimento indireto com a Síria, por intermédio dos Estados
Unidos. A ideia era se espelhar na bem"-sucedida política da linha
vermelha de 1976, segundo a qual Israel concordava com a presença
militar síria no Líbano, contanto que a Síria se mantivesse afastada da
área da fronteira israelense. O diplomata americano Richard Murphy
tentou implementar esse plano para Rabin, mas Assad não concordou. Depois
de superar o fiasco inicial de 1982, o presidente sírio se sentia
confiante e fortalecido. Passou a defender a ``paridade estratégica''
com Israel e não tinha intenção de aliviar o sufoco que o país vinha
enfrentando no Líbano. Rabin então decidiu que a única opção seria uma
retirada unilateral. Em janeiro de 1985, com o apoio de Shamir e de
outros dois ministros do Likud, conseguiu aprovação do gabinete para uma
retirada unilateral em etapas, que seria implementada ao longo de seis
meses. Israel protegeria sua fronteira ao norte com uma zona de
segurança de dez quilômetros de profundidade e o apoio de uma milícia
local avançada, o Exército do Sul do Líbano, comandado por Antoine
Lahad, ex"-general do Exército libanês. Ao norte da zona de segurança, a
força de manutenção da paz da \textsc{onu}, a \textsc{unifil}, criada em 1978, continuaria
sua missão.

A retirada das \textsc{idf} para essa nova linha terminou de ser implementada em
junho de 1985. Não foi uma retirada total, e a presença direta
e indireta de Israel na zona de segurança continuaria sendo um problema
difícil até o ano 2000. Logo após a retirada de 1985, o escopo completo
dessa questão duradoura ainda não estava claro. Dois grupos xiitas --- o
Amal, ligado à Síria, e o Hezbollah, um braço do regime iraniano ---, bem
como diversos grupos palestinos, disputavam o controle e estavam
preocupados, antes de mais nada, em consolidar sua posição e influência.
O Exército do Sul do Líbano não conseguia se manter de forma
independente e precisava do reforço das \textsc{idf}, presentes na zona de
segurança. Israel tinha de enfrentar provocações e baixas, mas o
problema parecia controlável.

Por menor que fosse o número de baixas israelenses no sul do Líbano
naquele momento, Rabin se mostrava extremamente sensível a todas as
perdas. Da mesma forma que fazia quando tinha sido chefe do Estado"-Maior
das \textsc{idf}, ele comandava os mínimos detalhes do reposicionamento de Israel
no Líbano e insistia em ir a campo para supervisionar a distribuição das
unidades do exército. Como fizera nos anos 1950 e 1960, passava horas e
horas com os oficiais subalternos e os soldados da tropa, ganhando
domínio sobre a realidade operacional, sem perder de vista o panorama
geral. Quando as \textsc{idf} sofriam baixas no Líbano, Rabin assumia
responsabilidade pública por elas. Aparecia na televisão e afirmava que,
como ministro da Defesa, tinha total responsabilidade pelas mortes. Não
eram atos calculados de um político perspicaz, mas o reflexo genuíno de
como Rabin concebia seu papel. Ele estava na flor da idade, e seu
aspecto viril, a voz rouca e o cigarro sempre presente numa das mãos
eram elementos que contribuíam para sua aura de autoridade. Isso
repercutia bem entre o público israelense e ajudava a compor o
personagem de líder íntegro e confiável. Sua liderança, contudo, logo
seria posta à prova com a eclosão da Primeira Intifada.

\section{Rabin e a Primeira Intifada}

Os acontecimentos desencadeados pela Primeira Intifada (ou levante
espontâneo, de acordo com o termo em árabe), causariam uma mudança
radical no ponto de vista e na estratégia política de Rabin,
transformando os rumos futuros da política israelense. Quando a Intifada
teve início, ele estava numa visita aos Estados Unidos e foi pego de
surpresa. Como muitos de seus colegas dos \textit{establishments} político
e de defesa em Israel, Rabin passara a acreditar --- depois de vinte anos
de relativa calma na Cisjordânia e na Faixa de Gaza --- que o \textit{status quo}
poderia ser mantido. O Fatah e outros grupos palestinos haviam tentado
incitar a população palestina por meio de seus ataques, mas a maior
parte do povo nunca se insurgiu. Assim, o levante espontâneo de dezembro
de 1987 surpreendeu Israel, e também a própria liderança da \textsc{olp}. A
princípio, o episódio não foi encarado como marco de uma mudança radical. Em
amplo contraste com os atos terroristas que a \textsc{olp} tinha tentado
desencadear em anos anteriores, a Intifada se manifestou por meio de
protestos e revoltas da população civil, incluindo mulheres e
crianças. No meio de sua viagem de trabalho aos Estados Unidos, Rabin
não sentiu necessidade de encurtar a visita. Mais tarde, seria criticado
por esse erro de julgamento.

De volta a Israel, sua resposta inicial mais uma vez comprovou a
tendência de pensar e agir de forma radical em termos militares e de
forma moderada em termos políticos. E o lado radical veio primeiro: Rabin
acreditava que os desafios à segurança deviam ser vencidos e, para
conter o levante, ordenou uma política linha"-dura, baseada no uso da
força. Acusaram"-no de ter dito aos soldados das \textsc{idf} que ``quebrassem os
ossos'' dos manifestantes, mas não foi bem assim. A história surgiu numa
reunião que ele teve com os soldados numa colina perto de Ramallah,
pouco depois de voltar dos Estados Unidos. Como de hábito, Rabin não
queria administrar o confronto com os palestinos a partir de seu
escritório, então foi a campo. Os soldados, que tinham recebido ordens
para não atirar, reclamavam de se sentir impotentes; diziam que estavam
sendo humilhados por sua incapacidade de lidar com os manifestantes,
incluindo mulheres e crianças, que atiravam pedras, batiam e cuspiam
neles. Rabin lhes disse que, em vez de ficar parados, eles deviam atacar os
manifestantes, usando a força e cassetetes se necessário. Verdadeira ou
não, a história e a ordem atribuída a ele assombraram"-no durante muitos
anos. Amnon Strashnov, advogado"-geral das \textsc{idf} naquele momento, descreve
que Rabin fez uma declaração pessoal no Knesset, em 11 de julho de 1990,
dizendo"-se responsável pela decisão de migrar para o uso da força e ao
mesmo tempo evitar a utilização de armas de fogo: ``E eu, como ministro
da Defesa na época, assumo total responsabilidade por isso, conforme as
diretrizes que proferi às forças. Se minha memória não falha, nunca
disse que deveriam quebrar ossos; as diretrizes que proferi do meu
gabinete para os comandantes das \textsc{idf} foram claras''. Strashnov questiona
a clareza dessas instruções e insinua que elas foram bastante ambíguas,
de modo que alguns oficiais se sentiram autorizados a fazer uso
excessivo da força.\footnote{Amnon Strashnov, \textit{Justiça sob fogo cerrado}.
  Tel Aviv: Yedi'ot Achronot Books, 2004 {[}em hebraico{]}.} Porém, mesmo
que não tenha dado essa declaração explícita, Rabin sem dúvida foi o
autor de uma política que buscava debelar a Intifada pelo uso da força,
mas minimizando o recurso a armas de fogo. Ele implementou
essa política muito tempo depois de já ter chegado à conclusão de que só
havia um caminho para lidar com a Intifada: por meio de uma solução
política. Para ele, primeiro a violência tinha de ser combatida, para
que só então um processo político fosse posto em ação; do contrário, o
episódio seria encarado como uma derrota israelense. Posteriormente,
vieram à tona vários casos de abuso, e diversos oficiais e soldados
foram levados a julgamento.

Em poucas semanas, Rabin compreendeu que a Intifada era impulsionada por
correntes profundas e forças poderosas, representando uma guinada
radical nas relações entre israelenses e palestinos. Durante vinte anos,
os israelenses tinham convivido com a ocupação da Cisjordânia e da Faixa
de Gaza porque eram poucos os impactos sobre a maioria da população do
país. Havia resistência, bem como episódios de terrorismo, mas os
israelenses, por conveniência, podiam fechar os olhos a essa realidade,
o que vinha sendo feito havia duas décadas. A revolta popular e uma nova
modalidade de resistência semiviolenta representavam um novo desafio.
Israel teria que reprimir a resistência por meio de oposição civil, e os
israelenses veriam seus filhos, soldados das \textsc{idf}, perseguindo
adolescentes que lhes arremessavam pedras em Nablus e Ramallah. Muitos
achavam esse cenário inaceitável e exigiam uma nova mentalidade,
abordagens diferentes e possivelmente novas políticas.

Um exemplo dessa necessidade de implementar mudanças urgentes foi uma
reunião inusitada convocada por Rabin poucas semanas depois da eclosão
da Intifada. Foram convidados generais do exército, oficiais ativos e
aposentados da Administração Civil na Cisjordânia e na Faixa de Gaza,
além de especialistas em questões árabes e do Oriente Médio, tanto do
serviço público quanto da academia. Rabin queria ouvir o que esses
especialistas tinham a dizer sobre a Intifada e a política que Israel
deveria adotar. A apresentação de maior peso foi feita pelo historiador
Shimon Shamir, professor emérito especializado em Oriente Médio, cujo
texto situou o dilema israelense em relação à Intifada no contexto da
experiência de outras sociedades ocidentais que tiveram de lidar com
insurreições populares em territórios sob seu controle. Shamir mostrou
que em praticamente todos os casos o uso da força para reprimir a
resistência não foi uma boa saída; nesses exemplos, foi preciso recorrer
a acordos políticos. A análise minuciosa e eloquente não chegava a
surpreender. O fato surpreendente foi que no dia seguinte apareceu no
escritório de Shamir um enviado de Rabin, que queria o texto. O general
Ehud Barak também mandou um emissário, com o mesmo objetivo, e o texto
acabou sendo amplamente divulgado nos setores de defesa e segurança de
Israel. O esforço para conter a Intifada prosseguiu, mas a lição ficou
registrada na mente de Rabin. Um ano depois, ele declarou: ``A solução
só pode ser política''. A fala foi o prenúncio de uma mudança iminente.
A partir de 1988, Rabin passou a ter como objetivo reativar a ideia de
autonomia palestina levantada por Begin uma década antes, nas
negociações de Camp David.

\section{O cancelamento do Projeto Lavi}

Uma das principais questões --- e controvérsias --- do mandato de Rabin
foi o Projeto Lavi. Tratava"-se de um esforço para produzir um avançado
avião de caça em Israel, ideia que surgira no final da década de 1970.
Moshe Arens, político do Likud e engenheiro aeronáutico por profissão,
era o principal defensor da ideia. Em 1978, ele liderou o Comitê de
Assuntos Estrangeiros e de Defesa do Knesset, para recomendar o projeto.
Em 1980, Ezer Weizman, ex"-comandante da força aérea israelense e
ministro da Defesa na época, tomou a decisão de seguir em frente. O
projeto sofreu grandes mudanças nos anos seguintes, tornando"-se mais
ambicioso e dispendioso. Quando Arens assumiu a pasta da Defesa em 1983,
estava numa boa posição para promover o projeto. Conseguiu garantir o
apoio americano, e o governo Reagan concordou em fornecer tecnologia
avançada e financiar a pesquisa e o desenvolvimento.

Contudo, em 1984 o cenário mudou. Rabin se tornou ministro da Defesa no
governo de unidade nacional. Desde os anos 1950, como já vimos, vinha
demonstrando sua preferência por armamentos sofisticados adquiridos no
exterior (de preferência nos Estados Unidos), em detrimento de projetos
locais, produzidos em Israel. Sua predileção acabou reforçada pela
realidade econômica, como parte do esforço hercúleo do governo para
combater a crise nesse setor. Rabin cortou o orçamento da Defesa e não
via motivos para arcar com os custos de um projeto cada vez mais
oneroso. Sua decisão não foi unilateral: parte da liderança da força
aérea também argumentava que a qualidade dos aviões americanos \textsc{f}"-15 e
\textsc{f}"-16 era superior, e eles não queriam sacrificar a qualidade em prol do
desenvolvimento tecnológico de Israel. Outros oficiais de alta patente
das \textsc{idf} temiam que os investimentos no projeto levassem a futuros cortes
no orçamento da Defesa. O mais importante, no fim das contas, foi que o
Pentágono acabou se declarando contrário ao projeto: não fazia sentido
os Estados Unidos ajudarem a financiar um avião de caça que
possivelmente competiria com produtos norte"-americanos no mercado
internacional de Defesa.

Em dezembro de 1986, o primeiro protótipo do avião Lavi fez um voo de
teste, mas o debate e a controvérsia só aumentaram. Os defensores do
projeto eram influentes e reuniram argumentos poderosos: o programa
empregava milhares de engenheiros e técnicos; se fosse encerrado,
haveria demissões em massa e, por fim, a migração de mão de obra de
primeira linha e de \textit{know"-how}. Além daqueles diretamente envolvidos no
projeto, a Israel Aircraft Industries empregava uma força de trabalho
expressiva, muito bem organizada em termos de \textit{lobby} político. Arens e
outros defensores do novo avião sustentavam que estava em jogo o futuro
de Israel como potência científica e tecnológica.

Conforme subia a pressão do Pentágono, Rabin tomou sua decisão final, e
em agosto de 1987 convenceu o gabinete a votar com ele pelo encerramento do
projeto. Ironicamente, Peres, padrinho das indústrias de defesa
israelenses, votou com Rabin. No entanto, a controvérsia prosseguiu. Os
defensores do programa continuavam argumentando que encerrá"-lo era um
erro terrível, que a tecnologia israelense tinha sofrido um golpe mortal
e que o \textit{know"-how} produzido em Israel havia se espalhado por outros
países. Com o tempo, porém, ficou claro que os milhares de engenheiros
dispensados no fim da década de 1980 acabaram aplicando seus
conhecimentos em outros setores dentro de Israel, contribuindo para o
subsequente florescimento da indústria de alta tecnologia local. Em
suma, Rabin demonstrou sua capacidade de tomar uma decisão incisiva e
pôr fim a uma controvérsia que afligia o \textit{establishment} de defesa
israelense havia muitos anos.

\section[O esforço para restabelecer o plano de autonomia]{O esforço para restabelecer o plano de\break autonomia e o fim do governo de unidade}

No final da década de 1980, a noção de unidade começava a se
desgastar. Para surpresa de muitos que não acreditavam que ele de fato
entregaria o poder a Shamir, ou que o aconselhavam energicamente a não
fazê"-lo, Peres cumpriu seu compromisso em outubro de 1986. Talvez tenha
agido assim apenas porque era a coisa certa a se fazer, ou porque
esperava garantir de uma vez por todas sua credibilidade perante o
povo israelense. Os papéis foram invertidos, e ele se tornou ministro
das Relações Exteriores. Considerando que as duas principais missões
iniciais do governo de unidade já tinham sido cumpridas --- livrar Israel
de grande parte de seu envolvimento no Líbano e estabilizar a economia
---, Peres dedicou sua atenção e energia para trazer de volta à pauta o
processo de paz entre árabes e israelenses.

Foi notável a transformação pela qual ele passou. Em meados da década de
1970, durante o governo Rabin, era o ministro da Defesa radical que
defendeu a criação de assentamentos na Samaria. Nesse momento
posterior, contudo, tornou"-se um defensor da paz, disposto a apoiar as concessões
exigidas pelo processo de pacificação. Essa postura o pôs em rota de
colisão com Shamir, o novo primeiro"-ministro, que era comprometido com o
\textit{status quo} territorial e contrário a qualquer concessão. Rabin, por uma
questão de preferência pessoal e graças ao novo cenário político, passou
a ocupar uma posição intermediária, ao centro.

O principal esforço de Peres voltado à paz foi investido em Hussein. Nos
primeiros meses de 1987, não obteve muito sucesso. O rei não demonstrava
entusiasmo pela ideia e insistia que qualquer negociação entre Jordânia
e Israel deveria ser conduzida apenas no âmbito de uma conferência
internacional. Como sua reivindicação quanto à Cisjordânia tinha sido
revogada pela cúpula árabe em 1974, isso lhe daria cobertura contra a
acusação de que estaria conduzindo uma ilegítima negociação à parte com
Israel. Da perspectiva israelense, tratava"-se de um enorme obstáculo.
Uma conferência internacional implicava a participação soviética e
síria, uma radicalização dos posicionamentos árabes pela dinâmica do
coletivo árabe e, claro, proeminência da questão palestina e de sua
representação. Em abril de 1987, Peres conseguiu avançar um pouco: numa
reunião em Londres, o rei concordou em definir a conferência
internacional de modo que, na prática, ela serviria de tênue e inofensiva
cobertura para uma negociação direta entre Israel e Jordânia. A
conferência internacional seria convocada, nos moldes da Conferência de
Genebra de 1973, mas as negociações verdadeiras seriam conduzidas entre
Israel e uma delegação jordaniano"-palestina. Assim, foi assinado um
breve documento que ficou conhecido como Acordo de Londres.

O documento representava um avanço importante, mas havia um problema
crucial: todo esse movimento tinha sido conduzido pelo ministro das
Relações Exteriores, sem que consultasse ou informasse o primeiro"-ministro.
Quando Peres voltou a Israel trazendo a notícia, Shamir, que nem sequer
recebeu uma cópia do documento, ficou furioso. Era contrário ao teor do
Acordo de Londres, bem como à forma como fora conduzido. O rei Hussein
insistia que o apoio dos Estados Unidos era uma condição essencial. A
fim de manifestar sua oposição, Shamir enviou Arens, seu homem de
confiança, aos Estados Unidos, para se encontrar com George Shultz,
secretário de Estado norte"-americano. Shultz queria ver progressos no
processo de paz, mas diante do veto do primeiro"-ministro, recusou"-se a
seguir em frente com o ministro das Relações Exteriores. O Acordo de
Londres acabou engavetado, e a relação entre Peres e Shamir sofreu um
dano irreparável. Rabin, por sua vez, ofereceu apoio sutil a Peres, mas
não compartilhou de seu entusiasmo inicial nem da subsequente frustração
e raiva. Rabin defendia que a questão palestina deveria ser resolvida
por meio de um acordo com a Jordânia, mas criticava o fato de que Peres,
como ministro das Relações Exteriores, confrontara o primeiro"-ministro
com um fato consumado; fora isso, também tinha dúvidas se o Acordo de
Londres era viável.

Com a eclosão da Primeira Intifada, em dezembro de 1987, aumentou a
pressão sobre a retomada do processo de paz. Shultz, amigo de Israel,
lançou uma iniciativa em março de 1988 que apelava para negociações
entre o Estado israelense e uma delegação jordaniano"-palestina sobre um acordo de
autonomia, a ser seguido por negociações envolvendo uma solução
permanente. Talvez a iniciativa não fosse aceita pelos possíveis
participantes árabes e pelo consenso árabe, mas, de todo modo, foi
rejeitada por Shamir. No final de 1988, às vésperas da transição do
governo de Reagan para George H. W. Bush, Shultz começou um diálogo com
integrantes do baixo escalão da \textsc{olp}, em Túnis. A mensagem era clara: se
não conseguisse entabular uma negociação entre israelenses e palestinos,
pelo menos queria facilitar a tarefa para os seus sucessores. O novo
secretário de Estado seria James Baker; ele e o presidente Bush estavam
ávidos por retomar o processo de paz entre israelenses e árabes, mas
consultas realizadas pelo novo governo apontaram para enormes
dificuldades envolvendo qualquer esforço dessa natureza.

Uma das dificuldades era a transformação política ocorrida dentro de
Israel. Nas eleições de novembro de 1988, o Likud, liderado por Shamir,
conseguiu quarenta assentos, enquanto o Partido Trabalhista, sob o comando de
Peres, só chegou a 39. Formou"-se um segundo governo de unidade, dessa
vez sem alternância. Rabin se manteve na pasta da Defesa, enquanto Peres
passou a ministro da Fazenda. Para o Ministério das Relações Exteriores,
o escolhido foi Arens. A paridade foi substituída por uma tênue
hegemonia do Likud. Contudo, as autoridades em Washington enxergaram
sinais positivos. Um deles foi uma tentativa preliminar do governo de
Shamir para iniciar novas negociações com os palestinos durante a
transição do governo Reagan para o governo Bush. A ideia de Shamir e
seus assessores era impedir uma guinada pró"-árabe de um governo que, na
perspectiva deles, era dominado por republicanos do Texas e pendia para
o lado dos árabes, em razão de interesses no petróleo. Em 1989, Shamir
apresentou a Baker um plano que o próprio Baker, em suas memórias,
chamou de ``obscuro plano de quatro pontos'', para que fossem realizadas
eleições na Cisjordânia e na Faixa de Gaza. Ainda em 1989, depois de
trabalhar com sua equipe para tentar dar mais substância às ideias de
Shamir e encontrar uma fórmula que pudesse ser aceita tanto por
israelenses quanto por árabes, Baker apresentou seu próprio ``plano de
cinco pontos''.

Havia dois assuntos espinhosos na negociação entre o governo Bush e o
governo israelense: a participação eleitoral de palestinos de fora
dos territórios ocupados e a questão dos residentes de Jerusalém
Oriental, vista por Israel como parte de seu território soberano. Do
lado israelense, o principal parceiro dos Estados Unidos era Rabin.
Depois de concluir que a Intifada não podia ser suprimida e teria de
ser resolvida por meio de negociações, ele estava empenhado em fazer as
conversas acontecerem. A cooperação entre Rabin e o governo Bush se dava
em sigilo. Ele às vezes ia até a residência do embaixador americano no
norte de Tel Aviv para falar com Dennis Ross, diretor de planejamento
de políticas de Baker, pela conexão criptografada do Departamento de
Estado. Por sua vez, o Departamento de Estado usava um codinome para
Rabin, a quem chamavam em seus telegramas e memorandos de ``o homem que
fuma''. Rabin chegou a uma solução para a questão da participação
palestina de fora dos territórios: permitir que um palestino deportado
voltasse à Cisjordânia e se tornasse um negociador ``externo'', o que de
fato acabou sendo aceito pelos palestinos.

Como já era típico na conturbada política desse suposto período de
unidade, o ministro Arens, das Relações Exteriores, só ficou a par da
situação quando foi a Washington para uma visita. William Brown,
embaixador americano em Israel naquela época, fez um nítido relato sobre
esse episódio:

\begin{quote}
Eu tinha uma linha de telefone criptografada em casa, que ficava dentro
de uma caixa"-forte. Convidava Rabin para tomar uns drinques, e ele
nunca recusava. Tomávamos uma taça disso ou daquilo, subíamos para essa
caixa"-forte, aí eu pegava o telefone especial e o desbloqueava. Depois
de ligar para Denis Ross, entregava o telefone a Rabin, e os dois
ficavam conversando. Eu nem me envolvia nessas conversas.
\end{quote}

Brown tinha de pisar em ovos ao lidar em separado com Shamir, Rabin,
Peres e Arens. Assim ele descreveu uma reunião que teve com Arens,
depois de outro telefonema difícil:

\begin{quote}
Quando eu disse boa noite a Arens, ele falou: ``Imagino que agora você
vai se encontrar com o Rabin'', ao que eu respondi: ``Verdade, Moshe''.
Então fui mesmo me encontrar com Rabin, segundo algumas instruções. Já
passava bastante da meia"-noite. Revi os pontos da conversa entre Baker e
Arens, conforme Dennis Ross me havia transmitido, bem como os pontos da
minha conversa com Moshe Arens. Em casa, Rabin ficou imóvel, absorvido
naquilo. Ao longo desse processo, algumas vezes senti como se estivesse
lidando com uns quatro governos israelenses diferentes {[}\ldots{}{]} Rabin deixou
muito claro para mim que estava reportando tudo ao primeiro"-ministro
Shamir. Ele disse assim: ``Quero que você e Washington saibam que não
estou jogando. Conto a Shamir tudo que está acontecendo''.\footnote{Association
  for Diplomatic Studies and Training (\textsc{nara}), entrevista com o embaixador William Brown.}
\end{quote}

Por fim, Rabin e Peres descobriram que Shamir e seu partido não estavam
dispostos a avançar. E mesmo quando Shamir queria demonstrar alguma
flexibilidade, a oposição linha"-dura dentro do Likud conseguia
bloqueá"-lo.

Num discurso proferido por Rabin em 26 de março de 1989, fica clara sua
visão de que, para enfrentar a questão palestina, era preciso implementar
o plano de autonomia proposto nos acordos de Camp David, assinados por
Begin. Ele também acreditava que os interlocutores de Israel deveriam
ser líderes locais da Cisjordânia e da Faixa de Gaza, e não a \textsc{olp}, e que
o Partido Trabalhista deveria defender a implementação dessa política,
apesar da oposição de seus parceiros do Likud no governo de união
nacional:

\begin{quote}
Nós e o Likud somos substancialmente diferentes em um aspecto: o
objetivo final, a solução definitiva, entre a ideia da Grande Israel ou
um Estado judeu democrático, sem o controle sobre um milhão e meio de
palestinos {[}\ldots{}{]} O que dizem os princípios básicos, com os quais concordo
totalmente? Oposição a um Estado palestino entre Israel e Jordânia.
Portanto, nada de negociações com a \textsc{olp} {[}\ldots{}{]} Se as eleições acontecerem
daqui a seis meses, não tenho certeza de que nos sairemos melhor do que
nas eleições anteriores {[}\ldots{}{]} Assim, me parece que deveríamos vir com uma
proposta, que já apresentei {[}\ldots{}{]} Acredito também que os Estados Unidos
possam concordar com ela e que possamos coordená"-la juntos {[}\ldots{}{]} Essa
política deveria se pautar nas seguintes fases: retomada da
tranquilidade, eleições, negociação de um período transicional de cinco
anos e, em no máximo três anos, inciar uma negociação para um acordo
definitivo {[}\ldots{}{]} Esse passo deve se basear nos acordos de Camp David
{[}\ldots{}{]} A
segunda questão é continuar com a política de uso da força contra
manifestações violentas {[}\ldots{}{]} E a terceira, a coordenação com os Estados
Unidos {[}\ldots{}{]} Mas e se o Likud não aceitar? {[}\ldots{}{]} Bom, é claro que podemos
deixar o governo. E o que acontecerá? Será formado um governo de
extrema"-direita, do Likud com os ortodoxos. E depois de três anos? O que
ocorreu nos assentamentos nos últimos anos se multiplicará.\footnote{Arquivos do
  Yad Tabenkin.}
\end{quote}

Com o passar do tempo, o Partido Trabalhista e Peres, mais do que Rabin,
foram ficando cada vez mais impacientes. A relutância de Shamir em
responder aos cinco pontos de Baker era apenas uma das manifestações da
guinada à direita do governo. No início de 1990, momento em que a
controvérsia sobre o processo de paz e a relação com Washington chegava
no limite, Peres estava certo de que poderia convencer os partidos
ultraortodoxos e talvez outros membros do Knesset a apoiá"-lo, a fim de
formar uma coalizão alternativa. Em março de 1990, a maioria de Shamir
rejeitou imediatamente a última formulação de acordo proposta por Baker,
e Peres passou a achar que tinha garantido o apoio parlamentar
necessário para formar seu próprio governo. Quando soube da movimentação
de Peres, Shamir o destituiu. Então os ministros do Partido Trabalhista
renunciaram em conjunto. Rabin, por sua vez, tinha dúvidas quanto à
conduta de Peres. Foi um dos momentos que revelaram as diferenças entre
os dois. Peres era corajoso, ousado até, enquanto os pés de Rabin
mantinham"-se fincados no chão. Além disso, Rabin se sentia confortável
com Shamir, às vezes até mais do que com Peres. Tentou, sem sucesso,
fazer uma mediação entre ambos. Porém, por mais insatisfeito
que estivesse com os rumos da situação, acabou se juntando a seus
colegas trabalhistas, em apoio a Peres. Este se envolveu, por fim, num
esforço hercúleo --- que se provou infrutífero --- para levar à sua
coalizão os ultraortodoxos e uma pequena facção de possíveis dissidentes
do Likud. Com a certeza de que os esforços de Peres seriam em vão, Rabin
se manteve a seu lado, sem entusiasmo, até o colapso final, em abril de
1990, o que ele mais tarde descreveria como ``a tentativa fétida'' ---
mais uma contribuição sua para o léxico político israelense.

Sob dois aspectos, os eventos de março de 1990 provaram ser um momento
de virada. Shamir se tornou o primeiro"-ministro de um governo
exclusivamente de direta. Também foi cercado, à direita, por Sharon e
outros líderes ultrarradicais do Likud. Não havia qualquer perspectiva
quanto a negociações de paz, e os assentamentos na Cisjordânia vinham se
multiplicando rapidamente. A tensão com o governo Bush se transformou em
ruptura escancarada, que só ficou temporariamente em segundo plano
quando surgiu a necessidade de unir forças após a invasão do Kuwait pelo
presidente iraquiano Saddam Hussein e durante a Primeira Guerra do
Golfo. Depois, a ruptura acabou vindo à tona de novo, no momento em que
Israel pediu ao governo Bush garantias de empréstimo para ajudar o país
a absorver a maciça onda de imigração oriunda da antiga União Soviética.
O governo americano rejeitou o pedido, sem poupar esforços para
demonstrar sua insatisfação com o governo de Shamir e suas políticas. O
secretário de Estado, Baker, chegou a anunciar na televisão o número de
telefone da Casa Branca, de modo que Israel soubesse para onde ligar
quando ``passasse a levar a sério a questão da paz''. Também impediu
Benjamin Netanyahu --- na época, vice"-ministro das Relações Exteriores ---
de entrar no prédio do Departamento de Estado depois de ele ter
criticado publicamente o governo dos Estados Unidos. Esse
desentendimento envenenava a política: o público israelense ainda
julgava o desempenho de seus líderes pela capacidade de se relacionarem
com o presidente dos Estados Unidos. De forma deliberada e efetiva, Bush
e Baker estavam minando a reputação de Shamir dentro de Israel.

Enquanto isso, a rivalidade entre Rabin e Peres voltava à cena. Ao longo
dos nove anos anteriores, Rabin tinha aceitado a senioridade de Peres, e
a relação entre os dois se mostrou, de certa forma, razoável. Desde
1980, Rabin mantinha seu próprio grupo dentro do Partido Trabalhista, o
que continuou se ampliando na década seguinte. Ele estava satisfeito com
o cargo de ministro da Defesa. Contudo, a crise de março e abril de 1990
fez com que perdesse o posto, e ele atribuiu a Peres a culpa pelo
conflito com Shamir, desnecessário e fadado ao fracasso. Segundo o já
falecido Gad Yaacobi, ministro trabalhista no governo de unidade, Rabin
teria dito a Peres durante uma reunião no dia 3 de maio: ``Estou cansado
das suas artimanhas. A verdade precisa ser dita. Eu sabia que aquilo
tudo {[}nos{]} levaria à oposição, e deixei isso bem claro. Tentei nos
poupar desse constrangimento, anulando a demissão, mas você
vetou''.\footnote{Gad Yaacobi, \textit{Dádiva do tempo}. Tel Aviv: \textit{Yedi'ot Achronot}, 1991, p. 356 {[}em
  hebraico{]}.} Rabin estava disposto, então, a enfrentar Peres. Mais
tarde, naquele mesmo dia, anunciou numa entrevista de televisão: ``Vai
haver mais de um candidato a primeiro"-ministro, e eu serei um
deles''.\footnote{Ibid.}

O fracasso da candidatura de Peres em março"-abril de 1990 marcou o fim
de sua liderança dentro do Partido Trabalhista. Depois de ter sido
derrotado em quatro eleições desde 1977 e ter exposto o partido aos
riscos de uma aposta política malconcebida, ele passou a ser visto por
muitos ativistas e membros do partido como incapaz de levá"-los de volta
ao poder. O apoio estava migrando para Rabin, que parecia ter chances de
vencer. Por seis anos, havia sido um ministro da Defesa confiável e
popular, e estava claro que só um candidato de centro, com fortes
credenciais de segurança, conseguiria vencer a resistência do público
israelense em aceitar a chancela trabalhista. Rabin estava pronto para
voltar ao ringue.

Ainda assim, levaria mais quase dois anos até que fosse indicado pelo
partido, nas eleições parlamentares de 1992. Ele cometeu o erro de
enfrentar Peres no diretório, em julho de 1990, num gesto prematuro. As
bases do partido estavam migrando o apoio na direção de Rabin, mas Peres
ainda controlava a máquina partidária. Ele venceu com 54\% dos votos,
contra 46\%, permanecendo como presidente do partido. Contudo, no
decorrer de 1991, a maré começou a virar. Rabin, em sua nova encarnação
como Sr. Segurança, popular graças a sua confiabilidade e credibilidade,
era o único candidato capaz de angariar um número suficiente de votos do
centro e da direita moderada, para levar os trabalhistas de volta ao
poder.

Embora tenha sido fundamental, a virada em si não garantiu Rabin como
candidato do Partido Trabalhista nas eleições de 1992. Porém, houve um
desdobramento crucial: o partido aprovou a resolução de eleger um
candidato por meio de primárias, em vez de fazê"-lo através do diretório.
Assim, Rabin conseguiu driblar o controle de Peres sobre a máquina
partidária e capitalizar o apoio das bases. A candidatura de outros dois
nomes também beneficiou Rabin: Israel Keisar recebeu cerca de 20\% dos
votos, muitos dos quais teriam ido para Peres. Com isso, Rabin conseguiu
pouco mais do que os 40\% necessários para garantir sua eleição na
primeira rodada. Portanto, em fevereiro de 1992, tornou"-se o candidato
do Partido Trabalhista ao cargo de primeiro"-ministro.

\section{As eleições de 1992}

O Partido Trabalhista adotou uma estratégia dupla nas eleições de 1992.
Em primeiro lugar, a campanha foi conduzida quase como uma corrida
pessoal de Rabin ao cargo de primeiro"-ministro. Essa tática explorava
sua popularidade, deixando em segundo plano o papel do próprio partido.
Naquele momento, já estava bem claro que grande parte do eleitorado
israelense relutava em votar no Partido Trabalhista; assim, tudo
indicava que seria mais fácil votarem no ``Trabalhista com Rabin''.
Enfatizou"-se o perfil que primava pela segurança da liderança do partido, e
durante a campanha os membros da ala mais à esquerda foram postos de
lado. A plataforma de Rabin combinava suas credenciais de segurança com
o objetivo de romper o impasse político"-diplomático e renovar um
movimento deliberado em direção à paz. O segundo tema relevante da
campanha foi a necessidade de ajustar a relação com os Estados Unidos.
Como se saíra muito bem em seu cargo de embaixador em Washington, Rabin
era mais do que qualificado para cumprir essa missão.

Apesar de todas essas vantagens, ele quase perdeu a disputa. Sua lista
conquistou 44 assentos; o Meretz, de esquerda, chegou a doze; e os
partidos árabes conseguiram cinco. Isso deu a Rabin o número necessário para
impedir que Shamir formasse um governo. Como os partidos
árabe"-israelenses não eram (e ainda não são) candidatos a compor uma
coalizão, Rabin obteve, naquele momento, uma coalizão de 56, mas o apoio
dos cinco membros árabes lhe garantiu a maioria de 61. Foi um marco
importante, pois permitiu que o Shas, partido ultraortodoxo,
participasse da coalizão. O Shas tinha sido fundado em 1982, como
partido sefaradi ultraortodoxo, e em 1984 eles concorreram pela primeira
vez ao Knesset. Para todos os efeitos, era um partido de direita, embora
seu líder espiritual, o rabino Ovadia Yosef, e seu líder político, Aryeh
Deri, na verdade apoiassem a continuidade do processo de paz. O Shas só
passou a compor a coalizão quando ela obteve o número necessário de
votos. Ao fazer isso, ampliou a base da aliança, mas sua participação
era tênue e nada garantida ao longo do tempo. Assim, Rabin começou seu
mandato determinado a fazer escolhas cruciais, apoiadas em bases
políticas limitadas.

Nesse segundo mandato, sua relação com Peres mudou para melhor, embora
persistissem a rivalidade e a antipatia. No dia da eleição, 23 de junho
de 1992, Rabin aguardava o resultado em seu quarto de hotel. Os
assessores o convenceram a evitar a todo custo o constrangimento de
declarar vitória antes da hora. Pela televisão, ele via Peres celebrando no
andar de baixo, com um grupo de ativistas do partido. Rabin manifestou
seu famoso temperamento assim que desceu. Nervoso, com o rosto vermelho,
declarou: ``Eu que vou pilotar''. Rabin precisou de um tempo para
digerir a ideia de que teria de oferecer a Peres um ministério
importante, bem como a posição de número dois \textit{de facto}. Porém,
seu segundo mandato começou guiado por algumas lições aprendidas com os
problemas do primeiro; uma delas era a necessidade de evitar que se
repetisse a rixa destrutiva com Peres. Se a presença de Peres era um
fato incontestável, eles precisariam estabelecer uma relação
profissional. Rabin manteve para si a pasta da Defesa, e Peres foi
nomeado ministro das Relações Exteriores. Na divisão de trabalho entre
os dois, coube ao primeiro"-ministro cuidar das negociações bilaterais
envolvidas no processo de paz e da relação com os Estados Unidos. Peres
ficou com a responsabilidade menos importante do diálogo multilateral.
Contudo, essa divisão mudou em poucos meses, quando Peres e seus
homens criaram o canal de Oslo, ganhando centralidade no processo de
paz.

\chapter[A política de paz de Rabin, 1992--95]{A política de paz de Rabin, 1992--95}
\markboth{A política de paz de Rabin}{}

No dia 26 de julho de 1994, o primeiro"-ministro Rabin e o rei Hussein
discursaram numa sessão conjunta do Congresso americano, e no dia 30 de
julho assinaram um acordo em Washington, encerrando o estado de guerra
entre Jordânia e Israel. Foi uma etapa intermediária, que abriu caminho
para um tratado de paz abrangente, assinado em 26 de outubro de 1994. A
Jordânia se tornou, então, o segundo Estado árabe depois do Egito a
assinar um tratado de paz total com Israel. No discurso ao Congresso,
Rabin disse: ``Eu, Yitzhak Rabin, \textsc{id} militar número 30.743,
tenente"-general aposentado, soldado das \textsc{idf} e do exército da paz, eu,
que já mandei tropas para o meio de disparos e soldados para a morte,
digo a você, rei da Jordânia, e a vocês, meus amigos americanos: hoje
estamos lançando uma guerra que não tem mortos nem feridos, não tem
sangue nem sofrimento, a guerra pela paz''.

No passado, Rabin não tinha se destacado como grande orador, mas em seu
segundo mandato, a colaboração com Eitan Haber, seu chefe de gabinete e
redator de discursos, levou a vários pronunciamentos memoráveis. Esse foi um
deles. Apesar da hipérbole retórica com a qual Rabin se sentia
visivelmente desconfortável, sua declaração refletia duas verdades
profundas. Para ele, fazer a paz naquele momento --- depois de ter
alcançado o auge de sua carreira e quando se aproximava do fim da vida
--- era uma extensão natural de seu início profissional como soldado.
Houve a época da guerra, e por fim havia chegado a época de fazer a paz.
Para o público israelense, era crucial que o processo de paz dos anos
1990 fosse conduzido por um ex"-chefe do Estado"-Maior das \textsc{idf} e
ex"-ministro da Defesa. Os israelenses estavam ansiosos e hesitantes
quanto ao processo de paz que começara em 1993 e só se mostravam
dispostos a seguir em frente nessa direção se fosse sob a liderança de
um primeiro"-ministro em quem confiassem profundamente e que lhes
reforçasse o sentimento de segurança enquanto eram feitas as concessões
necessárias.

Naquele momento, não havia nada controverso no que dizia respeito a
fazer a paz com a Jordânia. Israel e Jordânia tinham lutado no passado,
às vezes de forma brutal, mas o rei Hussein e a Jordânia hachemita eram
vistos na década de 1990 como potenciais aliados e parceiros da paz.
Israel enxergava a Jordânia como parte amigável do triângulo
Israel"-Jordânia"-palestinos, bem como protetor confiável de sua fronteira
terrestre mais extensa. Além disso, as concessões feitas para persuadir
a Jordânia a assinar o tratado de paz eram pequenas e indolores: Israel
devolveu à Jordânia um território que havia ocupado no sul, mas
conseguiu, em troca, que pudesse arrendar a parte cultivada. Porém, para
conseguir esse marco histórico de paz com a Jordânia --- transformando
mais de vinte anos de contatos clandestinos com o rei Hussein em
negociações de paz profícuas e oficiais ---, os Acordos de Oslo, muito
mais controversos, tiveram de ser assinados um ano antes. Depois que a
liderança do movimento nacional palestino assinou seu acordo com Israel
e as duas partes passaram por um processo de reconhecimento mútuo, o rei
da Jordânia obteve, por fim, legitimidade e senso de urgência para fazer a
paz com Israel.

Os Acordos de Oslo e o processo de Oslo foram os eventos mais
importantes do segundo mandato de Rabin como primeiro"-ministro. Ele
pagou por isso com a própria vida, mas passou a ser reconhecido como
estadista internacional, como líder capaz de tomar decisões corajosas,
indo contra sua própria essência e conseguindo apoio da opinião pública.

\section{O caminho para Oslo}

Quando Rabin deu início a seu governo, em 13 de julho de 1992, não
era muito nítido até que ponto estava disposto a ir na questão do processo
de paz. No discurso que fez nesse mesmo dia no Knesset, ele falou sobre
modificar a agenda nacional israelense. Estava determinado a encerrar o
que via como a hipoteca dos recursos e do futuro de Israel no projeto de
assentamento da Cisjordânia e de Gaza. Para ele, a campanha de
assentamento vinha drenando os recursos econômicos do país e
enfraquecendo seu posicionamento internacional. Sua ideia era destinar
os recursos antes comprometidos pelos governos do Likud com esse projeto
para novos investimentos: em infraestrutura, preparando Israel para o
século \textsc{xxi}; na absorção das ondas massivas de imigração oriunda da
antiga União Soviética; e na reparação da relação de Israel com os
Estados Unidos. Durante a campanha eleitoral, ele também prometera
concluir o acordo de autonomia com os palestinos num prazo de nove
meses. Em seu discurso, Rabin perguntou: ``A prioridade nacional deveria
ser investir pesado nos territórios ou na luta contra o desemprego? Em
educação? {[}\ldots{}{]} Na realidade atual, só há duas opções: ou nos esforçaremos
com seriedade para fazer a paz com segurança {[}\ldots{}{]} ou viveremos o resto da
vida em guerra''.

Rabin tinha suas reservas sobre o processo de Madri que herdara, lançado
pela administração Bush"-Baker em outubro de 1991, logo após a Primeira
Guerra do Golfo. Os Estados Unidos viviam então o auge de sua influência
sobre o Oriente Médio, depois de derrotarem Saddam Hussein, libertarem o
Kuwait e salvarem a Arábia Saudita de uma invasão iraquiana. A União
Soviética tinha acabado de entrar em colapso, e os Estados Unidos eram a
única superpotência mundial. Bush e Baker decidiram converter esses
ativos num novo esforço para resolver o conflito árabe"-israelense numa
conferência de paz em Madri e no subsequente processo de Madri. Foram
estabelecidos três canais de negociações diretas: entre Israel e a
Síria, Israel e o Líbano, e Israel e uma delegação conjunta
jordaniano"-palestina. A \textsc{olp}, enfraquecida pela aposta equivocada em
Saddam Hussein, teve de assumir uma posição subalterna no âmbito da
delegação jordaniano"-palestina. O componente palestino das delegações
era formado por residentes da Cisjordânia e da Faixa de Gaza e não
incluía ativistas da \textsc{olp}. Além dos canais diretos de negociação, foram
criados cinco grupos de trabalho para lidar com questões como segurança
regional, meio"-ambiente e refugiados.

As reservas de Rabin pautavam"-se no fato de que ele achava errado
fundamentar um processo de paz árabe"-israelense em uma negociação entre
Israel e o coletivo árabe. Como já vimos, desde sua primeira experiência
com a diplomacia entre Israel e os árabes em 1949 e em ocasiões
subsequentes, ele tinha entendido que qualquer conferência de paz que
contasse com a presença do coletivo árabe tendia a se radicalizar sob a
influência do participante árabe mais extremista. Para ele, seria muito
melhor que Israel negociasse em separado com cada Estado árabe,
individualmente. Mas não valia a pena tentar atrapalhar ou alterar o
marco de Madri logo de início, então Rabin se resignou a começar seu
processo de paz sob esse escopo.

Após a Conferência de Madri, as delegações do governo de Shamir se
reuniram algumas vezes, no final de 1991 e início de 1992, com seus
interlocutores árabes em Washington. Não houve progressos em nenhuma das
frentes, e as negociações foram suspensas em antecipação às eleições
israelenses de 1992. Rabin manteve os líderes originais de duas
delegações --- Elyakim Rubinstein como principal negociador com a
Jordânia e os palestinos, e Uri Lubrani como principal negociador com o
Líbano. Em substituição a Yossi Ben"-Aharon, diretor"-geral do gabinete de
Shamir que tinha sido o principal negociador com a Síria, Rabin indicou
o meu nome para o cargo. Eu era historiador especialista na Síria
moderna na Universidade de Tel Aviv, ex"-chefe do principal instituto de
pesquisa do país sobre questões de Oriente Médio e reitor da
Universidade de Tel Aviv. Em termos sociais e profissionais, minha
relação com Rabin era amigável, porém superficial. A partir daquele
trabalho conjunto, rapidamente desenvolveríamos uma relação próxima e
profunda. Quando era nomeado para uma nova função, Rabin costumava levar
consigo vários nomes de seu pessoal e tendia a trabalhar com a equipe
antiga, a menos que houvesse um motivo especial para introduzir alguma
mudança. Nesse caso em particular, ele achava que Ben"-Aharon,
colaborador próximo do primeiro"-ministro Shamir e notório radical, não
poderia continuar em seu papel de principal negociador com a Síria.
Rabin queria transmitir a mensagem de que levava as negociações a sério.

Desde o início, ficou claro que ele pretendia fazer progressos
significativos no processo de paz. Sentia que não ganhara uma segunda
chance, algo tão raro, para simplesmente passar mais alguns anos sentado
na cadeira de primeiro"-ministro; a seu ver, transformar a relação de
Israel com o entorno árabe seria a forma mais efetiva de deixar sua
marca. Rabin acreditava que as principais ameaças à segurança nacional
de Israel vinham do flanco oriental do Oriente Médio, do Irã e do
Iraque, e que transformar a relação de Israel com seus vizinhos
imediatos era uma boa oportunidade para lidar com desafios maiores
vindos do Leste.

Na conversa que tivemos em sua casa, por ocasião da minha nomeação, em
julho de 1993, Rabin me sinalizou estar disposto a fazer
grandes concessões nas negociações com a Síria. Contou que durante a
última visita de Baker à região como secretário de Estado, Hafez al"-Assad
tinha lhe dito que assinaria um acordo de paz com Israel, nos
moldes do acordo assinado em 1979 por Sadat, e que o governo Bush estava
pronto para subscrever a paz entre israelenses e sírios. Rabin disse que
minha missão era descobrir ao longo das negociações se isso de fato era
viável. Foi um momento interessante e significativo. Durante sua
campanha em 1992, ele declarara que Israel não deveria abrir mão
das Colinas do Golã. Porém, um acordo de paz entre Israel e Síria nos
moldes do acordo com o Egito implicaria uma retirada total em troca de
paz total. Na relação entre um líder e seu negociador, nem tudo é
esclarecido; algumas coisas ficam subentendidas. Ao me dizer que minha
principal missão seria descobrir se o acordo preconizado por Baker era
viável, Rabin na verdade demonstrava sua disposição de ir até o fim se
ficasse claro que haveria um parceiro sírio na jornada.

Naquele momento, Rabin ainda não tinha atribuído prioridade nem à frente
síria nem à frente palestina do processo de paz. Porém, havia uma nítida
vantagem da frente síria, pois o conflito entre Israel e Síria ---
essencialmente um conflito territorial acerca das Colinas do Golã --- era
mais simples do que o conflito entre Israel e os palestinos, que
envolvia um conflito nacional. A Síria era um país sob um regime
poderoso e autoritário, comandada por um presidente com reputação e
histórico de negociador difícil, cumpridor de seus acordos. O povo
palestino era um ator não estatal, sob o comando de um líder --- Yasser
Arafat --- cuja personalidade e credibilidade ainda precisavam ser
decifradas pela liderança israelense. O Líbano era um Estado cliente da
Síria, e o rei Hussein só faria qualquer movimento se suas ações
pudessem ser legitimadas por um acordo anterior com a Síria ou com os
palestinos. Nessa fase inicial, no verão de 1992, Rabin não fez uma
escolha entre a frente síria ou a frente palestina. Para que conseguisse
determinar as perspectivas de cada uma das frentes, primeiro suas
delegações teriam que retomar as negociações.

Na divisão de trabalho entre Rabin e Peres, seu ministro das Relações
Exteriores, Rabin assumiu a responsabilidade pelas negociações diretas e
atribuiu os grupos de trabalho multilaterais a seu parceiro e rival.
Ele acompanhava de perto as negociações bilaterais. Queria ser
consultado em assuntos sensíveis como território e segurança, mas não
ficava supervisionando cada detalhe. Por muitos meses, as conversas com
Washington progrediram devagar. A delegação palestina nitidamente
recebia instruções de Arafat e da \textsc{olp} em Túnis. A suposta autenticidade
dos palestinos era conveniente, como se Israel não estivesse negociando
com a \textsc{olp}, mas mesmo assim as negociações não deram em nada. Houve certo
progresso nas negociações com a Síria. Algumas declarações de Rabin e o
discurso inicial que eu fiz indicavam que Israel estava disposto a fazer
concessões não especificadas a fim de alcançar a paz com a Síria, com
frases como ``a abrangência da retirada refletirá a abrangência da
paz''. A atmosfera das negociações passou da franca hostilidade da época
de Shamir para uma conduta pragmática. A delegação síria botou sobre a
mesa um rascunho a respeito da paz. Era um documento curioso, pois
falava da paz mas não mencionava sequer uma vez a palavra ``Israel''.
Formulado em termos bastante vagos, ainda assim representava algum
progresso, uma vez que o termo ``paz'' era introduzido ao vocabulário
oficial sírio. Esse progresso, porém, estava longe de representar um
grande avanço. O principal obstáculo era que a Síria exigia o
compromisso de Israel com a retirada completa do Golã antes mesmo de as
duas partes começarem qualquer discussão concreta e detalhada sobre a
paz que a Síria pretendia oferecer em troca dessa retirada.

As conversas de Washington foram interrompidas em dezembro de 1992, como
indício inicial do severo impacto perturbador que o terrorismo teria
sobre a política de paz de Rabin. Um agente policial da fronteira
israelense foi sequestrado e morto pelo Hamas. O Hamas (acrônimo para
Movimento de Resistência Islâmica), braço palestino da Irmandade
Muçulmana egípcia, tornou"-se ativo na política palestina em dezembro de
1987, na época da eclosão da Primeira Intifada, e se apresentava como
alternativa islâmica à secular \textsc{olp}. Em 1989, passou a financiar inúmeras
atividades terroristas. Rabin queria adotar uma atitude radical para pôr
fim a essas atividades: expulsou o grosso dos integrantes do Hamas,
cerca de quatrocentos ativistas, para o sul do Líbano. No fim das
contas, a decisão acabou se provando equivocada e custosa. Em fevereiro
de 1993, Israel teve que admitir de volta os deportados, como condição
para que as conversas de paz de Washington fossem retomadas. Os
ativistas do Hamas voltaram para a Cisjordânia e a Faixa de Gaza, e o
período que passaram juntos no sul do Líbano acabou contribuindo de
forma decisiva para a consolidação da organização.

Em março de 1993, Rabin foi a Washington para sua primeira reunião de
trabalho com o presidente Bill Clinton. Não era o primeiro encontro dos
dois. Rabin já tinha se reunido com Clinton e seu vice, Al Gore, em
agosto de 1992, durante a campanha eleitoral para a presidência, mas
tinha sido uma reunião formal, não muito cordial. Uma relação calorosa e
amigável começou a se desenvolver entre Rabin e Clinton a partir do
segundo encontro. Rabin ficou impressionado pela rapidez com que Clinton
compreendia os assuntos e por sua intuição política, pelo carisma e pela
generosidade que ele exalava e por sua nítida preocupação com Israel.
Clinton, por sua vez, via Rabin como estadista experiente, maduro,
confiante e confiável, um soldado que agora desejava fazer a paz, um
líder disposto a subverter as convenções; admirava sua franqueza e
objetividade. Rabin tinha sofrido claramente uma transformação nesse
papel, e sua franqueza, que antes incomodava muita gente, agora se
tornava uma característica cativante. É difícil considerar os
desdobramentos dos três anos seguintes sem levar em conta a relação
pessoal entre esses dois homens. Cerca de dois anos depois dessa
reunião, em maio de 1995, quando os dois líderes apareceram juntos num
encontro da \textsc{aipac}, o \textit{lobby} israelense em Washington, Clinton assim
descreveu aquela primeira reunião: ``Quando nós nos encontramos pela
primeira vez, como já cansei de dizer, ele ficou me olhando, e eu fiquei
olhando para ele, e ele estava de certa forma estava medindo o meu
tamanho, mas eu já sabia que estava diante de um gigante''.\footnote{Citado
  em Itamar Rabinovich, \textit{The Brink of Peace: The Israeli"-Syrian
  Negotiations} {[}Às vésperas da paz: as negociações israelo"-sírias{]}. Princeton: Princeton University Press, 1998, p. 91--2.}

Rabin e o governo Clinton tinham visões parecidas sobre o processo de
paz. A visão do governo Clinton pautava"-se na ideia da ``dupla
contenção'', tanto do Irã como do Iraque, enquanto o núcleo do Oriente
Médio seria estabilizado pelo processo de paz entre árabes e
israelenses. Como Rabin estava disposto a fazer concessões
significativas a fim de alcançar o progresso que tinha em mente, a
relação entre ele e o governo Clinton era livre de tensões. Havia
discordâncias táticas, mas que acabavam minimizadas pela visão mais
ampla compartilhada por ambas as partes e pela relação calorosa entre os
dois líderes.

Durante a reunião de março, Clinton manifestou a nítida preferência de
seu governo pela frente síria. A equipe americana sabia que Rabin queria
lidar com uma frente de cada vez. Em termos políticos, ele conseguiria
enfrentar a oposição a um acordo com a Síria ou com os palestinos, mas
não com os dois ao mesmo tempo. Rabin, por sua vez, manifestou com
maestria sua disposição em fazer amplas concessões para alcançar a paz
com a Síria, mas se esquivou de um compromisso claro com a retirada
total. No início da reunião, Clinton apresentou de forma breve o
conceito de dupla contenção subjacente à política de seu governo para o
processo de paz e depois seguiu adiante, dizendo que, de acordo com seu
secretário de Estado, Warren Christopher, as perspectivas eram melhores
na frente síria. Rabin respondeu de forma bem extensa. As reuniões entre
Rabin e Clinton e a maioria das reuniões com
Christopher eram reuniões ``de um para um'', mas na prática isso
significava ``dois para dois'', porque sempre havia uma pessoa de cada
lado tomando notas. Nessas reuniões, tive a sorte de ser o tomador de
notas do lado israelense e testemunhei vários momentos importantes. De
acordo com o que anotei, aqui está a essência do que Rabin disse:

\begin{quote}
Sem sombra de dúvida, os dois pontos"-chave para avançarmos são a Síria e
os palestinos. Em termos estratégicos, a Síria é mais importante, mas em
termos da vida cotidiana do país, é de enorme importância um acordo com
os palestinos. Com eles, o problema é ao mesmo tempo político e
emocional. Já o problema com a Síria é o seguinte: existe um líder e
existe uma capacidade de tomar decisões, mas numa democracia como
Israel, o apoio público precisa ser mobilizado em prol de concessões
difíceis {[}\ldots{}{]} Há dois componentes aqui: (1) o que significa a paz? (2) a paz
{[}com a Síria{]} precisa ser autossuficiente, ou seja, não uma paz em
separado, mas uma paz que não dependa de outras frentes. Portanto, eu
não uso o termo ``retirada total''; não seria aconselhável que qualquer
{[}líder{]} israelense dissesse isso. É impossível concluir um acordo
com a Síria sem reuniões com os líderes. Isso não pode ficar a cargo de
pessoas que sejam, de certa forma, do baixo escalão {[}\ldots{}{]} Só teremos avanços reais
depois que eu e a opinião pública formos persuadidos a respeito de todos
os pontos importantes que mencionei, e a eles acrescento a questão da
segurança. O clima em Israel está complicado, por conta do terrorismo, o
que tem seu próprio impacto. Eu prefiro a implementação por etapas. Nós
oferecemos elementos concretos e recebemos papéis e palavras. Vivemos
numa região desprovida de democracia. O que acontecerá se fizermos
concessões hoje e Assad não estiver mais lá amanhã? Falamos à opinião
pública que conhecemos as condições na vizinhança, mas sem riscos não
existe paz.
\end{quote}

Clinton fez então mais uma tentativa de levar Rabin a dizer que estava
pronto para uma retirada total do Golã, mas Rabin se esquivou de novo.
Ele concordava que os sírios não aceitariam a paz total sem uma retirada
também total, mas não seria inteligente da parte de um israelense dizer isso de
forma oficial antes de entender se eles aceitariam as condições.
``Diante disso'', disse Rabin, ``por ora eu não gostaria de ir além do
que já disse até aqui''. Clinton insistiu: se a Síria respondesse em
relação à definição de paz, haveria uma força americana para a
manutenção da paz e garantias para a estabilidade do acordo, como ocorrera
em Camp David. Seria o suficiente para mudar a opinião
pública israelense? Rabin não se entusiasmava com a ideia de uma força
americana de manutenção da paz no Golã: ``Nós sempre nos orgulhamos de
nunca ter pedido a um soldado americano que arriscasse a própria vida
por Israel. É um elemento que reforça a nossa relação. O público
israelense prefere uma retirada parcial a uma retirada total apoiada
pelas forças americanas. Vou ser muito sincero: a questão depende de
como o público israelense enxerga as intenções de paz da Síria''. Rabin
achava importante fazer com que Assad se engajasse na diplomacia pública.
A seu ver, a paz que estabeleceria com a Síria seria, em grande medida, baseada
na paz que Begin fizera com o Egito. A tarefa de Begin tinha sido mais
fácil. Ele teve a vantagem de lidar com Sadat, que intuitivamente
entendeu que precisava ajudá"-lo a persuadir o público israelense de que
o Egito não era mais um país inimigo. Assad, ao contrário, não entendia o
papel da mídia nas democracias ocidentais e não queria de jeito nenhum
se aproximar do público israelense. O comentário de Rabin levou Clinton
a perguntar qual conduta síria seria capaz de transformar a percepção do
público israelense. A resposta de Rabin foi: Assad aparecer na televisão,
diante de seu próprio povo, o comportamento da Síria no Líbano e, acima
de tudo, atos públicos. Não uma viagem a Jerusalém, mas atos públicos.

Rabin depois mencionou que teria de submeter o acordo com a Síria a um
referendo. Clinton quis saber em que momento aconteceria esse referendo.
A resposta de Rabin: entre a fase das rubricas e a assinatura final.
Porém, repetiu ele, o referendo só poderia acontecer depois de
apresentados os termos da paz. ``Estive pessoalmente no Golã e não fui
recebido com flores.'' Clinton então respondeu: ``Esse é o preço da
liderança''. Em seguida, Rabin replicou: ``Estou com 71 anos de idade.
Já testemunhei muitas guerras e o preço de cada uma delas, então estou
disposto a correr riscos em troca da paz. Ainda temos um certo tempo até
que o fundamentalismo atinja seu auge e antes de o Irã obter mísseis e
armas de destruição em massa. Muita gente me pergunta qual o sentido de
fazer a paz no âmbito interno enquanto o âmbito externo se comporta como
o Irã. Eu respondo que fazer a paz no âmbito imediato reduzirá os riscos
no âmbito externo. Para Israel, também é importante preservar o regime
hachemita na Jordânia''.

Mais para o fim da reunião, Rabin compartilhou com Clinton uma ideia que
já tinha levantado junto ao secretário de Estado Christopher e a Tony
Lake, assessor norte"-americano de Segurança Nacional. Ele queria que os
Estados Unidos propusessem que Faisal Husseini fosse convidado a
integrar a delegação palestina nas conversas de Washington. Husseini era
um líder respeitável e razoavelmente independente em Jerusalém Oriental,
descendente de uma das famílias palestinas mais proeminentes. O curioso
é que seu pai, Abd al"-Qader Husseini, tinha sido o comandante dos
milicianos palestinos que lutaram contra os soldados de Rabin na estrada
para Jerusalém, em 1948, e acabou morto em ação. Rabin explicou que
Husseini era o último exemplo de um líder local capaz de enfrentar
Arafat e dar substância às conversas de Washington. Era o único líder
local na Cisjordânia e em Gaza que tinha ao mesmo tempo o prestígio e a
coragem de agir com independência em relação à \textsc{olp}. Como anexou
Jerusalém Oriental após a Guerra dos Seis Dias, Israel se recusava a
negociar com representantes palestinos que viviam lá. Em 1989, quando
era ministro da Defesa, Rabin tentou promover a ideia de negociar a
autonomia com líderes locais, em detrimento da \textsc{olp}; para lidar com a
questão, incluiu nas delegações de palestinos indivíduos que residiam em
Jerusalém Oriental e tinham um segundo endereço na Cisjordânia. Rabin
queria fazer o mesmo com Husseini em 1993. Também estava a par da
negociação informal com a \textsc{olp} em Oslo, autorizada por ele em fevereiro,
mas naquele momento ainda enxergava a frente de Oslo como um exercício
exploratório e as conversas de Washington como a via principal. Como era
típico, Rabin queria explorar várias opções em paralelo antes de tomar
uma decisão final.

A questão da participação da \textsc{olp} foi levantada em outra reunião
interessante que Rabin teve durante a visita de março. Foi em um café da
manhã com a equipe de paz dos Estados Unidos, composta por um grupo de
diplomatas que conduziam a política de Washington no processo de paz,
incluindo Edward Djerejian, secretário"-assistente de Estado; Sam Lewis,
diretor de planejamento de políticas no Departamento de Estado e
ex"-embaixador dos Estados Unidos em Israel; Martin Indyk, assistente
presidencial para questões do Oriente Médio no Conselho de Segurança
Nacional; Daniel Kurtzer e Aaron Miller, funcionários do Departamento de
Estado; e Dennis Ross, que naquele momento parecia estar a ponto de
abandonar o governo. Era um tanto incomum que o primeiro"-ministro de um
país estrangeiro tivesse um encontro informal com um grupo de diplomatas
e analistas, mas Rabin ficou bem confortável naquela situação. Questões
de hierarquia ou antiguidade nunca o incomodaram, e ele se comportou
como mais um analista em meio aos demais. Aprendia muito durante aquelas
conversas, que tinham enorme impacto sobre ele. O diálogo franco com a
equipe de paz dos Estados Unidos seguiu, \textit{mutatis mutandis}, essa
mesma linha. Perto do encerramento, Sam Lewis se dirigiu diretamente a
Rabin e disse que a conclusão inevitável da análise do primeiro"-ministro
acerca das negociações entre Israel e os palestinos era de que Israel
precisava conversar com a \textsc{olp}. Rabin esboçou um sorriso enigmático.
Poucos naquela sala sabiam que ele já estava negociando de forma
indireta com a \textsc{olp}.

Enquanto Rabin estava nos Estados Unidos, eclodiu em Israel uma onda de
esfaqueamentos, perpetrados por terroristas palestinos. Como já tinha
falhado uma vez, em dezembro de 1987, quando não retornou a Israel no
momento em que eclodiu a Primeira Intifada, dessa vez Rabin interrompeu
a visita e regressou imediatamente ao país, mostrando ter aprendido a
lição. Foi um importante alerta para a relevância que teria o terrorismo
no fracasso final do processo de paz da década de 1990.

Porém, apesar de tudo, a visita definiu importantes entendimentos entre
os Estados Unidos e Israel e instigou o governo Clinton a tentar seguir
em frente. Na primavera de 1993, como consequência direta das conversas
de março entre Clinton e Rabin, os Estados Unidos passaram a empreender
um esforço especial para tentar desbloquear as negociações entre a Síria
e Israel. Christopher se encontrou em Viena com Farouk al"-Shara,
ministro sírio das Relações Exteriores, e tentou em vão persuadi"-lo ---
e, por meio dele, persuadir também o presidente Assad --- a adotar uma
atitude mais flexível. Mais adiante, Djerejian, secretário"-assistente de Estado,
viajou num avião militar dos Estados Unidos numa missão
secreta a Damasco para se encontrar pessoalmente com Assad. Djerejian
tinha se saído muito bem como embaixador em Damasco e havia desempenhado
um importante papel, junto com o então secretário Baker, na missão de
persuadir Assad a participar da Conferência de Madri. Porém, dessa vez,
nem ele conseguiu demover o líder sírio de sua posição: Assad continuava
insistindo que só manifestaria seu conceito de paz depois que Israel se
comprometesse com a retirada total do Golã.

A guerra sem fim entre Israel e o Hezbollah no sul do Líbano e ao longo
da fronteira entre os dois países acrescentava outro importante
componente à relação trilateral entre Israel, Estados Unidos e Síria. No
final da década de 1980, o Hezbollah se tornou o principal ator no sul
do país quando, com ajuda iraniana e síria, derrotou Amal na luta por
hegemonia sobre a comunidade xiita do Líbano. O Hezbollah era uma
entidade peculiar: a um só tempo movimento político, guerrilha,
organização terrorista e braço do regime iraniano. No início da década
de 1990, intensificaram seus ataques contra os soldados das \textsc{idf} que
ficavam no sul do Líbano para dar reforço ao Exército do Sul do Líbano,
incapaz de enfrentar sozinho o Hezbollah. De tempos em tempos, o
Hezbollah também disparava foguetes Katyusha contra o norte de Israel. O
interlocutor (indireto) de Rabin para lidar com essa entidade era Damasco.
Era típico de Assad --- que acreditava em negociar a partir de uma posição
de força e explorar todos os ativos que tivesse --- fazer negociações com
Israel e, ao mesmo tempo, apoiar os ataques do Hezbollah contra o país. E
se Israel era sensível às baixas, havia mais motivo ainda para aumentar
a pressão no sul do Líbano. Mas a opinião dos israelenses sobre o
comportamento de Assad era bem diferente. Como Rabin explicou a Clinton
em março, a conduta de Assad no Líbano reforçava sua imagem negativa
junto ao público israelense e configurava um grande obstáculo a um
possível acordo com a Síria.

\section{O processo de Oslo}

O processo de Oslo teve início em dezembro de 1992, como um clássico
exercício diplomático de bastidores, e culminou oito meses depois, em
agosto de 1993, com um acordo formal entre Israel e a \textsc{olp}. O processo de
Oslo se deu graças à convergência de três fatores: o ativismo de Yossi
Beilin, vice"-ministro das Relações Exteriores de Israel; o empenho de
Terje Larsen, diretor do Instituto Norueguês de Ciências Sociais
Aplicadas, na intermediação do contato e das negociações entre
israelenses e palestinos; e a disposição da \textsc{olp} em iniciar um processo
de paz com Israel.

O processo começou com uma reunião em Londres entre Yair Hirschfeld,
acadêmico israelense e colaborador de Beilin, e Ahmad Qurei (``Abu
Alaa''), um dos principais líderes da \textsc{olp}. A reunião aconteceu logo antes
da revogação de uma lei israelense que proibia qualquer contato com a
organização palestina. Hirschfeld viajou então para Oslo, com a
permissão de Beilin e acompanhando de outro acadêmico israelense, Ron
Pundak, para uma reunião com Abu Alaa e dois de seus colegasda \textsc{olp}. As
primeiras conversas tiveram como foco a implementação do plano de
autonomia a partir da retirada israelense de Gaza, a cooperação
econômica e --- por insistência de Beilin --- a formulação de uma
Declaração de Princípios definindo as relações entre israelenses e
palestinos durante o período interino.

Beilin levou algum tempo até atualizar Peres, o que ocorreu em janeiro.
Peres, por sua vez, levou mais um tempo até atualizar Rabin, em
fevereiro. Para surpresa dos dois, Rabin deu sinal verde para seguirem
em frente. Preocupado com a ausência de progressos nas conversas de
Washington e, por conta de seu pragmatismo, dispunha"-se a testar o que
viria a ser conhecido como a frente de Oslo. Estava totalmente ciente da
importância de tratar com a \textsc{olp}, mesmo que por um canal informal. Também
estava ciente de que Peres adentrava naquele momento na corrente
dominante do processo de paz, por meio da frente de Oslo, deixando para
trás o escanteio ao qual tinha sido relegado no início do governo Rabin.
A frente de Oslo era mantida em sigilo absoluto. Rabin não informava nem
mesmo seus assistentes mais próximos, e o governo Clinton era atualizado
por relatórios entregues por Beilin e também pelo governo norueguês a
Daniel Kurtzer, membro da equipe de paz dos Estados Unidos. Houve novas
reuniões em fevereiro e março para debater a Declaração de Princípios.
Em maio, Rabin concordou que Uri Savir, então diretor"-geral do
Ministério das Relações Exteriores, integrasse as conversas de Oslo.
Assim, Oslo deixou de ser um canal informal. Rabin continuava
determinado a manter as conversas de Washington como seu canal
principal, mas diante da ausência de progressos nessa frente, passou a
atribuir um papel mais relevante à frente de Oslo. O interlocutor de
Savir era Abu Alaa. Em junho, o ministro das Relações Exteriores
convidou Joel Singer, de Washington, para examinar a Declaração de
Princípios que Pundak e Hirschfeld haviam preparado junto com seus
interlocutores palestinos nos meses anteriores. O nome de Singer foi uma
escolha interessante para essa função. Ele era coronel da reserva das
\textsc{idf}, ex"-diretor do Departamento Internacional da Advocacia Geral Militar
e atuava como advogado num escritório em Washington, além
de ter vasta experiência nas negociações entre árabes e israelenses (com
o Egito e o Líbano) do final das décadas de 1970 e 1980. Rabin tinha
plena confiança em antigos oficiais das \textsc{idf} e ficou feliz com a
participação de Singer.

Em junho de 1998, Singer publicou no jornal \textit{Haaretz} um artigo no
qual Savir faz um relato sobre o processo de paz da década de
1990.\footnote{Ver Uri Savir, \textit{O processo: 1.100 dias que mudaram o Oriente Médio}.
Tel Aviv: \textit{Yedi'ot Achronot}, 1998 {[}em hebraico{]}.} Vale a pena
citar um trecho longo, por conta do importante papel de Singer no
processo de Oslo e pela crítica inesperada que ele faz à forma como as
negociações foram conduzidas:

\begin{quote}
O esboço do Acordo de Oslo foi escrito de forma não profissional. As
ideias contidas ali me pareciam em parte boas, mas algumas delas tendiam
demais para o lado palestino. Eu tinha grande estima por Peres, Beilin e
seus assistentes, mas o que me preocupava era o entusiasmo deles com
o fato em si de terem chegado a um acordo com a \textsc{olp}, o que podia ofuscar seus
aspectos negativos. Não me impressionavam muito as histórias que eles
contavam sobre a química que prevalecera nas conversas de Oslo, e eu
acreditava que, para além da química, havia necessidade de física. Ou
seja, de ancorar de forma clara no acordo a proteção de interesses
israelenses. Decidi, então, alertá"-los, com total franqueza, contra a
assinatura do acordo {[}\ldots{}{]} No final da reunião, estava convencido de que
nunca mais veria o ``grupo de Oslo''. Mas descobri que estava errado.
Mais ou menos na mesma época, Rabin ordenou por escrito que Peres
encerrasse os contatos de Oslo, porque ele também estava com uma
impressão negativa sobre o acordo. Peres me convidou de novo para ir a
Israel e, dessa vez, chamou a mim e a Beilin para uma conversa com
Rabin. Por fim, depois de Peres ter sugerido meu nome para integrar a
equipe de Oslo, Rabin concordou em dar permissão ao grupo para começar
uma negociação formal com a \textsc{olp}, desde que eu preparasse para sua
aprovação um rascunho do acordo, pautado nas instruções dele. O mais
lógico a fazer seria recomeçar o trabalho, mas para a minha preocupação,
fui instruído a aceitar como premissa o rascunho já existente e a
acrescentar a ele apenas mínimas correções indispensáveis. O pessoal do
Ministério das Relações Exteriores me explicou que, como as negociações
já estavam em andamento havia seis meses, os palestinos esperavam que o
acordo fosse assinado dali a mais uma ou duas reuniões. O Acordo de Oslo
como um todo poderia se desintegrar se eu tentasse introduzir alterações
muito amplas. Apesar de todo o meu esforço para minimizar as correções,
os representantes da \textsc{olp} em Oslo ficaram chocados quando lhes apresentei
o texto, porque havia eliminado todos os benefícios que eles tinham
conseguido com seus interlocutores israelenses.\footnote{\textit{Haaretz},
  10 jun. 1998.}
\end{quote}

Como apontou Singer, em junho Rabin se mostrou descontente com os rumos
que as conversas de Oslo tinham tomado. Não estava seguro quanto à
seriedade dos negociadores palestinos e ficava preocupado, achando que
essa frente pudesse neutralizar as conversas de Washington que ele tanto
queria fazer avançar. No dia 6 de junho, escreveu a Peres, ordenando que
suspendesse as negociações:

\begin{quote}
Os contatos conhecidos como ``contatos de Oslo'' constituem no momento
atual um perigo à continuidade das negociações de paz {[}\ldots{}{]} Os
representantes de Túnis são os mais extremistas entre os palestinos que
querem um processo de paz e impedem que sujeitos mais moderados avancem
nas negociações conosco {[}\ldots{}{]} É possível que os representantes de Túnis
queiram sabotar qualquer possibilidade de alcançarmos uma negociação
real em Washington e queiram nos forçar a falar apenas com eles, o que
representaria um risco às negociações de paz com a Síria, o Líbano e a
Jordânia {[}\ldots{}{]} Peço que suspenda os contatos, até conseguirmos maiores
esclarecimentos.\footnote{\textit{Maariv}, 12 set. 2003.}
\end{quote}

Ele mudou de ideia no dia 10 de junho, quando Singer voltou de Oslo
assegurando"-lhe sobre a seriedade da delegação palestina; naquele dia,
aprovou a retomada da frente de Oslo. Rabin concordou com Peres e sua
equipe ao perceber que Oslo estava se mostrando uma alternativa mais
promissora do que as conversas de Washington, mas continuava em dúvida
quanto à ideia de fazer um acordo com a \textsc{olp} e mantinha suas
desconfianças em relação a Peres. Ele recebia atualizações sobre as
conversas de Oslo em reuniões semanais, sempre às sextas"-feiras de
manhã, no Ministério da Defesa, mas não tinha certeza se lhe contavam
todos os detalhes das negociações. Além disso, sabia perfeitamente bem
que um possível acordo em Oslo seria muito positivo para a imagem de
Peres, pois o levaria das margens para o centro do processo de paz.
Outra preocupação de Rabin era que uma eventual decisão de encerrar o
processo de Oslo pudesse levar Peres a desafiá"-lo dentro do Partido
Trabalhista, apresentando"-se como defensor da paz, enquanto Rabin seria
o obstáculo.

A ambivalência de Rabin se traduziu também em ações concretas. De um
lado, autorizou dois de seus homens de confiança a explorar canais
alternativos a Oslo. Um deles era Haim Ramon, ministro da Saúde e
influente líder do Partido Trabalhista. Ramon estabeleceu um canal com
Ahmad Tibi, político árabe"-israelense muito próximo à liderança da \textsc{olp}.
Tanto Rabin quanto Mahmoud Abas (``Abu Mazen'') usavam o canal Ramon"-Tibi
para checar as intenções de seus parceiros/rivais. Rabin queria
verificar se os relatos de Peres sobre as negociações de Oslo eram
acurados, enquanto Abu Mazen estava testando os relatos recebidos por
intermédio de Abu Alaa. De todo modo, Arafat não permitiu o
desenvolvimento de qualquer canal alternativo, pois esperava que Oslo
lhe fornecesse o melhor resultado possível.

Houve também outra iniciativa, por intermédio de Ephraim Sneh, alto
oficial da reserva das \textsc{idf} e político do Partido Trabalhista. Em maio de
1993, Sneh convenceu Rabin de que o caminho para desbloquear o impasse
nas conversas de Washington seria pautado em quatro fases: o
reconhecimento da \textsc{olp}, condicionado a um prazo de seis meses sem nenhum
ataque terrorista na Cisjordânia e em Gaza; a adoção de uma série de
gestos e medidas para a criação de confiança na Cisjordânia e em Gaza; a
realização de eleições para um parlamento palestino; e a transferência
do controle da Faixa de Gaza, como modelo a ser aplicado posteriormente
na Cisjordânia. A transição de uma fase para a seguinte dependeria
do sucesso da implementação de cada etapa.

O plano baseava"-se numa modificação mútua da Declaração de Princípios
proposta às duas partes pelos Estados Unidos, em maio, no contexto das
conversas de Washington. No dia 7 de junho, Sneh foi a Londres com
instruções detalhadas da parte de Rabin, para negociar o texto de uma
Declaração de Princípios com Nabil Shaath, alto dirigente da \textsc{olp}. Depois
de completarem a missão no dia 9 de junho, Sneh e Shaath voltaram para
casa, para atualizar Rabin e Arafat. Rabin compartilhou com Sneh um
relatório de inteligência que dizia que Arafat estaria aparentemente
determinado a evitar qualquer alternativa a Oslo, pois era ali que ele
esperava alcançar mais em troca de menos compromissos. Alguns dias
depois, Rabin contou a Sneh que Peres tinha se irritado com a
tentativa de abrir um canal alternativo, e que seu pessoal havia
repreendido Abu Mazen por querer se encontrar com Sneh como emissário
pessoal de Rabin. Como Sneh escreveu em seu livro, Rabin passou a
entender ``que Oslo era o único caminho possível''.\footnote{Ephraim
  Sneh, \textit{Navegando em águas perigosas}. Tel Aviv: Yedi'ot Achronot Books, 2002, p. 22--3
  {[}em hebraico{]}.}

Rabin continuou mantendo segredo sobre o canal de Oslo, inclusive com
seus assessores mais próximos, como seu principal negociador com os
palestinos nas conversas de Washington, Elyakim Rubinstein. Não é de
espantar que vários desses assessores tenham se tornado grandes críticos
do processo de Oslo. Em parte, o segredo era para impedir o vazamento de
informações, mas também refletia as dúvidas de Rabin em relação a Oslo.
Ele era ambivalente por natureza, o que se agravava ainda mais nesse
caso, uma vez que o acordo representava um afastamento radical em
relação à sua postura pública, não se firmava em bases sólidas e era
conduzido por seu maior concorrente. Ele também continuou mantendo seus
parceiros americanos apenas parcialmente informados a respeito do canal
de Oslo. A transmissão de informações de Beilin para Kurtzer foi então
suspensa, sendo renovada no final de julho, quando as negociações em
Oslo estavam a ponto de resultar em um acordo. Em certa medida, isso se
devia, de novo, à ambivalência de Rabin, mas era também reflexo de sua
convicção de que os Estados Unidos não deveriam se tornar mediadores nas
negociações árabe"-israelenses. Para ele, o cenário ideal era uma
negociação discreta e direta entre as duas partes; os Estados Unidos só
seriam convocados num estágio posterior, enquanto avalista e para
ajudar a desobstruir os obstáculos finais. O comprometimento e futuro
engajamento dos Estados Unidos eram, para Rabin, condição \textit{sine qua
non}. Ele disse a Peres que seu próprio aval sobre o acordo estava
sujeito ao aval de Washington.

As negociações de Oslo prosseguiram ao longo de junho e julho, com altos
e baixos. Um acordo começava a ganhar forma e era muito diferente do
produto das negociações iniciais, que ocorreram nos primeiros meses de
1993. A principal mudança foi a introdução da ideia de reconhecimento
mútuo entre Israel e a \textsc{olp}. O acordo pautava"-se no conceito de autonomia
do final da década de 1970, mas levava"-o muito mais além. Uma Autoridade
Palestina (\textsc{ap}) autônoma seria instaurada em Gaza, com um ponto de apoio
em Jericó, na Cisjordânia, indicando que o acordo não se limitava a
Gaza. O território sob autonomia palestina na Cisjordânia seria ampliado
gradualmente, com o passar do tempo. O acordo teria duração de cinco
anos, e ao final desse período negociações quanto a um \textit{status} definitivo
levariam a um acordo final. Arafat e seus homens seriam autorizados a
retornar de Túnis e assumir o controle da nova \textsc{ap}. Para Rabin e Peres, o
elemento mais difícil e talvez o mais promissor do acordo era o
reconhecimento mútuo entre ambas as partes e a nova parceria com a \textsc{olp}.
O reconhecimento de Israel por parte do movimento nacional palestino era
decisivo para encerrar o conflito central com os palestinos e
transformar a relação de Israel com o mundo árabe e o mundo muçulmano em
geral. Mas será que podiam confiar em Arafat?

A natureza transitória dos acordos trazia algumas vantagens; por ora, os
assentamentos não teriam que ser removidos. Mas depois de cinco anos, as
questões mais difíceis sobre um acordo de \textit{status} definitivo teriam de
ser enfrentadas. O termo ``Estado palestino'' não era mencionado no
acordo, mas os líderes israelenses sabiam que a \textsc{olp} queria o estatuto de
Estado e que uma \textsc{ap} totalmente autônoma na Cisjordânia e na Faixa de
Gaza era um grande passo nessa direção.

Na primavera de 1993, aumentaram significativamente os combates contra o
Hezbollah na zona de segurança no sul do Líbano e também o lançamento de
foguetes por parte do grupo para o norte de Israel. No fim de julho,
Rabin decidiu lançar uma operação militar de grande escala no sul do
Líbano. Batizada de Operação Acerto de Contas, consistia numa série de
ataques aos bastiões do Hezbollah e no esforço deliberado de levar uma
grande massa de civis para o norte, em direção a Beirute, sob a premissa
(equivocada) de que isso criaria pressão popular sobre o governo central
do Líbano para fazer valer sua soberania e restringir as atividades do
Hezbollah. Ainda no fim de julho, Rabin concluiu que a operação já tinha
cumprido seu propósito; não havia mais ganhos pela frente. Ele pediu ao
governo Clinton que usasse sua influência com Assad para negociar um
cessar"-fogo segundo termos que proporcionassem estabilidade ao sul do
Líbano. O cessar"-fogo foi conduzido por Christopher, secretário de
Estado, e por Dennis Ross (que continuaria fazendo a intermediação e
acabou se tornando o chefe da equipe de paz dos Estados Unidos), junto
com o interlocutor sírio de Christopher, Farouk al"-Shara. Foi uma
encenação e tanto; al"-Shara alegou, é claro, que a Síria não tinha
qualquer influência sobre o Hezbollah, mas de um jeito ou de outro ele
conseguiu desenrolar o acordo de cessar"-fogo. Christopher e Ross fizeram
por telefone o trabalho inicial relativo ao acordo e depois foram até a
região em agosto de 1993 para finalizá"-lo.

\section{A virada em agosto de 1993}

Quando embarcaram para o Oriente Médio, o secretário de Estado e sua
equipe não sabiam que aquela visita se tornaria um ponto de inflexão na
história do processo de paz. Rabin decidiu que tinha chegado a hora de
tomar decisões cruciais. Sabia que sem uma decisão dolorosa --- fosse na
frente síria, fosse na frente palestina ---, não haveria grandes avanços
no processo de paz e estava determinado a empreender esses avanços. Em
conversas particulares com pessoas de sua confiança, Rabin manifestava
preocupação com a opinião pública israelense e com a capacidade das \textsc{idf}
de pagar o preço de um conflito de longo prazo, aparentemente sem fim.
Preocupava"-se com a resposta da opinião pública israelense aos ataques
iraquianos com mísseis em 1991, quando um grande número de moradores de
Tel Aviv deixaram a cidade e foram para Jerusalém, onde havia mais
segurança, e também se preocupava com o desempenho medíocre das \textsc{idf} na
Guerra do Líbano em 1982. Se as condições em 1993 permitissem a Israel
transformar sua relação com o mundo árabe, a oportunidade deveria ser
aproveitada. Rabin também se sentia pressionado no sentido mais estreito
da noção de momento político. Tinha se comprometido com os palestinos a
chegar a um acordo de autonomia num prazo de nove meses. Para piorar as
coisas, sua coalizão de governo estava se desmantelando. Aryeh Deri,
líder do Shas, enfrentava acusações de corrupção, e o partido
provavelmente deixaria a coalizão comandada pelos trabalhistas. Era
muito melhor fazer as concessões necessárias a um acordo se na aliança
de governo houvesse um partido ortodoxo de direita do que com um governo
de centro"-esquerda que dependesse dos votos de membros árabe"-israelenses
do Knesset, o que seria um anátema à direita israelense.

A viagem de Christopher foi planejada de forma que ele e a equipe se
encontrassem primeiro com Rabin, para ouvir o posicionamento e as
questões do primeiro"-ministro, e em seguida fossem a Damasco, para ouvir
as respostas de Assad. Christopher passou pelo Cairo no dia 1\textsuperscript{o} de
agosto, para se informar e consultar com o presidente do Egito, Husni
Mubarak, importante parceiro político dos Estados Unidos na região, e
chegou a Israel no dia 2. A reunião com Rabin e sua equipe, no gabinete
do primeiro"-ministro em Jerusalém, foi na manhã do dia 3. Antes da
reunião principal, aconteceu a breve reunião tradicional com os dois
protagonistas e seus encarregados de tomar notas (no caso, Ross e eu).
Essa reunião prévia costuma ser rápida, mas não foi o que ocorreu
naquele dia. Rabin foi direto ao ponto: estava disposto a dar um passo
significativo em uma das frentes do processo de paz; preferia que fosse
na frente sírio"-libanesa e que aos palestinos fosse oferecido um acordo
limitado, como Gaza primeiro, sem Jericó. A respeito da Síria e do
Líbano, havia duas opções: primeiro o Líbano primeiro, ou seja, um
acordo inicial com o Líbano, Estado cliente da Síria, a ser seguido por
um acordo entre Israel e Síria, mas ele duvidava que Damasco
concordaria. No entanto --- e esse foi o momento dramático da reunião ---,
ele pediu a Christopher para explorar com Assad a seguinte possibilidade:
se a demanda do presidente sírio fosse atendida (da retirada total do
Golã), a Síria estaria pronta para: (a) assinar um tratado de paz com
Israel sem qualquer relação com o progresso de outras frentes?; (b) uma
paz \textit{verdadeira}, incluindo normalização, relações diplomáticas e
todas as demais parafernálias pertinentes?; e (c) oferecer elementos
dessa paz antes da implementação completa da retirada total? Rabin disse
a Christopher que imaginava um prazo de cinco anos para encerrar todo
esse processo e que, considerando o fato de que estavam querendo de
Israel elementos tangíveis em troca de elementos intangíveis, ele, por
sua vez, queria provas tangíveis da paz antes de empreender uma retirada
significativa. Sua opinião inspirava"-se no ajuste precedente entre Egito
e Israel, quando Sadat concordou com o estabelecimento de embaixadas
depois de completada a primeira fase da retirada.

Além dessas três questões a serem levadas a Assad, Rabin levantou outros
quatro pontos. Em primeiro lugar, queria contar com a participação dos
Estados Unidos no regime de segurança pós"-acordo. Em segundo lugar,
enfatizou que estava falando de ``uma suposição'' (ou seja, uma
declaração hipotética, e não um compromisso). Em terceiro lugar,
enfatizou também a confidencialidade absoluta desse exercício e disse
várias vezes a Christopher que sua sinalização deveria ficar no bolso do
secretário, sem ser ``posta na mesa''. Por fim, explicou a Christopher
que antes de assinar um acordo com a Síria pautado na retirada total,
teria que propor um referendo em Israel.

Rabin e Christopher debateram essas questões por um tempo, esclarecendo
e examinando os pontos envolvidos. Christopher queria saber se Jericó
(um possível ponto de apoio para a \textsc{olp} na Cisjordânia, para além de
Gaza) seria parte do concomitante ``acordo limitado'' com os palestinos
que Rabin tinha em mente; esse era um nítido eco das negociações
paralelas que vinham acontecendo com os palestinos, das quais o
secretário de Estado estava a par, pelo menos em termos gerais. Rabin
respondeu que, na eventualidade de um acordo com a Síria, o acordo com os
palestinos teria de se limitar a Gaza, mas que se o primeiro acordo
fosse feito com os palestinos, tanto Gaza quanto Jericó seriam
incluídos.\footnote{Rabinovich, \textit{The Brink of Peace}, \textit{op}.
\textit{cit}., p. 104--8.}

No curto trajeto do gabinete de Rabin até a sala de reuniões onde os
assessores de Rabin e Christopher aguardavam com impaciência, eu disse a
Ross que conseguia ouvir ali o bater de asas da história. Sabia que
Rabin tinha acabado de dar a Christopher o caminho das pedras para um
acordo de paz entre Israel e Síria. A estratégia dele foi muito
audaciosa. Sua atitude de confiar ao secretário de Estado americano a
disposição hipotética e condicionada de se retirar das Colinas do Golã
era tão radical quanto sua disposição de negociar com a \textsc{olp}. No passado,
ele tinha sido contrário a uma saída do Golã e disse isso com todas as
letras durante a campanha eleitoral. Como ex"-chefe do Comando Norte
das \textsc{idf} e chefe do Estado"-Maior nos anos que antecederam a Guerra dos
Seis Dias, seu envolvimento com a segurança da Alta Galileia era a um só
tempo profissional e emocional. Contava com forte apoio popular naquela
parte do país e nas Colinas do Golã, onde muitos dos colonos eram
eleitores do Partido Trabalhista. Rabin estava nitidamente fornecendo a
Christopher a chave para um acordo com a Síria e disse a ele sem rodeios
que esse acordo acabaria minimizando o que estava a ponto de ser
assinado em Oslo. Rabin também sabia que Peres não reagiria bem a
tamanha virada dos acontecimentos, mas, de posse do acordo com a Síria,
ele conseguiria evitar a disputa contra Peres dentro do partido. Rabin
preferia a opção da ``Síria primeiro'', entretanto, diante da possibilidade
clara de que as negociações com Assad não dessem em nada, decidiu se
voltar para Oslo.

Porém, antes de se comprometer definitivamente com Oslo, tentou
descobrir se havia ou não uma alternativa síria. Caso essa possibilidade
se mostrasse real, estaria disposto a pagar o preço necessário para
assegurá"-la. Depois de tomar a decisão fundamental de que uma grande
concessão teria de ser feita para conduzir o processo de paz por uma
frente significativa --- negociar com a \textsc{olp} ou sair das Colinas do Golã
---, Rabin estava pronto a se comprometer com qualquer uma das duas vias.

Foi preciso esperar para saber se havia de fato uma opção síria. No dia
4 de agosto, Christopher e sua equipe viajaram para Damasco com a
proposta no bolso e voltaram no dia 5. Rabin ficou muito decepcionado
com a resposta. Assad estava disposto a oferecer uma paz contratual
formal, em troca da retirada total e, a princípio, também estava
disposto a fazer com que o acordo ``se sustentasse por si só''. Contudo,
logo apareceu uma longa lista de ``senãos e poréns''. O ponto mais
relevante era que Assad não aceitava a demanda de Rabin de que o acordo
fosse implementado de forma a oferecer a Israel, logo de início, uma
ampla margem de normalização, em troca de uma retirada limitada.
Tampouco aceitava o horizonte de tempo de cinco anos e queria, em vez
disso, que o acordo fosse implementado num período de seis meses.
Christopher e Ross interpretaram como positiva a resposta de Assad, no
sentido de que ele aceitava ``a equação básica'', mas Rabin enxergou a
questão de forma muito diferente.

A resposta de Assad quanto à vinculação com outras frentes, embora
positiva a princípio, não era nada clara. Rabin sabia que a Síria
insistiria em se atrelar completamente à frente libanesa e entendia que
Assad precisava de algum progresso com os palestinos para legitimar seus
próprios passos, mas não estava nítido o escopo desse progresso. Assad
concordava com a ideia de paz total, mas disse a Christopher que tinha
restrições quanto ao termo ``normalização''. Ele rejeitava a ideia de
Rabin de estabelecer um canal discreto e direto; o máximo que aceitava
eram reuniões entre Muwaffaq Allaf, que era seu principal negociador, Walid Muallem,
seu embaixador em Washington, e eu, com a participação também
de um representante americano. A questão dos acordos de segurança nem
chegou a ser debatida em detalhes (pelo menos segundo os relatos de
Christopher e Ross), e não teve qualquer ressonância a insistência de
Rabin em repetir que, sem um investimento significativo por parte da Síria em
diplomacia pública, faltaria"-lhe base política para seguir em frente.

A partir dos relatos de Christopher e Ross, Rabin concluiu que sua
proposta não tinha ficado no bolso do secretário de Estado durante a
reunião com Assad, mas, pelo contrário, havia sido escancarada sobre a
mesa. Como seria de se esperar, em resposta a essa estratégia, embora sua
reação inicial tenha sido positiva, Assad logo começou a fazer barganhas.
Rabin inflara seu posicionamento, antecipando"-se a esse processo de
barganha, mas sentiu que tinham puxado seu tapete. Também ficou surpreso
ao ouvir de Christopher que ele e sua equipe voltariam aos Estados
Unidos para as férias de verão. A opção síria continuava valendo?

A decisão final de Rabin foi de que não havia essa opção. E não foi uma
decisão fácil. Assad tinha, ``a princípio'', dado uma resposta positiva,
interpretada pelo secretário de Estado como adequada. Ao decidir
abandonar a opção síria e se voltar às negociações de Oslo, Rabin tinha
tudo para irritar o governo americano. Era também um compromisso sério
com um cenário que deixava o próprio Rabin inseguro. Porém, a opção de
abandonar Oslo em troca de uma negociação lenta, árdua e possivelmente
infrutífera com Assad, na frente síria, parecia ainda pior. Além disso,
uma divergência explícita com Peres, decorrente da opção de abandonar
Oslo em troca de um acordo limitado com os palestinos como corolário
para um acordo com a Síria, provavelmente seria mais penosa do que ter
que lidar com a raiva de Christopher. Se Rabin conseguisse exibir
grandes progressos com a Síria, poderia superar a indignação de Peres e
a disputa potencial dentro do partido, mas a resposta de Assad e a
decisão de Christopher de voltar imediatamente aos Estados Unidos o
deixaram sem nada para apresentar. Havia outra vantagem na opção de
Oslo: por ser um acordo interino, as decisões mais difíceis podiam ser
adiadas por cinco anos. Num acordo com a Síria, as escolhas mais
complicadas teriam que ser feitas logo no início. Rabin acabou decidindo
dar sinal verde para a conclusão das negociações de Oslo e abandonou a
frente síria. Contudo, insistiu que, antes da formalização do acordo, era
preciso assegurar o aval de Washington.

De fato, os parceiros americanos de Rabin ficaram possessos com a
decisão. Sentiram"-se usados e ludibriados. Para eles, tinham recebido
uma proposta (à qual tendiam a se referir como ``compromisso''),
transmitido essa proposta e voltado com uma resposta que interpretavam
como positiva. Consideravam que tinham assumido um compromisso com a
Síria. Mas a raiva deles logo foi controlada. Para começar, considerando
a excelente relação já estabelecida entre o governo Clinton e o governo
Rabin, um incidente dessa magnitude podia ser facilmente contido. E o
mais importante: eles entenderam o enorme potencial do reconhecimento
mútuo entre Israel e o nacionalismo palestino e da criação de uma
entidade palestina autônoma, o que estava no cerne dos Acordos de Oslo.
Clinton e sua equipe preferiam a opção síria, mas sabiam que, apesar de
tudo, estavam diante de um grande avanço que os governos anteriores
nunca tinham conseguido alcançar. A perspectiva do governo Clinton é
descrita num livro de Martin Indyk, que era um ferrenho apoiador da
política da ``Síria primeiro''. Decepcionado, ele chegou à conclusão
(equivocada) de que Rabin queria desde o início priorizar o acordo com
os palestinos e tinha usado os Estados Unidos, perante a Síria, para
melhorar os termos do acordo. Indyk escreveu: ``Acreditávamos ter um
acordo sobre o caminho a seguir: um esforço comum para alcançar primeiro
a paz com a Síria, com base na retirada total de Israel do Golã. Foi uma
grande ingenuidade nossa não imaginar que Rabin usaria a influência
americana para seus objetivos próprios. O problema nesse caso foi de
falha de coordenação, e não de excesso. Clinton prometera a Rabin não
surpreendê"-lo, mas não deixou claro que, em troca, os Estados Unidos não
queriam ser surpreendidos por Israel. A coordenação precisava ser uma
via de mão dupla''.\footnote{Martin Indyk, \textit{Innocent Abroad: An
  Intimate Account of American Peace Diplomacy in the Middle East} 
  {[}Inocente no exterior: um relato intimista da diplomacia de paz
  dos Estados Unidos no Oriente Médio{]}. Nova
  York: Simon and Schuster, 2009, p. 91. Ver também Dennis Ross, \textit{The
  Missing Peace: The Inside Story of the Fight for Middle East} 
  {[}A paz ausente: a verdadeira história da luta pelo Oriente Médio{]}.
  Nova York: Farrar, Straus and Giroux, 2004.}

No dia 19 de agosto, tiveram início em sigilo os Acordos de Oslo,
durante uma visita de Peres à cidade. O passo seguinte era conseguir o
aval de Washington. Peres e seu interlocutor norueguês, Ian Holst,
viajaram à base dos fuzileiros navais em Point Mugu, perto de Santa
Bárbara, na Califórnia, para se encontrar com Christopher, que passava
férias em sua casa de campo. Eu cheguei a Santa Bárbara na véspera, para
participar dos preparativos para o encontro. Os dois estavam acompanhados de
uma pequena equipe de assessores. Peres estava ansioso e nervoso, pois
temia que a experiência negativa com Shultz, em 1987, se repetisse. Mas
não seria o caso com Christopher. Ele e sua equipe ouviram a explicação
de Singer sobre o acordo. Christopher então pediu licença e saiu para
telefonar para Clinton. Voltou dizendo que os Estados Unidos apoiavam o
acordo. Em pouco tempo, ficou decidido que os Estados Unidos se
tornariam um parceiro ativo e seriam também os anfitriões da cerimônia
de assinatura, nos jardins da Casa Branca, no dia 13 de setembro.

A cerimônia na Casa Branca seria precedida por uma crise na relação
entre Rabin e Peres. O plano original era que a cerimônia fosse
conduzida entre os ministros das Relações Exteriores, ou seja,
Christopher, Peres e Abu Mazen. Porém, conforme a data se aproximava,
dois fatores começaram a mudar esse plano. O primeiro foi o desejo de
Clinton de desempenhar um papel pessoal no que tinha tudo para ser um
grandioso evento internacional e um sucesso para seu governo. Em Israel,
alguns assessores de Rabin o pressionavam a viajar para Washington e
participar da cerimônia --- mas ele não estava muito entusiasmado para
encontrar Arafat e cumprimentá"-lo. Entretanto, entendia que a cerimônia e o
acordo podiam transformar totalmente a relação de Israel com os
palestinos e, portanto, ele deveria estar presente no evento. Havia também a questão
inevitável de sua relação com Peres: deveria lhe ceder o protagonismo?
Ou deveria ele mesmo ocupar o lugar central? Rabin tomou a decisão final de
ir para Washington praticamente no último minuto, na
noite de sexta"-feira, dia 10 de setembro. Peres ficou surpreso e
ofendido com a decisão de Rabin, ainda mais por tê"-la ouvido no rádio.
Furioso, deu uma entrevista no sábado de manhã, em seu apartamento
oficial em Jerusalém, para dois grandes jornalistas israelenses, Nahum
Barnea e Shimon Shiffer, comentando com raiva o modo como Rabin o
tratara. Disse enfaticamente aos dois que não iria para Washington.
Durante a entrevista, Peres pediu duas vezes que os jornalistas saíssem
da sala para que ele pudesse dar telefonemas confidenciais. Nem ele nem
os jornalistas perceberam que o gravador deixado por eles sobre mesa era
ativado por voz, e parte dos telefonemas foi gravado. Um dos
interlocutores era Giora Einy. Foi assim que o papel de Einy como
intermediário foi descoberto e acabou sendo relatado por Barnea. A
manchete de domingo do jornal \textit{Yedi'ot Achronot} anunciava a
decisão de Peres de não ir a Washington. No fim das contas, porém, ele
acabou indo, como vice. O jornal saiu praticamente na mesma hora em que
Peres aterrissava junto com Rabin na Base da Força Aérea de Andrews. O
desempenho de Rabin na cerimônia de assinatura ilustrou a importância de
seu papel individual no sentido de vender ao público israelense a
guinada revolucionária na política do país em relação aos palestinos.
Embora não fosse tido como grande orador, Rabin fez um discurso muito
impactante. Descobriu, também, a conduta perfeita para lidar com Arafat.
Apertou a mão do líder palestino, mas seu desconforto era evidente. A
expressão facial e a linguagem corporal de Rabin refletiam seu
mal"-estar. Para o público israelense, que precisava absorver a
transformação de Arafat --- de inimigo raivoso e condenável a parceiro
num processo de paz ---, Rabin encontrou o tom perfeito.

\section{1994: dificuldades iniciais e o ápice do processo de paz}

O avanço nos Acordos de Oslo e a euforia gerada pela cerimônia de
assinatura nos jardins da Casa Branca foram seguidos por meses difíceis,
com negociações complexas e extensas a respeito da implementação. Era
preciso fazer acordos sobre a chegada de Arafat com seus soldados de
Túnis --- para uma divisão de trabalho envolvendo segurança, para a
relação entre a \textsc{ap} e os assentamentos israelenses na Cisjordânia e na
Faixa de Gaza ---, sobre a travessia da fronteira com a Jordânia e sobre
questões econômicas. Rabin nomeou o general Amnon Shahak, vice"-chefe do
Estado"-Maior das \textsc{idf}, para negociar grande parte dessas questões com
Nabil Shaath, nomeado por Arafat. As questões econômicas foram delegadas
ao ministro da Fazenda, Avraham (Baiga) Schochat. Em 29 de abril foi
assinado o Protocolo de Paris, que regulamentava a relação econômica
entre a Autoridade Palestina e Israel; o acordo de implementação foi
firmado no Cairo, em 4 de maio de 1994. Essa cerimônia de assinatura
foi prejudicada quando Arafat a princípio se recusou a assinar e acabou
sendo persuadido pelo presidente Mubarak a fazê"-lo. Foi um indicativo
prévio das dificuldades que surgiriam à frente.

A principal negociação, conduzida entre Shahak e Shaath em Taba, no
Sinai, foi longa e árdua. Envolveu uma série de arranjos de ordem
prática e detalhes ínfimos --- por exemplo, a configuração exata dos
postos de controle para entrar e sair da \textsc{ap} --- entre duas partes que até
pouco tempo antes tinham sido ferozes inimigas. Israel precisava confiar
que seus antigos inimigos cuidariam de sua própria segurança contra
potenciais atos terroristas. Levou tempo até que os dois lados
desenvolvessem a confiança mútua necessária. Em 1994, acrescentou"-se ao
cenário a influência letal do terrorismo. No dia 25 de fevereiro, um
fanático judeu, Baruch Goldstein, massacrou 28 devotos
muçulmanos na Tumba dos Patriarcas, em Hebron. Nos dias 6 e 13 de abril,
o Hamas cometeu dois atentados suicidas certeiros em duas cidades
israelenses: Afula e Hadera. Foi somente em maio de 1994 que a maré
começou a virar rumo à paz verdadeira: o acordo de implementação foi
assinado no Cairo e o rei Hussein finalmente decidiu fazer um movimento
em direção à paz com Israel.

Outra dificuldade foi a pressão de Washington em apressar as negociações de
um acordo entre Israel e Síria, preferência inicial do governo
americano. Rabin preferia seguir em frente com a Jordânia, que prometia
ser um acordo mais fácil, exigindo concessões limitadas. Mas levou um
tempo até que o rei Hussein absorvesse o choque do acordo de Israel com
a \textsc{olp} e então chegasse à conclusão de que sua melhor escolha era se
integrar por completo ao processo. O governo Clinton não se opunha, é
claro, a uma negociação de paz entre Israel e Jordânia (um importante
avanço por si só), mas queria que ela viesse depois, e não antes da
negociação com a Síria. Os americanos sentiam"-se em dívida com Assad,
porque achavam que ele tinha respondido de forma positiva à sinalização
de Rabin. Nem é preciso dizer que Assad, por sua vez, estava uma fera
quanto aos Acordos de Oslo. Ele se considerava o principal parceiro
entre os representantes do mundo árabe que negociavam com Israel. A seu
ver, os grandes avanços tinham que ocorrer nas negociações com ele, e
não com os palestinos. Assad nunca confiou em Arafat e sentiu"-se traído
por ele. Ironicamente, a questão da vinculação entre a Síria e os
palestinos tinha sido removida da agenda entre Israel e Síria. Assad
estava livre de qualquer compromisso com os palestinos e concentrava"-se
na Síria e no Líbano. Ele respondeu à solicitação americana, se absteve
de denunciar o acordo entre Israel e os palestinos e mandou seu
embaixador a Washington para participar da cerimônia de assinatura.
Rabin, por sua vez, estava preocupado com o ímpeto do governo Clinton de
apaziguar Assad e agilizar o progresso na frente síria. Na véspera da
cerimônia de assinatura em Washington, Clinton ficou trinta minutos no
telefone com Assad e compartilhou seu entusiasmo com Thomas Friedman,
influente colunista do \textit{New York Times}: ``A cada dia que o acordo
ganha mais força, acho que possibilita ao governo israelense o
comprometimento com a Síria. Particularmente, acredito que isso seja
muito mais importante do que esse pedaço de terra nas Colinas do Golã ou
qualquer outra coisa''.\footnote{Rabinovich, \textit{The Brink of Peace},
  \textit{op}. \textit{cit}., p. 117.}

Nos meses seguintes, prosseguiram os esforços americanos para fazer
avançar a frente síria e a pressão sobre Rabin, que continuava
relutante. Os Estados Unidos ainda acreditavam que um acordo entre
Israel e Síria era fundamental para uma mudança estratégica em toda a
região e para uma paz abrangente entre Israel e o mundo árabe. Se Assad
se frustrasse, poderia ir tudo por água abaixo. Rabin pensava diferente:
a seu ver, tinha acabado de concluir um acordo bastante controverso com
a \textsc{olp} e precisava dar um tempo para que ele funcionasse e para que o
público israelense o digerisse, antes de fazer avanços com a Síria. Já
dera uma chance a Assad, que acabou desperdiçada.

Em janeiro de 1994, Clinton foi a Genebra para se encontrar com Assad e
discutir a substância do futuro acordo entre Israel e Síria e a
diplomacia pública que Rabin considerava crucial. Rabin havia dito
inúmeras vezes a Clinton, Christopher e à equipe deles que, sem um
esforço real da parte de Assad para persuadir o público israelense de que
ele queria mesmo a paz, ele, Rabin, não conseguiria reunir apoio para um
acordo com a Síria. Sabia que Assad não era Sadat, e que não se podia
esperar dele um gesto comparável à viagem de Sadat a Jerusalém, mas ele
podia fazer muita coisa, como por exemplo falar a seu povo
explicitamente sobre a paz com Israel, convidar jornalistas israelenses
para a Síria e ajudar a esclarecer o destino dos soldados israelenses
desaparecidos em combate durante a guerra de 1982 no Líbano. Clinton e a
equipe de paz do seu governo fizeram de tudo para persuadir Assad a se aproximar
do público israelense durante a coletiva conjunta de imprensa. Porém,
foi o próprio Clinton quem teve de tomar a iniciativa e declarar que Assad
decidira"-se a fazer a paz com Israel. O resultado foi no mínimo
controverso. Rabin não se sensibilizou com a demonstração cautelosa de
Assad, ainda que positiva, nem com o esforço evidente de Clinton de
florear sua mensagem. Quando Ross e Indyk foram de Genebra a Jerusalém
para deixar Rabin a par da reunião, ele se mostrou frio e desdenhoso.

De todo modo, prometeu ao governo americano que renovaria as negociações
com a Síria assim que o novo relacionamento com os palestinos estivesse
consolidado. Sua preferência, contudo, era acelerar as negociações com a
Jordânia. O rei Hussein tinha ficado surpreso e irritado com os Acordos
de Oslo. A Jordânia nunca abandonara completamente suas reivindicações
sobre a Cisjordânia nem seu papel naquele território. Considerando os
mais de 50\% de palestinos que viviam na margem oriental
do Jordão, o rei e a elite política de seu regime estavam plenamente
cientes das repercussões que qualquer importante mudança política na
Cisjordânia poderia ter sobre o futuro do reino. Enxergavam o surgimento
da Autoridade Palestina autônoma na Cisjordânia e na Faixa de Gaza como
potencial ameaça à sua estabilidade. Mas havia outras considerações a se
levar em conta. A assinatura do acordo entre Israel e os palestinos, bem como o
reconhecimento mútuo entre Israel e a \textsc{olp} abriam o caminho para que a
Jordânia fizesse seu próprio acordo. Por mais que estivesse insatisfeita
com o acordo firmado entre israelenses e palestinos, a Jordânia percebeu
que as regras do jogo tinham mudado; era preferível participar do jogo e
influenciar seu destino do que ficar só olhando de fora.

A Jordânia começou, então, a avançar num ritmo cuidadoso. No dia 14 de
setembro de 1993, os chefes das delegações israelense e jordaniana nas
conversas de Washington, Elyakim Rubinstein e Abdul Salam Majali,
assinaram uma agenda comum, que na verdade já tinha sido acordada
previamente. Isso representava um modesto progresso e era
importante para demonstrar que as relações entre Israel e Jordânia
também estavam progredindo. Em 1\textsuperscript{o} de outubro, Hassan, o então príncipe
herdeiro, e Peres, ministro das Relações Exteriores, tiveram um encontro
público na Casa Branca. Na sequência, houve mais duas reuniões. No
início daquele mês, Rabin foi até a Jordânia para se encontrar com o rei,
num esforço de restabelecer a relação e redefini"-la, uma vez que em
grande parte ela vinha se pautando na oposição comum à \textsc{olp}. A segunda
reunião aconteceu quando Peres, com aprovação de Rabin, foi à Jordânia
no dia 2 de novembro, para se encontrar com o rei Hussein. Delineou"-se
um roteiro para o acordo de paz entre as duas partes e foi incluída na
agenda de negociações a ideia de que Israel participaria das
conferências econômicas do Oriente Médio, algo até então impensável.

Entretanto, Peres, que teve uma ótima conduta durante a reunião com o rei,
não conseguiu resistir à tentação de assumir os créditos pelos
progressos que vinham sendo feitos com a Jordânia. Ele estava numa maré
de sorte desde os Acordos de Oslo: tinha sido o arquiteto do acordo e
acabou levando o mérito que havia solicitado pelos avanços empreendidos.
O equilíbrio entre ele e Rabin pendeu a seu favor, e ele continuava
fazendo progressos. Ele não se esforçou para ocultar a reunião
supostamente secreta com o rei, e a mídia logo repercutiu o assunto. O
rei ficou exasperado e disse a Rabin que se ele queria fazer avanços com
a Jordânia, teria de manter Peres fora da jogada, o que não representou o menor
problema para Rabin. Era a oportunidade perfeita de retomar para si a
iniciativa de liderar o processo de paz. Atribuiu a responsabilidade das
negociações com a Jordânia a Ephraim Halevi, agente sênior do Mossad que
mantinha uma estreita relação pessoal com o rei Hussein, e a Rubinstein.
Este último, por sua vez, ficou ofendido por não ter sido informado sobre o canal de
Oslo enquanto negociava com Washington, mas, como leal funcionário do
Estado, evitou qualquer comentário público e concordou em continuar como
negociador com a Jordânia. O general Danny Yatom, assessor militar de
Rabin, e Eitan Haber, seu diretor de gabinete, trabalharam junto com
Halevi e Rubinstein.

A questão de qual acordo deveria vir primeiro --- se com a Síria ou a
Jordânia --- surgiu diversas vezes no toma lá dá cá entre Rabin e Peres e
seus homólogos norte"-americanos. Rabin tinha plena noção da preferência
dos Estados Unidos por um acordo com a Síria e do compromisso que
sentiam ter com Assad, mas deixou claro que sua prioridade era avançar
primeiro com a Jordânia. Seu argumento era simples: um acordo de paz
entre Israel e Jordânia seria mais fácil, porque as concessões exigidas
de Israel seriam de menor relevância. Se chegasse a um acordo para um
tratado de paz com a Jordânia, Rabin conseguiria apresentar ao público
israelense um feito grandioso e incontestável. Diante de um cenário em
que a direita israelense fazia duras críticas aos Acordos de Oslo, um
pacto com a Jordânia seria um grande trunfo político para seu governo. O
ponto crucial para o rei Hussein aconteceu no dia 4 de maio, com a assinatura do
acordo de implementação entre israelenses e palestinos. Até aquele
momento, não estava convencido de que o acordo se concretizaria. 
Em 28 de maio, o rei e Rabin se
encontraram em Londres e estabeleceram as bases para o progresso
definitivo.

Embora as negociações tenham sido conduzidas diretamente entre Israel e
a Jordânia, o rei agora queria envolver os Estados Unidos. Era a
oportunidade de normalizar sua relação com Washington, onde muitos ainda
o criticavam por sua conduta durante a Primeira Guerra do Golfo, quando
foi tido como aliado de Saddam. O rei Hussein também queria obter dos
Estados Unidos o alívio da dívida jordaniana, então disse ao governo
americano que estava disposto a dar um importante passo junto a Israel,
rumo à paz total. Depois de uma breve barganha, em 25 de julho foi
assinada nos jardins da Casa Branca a Declaração de Washington, pondo
fim ao estado de guerra entre Jordânia e Israel. As negociações ainda
levaram mais três meses para se completar, e o acordo de paz entre os
dois países foi por fim assinado no dia 26 de outubro de 1994, na região
fronteiriça entre Israel e Jordânia, ao norte de Eilat. O presidente
Clinton e uma delegação americana compareceram à cerimônia.

Depois da solenidade, Clinton seguiu para Amã e Damasco. A viagem para
Damasco tinha como objetivo apaziguar Assad, que fora deixado de lado
pela segunda vez, mas criou um obstáculo político para
Washington. Como o presidente poderia visitar um país que aparecia na
lista do Departamento de Estado como uma das nações que financiavam o
terrorismo? Chegou"-se à seguinte solução: na coletiva de imprensa que
aconteceria ao final da visita, um jornalista americano faria uma
pergunta a Assad relativa ao terrorismo e o presidente sírio aproveitaria
a oportunidade para deixar claro seu distanciamento desse tipo de ato.
Porém, como aconteceu diversas vezes quando a imprensa estava envolvida,
as esperanças americanas de transformar Assad e seu regime em parceiros
efetivos foram por água abaixo, causando enorme constrangimento. Assad
foi questionado em relação ao terrorismo, mas em vez de condená"-lo,
insistiu na ideia de que ``o que para alguns é um terrorista, para
outros é um combatente pela liberdade''. O episódio provocou um raro
incidente de caráter privado entre Clinton e Rabin. Quando Clinton foi a
Jerusalém, ficou hospedado no hotel King David, e Rabin foi
encontrá"-lo em sua suíte; Indyk e eu participamos do encontro para tomar
notas. A primeira coisa que Rabin fez foi levar Clinton até a sacada que
dava para as muralhas da Cidade Velha e discorrer sobre um dos episódios
estruturantes de sua vida: o fracasso na tomada daquele local, em
1948. De volta ao quarto, Clinton pediu a Rabin que o ajudasse a lidar
com o constrangimento ocorrido em Damasco; na coletiva de imprensa do
dia seguinte, queria que ele dissesse que, a seu ver, não tinha havido
problema algum na coletiva de Damasco. Rabin, como era de se esperar,
disse que não poderia fazer isso, pois não era verdade. Enrubescido,
Clinton se levantou, dizendo que iria embora naquele minuto. Rabin
prontamente mudou de ideia e se encarregou de apaziguar os ânimos. Na
coletiva de imprensa do dia seguinte, falou de maneira positiva sobre a
visita de Clinton a Damasco.

Naquelas semanas, continuou havendo progresso em outras frentes. Em
maio, Arafat, seu séquito e sua milícia deixaram Túnis e começaram a
controlar a \textsc{ap} em Gaza e Jericó. No final de abril, Rabin cumpriu o
compromisso assumido com Christopher, o secretário de Estado,
convidando"-o para ir até a região e retomar as negociações entre
israelenses e sírios. Christopher foi primeiro a Damasco e, quando chegou
a Jerusalém, trazia más notícias: Assad agora insistia que a ``retirada
total'' significava recuar às linhas de 4 de junho de 1967, e não à
fronteira internacional de 1923 estabelecida entre o Mandato da Síria e
o Mandato da Palestina. A diferença não era grande em termos de
quilômetros quadrados, mas o que havia de importante era o significado
simbólico e a questão da soberania sobre o lago Tiberíades. A principal
motivação por trás da demanda de Assad era que ele queria superar Sadat.
Se o presidente egípcio conseguira de volta todo o Sinai, ele queria
ainda mais. Christopher se deu conta de que a demanda de Assad se
sustentava em bases pouco sólidas, mas, da perspectiva de uma
superpotência, a diferença entre as duas linhas parecia insignificante.
Esperava que Rabin aceitasse ampliar sua proposta original, garantindo
as linhas de 4 de junho.

Rabin se atormentou com esse dilema por várias semanas. As linhas de 4
de junho de 1967 dariam à Síria acesso direto ao lago Tiberíades,
principal reservatório hídrico de Israel. Do ponto de vista de Rabin, a
demanda de Assad não era legítima e tornaria ainda mais controversas as
concessões envolvendo um acordo sírio"-israelense. Por fim, acabou
autorizando Christopher a dizer a Assad que ``sua impressão'' era de que
Rabin aceitaria sua demanda. Assad então concordou em ampliar o formato
da negociação. Foi estabelecido, assim, um ``canal entre embaixadores'':
entre Walid Muallem, embaixador sírio em Washington, e eu. Como Assad
insistiu que nos reuníssemos apenas na presença de um diplomata
americano, Ross e Indyk se juntaram a nós. As reuniões foram boas.
Muallem queria fazer um acordo, especialmente com Washington, e muitos
temas foram contemplados. Porém, cheguei à conclusão de que só
conseguiríamos resolver as espinhosas questões de segurança se
fizéssemos reuniões com o alto comando militar. Foi necessária mais
pressão por parte dos Estados Unidos para que Assad concordasse em enviar
o chefe do Estado"-Maior do exército sírio, Hikmat Shihabi, para se
encontrar com seu homólogo israelense. Rabin enviou Ehud Barak, que
terminaria seu mandato como chefe do Estado"-Maior das \textsc{idf} em dezembro.

Rabin queria que a reunião entre os chefes de Estado"-Maior ficasse em
segredo, pois sua divulgação criaria em Israel a impressão de que um
acordo com a Síria era iminente, o que acabaria mobilizando a oposição.
Mas os sírios optaram por divulgá"-la, provavelmente para dar a ideia de
que a Síria estava no caminho de recuperar o Golã. Isso foi aproveitado
pelo movimento que crescia dentro de Israel contra a retirada. A
dimensão dos protestos e da oposição dentro do país era desproporcional
ao limitado progresso efetivo nas negociações com a Síria. Os colonos do
Golã tinham conseguido orquestrar uma campanha popular com o lema ``O
povo está com o Golã''. Em junho de 1994, foi criado um movimento
chamado Terceira Via, composto majoritariamente por membros e apoiadores
do Partido Trabalhista que criticavam o que acreditavam ser uma guinada
à esquerda da parte de Rabin, tanto na frente síria quanto na frente
palestina. Mas seu foco principal era a oposição à retirada do Golã.
Rabin ficou pessoalmente ofendido quando vários de seus amigos do
Palmach e colegas das \textsc{idf} se juntaram ao movimento. Eles criticavam a
estratégia do primeiro"-ministro, mas também se mostravam ofendidos pelo
fato de que, à medida que ele foi crescendo na hierarquia política, acabou se
distanciando dos antigos companheiros e passou a ter uma vida social com
um novo círculo de amigos ou, melhor, de conhecidos. O desafio político
se exacerbou ainda mais quando dois membros trabalhistas do Knesset
também passaram a integrar o movimento. A coalizão de Rabin estava por
um fio. Em setembro de 1994, os colonos do Golã tentaram passar no
Knesset a Lei de Entrincheiramento do Golã. Isso exigiria uma maioria
qualificada tanto para revogar a lei do Knesset de 1981 que aplicara a
lei israelense ao Golã quanto para aprovar o referendo que Rabin havia
prometido na eventualidade de um acordo com a Síria, incluindo a
retirada do Golã. Por sorte, Rabin conseguiu extinguir essa iniciativa,
mobilizando um número suficiente de votos no Knesset.

As dificuldades com a \textsc{ap} e o terrível impacto dos ataques terroristas
felizmente foram ofuscados pela crescente normalização das relações de
Israel com os países árabes do Golfo e do Norte da África. Realizou"-se
uma segunda Conferência Econômica do Oriente Médio, dessa vez na
Jordânia, com participação de uma grande delegação israelense. Desde
1948, uma das principais armas usadas pelo mundo árabe contra Israel
depois da derrota militar que sofreram era a rejeição da normalidade, ao
exercerem boicotes diretos e indiretos. Portanto, o contexto de
conferências econômicas no Oriente Médio em que representantes árabes e
israelenses e homens e mulheres de negócio se esbarravam uns nos outros
era uma manifestação impressionante do sucesso do processo de paz dos
anos 1990.

\section{Tendências contraditórias em 1995}

O processo de paz em 1995 era semelhante ao de 1993. Israel estava
negociando ao mesmo tempo com a Síria e com os palestinos e havia
dificuldades em ambas as frentes. Rabin jogava uma contra a outra e
acabou fazendo um segundo grande acordo com os palestinos, deixando Assad
de fora, furioso. Mas esse resultado não foi apenas consequência de uma
manipulação inteligente da parte de Rabin. Assad provou ser um parceiro
de negociações extremamente difícil, o que levou Rabin a concluir, por
fim, que o presidente sírio não estava interessado num acordo ou,
julgando com mais leniência, não estava suficientemente interessado a
ponto de fazer com que uma tratativa tradicional funcionasse. As
negociações entre israelenses e palestinos que levaram ao acordo de Oslo
\textsc{ii} concentraram"-se na expansão do domínio da \textsc{ap} sobre outros territórios
da Cisjordânia, para além do território original dentro e ao redor de
Jericó. Foram negociações demoradas e difíceis, que acabaram levando à
criação das áreas \textsc{a}, \textsc{b} e \textsc{c}
na Cisjordânia. A área \textsc{a} abrangia as
principais cidades da Cisjordânia; a maior parte da população nessa área
ficaria sob controle total da \textsc{ap}. A área \textsc{b} ficaria sob controle civil da
\textsc{ap}, mas sob controle militar de Israel. Já a área \textsc{c}, uma região grande e
pouco povoada, permaneceria sob controle israelense.

Assinado na Casa Branca em 24 de setembro de 1995, o acordo representou
um passo fundamental no caminho para um acordo definitivo entre Israel e
os palestinos. Segundo os que foi acordado inicialmente em Oslo, as negociações para
um acordo definitivo deveriam terminar no prazo de cinco anos a partir
da conclusão do pacto de implementação assinado em maio de 1994. Tanto
quem era favorável ao processo de paz entre israelenses e palestinos
quanto quem era contrário a ele entendia a magnitude das implicações de
Oslo \textsc{ii}. Dentro de Israel, a oposição à assinatura desse acordo atingiu
um nível inédito, sem precedentes, que traria consequências letais.

A frente síria, por sua vez, chegava a um beco sem saída. Depois da
reunião entre Barak e Shihabi, Assad passou a insistir que antes de outra
reunião fossem tratados ``os princípios dos dispositivos de segurança''
e que os dois lados tivessem tratamento igual. Enquanto Rabin
argumentava que os dispositivos de segurança deveriam compensar Israel
por abrir mão do controle das posições elevadas, Assad dizia que a Síria
era a parte sob ameaça. Foram meses e meses de exercícios linguísticos
antes de chegarem a uma fórmula de compromisso. No final de junho de
1995, Assad autorizou Shihabi a viajar para se encontrar com o novo chefe
do Estado"-Maior das \textsc{idf}, Amnon Shahak. Correu tudo bem na reunião, mas
quando Shihabi voltou a Damasco, Assad mais uma vez resolveu dizer que o
encontro não tinha sido bom. Foi nesse momento que Rabin concluiu que o
presidente sírio não estava de fato interessado em chegar a um acordo.
Assad se queixava do formato do processo de paz. Segundo ele, alguns
encontros sírio"-israelenses de alto nível não chegaram a produzir
avanços reais, mas acabaram legitimando a normalização das relações
árabe"-israelenses. Assad achava que Rabin lhe passara a perna. Sua
crítica mais dura ao processo de paz saiu numa entrevista concedida ao
jornal egípcio \textit{Al"-Ahram}, em 11 de outubro: ``Os israelenses estão
se aproveitando desses processos para exercer pressão sobre as outras
partes. Oslo exerceu pressão na Jordânia, e a Jordânia e a Palestina
exerceram pressão em outros {[}leia"-se, na Síria{]} {[}\ldots{}{]} Nós estamos no
processo de paz, mas não contribuímos para o esforço de exercer pressão
sobre nós mesmos''.

As críticas de Assad eram compartilhadas por proeminentes intelectuais
árabes, que cunharam o termo \textit{taharwul} {[}corrida{]} para
denunciar a pressa de alguns países árabes em reconhecer Israel e fazer
negócios com o país. Assad também encontrou um inesperado aliado no
governo do Cairo. O Egito tinha chegado a um acordo de paz com Israel em
1979, mas mantivera"-o como ``paz fria'' por conta do fracasso na
implementação do componente palestino. Já que as negociações entre
Israel e a \textsc{ap} estavam avançando, o Egito poderia, em tese, tirar da
geladeira as relações com Israel. Mas não foi o que aconteceu. O Egito
enxergava Israel como concorrente na luta pela hegemonia regional e
temia que uma normalização nas relações árabe"-israelenses pudesse
catapultar Israel a uma posição hegemônica. A ampla delegação israelense
que foi à primeira conferência econômica regional em Casablanca, no
Marrocos, serviu para alimentar essa ansiedade. O Egito decidiu usar a
questão nuclear para desacelerar a normalização. Tradicionalmente, o
Cairo tinha mantido uma postura contrária ao programa nuclear israelense, mas agora
se pronunciava energicamente contra a tentativa de Washington de
estender o Acordo de Não Proliferação, que expiraria em 1995. O governo
egípcio insistia na obrigatoriedade da inclusão de Israel no tratado.

No dia 5 de outubro de 1995, numa sessão especial do Knesset convocada
para ratificar Oslo \textsc{ii}, Rabin fez um discurso importantíssimo, no qual
explicou a essência do acordo e traçou seu roteiro para as fases
subsequentes da relação entre Israel e os palestinos. Depois de
descrever o acordo como ``um avanço significativo para solucionar o
conflito palestino"-israelense e pôr fim a décadas de terror e sangue'',
explicou seu ponto de vista sobre a solução permanente:

\begin{quote}
Nós enxergamos a solução permanente no seguinte contexto: o Estado de
Israel continuará com a maioria do território da Terra de Israel segundo
seus contornos sob o Mandato Britânico, e a seu lado a entidade
palestina será o lar da maioria dos palestinos que vivem na Faixa de
Gaza e na Cisjordânia. Queremos que essa entidade seja menos do que um
Estado e que controle, com independência, a vida dos palestinos que estejam
sob sua autoridade. As fronteiras do Estado de Israel na época da
solução permanente estarão para além das linhas de antes da Guerra dos
Seis Dias. Nós não retornaremos às linhas de 4 de junho de 1967.
\end{quote}

Rabin ainda mencionou as áreas que seriam acrescentadas ao território
israelense no acordo definitivo: a Grande Jerusalém, uma ampla fronteira
de segurança no Vale do Jordão, algumas áreas específicas perto de
Jerusalém, a leste da Linha Verde, e ``blocos de assentamentos
judaicos''. Explicou que seu governo escolhera a opção de um Estado
judeu, em detrimento da opção da Grande Israel. Os palestinos, disse
ele, ``não representaram, no passado, nem representam hoje uma ameaça à
existência do Estado de Israel''. A principal ameaça à implementação do
processo de paz com eles sempre veio das organizações terroristas. Sob a
liderança de Arafat, a \textsc{olp} parou de se envolver com o terrorismo, mas a
\textsc{ap} precisava fazer muito mais do que vinha fazendo contra esse tipo de
organização.

Rabin descreveu detalhadamente os planos 
para as áreas \textsc{a}, \textsc{b} e \textsc{c}, bem
como o processo de realocação gradual pelas \textsc{idf}. Este seria
dividido por fases, para permitir que Israel verificasse o desempenho da
\textsc{ap}. Ele manifestou sua decepção com a \textsc{ap}, por não ter conseguido até
aquele momento modificar a Carta Nacional Palestina e declarou que ``as
mudanças na Carta Palestina serão um importante critério para a decisão
de avançar com a implementação do acordo em sua totalidade''. Rabin
examinou em detalhe os demais aspectos do acordo e encerrou num tom
cauteloso: ``Ao que tudo indica, hoje entramos numa nova fase da
história do povo judeu e do Estado de Israel. Estamos cientes das
perspectivas e também dos riscos''. Agradeceu à equipe que tinha
participado da negociação do acordo e fez um agradecimento específico a
Peres. O gesto inaugurou uma nova fase na relação entre os dois. Em
junho, Peres se comprometeu por escrito a não competir com Rabin pela
liderança. Rabin, por sua vez, o aceitou como segundo homem na linha de
comando e ambos concordaram quanto ao caminho a ser seguido.

\chapter[Política, politicagem, incitação e assassinato, 1992--95]{Política, politicagem, incitação\\ e assassinato, 1992--95}
\markboth{Política, politicagem\ldots{}}{}

O genuíno desejo de Rabin de deixar um legado em seu segundo mandato
representava um desafio para um governo dependente de uma pequena
maioria. Com o Shas na coalizão, ele possuía uma maioria mínima de 62
mandatos. Os membros árabes do parlamento podiam ser levados em conta
para apoiar o processo de paz, mas o Shas era um sócio duvidoso. O rabino Ovadia Yosef, líder espiritual do Shas, publicou uma interpretação
segundo a qual era permitido devolver partes da Terra de Israel para
salvar vidas judaicas. Em outras palavras, deu prioridade à santidade da
vida sobre a santidade da terra --- em forte contraste com a doutrina do
Gush Emunim, que favorecia a Terra de Israel frente ao Estado de Israel
e, de certa forma, também seus habitantes. Mas os membros do Shas e a
maioria de seus eleitores eram radicais e, no final, o seu apoio ao
processo de paz tendia a ser frágil. Rabin também tinha que se preocupar
com as alas direita e esquerda dos 44 membros do Knesset
pertencentes ao seu próprio partido; seu apoio não estava garantido. Em
outras palavras, certo esforço e uma atenção permanente e concentrada eram
necessários para garantir a sobrevivência do governo. A ligeira maioria
com que contava o governo de Rabin demonstrou ser suficiente para tomar
e implementar decisões audaciosas, mas empenhos intermináveis eram
necessários para garantir uma maioria parlamentar. Uma das razões para o
senso de urgência nas semanas que antecederam os Acordos de Oslo era a
investigação criminal de Aryeh Deri, o líder político do Shas. Ele era
suspeito da prática de corrupção e foi, posteriormente, condenado.
Parecia provável que sua condenação retiraria seu partido da coalizão.
Quando votados no Knesset, os Acordos de Oslo foram aprovados por
uma maioria de 61 votos contra cinquenta, oito abstenções e
uma ausência. Os membros do Shas se abstiveram, assim como três membros
do Likud que decidiram ``dar uma chance a Oslo''. Em setembro, o Shas
abandonou a coalizão.

Os Acordos de Oslo e suas sequelas afetaram de maneira contraditória o
público israelense e seu sistema político. As tensões subjacentes entre
os grupos políticos de tendência de direita e esquerda exacerbaram"-se
com o significativo avanço do processo de paz. Apoiadores do acordo
foram galvanizados pelo avanço nas relações com os palestinos, a
expectativa de um avanço similar com a Síria, a paz com a Jordânia, a
dramática melhora da situação israelense no cenário internacional e
outras importantes manifestações da normalização das relações de Israel
com o mundo árabe em geral, como a participação de uma grande delegação
israelense na conferência econômica de Casablanca em outubro de 1994.
Esse segmento do povo israelense sentia que Rabin estava conduzindo o
país rumo a uma transformação de sua condição regional e internacional,
de um familiar estado de sítio para uma existência normal. A direita e
seus apoiadores, por outro lado, estavam horrorizados com as concessões
feitas em Oslo e que possivelmente seriam feitas em acordos futuros. Mas
o maior dano ao apoio popular ao processo de paz foi causado pela
série de atentados terroristas perpetrados pelo Hamas e pela Jihad
Islâmica, lançados contra cidades israelenses a partir de abril de 1994.
Para se ter uma ideia da escalada, é necessário comparar um pequeno
ataque terrorista em 1993 com os cinco ataques mortais em 1994 e outros
cinco em 1995. O impacto psicológico foi terrível. Cerimônias festivas
de assinatura dos acordos e a imagem de bandeiras israelenses hasteadas
no Golfo e nos países árabes do Norte da África eram uma coisa; a
possibilidade de ser morto em um ônibus no meio de uma cidade israelense
era outra, muito diferente. A acusação da direita, de que entregar a
segurança israelense nas mãos de Arafat e de sua Autoridade Palestina
era um ato temerário, convenceu a muitos. Dois termos e \textit{slogans}
transmitem a sensação do amargo debate travado naquela época sobre esses
temas. Um dos termos era ``vítimas da paz'', usado de forma sarcástica
pelos críticos da direita. O outro, ``não lhes deem armas'', uma linha
tomada de um poema da década de 1930, de autoria do poeta nacional israelense Nathan
Alterman, era utilizada pelos críticos dos Acordos de Oslo para reiterar o
fato de que parte da responsabilidade pela segurança israelense estava
agora nas mãos dos palestinos.

A oposição ao processo de paz e o desafio político à maioria de Rabin
intensificaram"-se em 1995, alimentados por duas fontes. Uma era a
crescente oposição a um acordo com os sírios, envolvendo uma retirada do
Golã. Era liderada pelos colonos do local, que tinham considerável apoio
no país, assim como no Partido Trabalhista e no novo movimento chamado a
Terceira Via. Em julho de 1995, uma moção foi apresentada no Knesset,
requerendo uma maioria especial, caso houvesse uma votação para aprovar
a retirada do Golã. Os assessores de Rabin conseguiram convencer dois
membros do partido direitista Tzomet, e o resultado foi um empate por
sessenta votos. A moção não foi aprovada, mas era uma clara indicação da
fragilidade da maioria parlamentar de Rabin.

Muito mais ameaçadora foi a agitação provocada pelos colonos da
Cisjordânia e a direita em geral, à espera da assinatura do acordo Oslo
\textsc{ii}. Eles sabiam que a assinatura representaria um importante avanço para
transformar os Acordos de Oslo em uma nova realidade, colocando a área
central da Cisjordânia sob controle de Arafat e da Autoridade Palestina.
A incitação contra o governo de Rabin e contra ele pessoalmente
começaram a atingir novos patamares, mais perigosos. A batalha contra o
acordo com os sírios deslocou"-se do Knesset para as ruas e praças das
cidades.

É impossível separar o processo de Oslo da complexa relação entre Rabin
e Peres. Os Acordos de Oslo serviram para ilustrar o que ambos podiam
conseguir trabalhando em conjunto. Peres trouxe a iniciativa ousada e
Rabin o escrutínio cuidadoso e a habilidade de engajar o público
israelense. Mas a velha rivalidade entre ambos, temporariamente
posta de lado com sua colaboração durante a primeira parte do segundo
mandato de Rabin, ressurgiu quando Peres tentou explorar sua
popularidade pós"-Oslo ao assumir o controle sobre as negociações
bilaterais, nos bastidores, com a Jordânia. Rabin deu cabo delas e
assumiu o controle das negociações. Após a assinatura do acordo de
não beligerância com a Jordânia, em julho de 1994, Rabin humilhou Peres
ao protestar contra ele na presença de vários jornalistas israelenses.
Em dezembro de 1994, Rabin e Peres foram agraciados com o Prêmio Nobel da
Paz, juntamente com Arafat. O comitê Nobel foi suficientemente astuto
para dividir o prêmio, mantendo velada a disputa. Mas Rabin e Peres
entenderam e aceitaram o fato de que estavam ligados de forma
indissolúvel --- aprenderam a viver e trabalhar em conjunto. Em junho de
1995 formalizaram o acordo, quando Peres aceitou a liderança de Rabin e
Rabin o aceitou como seu vice. Esse acordo foi fortalecido pelo avanço
que se desenhava, manifestado pela assinatura de Oslo \textsc{ii}.

\section{Políticas domésticas}

O segundo mandato de Rabin, entre 1992 e 1995, é e será lembrado
primordialmente por sua atuação em prol da paz, mas também foi um
governo que promoveu vigorosas políticas domésticas, parte de sua promessa
de reorganizar as prioridades nacionais. Foi um período de
rápida expansão econômica. No início de seu mandato, ele viajou aos
Estados Unidos e acordou com o governo Bush a liberação de uma garantia
de dez bilhões de dólares para empréstimos, o que havia sido recusado durante o
governo Shamir. Rabin também tomou rapidamente a decisão de congelar as
construções nos assentamentos. Era uma declaração política com
repercussões econômicas: os fundos alcançados através das garantias de
empréstimo e os obtidos com a economia na suspensão da construção nos
assentamentos foram investidos em novas prioridades. O orçamento da
educação cresceu 70\%, o sistema de educação superior expandiu"-se e novas faculdades
ofereceram maior acesso à educação universitária. Houve também um investimento maciço em novas estradas --- especialmente uma que cruzava todo o país ---, pontes e
acessos. O orçamento do
cientista"-chefe no Ministério da Indústria e Comércio foi reforçado para
permitir um maior investimento em pesquisa e desenvolvimento. E um
grande fundo de investimento, Yozmá {[}Iniciativa{]}, foi criado como
uma entidade estatal e depois privatizado. Ambas medidas tiveram papel
central no desenvolvimento da indústria israelense de alta tecnologia.

À medida que se acelerava o processo de paz, aumentava a sensação de
otimismo em relação ao futuro de Israel, dentro e fora do país. Os
investimentos estrangeiros cresceram dramaticamente. Empresas
multinacionais como Volkswagen e Nestlé investiram no mercado
israelense. Em menos de quatro anos, o investimento estrangeiro cresceu
de 180 milhões de dólares por ano para quase seis bilhões. A economia
israelense crescia a uma taxa anual de 6\% (exceto em 1993, devido à
moratória na construção dos assentamentos).

O ministro das Finanças de Rabin, Avraham (Baiga) Shochat, era um
importante auxiliar e aliado, mas o tema do desemprego era um foco de
conflito entre eles. Rabin dava grande importância ao combate ao
desemprego --- que em 1992 havia alcançado a marca de 11\% --- e exigiu que
Shochat promovesse projetos apoiados pelo Estado para diminuí"-lo.
Shochat e seus ``meninos das Finanças'' compartilhavam da visão de Rabin,
mas insistiam em apoiar somente projetos que tivessem ao menos algum
valor agregado. De qualquer forma, o desemprego foi reduzido a 6,5\%,
apesar da necessidade de integrar na força de trabalho a recente leva de
imigrantes oriundos da antiga União Soviética. Relacionava"-se ao tema do
desemprego a pressão exercida pela área econômica para abrir ao mundo a
economia e o mercado israelenses. Eles não viam vantagem em proteger
fabricas primárias, têxteis e de madeira compensada e argumentavam que a
abertura desses setores da economia a importações competitivas reduziria
o custo de vida em Israel e forçaria a indústria israelense a melhorar o
nível de seus produtos. Como alguém que havia crescido no seio do
Partido Trabalhista israelense, Rabin sentia ser parte de sua herança a
defesa dos trabalhadores. Mas Rabin também era influenciado pelo período
que passou nos Estados Unidos e entendia a necessidade de integrar a
economia israelense à economia global.

O segundo mandato de Rabin é visto pela minoria árabe israelense como um
período glorioso. Inicialmente, ele não foi bem recebido por esse
segmento, que dele se lembrava como o agressivo ministro da Defesa que
conteve a Primeira Intifada com violência. Mas, como primeiro"-ministro,
Rabin destinou recursos consideráveis ao setor. As alocações para
crianças de famílias árabes foram equiparadas às das famílias judias
(politicamente isso foi compensado com o aumento das alocações para os
soldados das \textsc{idf} ao final de seu serviço militar obrigatório). Os
recursos para o setor árabe foram alocados direta ou indiretamente
através do orçamento da educação e investimentos em infraestrutura nas
áreas árabes, incluindo a construção de estradas e a extensão do
suprimento de eletricidade.

\section{Crônica de um assassinato anunciado}

Em algum momento no final de 1994, Yehoshafat Harkabi, um professor
aposentado da Universidade Hebraica de Jerusalém, ex"-diretor do Serviço
Militar de Inteligência Israelense (Aman, no acrônimo em hebraico), e um dos mais proeminentes intelectuais
públicos, fez uma previsão profética e sombria conforme observava a
agitação promovida pela oposição, parlamentar e extra"-parlamentar,
contra Rabin e sua política: ``O debate interno será terrível, haverá
tentativas de assassinato, Rabin não morrerá de morte natural e o país
será afetado por um choque terrível. Alguns dirão que não estavam
enganados, que Rabin era culpado por ser demasiado generoso''.\footnote{Citado em Karpin
\& Friedman, \textit{Assassinato em nome de Deus: o plano para matar Yitzhak
Rabin}, \textit{op}. \textit{cit}., p. 9--10.}

O jornalista francês Victor Cygielman escreveu um artigo para o \textit{Le
Nouvelle Observateur}, publicado em 2 de novembro de 1995, dois
dias antes do assassinato de Rabin. Nele é relatada uma série de atos
violentos, formas extremas de incitação e cerimonias sinistras
realizadas por ativistas da direita radical: Rabin foi denunciado como
traidor, implícita e explicitamente; rabinos radicais declararam"-no
culpado de atos passíveis de morte e assassinato; ele foi exibido em
pôsteres vestindo um uniforme da \textsc{ss} nazista ou o turbante árabe e
retratado em atos equivalentes à realização de uma premonição. Cygielman
chegou à conclusão de que ``o palco havia sido preparado para o assassinato
de Rabin, e um atentado contra ele era somente questão de
tempo''.\footnote{\textit{Ibid}., p. 10. Esse livro, em conjunto com as obras de Yoram Peri, \textit{Irmãos na guerra} (Tel Aviv: Bavel, 2005) e de Carmi Gillon, \textit{A Shin Bet entre os cismas} (Tel
Aviv: Yedi'ot Achronot Books \& Chemed Books, 2000) {[}ambas em hebraico{]}, e várias entrevistas orais são a principal fonte para essa
passagem.} Realmente não era necessário ser um visionário para
perceber o curso dos eventos que levaram ao assassinato de Rabin. Após a
assinatura dos Acordos de Oslo, esses eventos se desenrolaram em várias
etapas: oposição política legítima e ilegítima; desacreditar e
deslegitimar o governo e seu líder; desumanização do rival político;
conduta simbólica e ritual assassino; conduta política violenta e,
finalmente, o assassinato.\footnote{Peri, \textit{Brothers at War}, \textit{op}.
\textit{cit}., p. 38.}

A oposição ao governo de Rabin em 1993 era liderada por Benjamin
Netanyahu, que fora eleito líder do Likud após as eleições de 1992. A
oposição, assim como o público israelense, ficou surpresa e chocada com
os Acordos de Oslo, e levou algum tempo até que se organizassem e
implementassem uma oposição legítima às políticas de governo. O Likud
sentia"-se fraco e juntou"-se a uma coalizão mais ampla que agregava a
direita israelense, uma aliança administrada por uma entidade
denominada Joint Staff e mais tarde Mathe Ma'amatz {[}Central do Esforço{]}.
Era composta pelo Conselho Yesha, a organização que representava os colonos
de Gaza e da Cisjordânia, pelo Likud e três outros partidos de direita, e
por vários representantes dos partidos religiosos e de grupos
extraparlamentares. O papel principal era o do Conselho Yesha, e a
coordenação entre eles e o Likud era responsabilidade de dois ativistas
do Likud, Tzachi Hanegbi e Reuven Tzadok. O Joint Staff organizava
demonstrações não autorizadas, marchas, vigílias e ações junto à mídia.
Em 1994, a oposição legítima foi transformada em uma série de atos que
visavam deslegitimar o governo e suas políticas, incluindo violentas
manifestações não autorizadas, perturbação de eventos organizados pelo
governo ou com a participação de Rabin, além de bloqueios de rodovias. A
estrutura flexível do Joint Staff permitia que seus líderes cooperassem
com grupos tais como os dos apoiadores de Meir Kahane (um judeu
norte"-americano ortodoxo radical que havia importado para Israel seu
movimento de ideologia ultranacionalista dos Estados Unidos), mantendo
uma negação plausível dessas posições.

Era óbvio nessa etapa que o governo e a oposição estavam em rota de
colisão. Rabin estava determinado a implementar os Acordos de Oslo e
avançar com os desdobramentos do processo. O movimento dos colonos
estava determinado a obstruí"-lo. Segundo eles, os Acordos de Oslo
ameaçavam não só a segurança e a existência do país mas também o
projeto de assentamento, no qual haviam investido suas vidas e sua
identidade. Sua oposição era derivada de uma profunda crença religiosa
na santidade da terra e na absoluta rejeição da demanda palestina por
ela. Conforme vimos, na visão do Gush Emunim e na teologia de seus
rabinos, a Terra de Israel é superior ao Estado de Israel e nenhum
governo israelense tem autoridade para abrir mão de qualquer parte dela.
Qualquer governo disposto a fazê"-lo era, por definição, ilegítimo. Já na
década de 1970 Rabin havia expressado sua percepção do movimento
político dos colonos como uma ``entidade cancerosa''. Em 1984, foi
exposta a assim denominada ``Resistência Judaica'', composta de colonos
radicais. Seus membros tentaram assassinar vários prefeitos da
Cisjordânia, e a ala mais radical conspirou para explodir a mesquita de
Al"-Aqsa, no Monte do Templo. Os membros do movimento clandestino vinham
da população de colonos, mas desviaram"-se das ideias originais,
radicalizando"-se. Foram detidos, julgados e presos, mas, em
1988, foram indultados como parte do acordo de coalizão que permitiu a
Yitzhak Shamir formar seu governo. O indulto indicava a atitude
tolerante da ala direita moderada e até de alguns elementos do Partido
Trabalhista em relação ao Gush Emunim, que os viam como novos pioneiros.
Os líderes do Gush eram exímios manipuladores do sistema governamental e
político e muito bons em contornar a lei, maximizando a reticência do sistema em
enfrentá"-los. O dinheiro era transferido a eles por governos israelenses
de direita, e todos os governos eram passivos ou impotentes quando
novos assentamentos eram criados ou quando se expandiam os já existentes. Em 1993,
esses líderes viam os Acordos de Oslo como a mais perigosa ameaça a seu
projeto e estavam claramente dispostos a fazer de tudo para obstruí"-los.
Rabin concordou, relutante, em participar de um encontro com os colonos
e tratar de chegar a um entendimento. O encontro foi um fracasso e os
esforços para manter o diálogo foram abandonados.

No final de 1993, o conselho dos colonos lançou uma campanha que tinha por objetivo
``acabar com o ânimo'' de Rabin. Seu modelo estava baseado no colapso de
Begin em 1983, logo após a Primeira Guerra do Líbano.\footnote{Karpin \& Friedman, \textit{Assassinato em nome de Deus},
\textit{op}. \textit{cit}., p. 96--102.} Quando
a tática falhou, a campanha contra Rabin se intensificou e ele passou a
ser chamado de traidor e exibido em cartazes usando um turbante árabe.
Mais tarde, analogias ao Holocausto foram introduzidas no discurso
público. O governo de Rabin foi apelidado de \textit{Judenrat}, o termo
pejorativo usado para judeus que haviam colaborado com os nazistas. E a
campanha se radicalizou com o impacto dos atentados suicidas perpetrados
pelo Hamas. Em 19 de outubro de 1994, 22 israelenses foram
mortos e dezenas ficaram feridos quando um ônibus explodiu na rua Dizengoff, em
Tel Aviv. O líder da oposição, Netanyahu, visitou o local e, frente às
câmaras de \textsc{tv}, acusou Rabin pela tragédia: ``O primeiro"-ministro optou
por dar preferência a Arafat e o bem"-estar dos habitantes de Gaza em
detrimento da população israelense''. Mas sua tática foi um tiro que
saiu pela culatra. Netanyahu foi severamente criticado por romper com o
código de conduta da política israelense segundo o qual vítimas,
militares e civis não devem ser exploradas com objetivos políticos. Ele
continuou a criticar e denunciar a Rabin e seu governo, mas absteve"-se
de aparecer nos locais em que ocorreram novos atentados.

Mas a incitação contra Rabin crescia a passos largos, tanto em
intensidade quanto em número de ações. Uma rádio pirata, Canal 7, que
transmitia de um ponto mais além das águas territoriais israelenses,
usava linguagem e terminologia das mais duras e ofensivas. Rabin era
acusado de bizarro, traidor, assassino e bêbado.\footnote{\textit{Ibid}., p. 121.} O
boletim \textit{Nekuda} {[}Ponto{]} dos colonos publicou, em julho de 1995, que Rabin
``atua como um ditador {[}\ldots{}{]} é imprudente {[}\ldots{}{]} doente.''\footnote{Citado em Peri, \textit{Brothers at War}, \textit{op}. \textit{cit}., p. 39.} Um
psicólogo de direita publicou um ``relatório'' do perfil psicológico de
Rabin, no qual determinou ser ele ``um esquizoide alheio à realidade''.
Sharon, amigo de Rabin no passado, juntou"-se à briga também utilizando
terminologia emprestada do Holocausto e dos crimes perpetrados por
Stalin. Sharon chamou o governo de Rabin de ``governo insano que limita
Israel às fronteiras de Auschwitz, um governo imprudente, submisso,
confuso, traiçoeiro, insano''.\footnote{Citado em Karpin \& Friedman, \textit{Assassinato em nome de Deus}, \textit{op}. \textit{cit}., p. 123--32, e também em Peri, 
\textit{Brothers at War}, \textit{op}. \textit{cit}., p. 39--47.} O colega de Rabin do Palmach, Rehavam Ze'evi, competia com Sharon em seus ataques verbais contra o
primeiro"-ministro.

No início do verão de 1995, um novo elemento foi introduzido quando um
movimento denominado Zo Artzeinu {[}Esse é Nosso País{]} tentou bloquear o
processo de Oslo por meio da desobediência civil. O movimento era
liderado por um colono chamado Moshe Feiglin, cujos pais haviam imigrado
da Austrália. Muitos de seus associados eram imigrantes da França e dos
Estados Unidos. Zo Artzeinu era um movimento organizado e dispunha de
recursos, conseguindo criar o caos em Israel ao bloquear rodovias e
cruzamentos, mas não conseguira impedir a assinatura dos Acordos
de Oslo. Feiglin finalmente juntou"-se ao Likud, registrando um grande
número de colonos e outros direitistas radicais para as primárias do
partido e cumprindo assim um papel importante na guinada do Likud para a
direita.

Em 1994, rabinos extremistas nos assentamentos da Cisjordânia e nos
Estados Unidos introduziram dois termos radicais no discurso da direita
radical: a ``Lei do Perseguidor'' (Din Rodef) e a ``Lei do Informante''
(Din Moser). Ambas foram adotadas da longa história dos judeus na
diáspora e sob jugo estrangeiro, e ambas sancionavam o assassinato de um
judeu que perseguisse ou processasse outros judeus, ou os entregasse às
autoridades gentias. O projeto foi administrado cuidadosamente: o grupo
de rabinos que lidou com os dois termos sabia muito bem que podia ser
acusado de atividade ilegal e incitação ao assassinato e era, portanto,
cauteloso e evasivo ao utilizar os termos. Eles foram mais tarde
pressionados por um rabino moderado dos assentamentos, Yoel Ben"-Nun que,
chocado com o assassinato de Rabin, exigiu que o movimento nacional
religioso ou o governo investigassem o uso dos termos por vários
rabinos. Ben"-Nun pagou um preço pessoal por atuar contra os membros de
seu próprio grupo (foi praticamente boicotado pela comunidade de colonos
na Cisjordânia, onde residia) e sua campanha terminou sendo fútil, mas
lançou uma luz sobre a atividade perniciosa que ajudou a estabelecer a
fundação moral para um assassinato ``legitimado'' por uma suposta
autoridade religiosa.

Em janeiro de 1995, uma carta foi enviada do assentamento de Har Bracha,
localizado na Samaria, para quarenta rabinos ortodoxos em Israel, nos
Estados Unidos, na Bélgica e no Canadá. Aos rabinos era solicitado
responder duas perguntas: tendo em conta os Acordos de Oslo, o primeiro"-ministro
de Israel e os membros de seu governo deveriam ser vistos como
``perseguidores'' de acordo com a lei judaica? E, consequentemente,
deveriam ser alertados de que sua punição havia sido estabelecida? A
carta foi iniciativa de um time de três rabinos liderados pelo radical
Eliezer Melamed, que era na época o rabino de uma pequena yeshivá em Har
Bracha, próxima a Nablus. Dos quarenta rabinos, onze responderam: dois
confirmaram que a ``Lei do Perseguidor'' não se aplicava a Rabin; sete
ofereceram respostas ambivalentes; e dois criticaram duramente os
autores da carta por misturar a lei judaica com política. Um deles
alertou: ``Vocês estão brincando com fogo''.\footnote{Karpin \& Friedman, 
\textit{Assassinato em nome de Deus}, \textit{op}. \textit{cit}., p. 156.}

Na atmosfera daqueles dias não era normal que rabinos radicais da
Samaria envolvessem rabinos da diáspora em suas campanhas. A oposição
parlamentar já havia levado para os Estados Unidos sua campanha contra
os Acordos de Oslo e a possibilidade de um acordo sírio"-israelense. Essa
campanha havia sido orquestrada por David Bar"-Ilan, um ex"-pianista de
concertos, ligado a Netanyahu. Bar"-Ilan tentou criar oposição ao
processo de paz junto à comunidade judaica, fundamentalistas cristãos
norte"-americanos e elementos de direita. Ele e seu time promoveram
a oposição a concessões territoriais na Cisjordânia (a Terra Santa, a
Terra de Israel para os cristãos fundamentalistas) e no Golã. Um acordo
de paz entre Israel e a Síria, argumentavam, requereria o estacionamento
de tropas norte"-americanas para a manutenção da paz no Golã e as
colocaria em perigo. Alguns rabinos ortodoxos radicais nos Estados
Unidos, sem ficar muito atrás de seus colegas da Cisjordânia, opunham"-se
e denunciavam qualquer concessão territorial na Terra de Israel. O
rabino Abraham Hecht, do Brooklyn, Nova York, declarou em setembro de
1995 que ``aquele que cede partes da Terra de Israel, o que primeiro o
matar será recompensado, mas eu não fui porque ele ainda está vivo''.
Vernon Kurtz, o rabino líder do Chabad na Flórida, decretou em outubro de 1995
que ``Rabin está qualificado como inimigo e portanto está sujeito à
regra de agir rapidamente para matar a pessoa que venha te
matar''.\footnote{Carmi Gillon, \textit{A Shin Bet entre os cismas}, 
\textit{op}. \textit{cit}., p. 240--1.}

No verão de 1995, a incitação e denúncia de Rabin foram transformadas em
uma chamada direta para o assassinato do primeiro"-ministro. Poucos meses
foram necessários para que fosse atendida e o homicídio fosse
perpetrado.

\section{A cadeia de eventos}

Na cadeia de eventos que culminou com o assassinato de Rabin,
destacam"-se três episódios que o anunciavam iminente. Em março de 1994,
próximo à cidade de Raanana, ao norte de Tel Aviv, uma marcha de protesto
foi organizada pelo Kahane Chai, o movimento fundado pelo rabino Kahane.
Netanyahu foi visto marchando à frente de um caixão no qual estavam
inscritas as palavras ``assassino do sionismo''. Em frente a Netanyahu
marchava uma pessoa que carregava uma forca. Era um exemplo extremo da
premonição anunciada. Quando questionado sobre sua participação no
evento, Netanyahu alegou que ``somente passava por lá'' e não estava
ciente do caixão e da forca.\footnote{Karpin \& Friedman, 
\textit{Assassinato em nome de Deus}, \textit{op}. \textit{cit}., p. 253.}

Em 5 de outubro, o dia da votação no Knesset sobre Oslo \textsc{ii}, o Likud
organizou uma maciça manifestação na praça Zion, em Jerusalém, com a
participação de cem mil pessoas. A liderança do Likud posicionou"-se na
sacada de um hotel local, com bandeiras do Estado de Israel, do Likud e
do grupo Kahane Chai hasteadas pela multidão, ao lado de cartazes que
retratavam Rabin vestido com um uniforme da \textsc{ss}. A manifestação naquele
momento transformou"-se em um tumulto, e a multidão começou a cantar
``Morte a Rabin!''. Em um determinado momento, a imagem de Rabin em um
uniforme da \textsc{ss} foi projetado na parede atrás de onde as pessoas
discursavam. Os oradores estimulavam a multidão com discursos
inflamados, alertando para os perigos inerentes dos Acordos de Oslo e
detonando o governo que os havia assinado. Netanyahu fez um discurso que
enfatizava o caráter não judaico da maioria em que se apoiava o governo
--- o fato de que se baseava nos votos dos árabes israelenses. Um líder
moderado do Likud, David Levi, foi vaiado quando abandonou, enojado, a
manifestação. Outros líderes moderados, como Dan Meridor, também se
retiraram. A multidão estava extasiada e violenta. A principal
reclamação contra Netanyahu não alegava, na verdade, que seu discurso era
censurável, mas sim por deixar de informar àquela multidão que sua
conduta era inaceitável. A omissão de Netanyahu a legitimava. O caos
gerado pela multidão continuou próximo ao Knesset. O ministro do
Trabalho, Fouad Ben Eliezer, foi atacado pela turba quando ia para o
Knesset. Uma vez dentro do edifício, encontrou Netanyahu e lhe disse:
``Você tem que conter o seu pessoal, caso contrario isso terminará em
assassinato. Eles já tentaram me matar''. Netanyahu respondeu com um
sorriso envergonhado, e Ben Eliezer continuou: ``Sugiro que você apague
esse sorriso da sua face. Seus seguidores são loucos. Se alguém for
morto, você será responsável''.\footnote{\textit{Ibid}., p. 135.}

Em 10 de setembro de 1995, Rabin foi ao Instituto Wingate, próximo a Tel
Aviv, para um encontro promovido pela Associação de Imigrantes dos
Estados Unidos e Canadá. Estavam presentes vários extremistas de
direita, como costumam ser alguns dos imigrantes. Um deles, o rabino
Natan Ophir, funcionário da Universidade Hebraica, chegou muito perto de
Rabin, ofendendo"-o e brigando com seus guarda"-costas. Era uma clara
indicação da ineficiência dos protetores de Rabin.

A única coisa necessária para converter esse potencial explosivo em um
assassinato era uma pessoa. E essa pessoa era Yigal Amir. Na época, ele
tinha 24 anos e era estudante de direito da Universidade
Bar"-Ilan, uma instituição de orientação nacional"-religiosa. Filho de
pais ortodoxos que haviam imigrado do Iêmen, vivia em Herzlia, ao
norte de Tel Aviv. A juventude e educação de Amir refletiam a diluição
da distinção entre os Haredi (ultraortodoxos) e o judaísmo
nacionalista"-religioso. Tendo recebido uma educação religiosa
tradicional, ele completou seu serviço militar em uma unidade de combate
e foi então aceito na Universidade Bar"-Ilan para estudar direito e
ciências da computação. Amir era um crente, um fanático teimoso que se
associou e foi influenciado pelos mais radicais defensores da ligação
messiânica com a Terra de Israel. Ele frequentava manifestações da
direita e logo chamou a atenção da divisão judaica do
\textsc{gss},  
o \textsc{fbi} israelense. Amir se confraternizava com, entre outros, Avishai Raviv, o controverso agente provocador do \textsc{gss}, que operava dentro do
núcleo radical da direita. Ele era o único agente que o \textsc{gss} fora capaz
de infiltrar nesse grupo e, para construir e manter sua credibilidade,
promovia e participava de atividades ilegais. Sua conduta problemática
tem sido explorada pelos radicais de direita que, desde novembro de
1995, propagaram uma série de teorias da conspiração alegando que o \textsc{gss}
havia sido a responsável pelo assassinato de Rabin.

Amir decidiu matar Rabin em setembro de 1993, após assistir à cerimônia
de assinatura dos Acordos de Oslo no jardim da Casa Branca. Levou dois
anos para que sua decisão maturasse, transformando"-se em uma resoluta
determinação para agir, e para que a oportunidade surgisse. Em três
ocasiões Amir levou sua arma ao local em que Rabin deveria estar
presente, mas, por várias razões, não se arriscou. Em uma manifestação pela
paz, a ser realizada em Tel Aviv em 4 de novembro de 1995, na grande
praça ao lado da prefeitura, haveria um discurso de Rabin e pareceu a
oportunidade perfeita para Amir. A manifestação foi um sucesso, com a
presença de um enorme número de apoiadores. O tamanho da multidão, o
entusiasmo, o apoio e simpatia que Rabin recebeu agradaram ao primeiro"-ministro,
tão criticado. Após proferir um breve discurso, Rabin, feliz e
acompanhado de um Peres exultante, preparava"-se para deixar o local e
começou a descer o pequeno lance de escadas que separava o terraço da
prefeitura em frente à praça (agora denominada Praça Rabin) dos carros.
Enquanto descia os degraus, Rabin disse a Peres que voltaria ao pódio na
sacada para agradecer apropriadamente aos organizadores. Peres decidiu
continuar e dirigiu"-se a seu carro. Na base da escada, em teoria uma
``área estéril'', estava Amir, armado com sua pistola. Um vídeo filmado
por um amador que vivia nas proximidades e estava presente do outro lado da rua 
mostra o rosto de Amir, aparentemente ponderando um dilema: ele poderia
matar Peres, mas então não seria capaz de matar Rabin. E este último era a
chave, o líder capaz de fazer com que Oslo funcionasse. Amir poupou
Peres. Poucos minutos depois, aproveitou a oportunidade e disparou três
tiros nas costas de Rabin --- que estava praticamente desprotegido.

Como pôde alguém como Yigal Amir, conhecido do \textsc{gss}, aproximar"-se tanto
do primeiro"-ministro para ser capaz de assassiná"-lo? Israel é
um país profundamente familiarizado com o terrorismo e a violência e no
qual as medidas de segurança são conhecidas. O sucesso de Amir pode ser
explicado pelo impacto de uma mentalidade, por uma série de acidentes e
falhas e por absoluta incompetência. A mentalidade comum na época,
compartilhada por Rabin, ditava que somente árabes/palestinos
representavam o risco de um atentado contra a vida do primeiro"-ministro
--- acreditava"-se que ``um judeu não assassinaria outro judeu''. Apesar
da feroz incitação por parte dos críticos e opositores de Rabin, ninguém
acreditava que haveria um atentado contra sua vida. Houve tentativas de
alterar essa mentalidade; na primavera de 1995, e no verão do mesmo ano,
o chefe da divisão judaica do \textsc{gss} alertou sobre o perigo de uma
tentativa de assassinato contra Rabin e, especificamente, que poderia
ser cometida dentro da ``linha verde'' (Israel da fronteira pré"-junho de
1967) e não por um dos colonos. Os alertas infelizmente não foram
transformados em ações pela unidade do \textsc{gss} responsável pela segurança
\textsc{vip}. O próprio Rabin se recusou a utilizar um colete à prova de balas,
preferiu não fazer uso do Cadillac blindado que foi adquirido para
garantir a sua segurança e insistiu em participar dos eventos que
envolviam multidões, misturando"-se com o público, como sempre havia
feito.

\section{A repercussão}

Mais de vinte anos depois, a repercussão do assassinato de Rabin tem
sido marcada pelas eleições parlamentares de maio de 1996 que levaram
Netanyahu ao poder. Os seis meses que separam a morte de Rabin da
transferência de poder ao líder da oposição que buscou minar a Rabin e
suas políticas formam uma unidade temporal que dota o homicídio de
seu impacto de curto prazo. A transferência de poder ampliou o impacto
do assassinato como um marco da deriva de Israel rumo à direita e de seu
afastamento, ao menos temporário, da política de paz arquitetada por
Rabin. Não deixa de ser uma amarga ironia que ele tenha sido substituído
pelo líder da oposição que, ainda que não tenha participado diretamente
da incitação contra ele, tampouco dela se afastou. O fracasso do
\textit{establishment} do centro e da esquerda de conduzir o país através
de um verdadeiro exame de consciência após o trauma nacional de 4 de
novembro de 1995 também contribuiu para esse desfecho e, eventualmente,
permitiu à direita radical, aos colonos e seus aliados em Israel
manterem suas posições, reagrupar"-se e dominar a formulação da política
no país.

Essas mudanças radicais não ficaram aparentes de imediato. O país estava
em choque, tomado pela dor. Havia uma sensação generalizada de crise e
perigo, de uma potencial guerra civil. O principal setor da direita
adotou um modo defensivo e arrependido. O esforço central de Netanyahu
naquele momento foi feito para distanciar"-se da direita radical e dos
círculos associados ao assassinato. O campo religioso sionista em Israel
e nos assentamentos fez algum exame de consciência --- em alguns casos
sincero; em outros, simbólico. A centro"-esquerda dividiu"-se entre a
vontade de punir o setor associado ao assassino (e o temor de uma
guerra civil) e a percepção de que cerrar fileiras e buscar a união
nacional era o certo a se fazer.

Mas, no momento que se seguiu à atrocidade, Israel estava sem dúvida
tomado pela tristeza e pelo luto. O corpo de Rabin foi velado em frente
ao Knesset e dezenas de milhares de pessoas de todo o país vieram se
despedir dele. Quase tudo na vida e na carreira de Rabin havia
acontecido tardiamente, e isso também se refletia na ligação emotiva do
público com ele. Em seu auge, ele foi respeitado, mas não amado pelo
povo e, durante suas últimas horas, foi tomado pela recepção
calorosa que o envolveu na praça da prefeitura em Tel Aviv. Após o
assassinato, o calor se transformou em uma onda emotiva de amor,
adoração e uma profunda sensação de perda. Milhares de jovens, homens e
mulheres, mantiveram uma vigília próximo ao lugar do assassinato e à sua
casa, portando velas: a ``geração das velas'', como veio a ser
conhecida. Eles choraram a sua morte e sentiam que havia se interrompido
o avanço do país em direção a um futuro melhor.

O funeral de Rabin foi um evento impressionante. Setenta e oito países
se fizeram representar, um número enorme para um país acostumado ao
isolamento. Dois chefes de Estado árabes estavam presentes: o rei
Hussein da Jordânia, amigo de Rabin, e o presidente Mubarak do Egito.
O Egito havia mantido a paz com Israel desde 1979, mas um presidente
egípcio não havia visitado Israel desde a histórica jornada de Sadat em
novembro de 1977, um sintoma da ``paz fria'' entre os dois países. A
maciça presença internacional refletia a importância de Rabin e era
também parte de um esforço realizado para que Peres se mantivesse na
rota rumo à paz. Dos pronunciamentos em homenagem a Rabin durante a cerimônia
destacaram"-se o emotivo e eloquente tributo do presidente Clinton --- com
o proverbial ``Shalom Haver'' {[}``Adeus, meu amigo''{]} --- assim como o da
neta de Rabin, Noa, cujo discurso enfatizou a calidez pessoal de um
líder que muitas vezes havia sido visto como duro e autoritário.

Foram ofuscadas por esses ápices emotivos as decisões e medidas tomadas
para colocar o país e suas políticas em um rumo estável. Peres
substituiu Rabin como primeiro"-ministro e tomou três importantes
decisões inter"-relacionadas no início de seu mandato: não realizar
eleições imediatamente logo após o assassinato; tentar obter um rápido
acordo com a Síria, como principal esforço do processo de paz; e buscar
uma acomodação com o setor moderado do sionismo religioso.

Uma eleição imediata teria dado a Peres uma vitória maciça, e
reforçado sua posição como primeiro"-ministro eleito. Mas ele decidiu
manter inalterada a data da próxima eleição parlamentar, marcada para
outubro de 1996. Estava claro que Peres queria ser eleito por seu
próprio mérito e não como o vingador de Rabin. Ele buscou cerrar
fileiras em lugar de acertar contas com a direita radical e seus
apoiadores mais brandos. A decisão de manter a data de outubro para o
pleito parlamentar indicava que Peres esperava realizar um
movimento importante no processo de paz, cujo sucesso serviria de base para sua campanha eleitoral. A eleição referendaria o novo acordo, que poderia ser
assinado com a Síria ou com a Autoridade Palestina.

Essas decisões foram logo implementadas através de novas políticas. A
comissão chefiada pelo juiz Meir Shamgar, ex"-presidente da Suprema
Corte, recebeu a missão de investigar as falhas na segurança que
permitiram o assassinato de Rabin. O governo solicitou à comissão,
especificamente, não lidar com a incitação ou com os antecedentes
políticos do assassinato, mas priorizar as questões de segurança.
Tendo em conta essa abordagem, o rabino Yehuda Amital, um líder sionista
religioso moderado e muito respeitado, foi convidado a juntar"-se ao
gabinete. As negociações entre Israel e a Síria foram retomadas, com
Peres e o governo Clinton buscando um rápido avanço. Mas, no inverno de
1996, as coisas começaram a desandar: as negociações retomadas com a
Síria avançavam lentamente; Assad não tinha pressa em concluir um acordo
com um primeiro"-ministro que tinha de ser reeleito em outubro de 1996.
Quando Peres percebeu que não conseguiria chegar a um acordo com a Síria
a tempo das eleições de outubro, decidiu adiantá"-las para maio de 1996.
Assad, irritado e transtornado com a suspensão \textit{de"-facto} das
negociações, permitiu --- ou talvez tenha encorajado --- o Hezbollah a
reiniciar o lançamento de foguetes contra o norte de Israel, fisgando
Peres, que lançou, em abril de 1996, a operação Vinhas da Ira contra
as bases do Hezbollah no sul do Líbano. Não foi uma operação muito bem"-sucedida
e se encerrou tragicamente quando um obus da artilharia
israelense atingiu e matou, por engano, um grupo de refugiados civis na
aldeia de Kufr Kana. O evento foi precedido de uma onda de ataques
suicidas perpetrados pelo Hamas em Tel Aviv e Jerusalém, nos quais
morreram 59 israelenses em uma semana.

A sequencia de trágicos eventos cobrou um alto preço político. Peres,
que liderava por ampla margem nas pesquisas contra Netanyahu poucas
semanas antes das eleições, perdeu a maior parte de seu apoio. Eleitores
judeus ficaram indignados com a onda de ataques terroristas, enquanto os
eleitores árabes foram alienados pelo incidente de Kufr Kana. A campanha
de Peres e do Partido Trabalhista foi mal"-administrada. Para dar um
exemplo, o debate televisivo entre Peres e Netanyahu, que terminou
favorecendo o apoio a este último, foi mal preparado pelo time dos
trabalhistas. Em maio, Netanyahu e o Likud venceram as eleições, e Israel
começou a se afastar do caminho trilhado por Rabin.
