\chapter{Prefácio à edição brasileira}

\begin{flushright}
\textsc{samuel feldberg}

\textsc{debora fleck}
\end{flushright}


\noindent{}Novembro de 2020 marca o vigésimo quinto aniversário do assassinato de
Yitzhak Rabin. Como vem acontecendo em todos os anos desde então, a
chamada ``geração das velas'' se reunirá para lembrar aquele que foi um
ícone da história do jovem Estado.

Conheci o professor Itamar Rabinovich quando ele era presidente da
Universidade de Tel Aviv e já tinha escrito diversos livros que tratavam
das negociações entre Israel e a Síria, nas quais foi um dos principais
protagonistas. Quando o livro \textit{Yitzhak Rabin} foi originalmente publicado 
em inglês, não tive dúvida de que
seria um título obrigatório nas estantes daqueles que se interessam por
Israel e pelo Oriente Médio.

Rabin se tornou um mito da esquerda e de todos que idealizam o processo
de paz por ele iniciado. Mas, como bem descreve Rabinovich, Rabin foi o
político que fez a transição entre a velha e a nova guarda na política
israelense. Como militar, desde o início de sua carreira foi duro com os
árabes durante a Guerra de Independência e também na Primeira Intifada,
quarenta anos depois. Apesar disso, soube ganhar o apoio deles para
aprovar os Acordos de Oslo, que lhe custaram a vida.

Em um livro breve e completo, Rabinovich soube descrever os principais
dilemas vividos por Rabin, seus fracassos e sucessos. Como embaixador em
Washington, pôde colocar"-se na função exercida antes pelo próprio Rabin.
Mais tarde, tornou"-se membro da equipe do
primeiro"-ministro, com quem trabalhou intimamente tentando negociar a paz
com a Síria.

Os aniversários da morte de Rabin e dos Acordos de Oslo sempre renovam a
dúvida do que teria ocorrido se ele tivesse conseguido cumprir sua
missão. E esta biografia nos deixa mais perguntas do que respostas.
Rabin era cauteloso, meticuloso e, embora fosse consciente de que Israel não poderia continuar a dominar milhões de palestinos, em nenhum momento ofereceu a Yasser Arafat concessões como as que seriam feitas por Ehud Olmert e Ehud Barak. Cria de Yigal Alon, foi profundamente influenciado pelo Plano Alon, tão em pauta nesses dias em que se discute o Plano 􏰃Trump ou a paz com os Emirados Árabes Unidos.

Por ter convivido com Rabin, o professor Rabinovich, além de nos
oferecer uma fascinante jornada através da história de Israel, nos
garante também uma visão única sobre a vida de um de seus maiores
estadistas.

Foi um enorme prazer realizar esta tradução, garantindo assim que o
público brasileiro tenha acesso a uma obra tão importante.

Agradecemos vivamente ao apoio de Jorge Feffer, que possibilitou a
tradução e publicação deste título.

%\bigskip

%\hfill{}São Paulo, setembro de 2020

