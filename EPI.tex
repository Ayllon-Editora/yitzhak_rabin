\chapter[Epílogo\medskip]{Epílogo}

Logo após o vigésimo aniversário do assassinato de Rabin, em 26 de
outubro de 2015, Benjamin Netanyahu dirigiu-se ao Comitê de Defesa e de
Relações Exteriores do Knesset. Ele se referiu a Rabin, dizendo:
``nesses dias muito se fala sobre o que teria acontecido se essa ou
aquela pessoa ainda estivessem aqui {[}\ldots{}{]} é irrelevante; nós sempre
viveremos em guerra''.\footnote{\emph{Haaretz}, 26 out. 2015.} Netanyahu estava claramente incomodado
pela onda de nostalgia por Rabin, gerada pelo aniversário de seu
assassinato, e que incluía uma série de filmes e documentários, programas
de televisão, livros, artigos na imprensa e uma grande manifestação na
qual Bill Clinton foi o principal expoente. A nostalgia era
implicitamente e, ocasionalmente, até explicitamente, crítica de
Netanyahu --- que no momento está solidamente no poder, em seu quarto
mandato, sendo três consecutivos, e não há nenhum rival à altura. Mas
está ciente da insatisfação que afeta parte significativa do povo
israelense: esta é produzida pela falta de contentamento com um líder
que prospera mantendo o \emph{status quo}, evitando decisões importantes;
um mestre da sobrevivência política que não foi capaz de mostrar uma
capacidade de liderança à altura de Rabin para lidar com os principais
problemas que afetam Israel.

A declaração de Netanyahu, na verdade, representa perfeitamente o choque
de narrativas sobre a herança, o legado e a memória de Rabin. Uma das
narrativas refere-se ao debate a respeito do impacto de Yigal Amir sobre
o processo de paz árabe-israelense na década de 1990. O que teria
acontecido se Yigal Amir não tivesse tomado a sua fatídica decisão de
assassinar Rabin? Pode-se argumentar que as balas do assassino não
mataram somente Rabin, mas também o processo de paz que ele conduzia.
Essa narrativa combina principalmente com a opinião do campo liberal, à
esquerda do centro em Israel e no exterior, que alega que, se Rabin não
tivesse sido assassinado, teria vencido as eleições programadas para
outubro de 1996 e chegado a um acordo com Arafat nas negociações para um
acordo final, com término previsto para maio de 1999. Essa opinião foi
articulada da forma mais eloquente por Thomas Friedman, colunista do New
York Times, em um artigo intitulado ''Relações Exteriores; \ldots{} E alguém
votou duas vezes''. Netanyahu, por outro lado, articulava a ideia da
direita israelense segundo a qual o processo de paz estava, de qualquer
maneira, destinado ao fracasso, porque Arafat não era um parceiro genuíno
para a paz e tinha seus próprios limites, que não estava disposto a
ultrapassar; e que, no momento decisivo, as negociações fracassariam. O
conflito israelo-palestino, segundo essa visão, é um conflito nacional
interminável e insolúvel.

Qual das narrativas é, então, mais plausível? Lidar com esses pontos de
vista opostos representa um exercício de historia alternativa. A
hipótese de que Rabin teria vencido as eleições de 1996 é bastante
realista, muito mais do que a que propõe que ele teria chegado a um acordo
com Arafat. Rabin era crítico da derrota de Arafat em cumprir com seu
compromisso, especialmente no que se referia à repressão do Hamas.
Rabin pode ter tido suas dúvidas sobre o comprometimento total de Arafat
com a solução de dois Estados, mas esperava testá-lo no momento
apropriado. Mas é provável que, na falta de um acordo final, Rabin
pudesse alcançar objetivos menos ambiciosos e evitar um embate frontal da
intensidade da Segunda Intifada que eclodiu após a cúpula de Camp David,
em julho de 2000. A viabilidade desse cenário alternativo é composta de
vários fatores históricos e eventos que de fato ocorreram: a vitória de
Netanyahu em 1996 se deu por margem estreita e certamente poderia ter sido
evitada; o processo de paz continuou em 1996, com a evacuação, por parte
de Netanyahu, da cidade de Hebron, cumprindo o acordo Oslo \textsc{ii} e
concordando (ainda que sem implementar), em 1998, em entregar aos palestinos
mais 13\% da Cisjordânia; Ehud Barak, em 2000, e Ehud Olmert, em 2008,
fizeram propostas generosas aos palestinos, muito além daquilo que Rabin
havia oferecido em outubro de 1995; e Ariel Sharon, quem diria, retirou
Israel de Gaza de forma unilateral em 2005 e estava disposto a fazer
retiradas unilaterais menores na Cisjordânia antes de sofrer um derrame.

Apesar disso, pode-se dizer que Amir realmente infligiu um pesado
golpe contra o processo de paz dos anos 1990. Rabin combinava, de
maneira única, a determinação de avançar com o processo de paz como
especialista na área de segurança nacional israelense, com o apoio e a
confiança de grande parte da nação, o respeito de Arafat e
uma excelente relação com o presidente Clinton. Sua remoção da cena --- e
eventual substituição por Netanyahu, que prometeu respeitar os Acordos
de Oslo, mas na verdade emasculou o processo dessas tratativas --- interrompeu o
processo histórico iniciado em 1992 e destruiu sua inércia. Um esforço
concentrado para solucionar os conflitos entre Israel e a Síria e os
palestinos foi retomado por Barak em 2000, mas o hiato de quatro anos
teve um efeito devastador sobre um processo inerentemente frágil. Israel
e os palestinos ainda estão presos ao conflito.

O assassinato também teve um profundo impacto na sociedade e na política
israelenses. Desde 1977 o espectro político israelense tem se desviado
para a direita. Com exceção de 1992-1996 e 1999-2001, o Partido Trabalhista
não conseguiu ganhar eleições e manter-se no poder. Sharon e Olmert, que
estiveram no poder de 2001 a 2008, moveram-se da direita para o centro e
buscaram, de diferentes maneiras, solucionar ou ao menos consolidar as
relações israelenses com os palestinos. Mas, durante a maior parte dos
quase quarenta anos desde 1977, o poder esteve nas mãos do Likud, apoiado por
seus ``aliados naturais'' (na nomenclatura política israelense): os
colonos e os partidos ortodoxos e ultraortodoxos. As vitórias
eleitorais de Rabin e Barak, em 1992 e 1999, demonstraram que os
trabalhistas podiam vencer somente quando liderados por um líder do
centro; um líder competente, com credenciais na área da segurança, que
pudesse garantir ao ansioso eleitorado israelense que era capaz de
avançar em direção à paz, sem minar a segurança. Ao comemorar-se o
vigésimo aniversário do assassinato, estava claro que o centro-esquerda
israelense teria dificuldade em desafiar Netanyahu e seus aliados da
direita.

A morte de Rabin paira como um marco importante na guinada de
Israel rumo à direita, distanciando-se de uma genuína busca pela solução
de dois Estados. Entre os eventos que aceleraram e aprofundaram a
direção da jornada estão o fracasso do Estado israelense em punir o
grupo mais amplo que incitou e clamou pelo assassinato de Rabin; a falta
de um exame de consciência por parte da sociedade após o homicídio;
o fato de que os sionistas religiosos tampouco tenham feito esse exame de consciência; 
que um limite tenha sido cruzado com o assassinato de um primeiro"-ministro; que
o assassino e seu grupo tenham sido, na verdade, recompensados pelo
crime.

Por vezes, argumenta-se que a liderança dos colonos ficou chocada com o
assassinato e moderou seu comportamento; e que sob o impacto do crime
conteve os seus seguidores quando, em 2005, Sharon evacuou e destruiu os
assentamentos em Gaza, para evitar o derramamento de sangue e uma
potencial guerra civil. Pode ser que sim. Alguns argumentam que a contenção
exercida pela liderança dos colonos em 2005 levou a uma futura
radicalização do grupo nos anos seguintes. De qualquer forma, o projeto
de assentamentos na Cisjordânia continua a se expandir e a liderança
tradicional já foi superada, à direita, por grupos messiânicos,
violentos. Esses elementos podem não ser todos resultantes do
assassinato de Rabin, mas são definitivamente produto da abertura das
comportas em 4 de novembro de 1995 e nos meses que o antecederam.

Após o assassinato havia aqueles, tanto na direita quanto na esquerda,
que esperavam que o ato de violência tivesse um efeito unificador. Eles
argumentavam que o temor da divisão e da ruptura levaria à união dos dois campos
em torno do legado de Rabin e da determinação de extirpar a
violência política. Mas nada disso aconteceu. Em lugar disso, alguns
elementos da esquerda acreditavam que havia contas a acertar, insistiam
em que o verdadeiro legado de Rabin seria sua política de paz e
que ele deveria ser lembrado e louvado por sua luta para solucionar a
questão palestina. Para a direita esse era um legado inaceitável, e a
memória de Rabin e sua celebração tornou-se um tema de confronto em uma
Israel profundamente dividida.

Quando presidentes e primeiros"-ministros de Israel falecem, eles são
homenageados pelo Estado de Israel através do Conselho para a Celebração
de Presidentes e Primeiros"-Ministros, um órgão do gabinete ministerial
liderado primeiro"-ministro em exercício. As comemorações
são, naturalmente, diferentes e em distintas escalas. Para três dos
primeiros"-ministros, leis especiais foram votadas no Knesset. O primeiro
foi o fundador de Israel, David Ben Gurion. Em 1996, uma lei foi aprovada
para lembrar a Rabin e apoiar a criação do Centro Rabin e sua
manutenção através de um orçamento estatal. Também foi determinado que
uma cerimônia estatal em sua memória seria realizada em seu túmulo no
aniversário de seu assassinato, assim como uma sessão em sua homenagem
no Knesset. Essas decisões tiveram amplo apoio, um reflexo da percepção
de que o assassinato de um primeiro"-ministro tem de ser reconhecido e
relembrado em um nível distinto de outras mortes. Independentemente da
decisão do governo, várias instituições públicas foram nomeadas
homenageando Rabin.

Tendo em conta que as comemorações em homenagem a Rabin foram assumidas
pelo Estado, que tem sido controlado pela direita na maior
parte dos últimos vinte anos, a tensão subjacente envolvendo a memória
do primeiro"-ministro assassinado emergiu em diversas ocasiões. Tem sido difícil, portanto, para
a família e os seguidores de Rabin, que mantêm a chama acesa, aceitar
Netanyahu como um dos oradores na cerimônia realizada todos os anos
junto a seu túmulo. Na edição de 2014, Rachel, irmã de
Rabin, uma mulher impressionante por seus próprios méritos, dirigiu, em nome de sua família, um
comentário ácido a Netanyahu quando falou depois dele:
``Nós lembramos quem estava naquele terraço'', disse
ela, referindo-se à violenta manifestação de outubro de 1995. Ainda
assim, Netanyahu e outros líderes da direita encontraram uma forma, e são
obrigados a fazê-lo, de atender aos eventos em memória do falecido,
reconhecendo também a necessidade de manter o Centro Rabin.

Lidar com a memória de Rabin no sistema educacional israelense tem sido
outra fonte de tensão, que irrompeu fortemente em 2016 quando o conteúdo
de um novo texto escolar foi vazado para a imprensa. O projeto havia
sido iniciado cinco anos antes e se prolongou em um Ministério da
Educação que teve três ministros: um de direita, um de centro e um
radical de direita. Naftali Bennett é o líder da nova encarnação do
partido dos colonos, o Lar Judaico, e o material foi finalizado sob seu
comando. Nele, o assassinato de Rabin é tratado de duas maneiras: em uma
parte é inserido no contexto da história da violência política em Israel
e no período anterior à criação do Estado, juntamente com o caso
\emph{Altalena} e o assassinato do pacifista Emil Grinzweig. Conforme
mencionado anteriormente, a responsabilidade pelo caso \emph{Altalena}
alternou-se ao longo dos anos entre Ben Gurion e Rabin, e a justaposição
do seu assassinato com aquele episódio gerou uma tormenta de protestos.
Em outro segmento do material, os autores citam duas opiniões
relacionadas com a incitação que precedeu o assassinato, uma delas de
um ex"-procurador'-geral que alegou não haver prova legal que associasse o
assassinato à incitação. Essa citação em particular, e a equivalência
moral criada ao citar duas perspectivas opostas sobre a incitação,
geraram outra onda de ruidosos protestos. Em função disso, a publicação
da apostila foi suspensa temporariamente, mas o evento refletiu a
influência obtida pela direita sobre o aparelho do Ministério da
Educação, assim como o constante desconforto da direita radical com o
assassinato e com o papel da incitação a ela atribuído por seus
detratores.

Ainda assim está claro que, ao longo de mais de vinte anos, a veneração
e a sensibilidade da direita em relação à morte de Rabin tem
diminuído, especialmente por parte de Netanyahu e do Likud. Nos
primeiros meses após o assassinato, Netanyahu sentiu-se vulnerável em
relação à acusação de que tivesse tido parte, direta e indiretamente, na
incitação. Ele se preocupava especialmente com a possibilidade de que isso
afetasse suas chances nas eleições de maio de 1996. Sua vitória nas
eleições e sua subsequente capacidade de voltar ao poder e mantê-lo
reduziram sua sensibilidade. Ele tampouco é totalmente indiferente a
comparações com a liderança e estatura de Rabin, que lançam uma luz
negativa sobre a avaliação de seu governo; mas, no contexto geral, esta já
não é uma preocupação central. Por outro lado, os colonos e o sionismo
religioso em geral têm demonstrado mais sensibilidade às acusações de
terem sido os responsáveis pelo assassinato. Eles respondem
argumentando que é errôneo responsabilizar todo um setor da população
israelense pelos atos criminais de uma pequena minoria, defendendo uma
teoria da conspiração que utiliza o papel do agente provocador, Avishai
Raviv, como uma espécie de prova, e explorando o caso \emph{Altalena}
para apresentar Rabin como alguém responsável pelo assassinato de outros
judeus. Nesse contexto, o próprio Amir, durante seu interrogatório,
disse: ``Rabin foi responsável pelo disparo de canhões e pelo
afundamento do navio \emph{Altalena} que, em 1948, levava armas para as
forças do Irgun na Terra de Israel''.\footnote{Citado por Amnon Barzilai 
em \emph{Haaretz}, 14 jun. 2013.} O fracasso do sionismo
religioso em fazer seu exame de consciência após o atentado tem
perseguido esse segmento da população judaica em Israel. O cerne da
liderança dos colonos foi abalado em 2015, quando tiveram que lidar com
elementos revolucionários em sua própria ala direita. Foram
constrangidos ao descobrir que violentos elementos radicais afiliados ao
movimento Kahane não só haviam cometido crimes contra os palestinos na
Cisjordânia mas também haviam criado uma organização clandestina que
visava substituir o Estado de Israel por aquilo que denominavam o Reino
de Israel.

As desavenças entre os admiradores de Rabin sobre a forma apropriada de
lembrá-lo são menos dramáticas, mas existem. Elas se manifestam nas
discussões sobre a função do Centro Rabin, o principal memorial, e sobre
que caráter deve ter a manifestação anual realizada na Praça Rabin no
dia do assassinato. Os organizadores desse evento, financiado por
doadores privados, têm lhe dado uma orientação fundamentalmente
política, tentando transformá-lo em pressão por uma nova iniciativa de
paz. Outros têm alegado que o seu conteúdo político deveria ser atenuado
e que Rabin deveria ser celebrado como um grande líder nacional com um
amplo apelo para todo o público israelense.

Efetivamente, é equivocado lembrar e celebrar Rabin como um líder
moderado. Ele era um líder de centro, preocupado com a segurança
israelense, e chegou à conclusão de que o país deveria buscar moderar e
eventualmente resolver o conflito com seus vizinhos árabes. Para ele, a
busca pela paz estava intimamente conectada à busca pela segurança. Ele
estava disposto a fazer concessões dolorosas mas examinava tais
concessões através da lente da segurança. Sua disposição em atendê-las
foi difamada pelos críticos da direita e no final resultaram em seu
assassinato. A esquerda, por sua parte, tendia a apresentá-lo como muito
mais moderado do que realmente era. Ele foi um estadista que quis por
fim aos conflitos israelenses com seus vizinhos através de um acordo de
paz, baseado em sólidos mecanismos de segurança. Durante um evento em
sua memória, realizado em Boston logo após o assassinato, até Henry
Kissinger, amigo e admirador de Rabin, sentiu-se incomodado enquanto um orador
seguido de outro orador homenageavam um Rabin moderado, louvando sua absoluta devoção
às negociações de paz com os palestinos. ``Yitzhak não era ingênuo'',
murmurou ele baixinho.

Em algum momento, considerei como um sub-título para esse livro uma linha
escrita pelo poeta Shaul Tchernichovsky: ``A imagem de sua paisagem
nativa''. Rabin era, em muitos aspectos, o \emph{Sabra} por excelência, o
israelense nato: infância e adolescência no cerne do movimento
trabalhista, a escola Kadoorie, o Palmach, a geração de 1948, uma
fachada rude escondendo uma sensibilidade interior.

A guerra de 1948 foi a experiência de formação da vida de Rabin e ele
dedicou os dezenove anos seguintes de sua vida à construção de uma poderosa
força armada que levou à espetacular vitória de junho de 1967. O pequeno
e vulnerável Estado de Israel, que emergiu da guerra de 1948, havia se
transformado na poderosa Israel pós"-1967. Mas a vitória tinha um preço.
Durante os últimos mais de quarenta anos, Israel tem se beneficiado de
algumas das conquistas daquela guerra, mas também teve de lidar com
seus outros resultados, principalmente a nova encarnação da questão
palestina. E o país também mudou. A ``paisagem nativa'' que moldou o
jovem Rabin estava desaparecendo. Seu segundo mandato também pode ser
visto como o valente esforço de um soldado, transformado em estadista,
em usar o território que capturou em 1967 para consolidar, economizar e
preservar aquela paisagem original. Seu assassinato tornou-se um passo
crucial no caminho para transformá-la.

\chapter{Agradecimentos}

Sou grato à Yale University Press por me encomendar a biografia de
Yitzhak Rabin. Trabalhei muito próximo a Rabin e acredito tê"-lo
conhecido bem. Entretanto, enquanto pesquisava e escrevia este livro,
aprendi muito mais sobre sua vida, sua carreira, seu caráter e o real
significado da perda sofrida por Israel em 4 de novembro de 1995.

Quero agradecer especificamente ao diretor da Yale University Press, sr.
John Donatich; ao sr. Leon Black, que iniciou a série sobre vidas
judaicas; aos três editores gerais da série, profa. Anita 
Shapira, sra. Ileene Smith, e prof. Steven Zipperstein; à minha editora,
sra. Erica Hanson; à sra. Marika Lysandrou, sra. Margaret Otzel, sra.
Heather Nathan e sra. Elizabeth Pelton.

Minha pesquisa foi facilitada por três membros da família Rabin: 
a irmã de Yitzhak Rabin, sra. Rachel Rabin"-Yaacov, e seus filhos, Dalia
and Yuval. Os funcionários do Centro Rabin, 
sra. Dorit Ben"-Ami e sra. Nurit Cohen"-Levinovsky, foram amáveis e
prestativos.

Agradeço especialmente a meus três assistentes de pesquisa: Arik
Rudnitzky, Revital Yerushalmi and Maddy Taras.

A dra. Tamar Yegnes e a sra. Hanne Tidnam muito me ajudaram, assim como
o fizeram em meus livros anteriores. A transcrição de dezenas de
entrevistas mantidas no Centro Rabin representaram uma fonte de
inestimável valor para minha pesquisa. Também me beneficiei de
entrevistas pessoais e conversas com um grande número de indivíduos, aos
quais sou extremamente grato:

Sr. Uri Avneri, gen. Yaacov Amidror, sr. Uzi Baram, sr. Nachum Barnea,
dr. Michael Bar"-Zohar, dr. Yossi Beilin, prof. Haim Ben"-Shahar, gen.
Amos Chorev, sr. Eitan Haber, dr. Yair Hirschfeld, sr. Amos Eran, dr.
Oded Eran, gen. Shlomo Gazit, prof. Moti Golani, sr. Haim Guri, dr.
Martin Indyk, sr. Chezi Kalo, dr. Igal Kipnis, sra. Niva Lanir, sr. Dani
Litani, sr. Dan Margalit, sr. Dan Meridor, sr. Shlomo Nakdimon, sr. Amir
Oren, sr. Amos Oz, gen. Elad Peled, sr. Yaacov Peri, prof. Yoram Peri,
sr. Zvi Rafiah, embaixador Dennis Ross, o finado sr. Yossi Sarid, sr.
Uri Savir, o juiz Meir Shamgar, sr. Shimon Sheves, sra. Ora Teveth, o
finado sr. Dov Tzamir, dr. Hagai Tzoreff, e o sr. Dov Weissglas.

Além do material de arquivo e outras fontes primárias, utilizei
extensamente a literatura secundária sobre a vida e a morte de Yitzhak
Rabin. Nesse contexto, vários autores merecem minha gratidão por sua
contribuição: Dan Efron, Yossi Goldstein, Michael Karpin e Ina Friedman,
Yoram Peri e Robert Slater.

\bigskip

\hfill{}Itamar Rabinovich

\hfill{}Tel Aviv, Agosto de 2016

