\textbf{Yitzhak Rabin} (1922--1995) prêmio Nobel da Paz de 1994, foi um político e general israelense, conhecido principalmente por fazer parte dos Acordos de Oslo (1993). Atuou como ministro da defesa de 1984--1990, embaixador de Israel em Washington de 1968--1973, e primeiro ministro entre 1974--1977 e 1992--1995, este último mandato encerrado por seu assassinato --- quando um judeu nacionalista fanático disparou tiros contra suas costas após discurso em praça pública. Preocupado com o Estado que governava, a busca pela paz vinha ligada à segurança, principalmente à resolução do conflito com os vizinhos árabes. Foi criticado pela direita, que resultou ao final em uma morte violenta, e exaltado pela esquerda, que o apresentava como mais ingênuo do que realmente era, criando uma figura ambígua porém lembrada até hoje como um dos marcos mais próximos de Israel rumo à solução de dois Estados.

\textls[-5]{\textbf{Yitzhak Rabin: uma biografia} é um documento fascinante sobre a vida de Rabin, um israelense nato que cresceu junto à elite pré"-criação do Estado, à escola do movimento trabalhista, ao Palmach, à Guerra da Independência (1948) e de uma ascendente carreira militar. Seu talento e perseverança --- e ocasional golpe de sorte --- o conduziriam ao topo da pirâmide militar, até o mais alto posto: o de primeiro ministro. Apesar de todas as consequências do assassinato político de Rabin, foi sua vida, as ações e decisões, e não sua morte, que definiu seu legado: a política de paz, as decisões ousadas que tomou em relação às negociações com a Síria e à Palestina, e a própria condução de um país.}\looseness=-1

\textbf{Itamar Rabinovich} (1942) foi embaixador de Israel em Washington (\textsc{eua}) na década de 1990 e fez parte da equipe de Rabin durante seu segundo mandato. Escreveu diversos livros que tratam da temática das negociações de paz entre Israel e Síria --- das quais foi um dos principais protagonistas. Foi também presidente da Universidade de Tel Aviv de 1999--2007, onde atualmente é professor emérito de História do Oriente Médio na Universidade, além de professor na Universidade de Nova York e bolsista na Brookings Institution.

\pagebreak
\thispagestyle{empty}

\textbf{Debora Fleck} é formada em Administração (\textsc{puc"-rio}) e mestre em Literatura Brasileira (\textsc{uerj}). Trabalhou por seis anos no mercado editorial antes de se dedicar exclusivamente à tradução e revisão. É sócia e uma das fundadoras da Pretexto, espaço físico e online dedicado a oficinas e palestras sobre o universo da tradução e afins.

\textbf{Samuel Feldberg} é graduado em Ciência Política e História (Universidade de Tel Aviv), além de doutor em Ciência Política (\textsc{usp}). Vive alternadamente entre Israel e o Brasil, e é professor do curso de Relações Internacionais e pesquisador do Centro Moshe Dayan da Universidade de Tel Aviv. É autor de \textit{Estados Unidos da América e Israel: uma aliança em questão} (2008), além de tradutor de \textit{Holocausto} (2010) e \textit{Israel, uma história} (2018). Colabora constantemente com artigos sobre Oriente Médio e conflitos internacionais para veículos de imprensa e é comentador midiático sobre Relações Internacionais.

\textbf{Coleção Índigo} reúne publicações que contribuem para a qualificação do debate público sobre o Estado de Israel, sionismo e judaísmo através de textos que promovem a abertura de caminhos para o diálogo e espaço para a diversidade de opiniões. É fruto de uma parceria com o Instituto Brasil"-Israel (\textsc{ibi}).



